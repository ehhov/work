\documentclass[a4paper, 12pt]{article}

% Configuration {{{
\usepackage[utf8]{inputenc}
\usepackage[T2A]{fontenc}
\usepackage[russian, english]{babel}

\usepackage[
	vmargin=1in,
	hmargin=1in
]{geometry}
\linespread{1.3}
\usepackage{xcolor}
\definecolor{allrefs}{HTML}{1010aa}
\usepackage[
	colorlinks=true,
	allcolors=allrefs
]{hyperref}
\usepackage{indentfirst}
\usepackage{graphicx}
\usepackage[multidot]{grffile}
\usepackage[labelsep=period]{caption}
\usepackage{enumitem}
\setlist{nolistsep}
\usepackage{mathtools}

%\usepackage{times}
\usepackage{multicol}
\usepackage{lipsum}
\setlength{\columnsep}{.25in}

\def\task#1{\begin{center}\it #1\end{center}}
\def\ans#1{\textit{#1}}

\newif\ifcols
\colsfalse
%}}}

\begin{document}

\noindent
Kerim Guseynov, 113M group
\hfill
Dec 18

\task{P. 97--98, translation}
% {{{

\begin{otherlanguage}{russian}
	\textbf{\textit{\large Важность асимметрии}}

	К счастью для нас, вселенная не симметрична как минимум на субатомном 
	уровне. Если бы была, при ее рождении только что образованная материя 
	аннигилировала бы с равным и противоположным количеством антиматерии, 
	и ничего бы не осталось. Вместо этого небольшой дисбаланс или 
	асимметрия в количестве созданного вещества и антивещества привело 
	к небольшому избытку вещества, из которого мы в конечном итоге 
	и образовались. Такая нарушенная симметрия -- ключ к нашему 
	существованию.

	Понимание симметрии или ее отсутствия -- текущая задача, 
	и Нобелевская премия по физике 2008 года присуждается за два 
	открытия, рассматривающих нарушение симметрии в физике частиц. 
	В 1960-х Йоширо Намбу, работавший с асимметрией, обуславливающей 
	сверхпроводимость, первым смог найти модель, объясняющую, как 
	нарушение симметрии может произойти спонтанно на субатомном уровне. 
	Сформулированные им математические описания помогли улучшить 
	Стандартную модель физики частиц -- современную теорию, наилучшим 
	образом объясняющую многое, но не все, в том, как фундаментальные 
	частицы и силы, управляющие их поведением, взаимодействуют, создавая 
	известную нам вселенную.

	В начале 1970-х Кобаяши и Маскава сформулировали модель, объясняющую 
	определенные нарушения симметрии, недавно удивлявшие наблюдателей 
	в экспериментах в физике частиц. Их модель предполагала, что набор 
	субатомных частиц, известный в то время, был недостаточным для 
	объяснения наблюденного поведения, и предсказала существование до тех 
	пор не открытых элементарных частиц. Модель, однако, не уточняла, 
	какую именно форму должны принимать эти частицы. Кобаяши и Маскава 
	предположили существование третьего поколения кварков, которые 
	являются одними из строительных блоков, образующих всю материю 
	и антиматерию. Затем им пришлось ждать почти три десятилетия, пока 
	экспериментальные результаты проверят их гипотезу. Существование всех 
	трех поколений было наконец подтверждено, когда последнее 
	составляющее было наблюдено в середине 1990-х.
\end{otherlanguage}
% }}}

\end{document}

\task{What I Know and Can Say about Symmetry}
% {{{

Symmetry is what always surrounds us in everyday life. The human eye 
and perception are so used to it that it's even less pleasant to look 
at asymmetric objects. Spheres, cylinders, and rectangles are 
everywhere. These shapes are the ones that we see the most and like the 
most. It's even harder to imagine a piece of furniture with the top 
smaller than the bottom. And the same applies to physics and math. 
A physicist is happier if he or she sees a system with some kind of 
symmetry. They know that this symmetry will simplify long formulas and 
possible calculations. Some symmetries even let us solve part of 
a problem in general terms and just write part of the answer straight 
away in more specific cases. In fact, the first thing a physicist does 
when they approach a system is try to figure out its symmetries to 
split the problem and solve its parts separately, as they are generally 
easier. I personally can't even remember any system where there isn't 
any symmetry. Its importance is huge.

However, this picture comes from our everyday experience, which is 
classical. It deals with large objects that have precise placement, 
size, and speed. When we dive into the world of subatomic particles, we 
find that the laws are quite different. Objects there don't have any 
precise placement in the space, and neither do they have precise 
momentum. This means that some of the parameters we are used to cannot 
even be applied to the microworld. Nonetheless, physicists had to start 
from something, and the only thing they had was the classical case. So 
they started from the classical picture and its laws. Being moved to 
the microworld, operations on a system look differently. Symmetry 
itself is the way the system changes (or rather stays the same) when 
its parameters undergo a specific transformation. In classical physics, 
such transformations are just variable transformations in corresponding 
functions and equations. In quantum physics, most physical entities 
become operators, and these transformations are no exception. Looking 
at an operator, it's difficult to tell whether it behaves the same as 
in classical physics, whether it preserves the symmetry or not. Since 
we are used to them so much, physicists at first didn't even think that 
the symmetry can change. But change it did. And when scientists 
realized that it may not be the same, they started searching for 
a method to verify the classical conservation laws. The easiest was the 
mirror symmetry. Two Chinese theorists Tsung Dao Lee and Chen Ning Yang 
came up with the idea of an experiment that could verify the mirror 
symmetry is broken or not. And another Chinese physicist, Chien-Shiung 
Wu, conducted such an experiment and showed that for subatomic 
particles the right is not the left, and the mirror symmetry is 
violated. If a system emitted an electron to the left, then the same 
system being mirrored would not necessarily emit an electron to the 
right.

After this discovery, physicists hoped that another operator, called 
the charge conjugation operator, would help, and the double 
charge-mirror symmetry would be preserved. But then came the kaon 
experiments that showed even its violation. And theorists began working 
on a theoretical explanation of this effect. The correct one was 
suggested by Japanese scientists Makoto Kobayashi and Toshihide 
Maskawa. They proposed the existence of three more quarks and 
introduced a 3x3 quark mixing matrix that could be the source of 
CP-violation. Their predictions were confirmed 30 years after, and now 
the Standard Model of particle physics incorporates their theory.

Another theoretical challenge for the Standard Model was the masses of 
fundamental particles. In order to obey some necessary symmetries, it 
had to imply that all particles have zero masses, which is obviously 
not the case. However, a genius theorist Peter Higgs proposed the 
existence of a separate field responsible for particle masses. An 
unusual requirement to that field is that it has to be distributed 
throughout the Universe extremely evenly, since the mass of a particle 
does not depend on its location. This requirement may seem bold and 
unlikely, but what it ultimately means is that the Higgs field is in 
its ground state with zero excitations. Higgs's explanation introduced 
a mechanism of symmetry breaking that gave different particles 
different masses but once had a symmetric state where they were all 
massless. The essence of the explanation is that this symmetric state 
has higher energy and is hence not the vacuum.

This way, the Standard Model of particle physics has been extended with 
many sub-models to explain our world piece by piece. There are still 
some mysteries and facts that are not yet clear and described by the 
model, but physics is developing. New experiments are being conducted 
and new theories are being formulated every year. Our work is to see 
which of them are closer to being correct and find the right way to 
the next Standard Model extension.
% }}}

\end{document}

New words:
- feasible -- осуществимый
