\documentclass[a4paper, 12pt]{article}

% Configuration {{{
\usepackage[utf8]{inputenc}
\usepackage[T2A]{fontenc}
\usepackage[russian, english]{babel}

\usepackage[
	vmargin=1in,
	hmargin=1in
]{geometry}
\linespread{1.3}
\usepackage{xcolor}
\definecolor{allrefs}{HTML}{1010aa}
\usepackage[
	colorlinks=true,
	allcolors=allrefs
]{hyperref}
\usepackage{indentfirst}
\usepackage{graphicx}
\usepackage[multidot]{grffile}
\usepackage[labelsep=period]{caption}
\usepackage{enumitem}
\setlist{nolistsep}
\usepackage{mathtools}

%\usepackage{times}
\usepackage{multicol}
\usepackage{lipsum}
\setlength{\columnsep}{.25in}

\def\task#1{\begin{center}\it #1\end{center}}
\def\ans#1{\textit{#1}}

\newif\ifcols
\colsfalse
%}}}

\begin{document}

\noindent
Kerim Guseynov, 113M group
\hfill
Nov 13

\task{Unit 7, p. 79--80, translation}
\ifcols\begin{multicols}{2}\fi% {{{
\begin{otherlanguage}{russian}
	\begin{large}
		\textbf{Нобелевская премия по физике 2007}
	\end{large}
	\\\textbf{Пресс релиз}
	\\\textbf{9 октября 2007}

	Шведская королевская академия наук решила присудить 
	нобелевскую премию по физике в 2007 году совместно Альберту 
	Ферту, Университет Париж-юг, Орсе, Франция, и Питеру 
	Грюнбергу, Исследовательский центр Юлих, Германия, ``за 
	открытие гигантского магнетосопротивления''.

	\textbf{Нанотехнологии дают нам чувствительные считывающие 
	головки для компактных жестких дисков}

	В этом году премия по физике присуждается за технологию, 
	используемую для считывания данных с жестких дисков. Именно 
	благодаря этой технологии стало возможным радикально 
	уменьшить жесткие диски в последние годы. Чувствительные 
	считывающие головки нужны для считывания данных с компактных 
	жестких дисков, используемых в ноутбуках и некоторых 
	музыкальных проигрывателях.

	В 1988 француз Альберт Ферт и немец Питер Грюнберг 
	независимо друг от друга открыли совершенно новый физический 
	эффект -- гигантское магнетосопротивление, или ГМР. 
	В системе с ГМР очень слабые магнитные изменения приводят 
	к существенным разницам в электрическом сопротивлении. 
	Система такого типа -- идеальное приспособление для 
	считывания данных с жестких дисков, когда информацию, 
	определяемую магнитно, нужно перевести в электрический ток. 
	Вскоре исследователи и инженеры начали пытаться применить 
	этот эффект в считывающих головках. В 1997 году впервые была 
	запущена считывающая головка, основанная на эффекте ГМР, 
	а вскоре это стало стандартной технологией. Даже самые 
	современные техники считывания сегодня являются дальнейшими 
	разработками того же эффекта.

	Жесткий диск хранит информацию, например как музыка, в виде 
	микроскопически маленьких площадей, намагниченных в разных 
	направлениях. Информация получается с помощью считывающей 
	головки, которая сканирует диск и регистрирует магнитные 
	изменения. Чем меньшие и более компактны жесткие диски, тем 
	слабее индивидуальные магнитные области. А значит, более 
	чувствительные считывающие головки нужны для упаковывания 
	информации еще плотнее на жестком диске. Считывающая 
	головка, основанная на эффекте ГМР, может переводить очень 
	слабые магнитные изменения в ток, испускаемый считывающей 
	головкой. Ток есть сигнал от считывающей головки, 
	а различные его силы соответствуют единицам и нулям.

	Эффект ГМР был открыт благодаря новым методам, развитым 
	в 1970х, позволяющим производить очень тонкие слои разных 
	материалов. Для существования ГМР нужны структуры, состоящие 
	из слоев толщиной в несколько атомов. В связи с этим эффект 
	ГМР можно считать одним из первых реальных применений 
	многообещающей области нанотехнологий.
\end{otherlanguage}
\ifcols\end{multicols}\fi% }}}

\task{Ex. 1, p. 80}
\ifcols\begin{multicols}{2}\fi% {{{
	\begin{enumerate}[label=\alph*)]
		\item \textit{It is thanks to Giant Magnetoresistance that} it has 
			been possible to \textit{miniaturize hard disks} so radically in 
			recent years.

		\item In 1988, the Frenchman Albert Fert and the German 
			Peter Gr\"{u}nbergeach \textit{independently discovered} 
			a totally new physical effect --- Giant 
			Magnetoresistance.

		\item A GMR system consists of layers that are only a few 
			atoms thick. Such layers could only be produced using 
			the state of the art technology of that time. This is 
			why the GMR effect is \textit{one of the first real 
			applications of nanotechnology}.

		\item In a GMR system, very weak magnetic changes 
			\textit{give rise to major differences} in electrical 
			resistance. It makes such systems \textit{the perfect 
			tool for reading data} from hard drives. The 
			\textit{information is retrieved} by a read-out head 
			that scans the disk and registers the magnetic changes 
			which are then \textit{converted to electric current}. 
			To make disks smaller, we need to \textit{pack information more 
			densely on a hard disk}, so more sensitive read-out heads 
			are required. A read-out head based on the GMR effect 
			can \textit{convert very small magnetic changes into 
			differences in electric resistance} and therefore 
			\textit{read data} from a smaller disk.

			For the first time, read-out heads based on the GMR 
			effect were used in 1997 and quickly \textit{became the 
			standard technology}. Even now, modern read-out 
			techniques are only \textit{further developments of 
			GMR}.
	\end{enumerate}
\ifcols\end{multicols}\fi% }}}

\task{P. 81--82, translation, three questions}
\ifcols\begin{multicols}{2}\fi% {{{
\begin{otherlanguage}{russian}
	\textbf{Гигант внутри малых устройств}

	\textit{Внутри компьютера, который вы используете для чтения 
	этой статьи, находится система получения памяти, основанная 
	на открытия, за которые Нобелевская премия по физике в 2007 
	году была дана Альберту Ферту и Питеру Грюнбергу. Они 
	открыли, независимо друг от друга, новый способ 
	использования магнетизма для контроля потока электрического 
	тока через слои металлов, построенных в нанометровой шкале.}

	150 лет назад Уиллиям Томсон наблюдал очень малые 
	изменения в электрических свойствах металлов, когда их 
	помещали в магнитное поле, феномен, названный им 
	магнетосопротивление. В сове время его открытие нашло 
	применение, магнитно индуцированные колебания тока стали 
	принципом, лежащим в основе считывания компьютерной памяти. 
	Затем, в 1988 году, Ферт и Грюнберг, работая со специально 
	собранными кипами чередующихся слоев очень близко 
	расположенных железа и хрома, неожиданно открыли, что можно 
	использовать магнитное поле для гораздо более сильного 
	увеличения электрического сопротивления, чем Томпсон или 
	кто-либо еще наблюдал. Осознавая новизну эффекта, Ферт 
	назвал его гигантским магнетосопротивлением, и это произошло 
	лишь за несколько лет до того как улучшение 
	и миниатюризация, предлагаемые этим эффектом, привели 
	к замене классического магнетосопротивления.

	Гигантское магнетосопротивление по своей сущности 
	квантовомеханический эффект, зависящий от свойств спина 
	электрона. Прикладывая магнитное поле для изменения 
	направления спина электронов в атомах чередующихся 
	металлических пластин, можно уменьшить количество 
	электрического тока аналогично тому, как перпендикулярные 
	поляризаторы блокируют солнечный свет. Однако когда 
	магнитное поле располагает спины электронов сонаправленно, 
	ток проходит проще, точно как свет проходит через 
	параллельные поляризаторы.

	Применение этого открытия было быстрым и широким 
	и существенно улучшило объемы памяти во многих устройствах, 
	от компьютеров до тормозов на машинах. И тихо проникая 
	в технологии за нашей ежедневной жизнью, принципы 
	гигантского магнетосопротивления сейчас используются для 
	решения проблем в более широких областях, например, для 
	селективного разделения генетического материала.
\end{otherlanguage}

	\textbf{Questions}
	\begin{enumerate}
		\item Who was the first to observe megnetoresistance?

			William Thomson discovered magnetoresistance by 
			observing  very small changes in the electrical 
			properties of metals when they were placed in a magnetic 
			field.

		\item What system were Fert and Gr\"{u}nberg working on 
			when they made the discovery?

			They were working with specially-constructed stacks made 
			from alternating layers of very thinly-spread iron and 
			chromium.

		\item What is the giant magnetoresistance effect based on?

			It is based on the quantum properties of the electron 
			spin. We can manipulate it using an applied magnetic 
			field and make electrons move easily (conduct) or not.
	\end{enumerate}
\ifcols\end{multicols}\fi% }}}


\end{document}
