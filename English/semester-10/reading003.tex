\documentclass[a4paper, 12pt]{article}

% Configuration {{{
\usepackage[utf8]{inputenc}
\usepackage[T2A]{fontenc} % T1 for English
\usepackage[russian, english]{babel}

\usepackage{enumitem}
\setlist{nolistsep}
\usepackage{mathtools}
\usepackage{xcolor}
\definecolor{dimblue}{HTML}{1010aa}
\usepackage[
	colorlinks=true, 
	allcolors=dimblue
]{hyperref}
\usepackage[
	vmargin=1in,
	hmargin=1in
]{geometry}
\linespread{1.3}
\usepackage{indentfirst}
\usepackage{graphicx}
\usepackage[multidot]{grffile}
\usepackage[labelsep=period]{caption}
\usepackage{subcaption}

%\usepackage{times} % for English
% }}}

\begin{document}

Kerim Guseynov, group 113M
\hfill
\today

\begin{center}
	\begin{large}
		\textbf{Ground-State Properties of the Hydrogen Chain: Dimerization, Insulator-to-Metal Transition, and Magnetic Phases}
	\end{large}

	Sections I--IV.

	The text is about 29k symbols.
\end{center}

\begin{center}
	\textbf{Vocabulary}
	\vskip.3\baselineskip
	\begin{tabular}{rl|c}
		\hline\hline
		retain & сохранять & page 1, abstract \\
		intricate & сложный, запутанный & page 1, abstract \\
		intertwined & переплетенные & page 1, abstract \\
		profoundly & очень, глубоко & page 1, par. 1 (right) \\
		\hline
		tractability & управляемость & page 2, par. 3 (left) \\
		hallmark & отличительная черта & page 2, par. 3 (left) \\
		synergistic & синергетический & page 2, par. 4 (left) \\
		\hline
		diffuse & расплывчатый, размытый & page 3, par. 3 (left) \\
		envelope & огибающая & page 3, last par. (left) \\
		illuminating & хорошая иллюстрация & page 3, last par. (right) \\
		\hline
		bond & связь & page 4, par. 2 (right) \\
		\hline
		commensurate & соизмеримый, соразмерный & page 5, last par. (left) \\
		\hline
		cusp & пик, острие & page 6, par. 1 (left) \\
		degeneracy & вырождение & page 6, par. 1 (left) \\
		kink & излом & page 6, par. 1 (right) \\
		\hline
		instantiate & приводить в пример & page 7, last par. (left) \\
		in silico & путем моделирования & page 7, par. 2 (right) \\
		\hline\hline
	\end{tabular}
\end{center}

\begin{center}
	\textbf{Questions}
\end{center}
\vskip-.5\baselineskip
\begin{enumerate}
	\item What system do the authors investigate?
	\item How is dimerization first probed in this work?
	\item How does it manifest itself apart from the electron density?
	\item What type of materials often exhibits metal-insulator 
		transitions?
	\item What parameters should be varied if metal-to-insulator 
		transition is to be observed?
	\item What is the ground state of a one-band model with commensurate 
		filling?
	\item Where does the metal-to-insulator transition arise from in the 
		hydrogen chain?
	\item What is responsible for metallic behavior in multi-band systems?
	\item What effects are absent in Hubbard model which give rise to the 
		hydrogen chain conductivity?
	\item Who pioneered the study of the metal-to-insulator transition?
	\item What was the revolutionary approach to constructing electron 
		wave-functions?
	\item How can the physics described in the article be realized 
		experimentally?
\end{enumerate}

\begin{center}
	\textbf{Overview}
\end{center}

Regarding electrical conductivity, here exist three types of materials: 
insulators, semiconductors, and conductors (also called metals). Of the 
three of them, the most interesting and complex is of course 
semiconductors. But the other two kinds are far from being fully 
understood and described. We know that insulators have electron 
densities localized in the vicinity of atoms or inside molecules. 
Considering metals, we know that they form a unique lattice structure of 
positively charged ions and almost completely free electrons between 
them that can be even perceived as a gas. For a more comprehensive look 
at the nature of electrical conductivity, we have to simulate the whole 
atomic system and derive conditions under which it behaves as a metal or 
as an insulator. The difficult part here is in calculations. Quantum 
systems, which atomic materials obviously are, need to satisfy specific 
symmetries of their wave-function, and the more electrons they contain, 
the more complex are the bounds and hence the calculations. The 
complexity grows exponentially with the size of the system. Quantum 
computing may help us perform such computations more easily, but for now 
we have to work with smaller systems.

The authors work with a system of N identical hydrogen atoms 
equidistantly located along the z axis. To further simplify the system, 
the protons are considered fixed in place, so there are just N electrons 
moving in a certain potential. The aim was to analyze how the electrons 
behave for different proton separations and various numbers of atoms. 
Even with this kind of simplification, regular straightforward methods 
don't yield any rational results, so the authors used several numerical 
methods and tried to identify a consistent result.

The authors started from a large separation. In this case, each electron 
is localized near the corresponding proton, and their energy levels are 
almost the same as of isolated atoms. However an interesting effect 
occurs anyway. Although energy levels are almost the same, the 
electrons, being part of a single quantum system, feel each other by 
symmetries of the wave-function. So at large separations, when the 
Coulomb interaction between the electrons is weak, they align their 
spins in opposite directions. This way, the hydrogen chain exhibits 
antiferromagnetic correlations which fade a bit faster than a simple 
approach yields; probably due to the finite size of the chain.

When the protons are brought closer, they start forming pars called 
dimers. The most obvious consequence of dimerization is that the 
electron density between a proton and its right neighbor differs from 
the density between it and the left neighbor. This effect is very subtle 
and hard to capture numerically, but three of five fundamentally 
different methods used by the authors gave approximately the same result 
for the difference in electron density depending on the length of the 
hydrogen chain, which is very impressive. More evidence of the 
dimerization is contained in kinetic energy and entanglement entropy.

For short proton separations, the most interesting effect to study is 
the metal-to-insulator transition (MIT). According to some of the most 
successful phenomenological models of three-dimensional conductors, 
there should be no such transition at all. However, it occurs for two 
reasons. The first is that single-atom energy levels broaden in presence 
of other atoms. The closer they are, the wider the levels become. And 
there comes a point when the highest occupied level overlaps with the 
lowest vacant level, and electrons don't need much energy to travel the 
chain. The second reason is a bit more technical. The quasi-wave-vector 
induced by the periodic structure gains a second allowed value closer to 
zero caused by self-doing. This allows some electrons diffuse along the 
chain axis and thus conduct electricity. It should be noted that fewer 
methods yielded consistent results here, but it's a generally more 
complex task, so the agreement is satisfactory.

\end{document}
