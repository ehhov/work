\documentclass[a4paper, 14pt]{extarticle}

% Configuration {{{
\usepackage[utf8]{inputenc}
\usepackage[T2A]{fontenc} % T1 for English
\usepackage[english, russian]{babel}

\usepackage{enumitem}
\setlist{nolistsep}
\usepackage{mathtools}
\usepackage{xcolor}
\definecolor{dimblue}{HTML}{1010aa}
\usepackage[
	colorlinks=true, 
	allcolors=dimblue
]{hyperref}
\usepackage[
	left=30mm, right=10mm,
	top=15mm, bottom=20mm,
]{geometry}
\frenchspacing
\linespread{1.43}
\usepackage{indentfirst}
\usepackage{graphicx}
\usepackage[multidot]{grffile}
\usepackage[labelsep=period]{caption}
\usepackage{subcaption}

%\usepackage{times} % for English

\usepackage[style=russian]{csquotes}
\usepackage{longtable}
% }}}

\begin{document}

% Title Page & ToC {{{
\thispagestyle{empty}

\begin{center}
	\textbf{МОСКОВСКИЙ ГОСУДАРСТВЕННЫЙ УНИВЕРСИТЕТ имени~М.\,В.~ЛОМОНОСОВА}
	\vskip.5\baselineskip
	\textbf{ФИЗИЧЕСКИЙ ФАКУЛЬТЕТ}
	\vskip.5\baselineskip
	Кафедра общей ядерной физики
\end{center}

\vfill
\begin{center}
	РЕФЕРАТ\\по дисциплине\\\enquote{Английский язык}\\на тему\\
	\enquote{ИЗУЧЕНИЕ ХАРАКТЕРИСТИК И КОНТРОЛЬ \\ КВАНТОВЫХ СИСТЕМ}
\end{center}

\vspace{\baselineskip}

\begin{flushright}
	Студента 113М группы\\Гусейнова А-К. Д.
	\vskip.5\baselineskip
	Научный руководитель\\канд. физ.-мат. наук Горелов И. В.
	\vskip.5\baselineskip
	Преподаватель\\ст. преп. Назарова Е. Н.
\end{flushright}

\vfill

\begin{center}
Москва --- 2021
\end{center}

\clearpage
\tableofcontents
\clearpage

% }}}

\section{Введение}
% {{{

Фундаментальные аспекты квантовой физики были развиты и сформулированы 
в начале XX века, когда классические подходы не смогли описать некоторые 
явления, включающие спектр теплового излучения абсолютно черного тела, 
испускание фотоэлектронов и комптоновское рассеяние. Классическая физика 
также не могла объяснить спектры некоторых атомов во внешнем магнитном 
поле, но эта проблема не выглядела настолько критичной, и ученые 
надеялись найти решение в рамках классической теории.

Квантовая механика заставила человечество посмотреть на микромир 
совершенно по-другому. Она показала сложность применения понятий, 
которыми мы с легкостью оперируем в повседневной жизни и в классической 
физике, к системам маленьких размеров. Было обнаружено множество 
явлений, совершенно не согласующихся со старыми представлениями. 
Оказалось, что свет одновременно со всеми характеристиками волны при 
более внимательном изучении обладает всеми характеристиками частицы. То 
же самое относится и к объектам, изначально считавшимся частицами, таким 
как электроны, протоны и даже большие молекулы. Этот эффект называют 
корпускулярно-волновым дуализмом, и он хорошо иллюстрирует, насколько 
велико различие между квантовой и классической физикой.

Изучение квантовых систем уже привело к созданию уникальных методов 
и технологий, повсеместно используемых в повседневной жизни каждого 
человека на Земле. При этом темп развития совершенно не замедляется. 
Перед учеными встают все новые и новые вопросы по мере нахождения 
объяснений для старых. Основой проблемой современных работ в области 
квантовой физики является сложность любых разумных вычислений. 
Математики и физики пытаются найти решение проблемы, прибегая 
к принципиально новому способу вычислений, использующему квантовые 
системы в своей сути. Но на данный момент каждая задача и каждый вопрос 
требует практически диаметрально противоположного подхода. В упоминаемых 
в реферате работах эта особенность квантового мира особенно заметна.

% }}}

\clearpage \section{Коэффициент преломления сред}
% {{{

Прозрачные материалы по-разному пропускают видимый свет. Их 
разнообразные свойства включают прозрачность, коэффициент отражения, 
закон дисперсии и коэффициент преломления. Большинство из них 
обусловлены оптическим откликом вещества, который представляет собой 
фундаментальную характеристику, определяемую атомной структурой. 
Несмотря на то, что оптический отклик одиночного атома довольно просто 
вычислить, любые попытки вывести из него показатель преломления атомного 
вещества приводили к нереалистичным результатам. В этой статье 
приводится принципиально новый подход к вычислению оптического отклика 
и показателя преломления атомного вещества и проливается свет на 
загадку, почему величины показателей преломления всегда порядка единицы.

% optical response -- оптический отклик
% atomic medium -- атомное вещество
% refractive index
% reflectivity
% dispersion law
% opacity

% Transparent materials let the visible light come through differently. 
% Some of their properties include opacity, reflectively, dispersion 
% law, and refractive index. Most of them are related to the optical 
% response of the material, which is its low-level characteristic 
% resulting from the atomic structure. While the optical response of 
% a single atom is easy to calculate, any attempts to derive the 
% refractive index of an atomic medium have led to unrealistic results. 
% This paper presents a completely new approach to calculating the 
% optical response and the refractive index of an atomic medium and 
% throws light on the mystery of why refractive indexes are of the order 
% of unity.

Чтобы понять проблему, нужно сначала проследовать за традиционным 
подходом и обнаружить его слабости. Для избежания самых сложных 
и требующих наибольшее компьютерное время вычислений атомы приближаются 
полуклассическими двухуровневыми системами. Это сравнительно близко 
к правде и охватывает наиболее важные характеристики атома. Отклик 
одиночного атома хорошо известен, так что наша задача лишь в том, чтобы 
применить его каким-либо образом к системе многих атомов. Традиционный 
путь заключается в том, чтобы забыть о дискретной, зернистой атомной 
структуре вещества, ввести его коллективные характеристики, такие как 
плотность, поляризуемость и так далее, и оперировать с точки зрения 
непрерывной среды. На этом этапе читатель уже, возможно, выявил 
наибольшую слабость данного подхода, но, тем не менее, стоит довести 
логическую цепочку до результата. Одной из отличительных особенностей 
сплошной среды оказывается то, что электрическое поле, 
распространяющееся в ней, постоянно и непрерывно находится под ее 
влиянием и таким образом оставляет невозможным множественное рассеяние 
и интерференцию. Это приводит к тому, что коэффициент преломления 
в лидирующем порядке представляет собой лишь коэффициент преломления 
единичного атома, масштабированный множителем, пропорциональным 
квадратному корню числа атомов, разделенного на объем. Таким образом, 
в резонансе коэффициент преломления бесконечно растет с увеличением 
плотности, что, очевидно, не наблюдается в эксперименте.

% continuous medium -- непрерывная среда
% polarizability, density, discrete
% leading order
% to grow indefinitely

% To understand the problem, we should first follow the conventional 
% approach and find its weak point. To avoid very complex and 
% time-consuming calculations, atoms are approximated as semi-classical 
% two-level systems. This is fairly precise and incorporates the most 
% important characteristics of the atom. The reaction of a single atom is 
% well known, so our work is to apply it somehow to a system of many 
% atoms. The conventional way is to forget about the atomic, discrete, 
% granular structure of the matter, introduce its collective 
% characteristics such as density, polarizability, etc., and operate in 
% terms of a continuous medium. At this point you may have already 
% gotten the largest weakness of the approach, but we should still reach 
% its result. One distinction of a continuous medium is that the 
% electric field traversing it is always and constantly influenced by 
% it, thus making it impossible to experience multiple scattering and 
% interference. The result of that is that the refractive index in the 
% leading order is just the index of a single atom scaled by a factor 
% proportional to the square root of the number of atoms divided by the 
% volume. So at the resonance, the refractive index just grows 
% indefinitely with density, what is obviously not observed in real 
% life.

Подход, предложенный в статье~\cite{one}, однако, не пренебрегает 
атомной природой любого вещества. Он осуществляется для ансамбля 
абсолютно одинаковых атомов, каждый из которых имеет одну и ту же 
резонансную частоту, но остается отдельной точечной частицей. Основной 
причиной рассмотрения приближения непрерывного вещества во всех 
предшествовавших теоретических работах является то, что оно колоссально 
упрощает все формулы и даже позволяет достичь результата аналитически. 
При сохранении атомной структуры приходится вычислять множество больших 
сумм по всем атомам, при этом зная величины электрического поля в точках 
положения каждого атома. Еще сильнее усложняет задачу то, что 
в правильном, то есть квантовом, подходе необходимо принять во внимание 
смешивание состояний атомов, обусловленное принципом Паули, и оно 
добавляет определитель размера $N\times N$ в каждое уравнение, 
экспоненциально увеличивая затрачиваемое на вычисления время в случае 
больших систем. Для значений $N$ даже порядка $10^3$ на вычисления 
требуется колоссальное время. Тот факт, что в рассматриваемой задаче 
атомы представляются простыми системами с лишь двумя энергетическими 
уровнями, безусловно, упрощает работу, но недостаточно для полноценного 
масштабного анализа. Однако, авторы также предлагают довольно элегантный 
способ избавиться от такой сложности. Вскоре он будет описан, но для 
начала нужно понять, как ведут себя два атома при попадании на них 
электромагнитного излучения.

% ensemble
% large-scale масштабный
% incident падающий

% The approach suggested in this paper, however, does not forget about the 
% atomic nature of any medium. It is implemented for an ensemble of 
% perfectly identical atoms, each of which has the same resonant 
% frequency, but remains a distinct and point-like particle. The main 
% reason why theoretical studies have always considered the smooth medium 
% approximation is that it dramatically simplifies all formulas and even 
% lets one achieve the result analytically. To keep the atomic structure, 
% one has to calculate large sums over all atoms including the values of 
% the electric field at each atom's position. To make matters worse, in 
% the correct, quantum approach, one has to take into account the mixing 
% of atomic states due to the Pauli principle, which adds an 
% N x N determinant to every equation, making it exponentially more 
% time-consuming for larger systems. And for N even about 10\^3, it takes 
% an enormous amount of time and work. The fact that in the system 
% considered atoms have only two energy levels makes things easier, but 
% not enough for a complete large-scale study. However, an elegant way out 
% is also suggested by the authors. I will explain it in a minute, but 
% first, you should understand how two atoms behave under a incident plain 
% electromagnetic wave.

Если два абсолютно идентичных атома с резонансной частотой $\omega_0$ 
взаимодействуют друг с другом, их оптический отклик не окажется функцией 
Лоренца с центром на частоте $\omega_0$ и высотой вдвое больше, чем для 
одного атома. Вместо этого, отклик представляет собой два значительно 
удаленных друг от друга лоренцевых пика обычной высоты с центрами на 
смещенных частотах $\omega_0 + \delta\omega$ и $\omega_0 
- \delta\omega$., где $\delta\omega$ зависит от расстояния между атомами 
и силы взаимодействия между ними. Основываясь на оптическом отклике, 
который определяет реакцию системы на падающий видимый свет, два 
идентичных атома ведут себя почти как два атома, чьи резонансные частоты 
слегка смещаются по сравнению с изначальной частотой атомов.

% Lorentzian

% If there are two perfectly identical atoms with the resonant frequency 
% omega zero that interact with each other, their optical response is not 
% a Lorentzian centered at omega zero with the height twice as large as 
% for a single atom. Instead, the response is two substantially distant 
% Lorentzian peaks of normal height centered at shifted frequencies of 
% omega zero plus-minus delta omega, where delta omega depends on the 
% distance between the atoms and their interaction strength. Based on the 
% optical response, which defines the reaction of the system to visible 
% light, two identical atoms behave almost like two atoms whose resonant 
% frequencies are slightly shifted from the original atoms' frequency.

Таким образом, можно ожидать, что набор идентичных атомов может быть 
представлен в виде набора разных атомов с эффективными резонансными 
частотами, зависящими от параметров пространственного распределения 
исходной системы. В самом деле, авторы сравнивают приведенное упрощение 
с прямым вычислением упомянутых выше объемных формул для малых 
плотностей и получают хорошее согласие между ними. Кроме того, сравнение 
показывает, что показатель преломления атомного вещества достигает 
предельного значения по мере увеличения плотности. Для больших 
плотностей возможно применение только упрощенного метода, но он 
подтверждает, что показатель преломления испытывает насыщение 
и достигает величины порядка 1.7, которая, очевидно, порядка единицы.

% This way, one can expect that a set of identical atoms can be 
% approximated by a set of different atoms with effective resonant 
% frequencies depending on spatial parameters of the original system. And 
% indeed, as the authors compare the simplification with a straightforward 
% calculation of the aforementioned large formulas for smaller densities, 
% this test shows good agreement between them. And it also shows that the 
% refractive index n of an atomic medium approaches a limit as the density 
% increases. For larger densities, only the simplified technique is 
% applicable, but it confirms that the index saturates and reaches a value 
% of about 1.7, which is of the order of unity.

Результат анализа объясняет универсальное правило, что показатель 
преломления любого наперед заданного вещества всегда имеет порядок 
единицы. Это обусловлено зернистостью и атомной структурой всего во 
Вселенной.

% The result of the study explains the universal rule that the refractive 
% index is always of the order of unity. It is because of the granularity 
% and atomic structure of everything in the Universe.

% }}}

\clearpage \section{Наименьшее время перемещения квантовой системы}
% {{{

С самого зарождения квантовой физики в начале XX века было открыто 
множество интересных, интригующих, необычных и странных явлений. 
Некоторые из них оказались весьма полезными в приложениях. Сегодня, 
целью многочисленных современных физических экспериментов служит 
регулировка и контроль квантовых систем для полноценного использования 
всех их свойств. Одной из таких целей является транспортировка квантовой 
системы в пространстве. Разумеется, критически важно осуществлять 
перенос когерентно, иначе состояние оказывается искаженным, а задача 
транспортировки на этом провалена. Итак, должен существовать какой-то 
естественный придел на длительность когерентной транспортировки. Эта 
тема уже давно описана для простейших двухуровневых систем, но они редко 
появляются, и люди в основном работают с более сложными системами. Можно 
было бы ожидать, что результаты для сложных систем будут представлять 
собой какую-либо модификацию более простых результатов, но эксперименты 
свидетельствуют о существенной разнице. Таким образом, интересно как 
понять, откуда идет эта разница, так и найти осмысленный нижний предел 
этой длительности. В дополнение к этому было бы также интересно сравнить 
результаты с классической физикой.

% coherent, distort

% Since the birth of quantum physics in the early 20th century, there have 
% been discovered many interesting, unusual, and bizarre phenomena. Some 
% of them were then found to be quite useful for applications. The 
% objective of many modern physics experiments is to manage the control of 
% quantum systems to fully exploit their properties. One of such 
% objectives is to transport a quantum system in space. Of course, it's 
% very important to do that coherently, or otherwise the state is 
% distorted and the task of transporting it is failed. So there has to be 
% some natural limit for the duration of coherent transportation. This 
% topic is long understood for simple two-level systems. But those are 
% rare, and we mostly deal with more complex ones. Some would expect 
% complex systems to have some kind of modification of the simpler 
% results, but experiments have shown a difference. So it is interesting 
% both to understand where it comes from and to find a meaningful lower 
% bound for the duration. As a complementary task, it would also be 
% interesting to compare the results with classical physics.

Задача оптимизации пути объекта с целью достижения минимального времени 
была впервые поставлена Бернулли для массивного объекта, падающего под 
действием однородного гравитационного поля. Решение хорошо известно 
и называется брахистохроной, что с греческого переводится как 
``кратчайшее время''. Эту задачу можно обобщить до нахождения 
наименьшего времени, за которое физическая система может быть изменена 
из ее начального состояния до определенного желаемого состояния. 
Наименьшее время в таком случае зависит от доступного количества энергии 
и типа управления.

% brachistochrone, uniform, generalize

% The task of optimizing the path of an object for minimal time was first 
% posed by Bernoulli for a massive object falling under the influence of 
% a uniform gravitational field. The solution is well known and called 
% a brachistochrone curve. This problem can be generalized to finding the 
% shortest time at which a physical system can be changed from its initial 
% state to a desired final state. The shortest time would depend on the 
% amount of energy available and the type of control.

В квантовой физике такой предел скорости был выведен Мандельштамом 
и Таммом~\cite{two.one} и напоминает принцип неопределенности 
Гейзенберга $\tau_{\mathrm{MT}} = \frac{\pi\hbar}{2\Delta E}$. Он 
показывает, что длительность перемещения не может стремиться к нулю, 
если нет доступа к неограниченным запасам энергии. Этот предел был 
экспериментально подтвержден для двухуровневых систем, однако для 
сложных систем реальный предел оказывается существенно больше, 
и существующий теряет свой смысл. Проблема здесь в том, что 
двухуровневой системе желаемого состояния можно достичь одной 
осцилляцией Раби, которая оказывается фундаментально неприменимой в для 
непрерывного набора состояний, как в случае пространственного положения.

% oscillation, continuous states

% In quantum physics such a speed limit was derived by Mandelstam and Tamm 
% and is reminiscent of Heisenberg's uncertainty principle. It shows that 
% the duration cannot vanish unless one has access to unlimited energy. 
% This limit was experimentally confirmed for two-level systems, however, 
% for complex systems the real limit is much greater, making the existing 
% one meaningless. The problem here is that two-level systems can be 
% converted through a Rabi oscillation, which is fundamentally 
% inapplicable to continuous states like the spatial position.

В работе~\cite{two} конвейерная лента, основанная на оптической решетке 
используется для транспортировки атомного волнового пакета на расстояние 
примерно в 15 раз больше его размера. Потенциал ловушки настраивается 
таким образом, чтобы глубина была достаточной для подавления квантового 
туннелирования и чтобы траектория переносимой системы была оптимальной. 
Оптимальная траектория определяется на основе численного моделирования.

% conveyor belt, lattice, quantum tunneling,
% numerical simulation численное моделирование

% In this work, a conveyor belt based on optical lattice is used to 
% transport an atomic wave packet on a distance 15 times its size. The 
% trap potential is adjusted to be deep enough to suppress quantum 
% tunneling and also so that the trajectory of the transported system is 
% optimal. The optimal trajectory is obtained from numerical simulations.

Точность транспортировки представляет собой пересечение между 
фактическим и целевым конечными состояниями, со значением 1 отвечающим 
идеальному результату, и определяется как функция длительности. Было 
найдено, что точность испытывает насыщение для длительностей больше, чем 
период квазигармонических осцилляций потенциала ловушки (этот же период 
также ответственен за неопределенность энергии в транспортируемой 
системе). Чтобы убедиться, что транспортировка осуществляется 
когерентно, авторы приготовляют конкретную суперпозицию в качестве 
начального состояния, затем транспортируют часть ее в целевое состояние 
и обратно, а затем извлекают постоянную интерференции. Было найдено, что 
постоянная интерференции в точности совпадает с чистотой, измеренной 
ранее, что доказывает сохранение когерентности при использовании данного 
метода.

% superposition

% The fidelity of the transportation, which is the overlap between the 
% actual and targeted final states, with 1 being the perfect result, is 
% obtained as a function of duration. It is found that the fidelity 
% saturates for durations larger than quasi-harmonious oscillation period 
% of the trap potential (which is also responsible for energy uncertainty 
% of the system to transport). To ensure that the transportation is 
% coherent, the authors prepared a specific superposition as an initial 
% state, then transported part of it to a target state and back, and then 
% extracted the interference constant. The constant was found to be 
% exactly the same as the fidelity measured earlier, proving that the 
% method keeps coherence.

Такие же измерения и симуляции были осуществлены для нескольких 
потенциалов ловушки с разными глубинами, и было показано, что 
минимальное время обратно пропорционально глубине. Также интересен тот 
факт, что классический и квантовый случаи различаются только на 25\%, 
а значит, предел длительности по большей части обусловлен Гильбертовой 
метрикой квантовых состояний.

% metric

% The same measurements and simulations were made for several trap 
% potential depths, and the minimal time is shown to be inversely 
% proportional to the depth. Interestingly, the classical and quantum 
% cases differ only by 25\%, which shows that the limit comes mostly from 
% the Hilbertian metric of quantum states.

% }}}

\clearpage \section{Свойства основного состояния водородной цепи}
% {{{

С точки зрения электрической проводимости существует три типа 
материалов: диэлектрики, полупроводники и проводники (также называемые 
металлами). Среди них трех наиболее интересными и сложными, конечно же, 
являются полупроводники, однако даже два других типа далеко не полностью 
поняты и описаны. Известно, что у диэлектриков электронная плотность 
локализована вблизи атомов или внутри молекул. Касательно металлов 
известно, что они формируют уникальную решеточную структуру положительно 
заряженных ионов и практически свободно движущихся между ними 
электронов, которые даже можно считать газом. Для более исчерпывающего 
описания природы электрической проводимости необходимо симулировать всю 
атомную систему целиком и вывести условия, при которых она ведет себя 
как металл или как диэлектрик. Самая сложная часть этой задачи 
в вычислениях самих по себе. Квантовые системы, очевидно, включающие 
в себя атомные вещества, должны соответствовать определенным симметриям 
волновой функции, и чем больше в системе электронов, тем сложнее эти 
ограничения, а значит и вычисления. Сложность возрастает экспоненциально 
с размером системы. Квантовые компьютеры могут значительно упростить 
вычисления такого рода, но на данный момент для работы остаются лишь 
малые системы.

% wave-function, insulator

% Regarding electrical conductivity, here exist three types of materials: 
% insulators, semiconductors, and conductors (also called metals). Of the 
% three of them, the most interesting and complex is of course 
% semiconductors. But the other two kinds are far from being fully 
% understood and described. We know that insulators have electron 
% densities localized in the vicinity of atoms or inside molecules. 
% Considering metals, we know that they form a unique lattice structure of 
% positively charged ions and almost completely free electrons between 
% them that can be even perceived as a gas. For a more comprehensive look 
% at the nature of electrical conductivity, we have to simulate the whole 
% atomic system and derive conditions under which it behaves as a metal or 
% as an insulator. The difficult part here is in calculations. Quantum 
% systems, which atomic materials obviously are, need to satisfy specific 
% symmetries of their wave-function, and the more electrons they contain, 
% the more complex are the bounds and hence the calculations. The 
% complexity grows exponentially with the size of the system. Quantum 
% computing may help us perform such computations more easily, but for now 
% we have to work with smaller systems.

В статье~\cite{three} авторы работают с системой $N$ идентичных атомов 
водорода, размещенных вдоль оси $z$ на одинаковом расстоянии. Для еще 
большего упрощения системы, протоны считаются неподвижными, так что 
задача представляет собой лишь движение $N$ электронов в определенном 
потенциале. Целью является анализ того, как электроны ведут себя 
в случае различных расстояний между протонами и различных значениях $N$ 
числа атомов. Даже при таком сильном упрощении обыкновенные 
прямолинейные подходы не предоставляют каких-либо разумных результатов, 
поэтому авторы использовали несколько численных методов и пытались 
выявить согласующиеся результаты.

% consistent, equidistantly

% The authors work with a system of N identical hydrogen atoms 
% equidistantly located along the z axis. To further simplify the system, 
% the protons are considered fixed in place, so there are just N electrons 
% moving in a certain potential. The aim was to analyze how the electrons 
% behave for different proton separations and various numbers of atoms. 
% Even with this kind of simplification, regular straightforward methods 
% don't yield any rational results, so the authors used several numerical 
% methods and tried to identify a consistent result.

Авторы начали с больших расстояний между протонами. В этом случае каждый 
электрон локализован вблизи соответствующего протона, а их 
энергетические уровни практически совпадают с уровнями изолированных 
атомов. Однако, интересный эффект наблюдается все равно. Хотя 
энергетические уровни и почти совпадают, электроны, являясь частями 
одной единой квантовой системы, ощущают друг друга из-за симметрий 
волновой функции. Таким образом, при больших расстояниях когда 
кулоновское взаимодействие между электронами мало, они располагают свои 
спины в противоположных направлениях. В результате цепочка атомов 
водорода проявляет антиферромагнитные корреляции, которые затухают 
слегка быстрее по сравнению с простейшими соображениями, что может быть 
обусловлено конечным размером цепочки.

% antiferromagnetic, localized, Coulomb interaction, exhibit, fade, 
% yield, finite

% The authors started from a large separation. In this case, each electron 
% is localized near the corresponding proton, and their energy levels are 
% almost the same as of isolated atoms. However an interesting effect 
% occurs anyway. Although energy levels are almost the same, the 
% electrons, being part of a single quantum system, feel each other by 
% symmetries of the wave-function. So at large separations, when the 
% Coulomb interaction between the electrons is weak, they align their 
% spins in opposite directions. This way, the hydrogen chain exhibits 
% antiferromagnetic correlations which fade a bit faster than a simple 
% approach yields; probably due to the finite size of the chain.

По мере приближения протонов, они начинают формировать пары, называемые 
димерами. Наиболее очевидное следствие димеризации в том, что 
электронная плотность между протоном и его соседом справа отличается от 
плотности между ним и соседом слева. Это очень тонкий эффект, и его 
сложно уловить численно, но три из пяти принципиально различающихся 
метода, использованных авторами, дали приблизительно одинаковые 
результаты для разности электронной плотности в зависимости от длины 
водородной цепи, что, безусловно, впечатляет. Другие свидетельства 
димеризации содержатся в кинетической энергии электронов и энтропии 
спутывания.

% dimer, electron density, entropy, entanglement, fundamentally, 
% evidence, kinetic energy

% When the protons are brought closer, they start forming pars called 
% dimers. The most obvious consequence of dimerization is that the 
% electron density between a proton and its right neighbor differs from 
% the density between it and the left neighbor. This effect is very subtle 
% and hard to capture numerically, but three of five fundamentally 
% different methods used by the authors gave approximately the same result 
% for the difference in electron density depending on the length of the 
% hydrogen chain, which is very impressive. More evidence of the 
% dimerization is contained in kinetic energy and entanglement entropy.

Для малых расстояний между протонами наиболее интересным для изучения 
явлением представляется переход от диэлектрика к проводнику. Согласно 
некоторым из наиболее успешных феноменологических моделей трехмерных 
проводников, в исследуемой системе не должно быть такого перехода 
вообще. Однако, он происходит по двум причинам. Во-первых, уровни 
энергии одиночных атомов уширяются в присутствие других атомов. Чем 
ближе они располагаются, тем шире становятся уровни. А значит, должен 
быть момент, когда наивысший занятый уровень перекрывается с самым 
низшим свободным, и электронам уже не нужна большая энергия, чтобы 
перемещаться по цепи. Вторая причина несколько более техническая. 
Квазиволновой вектор, порожденный периодической структурой, приобретает 
второе возможное значение, вызванное механизмом самодопинга, вблизи 
нуля. Это позволяет электронам диффундировать вдоль оси цепочки и, таким 
образом, проводить электричество. Необходимо отметить, что в этом случае 
меньшее количество методов дали согласующиеся результаты, но эта задача 
в целом более сложна, так что согласие считается удовлетворительным.

% phenomenological, quasi-wave-vector, self-doping, overlap, diffuse, 
% axis, conductivity, satisfactory

% For short proton separations, the most interesting effect to study is 
% the metal-to-insulator transition (MIT). According to some of the most 
% successful phenomenological models of three-dimensional conductors, 
% there should be no such transition at all. However, it occurs for two 
% reasons. The first is that single-atom energy levels broaden in presence 
% of other atoms. The closer they are, the wider the levels become. And 
% there comes a point when the highest occupied level overlaps with the 
% lowest vacant level, and electrons don't need much energy to travel the 
% chain. The second reason is a bit more technical. The quasi-wave-vector 
% induced by the periodic structure gains a second allowed value closer to 
% zero caused by self-doping. This allows some electrons diffuse along the 
% chain axis and thus conduct electricity. It should be noted that fewer 
% methods yielded consistent results here, but it's a generally more 
% complex task, so the agreement is satisfactory.

% }}}

\clearpage \section{Заключение}
% {{{

Большой интерес представляют явления, происходящие на стыке дисциплин. 
Коэффициент преломления с точки зрения микромира обусловлен законами 
квантовой механики, волновыми функциями и очень тонкими характеристиками 
межатомных взаимодействий. Атомы в этом случае являются сложными 
квантовыми системами, отклик которых на электромагнитное излучение 
в видимом диапазоне длин волн довольно причудливо. С другой стороны, 
коэффициент преломления, являясь внутренней характеристикой среды, 
описывает поведение света при прохождении через нее. На протяжении 
долгого времени не удавалось связать эти две стороны показателя 
преломления. Но новый подход к вычислениям, предложенный авторами 
статьи, позволяет работать с большими количествами частиц, очевидно, 
более подходящими для описания реальных систем, которые состоят из 
порядка $10^{23}$ атомов. Важнейшим результатом их анализа является 
достижение среды с показателем преломления разумной величины даже для 
большой плотности вещества.

Управление квантовыми системами без затрагивания классической стороны 
также представляет большой интерес. При обращении с квантовыми системами 
поддержание чистоты квантового состояния -- одна из самых сложных задач. 
Особенно сложно это осуществить в случае непрерывного набора состояний, 
как для пространственного положения волнового пакета. Таким образом, 
оказывается проблематично перемещать квантовую систему в пространстве. 
С точки зрения фундаментальной физики, интересно найти минимальный 
предел времени транспортировки квантовой системы с сохранением 
когерентности. Авторы статьи назвали эту задачу квантовой брахистохроной 
по аналогии с подобной задачей, поставленной Бернулли. Авторы путем 
моделирования выявили оптимальную траекторию, по которой должен 
двигаться волновой пакет, чтобы сохранить чистоту состояния. При 
движении по этой траектории искажение волнового пакета экспоненциально 
убывает при достижении определенного предела. Таким образом, этот предел 
и является наименьшим временем, требуемым для когерентного 
пространственного переноса квантовой системы.

Очередной системой, находящейся на стыке областей физики, является 
твердое тело. С одной стороны, твердое тело -- это просто набор атомов, 
находящихся под влиянием друг друга так, что они формируют равномерную 
структуру. Все свойства твердого тела происходят от взаимодействия между 
электронами разных атомов, которые объединены в одну систему. Наличие 
общей системы приводит к множеству интересных корреляционных эффектов, 
включая антиферромагнетизм, магнетизм вообще и электрическую 
проводимость. Касательно проводимости хорошо разработанные теории 
в состоянии лишь в общих чертах обосновать свободу движения электронов 
и как она появляется при переходе от диэлектрического состояния 
к металлическому. Авторы получают общую картину такого перехода 
и объясняют на уровне одномерных систем, каким образом электроны 
приобретают возможность беспрепятственно двигаться вдоль цепочки атомов 
водорода. Это представление позволяет по-новому взглянуть явление 
проводимости в металлах.

% }}}

% Vocabulary {{{
\clearpage
\phantomsection
\addcontentsline{toc}{section}{Глоссарий}
\section*{Глоссарий}

\begin{longtable}{r@{ -- }l}
	advent & пришествие \\
	antiferromagnetic & антиферромагнитный \\
	to approximate & приблизить \\
	atomic medium & атомная среда (состоящая из атомов) \\
	attenuation & затухание \\
	axis & ось \\
	bond & связь \\
	brachistochrone & брахистохрона \\
	bulk & большой, крупный \\
	coherent & когерентный \\
	commensurate & соизмеримый, соразмерный \\
	conductivity & проводимость \\
	to confine & ограничивать \\
	conjecture & предположение \\
	consistent (results) & согласующиеся (результаты) \\
	continuous medium & непрерывная среда \\
	continuous states & непрерывный набор состояний \\
	conundrum & загадка, парадокс \\
	conveyor belt & конвейерная лента \\
	to corroborate & подтверждать \\
	Coulomb interaction & кулоновское взаимодействие \\
	crest & гребень, вершина \\
	cusp & пик, острие \\
	to decompose & раскладывать \\
	degeneracy & вырождение \\
	density & плотность \\
	diametrically opposite & диаметрально противоположный \\
	diffuse & расплывчатый, размытый \\
	to diffuse & диффундировать \\
	dilute gas & разреженный газ \\
	dimer & димер \\
	discrete & дискретный \\
	dispersion law & закон дисперсии \\
	distorted & искаженный \\
	dyadic & состоящий из двух частей \\
	electron density & электронная плотность \\
	to encode & представлять \\
	ensemble & ансамбль \\
	entanglement & спутанность \\
	entropy & энтропия \\
	envelope & огибающая \\
	equidistantly & эквидистантно \\
	evidence & свидетельство \\
	to exhibit & проявлять \\
	exquisitely & изящно \\
	to fade & затухать \\
	finite & ограниченный \\
	to foster & стимулировать \\
	fundamentally & принципиально \\
	to generalize & обобщить \\
	granularity & зернистость, дискретность \\
	to grow indefinitely & неограниченно расти \\
	hallmark & отличительная черта \\
	hatched & заштрихованный \\
	(it's) illuminating & хорошая иллюстрация \\
	implication & последствие \\
	in silico & путем моделирования \\
	incident & падающий (свет) \\
	inset & вставка \\
	to instantiate & приводить в пример \\
	insulator & диэлектрик \\
	interrogation & извлечение данных \\
	intertwined & переплетенные \\
	intricate & сложный, запутанный \\
	kinetic energy & кинетическая энергия \\
	kink & излом \\
	large-scale & масштабный \\
	lattice & решетка \\
	leading order & ведущий порядок (в теории возмущений) \\
	localized & локализованный \\
	Lorentzian & функция Лоренца \\
	metric & метрика \\
	to necessitate & требовать \\
	numerical simulation & численное моделирование \\
	opacity & непрозрачность \\
	optical response & оптический отклик \\
	oscillation & колебание \\
	to overlap & перекрываться \\
	paradigmatic & характерный \\
	phenomenological & феноменологический \\
	pictorial & наглядный \\
	polarizability & поляризуемость \\
	to prescribe & описывать, определять \\
	profoundly & очень, глубоко \\
	quantum tunneling & квантовое туннелирование \\
	quasi-wave-vector & квазиволновой вектор \\
	ramp & наклонная плоскость \\
	readily & охотно \\
	recoil energy & энергия отдачи \\
	to reconcile & согласовать \\
	reflectivity & коэффициент отражения \\
	refractive index & коэффициент преломления \\
	to render & превратить \\
	to retain & сохранять \\
	satisfactory & удовлетворительный \\
	self-doping & самодопинг \\
	spatial & пространственный \\
	speckles & пятна \\
	superposition & суперпозиция \\
	swift & быстрый, резкий \\
	synergistic & синергетический \\
	tractability & управляемость \\
	trough & впадина, минимум \\
	uniform field & однородное поле \\
	wave-function & волновая функция \\
	wave-particle duality & корпускулярно-волновой дуализм \\
	to yield & давать, приносить \\
\end{longtable}
% }}}

% Bibliography {{{
\clearpage
\phantomsection
\addcontentsline{toc}{section}{Список литературы}
\begin{thebibliography}{9}
	\bibitem{one} F. Andreoli \textit{et al}., \textit{Maximum Refractive Index of an Atomic Medium}, \href{http://dx.doi.org/10.1103/PhysRevX.11.011026}{Phys. Rev. X \textbf{11}, 011026 (2021)}.

	\bibitem{two.one} L. Mandelstam and I. Tamm, \textit{The Uncertainty Relation between Energy and Time in Non-Relativistic Quantum Mechanics}, \href{https://doi.org/10.1007/978-3-642-74626-0_8}{J. Phys. (Moscow) 9, 249 (1945)}.

	\bibitem{two} M. R. Lam \textit{et al}., \textit{Demonstration of Quantum Brachistochrones between Distant States of an Atom}, \href{http://dx.doi.org/10.1103/PhysRevX.11.011035}{Phys. Rev. X \textbf{11}, 011035 (2021)}.

	\bibitem{three} M. Motta \textit{et al}. (Simons Collaboration on the Many-Electron Problem), \textit{Ground-State Properties of the Hydrogen Chain: Dimerization, Insulator-to-Metal Transition, and Magnetic Phases}, \href{http://dx.doi.org/10.1103/PhysRevX.10.031058}{Phys. Rev. X \textbf{10}, 031058 (2020)}.
\end{thebibliography}
% }}}

\end{document}

% backup {{{
		approximate to & приблизить \\
		atomic medium & атомная среда (состоящая из атомов) \\
		continuous medium & непрерывная среда \\
		density & плотность \\
		discrete & дискретный \\
		dispersion law & закон дисперсии \\
		ensemble & ансамбль \\
		granularity & зернистость \\
		grow indefinitely to & неограниченно расти \\
		incident & падающий (свет) \\
		large-scale & масштабный \\
		leading order & ведущий порядок (в теории возмущений) \\
		Lorentzian & функция Лоренца \\
		opacity & непрозрачность \\
		optical response & оптический отклик \\
		polarizability & поляризуемость \\
		reflectivity & коэффициент отражения \\
		refractive index & коэффициент преломления \\
		spatial & пространственный \\

		attenuation & затухание \\
		bulk & большой, крупный \\
		decompose & раскладывать \\
		dilute gas & разреженный газ \\
		dyadic & состоящий из двух частей \\
		encode & представлять \\
		exquisitely & изящно \\
		foster & стимулировать \\
		granularity & гранулярность, дискретность \\
		implication & последствие \\
		inset & вставка \\
		necessitate & требовать \\
		pictorial(ly) & наглядный \\
		prescribe & описывать, определять \\
		readily & охотно \\
		reconcile & согласовать \\
		retain & сохранять \\
		speckles & пятна \\

		\hline

		brachistochrone & брахистохрона \\
		coherent & когерентный \\
		continuous states & непрерывный набор состояний \\
		conveyor belt & конвейерная лента \\
		distorted & искаженный \\
		generalize to & обобщить \\
		lattice & решетка \\
		metric & метрика \\
		numerical simulation & численное моделирование \\
		oscillation & колебание \\
		quantum tunneling & квантовое туннелирование \\
		superposition & суперпозиция \\
		uniform & однородный \\

		advent & пришествие \\
		confine & ограничивать \\
		conjecture & предположение \\
		conundrum & загадка, парадокс \\
		conveyor belt & ленточный конвейер \\
		corroborate & подтверждать \\
		crest & гребень, вершина \\
		hatched & заштрихованный \\
		interrogation & извлечение данных \\
		paradigmatic & характерный \\
		ramp & наклонная плоскость \\
		recoil & отдача \\
		render & превратить \\
		swift & быстрый, резкий \\
		trough & впадина, минимум \\

		\hline

		antiferromagnetic & антиферромагнитный \\
		axis & ось \\
		conductivity & проводимость \\
		consistent (results) & согласующиеся (результаты) \\
		Coulomb interaction & кулоновское взаимодействие \\
		diffuse to & диффундировать \\
		dimer & димер \\
		electron density & электронная плотность \\
		entanglement & спутанность \\
		entropy & энтропия \\
		equidistantly & эквидистантно \\
		evidence & свидетельство \\
		exhibit to & проявлять \\
		fade to & затухать \\
		finite & ограниченный \\
		fundamentally & принципиально \\
		insulator & диэлектрик \\
		kinetic energy & кинетическая энергия \\
		localized & локализованный \\
		overlap to & перекрываться \\
		phenomenological & феноменологический \\
		quasi-wave-vector & квазиволновой вектор \\
		satisfactory & удовлетворительный \\
		self-doping & самодопинг \\
		wave-function & волновая функция \\
		yield to & давать, приносить \\

		bond & связь \\
		commensurate & соизмеримый, соразмерный \\
		cusp & пик, острие \\
		degeneracy & вырождение \\
		diffuse & расплывчатый, размытый \\
		envelope & огибающая \\
		hallmark & отличительная черта \\
		illuminating & хорошая иллюстрация \\
		in silico & путем моделирования \\
		instantiate & приводить в пример \\
		intertwined & переплетенные \\
		intricate & сложный, запутанный \\
		kink & излом \\
		profoundly & очень, глубоко \\
		retain & сохранять \\
		synergistic & синергетический \\
		tractability & управляемость \\

		
% }}}
