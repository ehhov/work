\documentclass[a4paper, 12pt]{article}

% Configuration {{{
\usepackage[utf8]{inputenc}
\usepackage[T2A]{fontenc} % T1 for English
\usepackage[russian, english]{babel}

\usepackage{enumitem}
\setlist{nolistsep}
\usepackage{mathtools}
\usepackage{xcolor}
\definecolor{dimblue}{HTML}{1010aa}
\usepackage[
	colorlinks=true, 
	allcolors=dimblue
]{hyperref}
\usepackage[
	vmargin=1in,
	hmargin=1in
]{geometry}
\linespread{1.3}
\usepackage{indentfirst}
\usepackage{graphicx}
\usepackage[multidot]{grffile}
\usepackage[labelsep=period]{caption}
\usepackage{subcaption}

%\usepackage{times} % for English
% }}}

\begin{document}

\noindent
Kerim Guseynov
\hfill
Apr 17

\begin{center}
	\textit{Ex. 2, p. 158, orally}
\end{center}

\begin{enumerate}[label=\alph*)]
	\item What did the development of efficient blue LEDs require?

		It required the production of GaN-based alloys with different 
		compositions and their integration into multilayer structures such 
		as heterojunctions and quantum wells.

	\item What technologies are used in today's high-efficiency white electroluminescent light sources?

		Two different technologies are used. The first is to direct blue 
		light through phosphor, which excites and emits red and green lights 
		that combine with the original blue light to create what we see as 
		white. And the second is to use separate LEDs for three 
		complementary colors (e.g. red, green, and blue).

	\item Why does the use of efficient blue LEDs lead to significant energy savings?

		It does because efficient blue LEDs use ten times less energy then 
		conventional light bulbs, and lighting accounts for about 20--30\% 
		of our total energy consumption.

	\item Whose theoretical developments took place prior to the formulation of the modern theory of electronic structure of solid-state materials?

		Theoretical developments of H. J. Round and O. Losev did it.

	\item Who realized that a p-n junction could be an interesting device for light emission?

		Kurt Lehovec and co-workers of the Signal Corps Engineering 
		Laboratory in the USA realized it.

	\item Why was GaAs attractive in developing techniques to make efficient p-n junctions?

		GaAs was attractive because of its direct band gap, which enables 
		recombination of electrons and holes without involvement of phonons.

	\item Why is it important that the semiconductors have direct band gaps?

		Such semiconductors are much more efficient because ones with 
		indirect band gaps require phonon-assisted recombination.

	\item What is the quantum efficiency of an LED?

		It is the ratio of the number of emitted photons to the number of 
		electrons passing through the contact in a given time.

	\item Which research groups made progress in making efficient LEDs using GaP at the end of the 1950s and what experiments did they conduct?

		There were three groups: Philips Central Laboratory in Germany, 
		Services Electronics Laboratories in the UK, and Belle Telephone 
		Laboratories in the USA. They used different dopants at various 
		concentrations to generate wavelengths from red to green.

	\item What are the basic properties of gallium nitride?

		GaN is a semiconductor of the III-V class, with Wurtzite crystal 
		structure. It can be grown on a substrate of sapphire (Al2O3) or 
		SiC, despite the difference in lattice constants. GaN can be doped, 
		e.g. with silicon to n-type and with magnesium to p-type. But doping 
		interferes with the growth process so that the GaN becomes fragile. 
		In general, defects in GaN crystals lead to good electron 
		conductivity, i.e. the material is naturally of n-type. GaN has 
		a direct band gap of 3.4 eV, corresponding to a wavelength in the 
		ultraviolet.

	\item What research did Philips Research Laboratories carry out using gallium nitride?

		They were developing a new lighting technology and obtained 
		efficient photoluminescence from GaN over a wide spectral range.

	\item What new crystal growth technique did Shuji Nakamura develop?

		He developed a technique for growing GaN crystals, in which a thin 
		layer of GaN is first grown on a substrate of sapphire at low 
		temperature and then heated to grow the rest of the crystal.

	\item What important observation did Amano, Akasaki and their colleagues make in connection with the doping of GaN?

		They noted that when Zn-doped GaN was studied with a scanning 
		electron microscope, it emitted more light, thus indicating better 
		p-doping. It was a breakthrough in p-n-junction development.

	\item Which alloys are necessary in order to produce heterojunctions?

		The necessary alloys are AlGaN and InGaN.

	\item What combinations did Nakamura exploit for producing heterojunctions and quantum wells?

		He exploited the combinations InGaN/GaN and InGaN/AlGaN.

	\item What are the basic applications of LEDs?

		LEDs are very efficient for lighting. They are also used for liquid 
		crystal displays. UV-emitting diodes are used in DVDs and may be 
		used for water purification in the future.
\end{enumerate}

\end{document}
