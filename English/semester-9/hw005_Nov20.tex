\documentclass[a4paper, 12pt]{article}

% Configuration {{{
\usepackage[utf8]{inputenc}
\usepackage[T2A]{fontenc}
\usepackage[russian, english]{babel}

\usepackage[
	vmargin=1in,
	hmargin=1in
]{geometry}
\linespread{1.3}
\usepackage{xcolor}
\definecolor{allrefs}{HTML}{1010aa}
\usepackage[
	colorlinks=true,
	allcolors=allrefs
]{hyperref}
\usepackage{indentfirst}
\usepackage{graphicx}
\usepackage[multidot]{grffile}
\usepackage[labelsep=period]{caption}
\usepackage{enumitem}
\setlist{nolistsep}
\usepackage{mathtools}

%\usepackage{times}
\usepackage{multicol}
\usepackage{lipsum}
\setlength{\columnsep}{.25in}

\def\task#1{\begin{center}\it #1\end{center}}
\def\ans#1{\textit{#1}}

\newif\ifcols
\colsfalse
%}}}

\begin{document}

\noindent
Kerim Guseynov, 113M group
\hfill
Nov 20

\task{P. 82--84, the main thought of each paragraph}
\ifcols\begin{multicols}{2}\fi% {{{
	Magnetoresistance is the change of conductivity in the presence 
	of an external magnetic field. In ferromagnetic materials, it is 
	anisotropic and is only a few percent in magnitude.

	Even though it was only a few percent, the technology was being 
	used for readout heads for magnetic disks. However, there was 
	not seen any performance improvement.

	The discovery of GMR was a great surprise. It appears in systems 
	composed of stacked layers of ferromagnetic and 
	non-ferromagnetic metals with each layer only a few atoms thick.

	The origin of magnetoresistance in GMR systems is completely 
	different. Gr\"{u}nberg had also realized the significance of 
	his discovery and patented it.

	The GMR effect originates from the difference in energy levels 
	of spin-up and spin-down electrons in an external magnetic 
	field. When electrons propagate between layers, they acquire 
	different electric resistance depending on the magnetic field.

	In the presence of an external magnetic field, polarizations of 
	ferromagnetic layers align in the same direction, and the 
	resistance is expressed through spin-up and spin-down 
	resistances.

	The conductivity of spin-up and spin-down electrons depends on 
	their densities at the Fermi energy, which can be quite 
	different. 100\% spin polarization was predicted and then 
	observed for CrO$_2$.

	To create layered systems, one can also grow alternating layers 
	of a metal and an insulator, but the insulator has to be only 
	a few nanometers thick so that electrons can tunnel it.

	GMR is a good example how a fundamental discovery can change 
	technology. It started the field of megnetoelectronics, using 
	the electron charge and spin at the same time, and now drives 
	the development of nanotechnology.
\ifcols\end{multicols}\fi% }}}

\task{Ex. 2, p. 85}
\ifcols\begin{multicols}{2}\fi% {{{
	\begin{enumerate}[label=\alph*)]
		\item About 150 years ago W. Thomson (Lord Kelvin) measured 
			the behavior of the resistance of iron and nickel in the 
			presence of an external magnetic field. He found that the 
			resistance increases along the field lines and decreases 
			across them. He discovered magneroresistance.

		\item Anisotropic magnetoresistance is the dependence of 
			electric resistance in a specific direction on the angle 
			between it and the external magnetic field lines. Giant 
			magnetoresistance is the effect similar to 
			magnetoresistance, but originates differently. It comes from 
			the electron spin and the difference in Fermi energies of 
			spin-up and spin-down electrons.

		\item Peter Gr\"{u}nberg with his colleagues used a trilayer 
			system Fe/Cr/Fe, and Albert Fert used multilayer systems of 
			the form (Fe/Cr)$_n$, where n could be up to 60. They 
			achieved very large dependency of electric resistance on the 
			external magnetic field, very large magnetoresistance.

		\item The GMR device is a stack of alternating layers of 
			a ferromagnetic and a non-conducting materials, each about 
			a few atoms thick. When an external magnetic field is 
			applied, the Fermi levels of spin-up and spin-down electrons 
			are distorted, and electron densities change. This leads to 
			differences in electric resistance.

		\item Half-metals are materials that behave like metals (i.e. 
			conduct electricity) in one direction and like insulators in 
			another.

		\item The tunneling magnetoresistance is the effect of passing 
			thin insulator layers by electrons when they traverse from 
			one metal to another. This effect leads to 
			megnetoresistance.

		\item The discovery of GMR led to the emergence of 
			magnetoelectronics, where both the electric charge and the 
			spin of electrons are used.

		\item The GMR effect made it possible to miniaturize hard 
			drives, read data more effectively, and create much more 
			precise measuring devices.
	\end{enumerate}
\ifcols\end{multicols}\fi% }}}

\task{Ex. 3, p. 85}
\ifcols\begin{multicols}{2}\fi% {{{
	\begin{otherlanguage}{russian}
		\begin{enumerate}[label=\alph*)]
			\item Ферт и Грюнберг не только существенно улучшили 
				магнеторезисторы, но и увидели эти наблюдения как новое 
				явление, в котором происхождение магнетосопротивления было 
				совершенно другим.

			\item Поскольку магнетосопротивление связано с электрической 
				проводимостью, очевидно, что именно поведение электронов 
				на поверхности Ферми (определяемой энергией Ферми) 
				наиболее интересно.

			\item Здесь изолирующий материал должен быть всего несколько 
				атомных слоев в толщину, чтобы электроны имели 
				значительную вероятность квантовомеханически 
				протуннелировать через изолирующий барьер.
		\end{enumerate}
	\end{otherlanguage}
\ifcols\end{multicols}\fi% }}}

\task{P. 86, translation}
\ifcols\begin{multicols}{2}\fi% {{{
	\textbf{The contribution of Albert Fert and Peter Gr\"{u}nberg}

	Albert Fert with his colleagues studies a system of a few tens 
	of alternating layers of iron and chromium. To get the expected 
	effect, scientists conducted the experiment in almost complete 
	vacuum and at low temperature. Peter Gr\"{u}nberg's group dealt 
	with a simpler system consisting of only two or three iron 
	layers separated by chromium layers.

	Fert found that the electric resistance of films decreases by 
	50\%, when the relative magnetization of ferromagnetic layers 
	changes from antiparallel to parallel, while an external 
	magnetic field is applied at low temperature. Gr\"{u}nberg's 
	figures are smaller --- only 1.5\% --- but at room temperature 
	(this figure rose to 10\% at 5 K). The physical nature of the 
	phenomenon seen by the two groups independently turn out to be 
	the same. The scientists concluded they observed a completely 
	new phenomenon. Albert Fert was one of those who suggested 
	a theoretical explanation of Giant Magnetoresistance, and in his 
	first publication, 1988, noted that the discovery can be 
	important for application. Peter Gr\"{u}nberg also noted the 
	practical potential of the phenomenon and together with 
	publishing his research in 1989 also patented his work in 
	Germany, Europe, and USA.

	But for wide use of the new technology, it was required to 
	develop an industrial way to achieve such thin layers. The 
	method used by Gr\"{u}nberg and Fert was pretty complex and 
	expensive. It best suited for laboratory investigations, but not 
	large scale industrial manufacturing. The works of an Englishman 
	Stuart Parkin helped to embody the fundamental developments. He 
	showed that the magnetron sputtering technique can be used to 
	create thin-layered magnetic sandwiches, even at room 
	temperature. The manufacturing of GMR reading heads started in 
	1997 and let us increase the capacity of hard drives manifold.
\ifcols\end{multicols}\fi% }}}

\end{document}
