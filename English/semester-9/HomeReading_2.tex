\documentclass[a4paper, 12pt]{article}

% Configuration {{{
\usepackage[utf8]{inputenc}
\usepackage[T2A]{fontenc} % T1 for English
\usepackage[russian, english]{babel}

\usepackage{enumitem}
\setlist{nolistsep}
\usepackage{mathtools}
\usepackage{xcolor}
\definecolor{dimblue}{HTML}{1010aa}
\usepackage[
	colorlinks=true, 
	allcolors=dimblue
]{hyperref}
\usepackage[
	vmargin=1in,
	hmargin=1in
]{geometry}
\linespread{1.3}
\usepackage{indentfirst}
\usepackage{graphicx}
\usepackage[multidot]{grffile}
\usepackage[labelsep=period]{caption}

%\usepackage{times} % for English
\setlength{\parskip}{10pt}
\setlength{\parindent}{0pt}
% }}}

\begin{document}

% Heading {{{
\begin{center}
	\begin{large}
		\textbf{Home Reading} \linebreak
		\textbf{Nucleon Resonances and Quark Structure}
	\end{large}
	\\\textit{\small Pages 10--18 of the PDF file, numbered 1144--1152 in the header, sections 3.2--4.2 inclusive}

	Kerim Guseynov \\
	\textit{Group 113M} \\
	November 6, 2020
\end{center}
% }}}

% Vocabulary {{{
\begin{center}
	\textbf{Vocabulary}
	\\\textit{\small (words I didn't know the meaning of)}
%\end{center}
%\begin{center}
	\par\begin{tabular}{r@{ --- }l c}
		formidable    & внушительный    & (section 3.2) \\
		viable        & приемлемый      & (section 3.2) \\
		lattice sites & границы решетки & (section 3.2) \\
		amass         & накопить        & (section 4.2) \\
	\end{tabular}
\end{center}
% }}}

% Questions {{{
\begin{center}
	\textbf{Questions}
\end{center}
\begin{enumerate}
	\item What is the modern theory describing quark interactions?

		It is called quantum chromodynamics, and it's a non-Abelian 
		gauge theory, in which quarks interact through the exchange 
		of colored gluons.

	\item What was the first evidence of the existence of 
		the structure of the nucleon?

		The first evidence of nucleon structure came from deep 
		inelastic scattering of electrons from protons, it showed 
		that protons have point-like objects inside.

	\item By means of what do proton and electron interact in 
		deep inelastic scattering?

		They interact by exchange of a virtual photon that carries 
		large four-momentum.

	\item How does the point-like structure of the proton appear 
		in its structure functions?

		The structure function $F_2$ extracted from the 
		experimental data remains constant over five orders of 
		magnitude of $Q^2$.

	\item What is the parton distribution function and how can we 
		use it?

		The parton distribution function is the probability of 
		finding a parton of a specific kind and momentum fraction 
		in the hadron. We use them to apply the theory of quarks 
		to hadrons.

	\item Do parton distribution functions depend on $Q^2$?

		They depend on $Q^2$, but only slightly, logarithmically. 
		And the evolution equations are quite different for sea 
		quarks and valence quarks.
\end{enumerate}
%}}}

% Main thoughts {{{
\newpage
\begin{center}
	\textbf{Main thoughts}
\end{center}
\vskip-.5\baselineskip
Quarks cannot be observed on their own, neither do they have 
integer electric charges. Hence they were not believed to 
exist at first. But as experiments revealed a point-like 
structure of the proton, and as theorists provided a reasonable 
explanation of quark confinement, scientists started to 
consider quarks real. Later, three more quarks have been 
discovered, and a complete theory of six quarks grouped into 
three pairs, called quantum chromodynamics, was built. Within 
the theory, quarks have two major features resulting from the 
force carrier properties: asymptotic freedom and confinement. 
Another important effect of QCD is that apart from three 
valence quarks in the proton there are sea quarks and gluons. 
To describe them all, parton distribution functions were 
introduced and are now used to calculate hadron interactions.
%}}}

% Presentation speech {{{
\begin{center}
	\textbf{Presentation speech}
\end{center}
\vskip-.5\baselineskip
After many particles had been discovered in the 20th century, 
there was a need for an approach to describe all of them 
consistently. This is when the quark model was born. Despite 
its obvious success, it contradicted some fundamental 
expectations from a theory and was not immediately accepted. 
First, quarks did not exist outside hadrons and hence were 
unobservable. Second, quarks had non-integer electric charges, 
so the previously known elementary charge seemed to cease to 
be elementary. And third, there were baryons consisting of 
three quarks of the same flavor, and since quarks were 
fermions, it violated the very fundamental Pauli principle. 
However, the proton had to have a point-like structure seen in 
the deep inelastic scattering experiments of that time. So 
scientists throughout the world tried to solve all the 
problems mentioned above.

Of course, as we all now know, scientists did find the 
solutions, and a working theory was created. It introduced new 
force carriers, called gluons, that unlike photons, interact 
with each other. This causes some extraordinary effects acting 
as the solutions. First, when we try to pull apart two quarks, 
the gluon field between them, due to its own interaction, 
becomes tremendously strong and produces quark-antiquark pairs 
thus breaking the string and not letting any quarks be free. 
This property is called quark confinement. Second, since there 
are no free quarks, the old elementary charge stays 
elementary, and nothing contradicts all the old experiments 
about it. And third, what is not a result of gluon interactions, 
but more like a reason for them, quarks have an additional 
quantum number called color, which can take three values, and 
the color of all hadrons has to be neutral. This lets 
particles like the $\Delta^{++}$ resonance exist.

Since its first appearance, the quark model has been extended 
with three more quarks and now has three light quarks called 
$u$, $d$, and $s$, two heavy $c$ and $b$ quarks, and one 
super heavy $t$ quark that decays so fast that it doesn't even 
have any bound states. This all led to the creation of an even 
more comprehensive theory, called quantum chromodynamics. It 
let us further understand the physics of quarks and hadrons. 
For example, it was found that a large portion of the hadron 
mass is created by so called sea quarks being produced inside 
the hadron due to gluon radiation.

This effect causes another difficulty for the description of 
hadrons. When we try to apply our beloved perturbation theory 
to the proton, we see that the contribution of sea quarks is 
more than 98\%, and the series doesn't converge. However, when 
we consider heavy quarks and hadrons comprising them, we see 
that what was 98\% for the proton is now only 6\%, and this 
opens new possibilities to test whether our theory and 
understanding are correct or not.

What should also be mentioned is that we are not helpless to 
describe the structure of the proton. To consider all the 
quarks and gluons inside it, we can introduce momentum 
distributions called parton distribution functions. By means 
of them, at high energies, thanks to asymptotic freedom of 
quarks, we can calculate some properties and reactions. This 
way we can understand what exactly is in the proton. But as 
you can see, such situations are very specific and demand 
a lot from the experiment. And the results of experiments show 
that we are correct and our theory is correct, at least for 
today. Otherwise we wouldn't learn it.
%}}}
\end{document}

% Backup stuff {{{



\begin{center}
	\textbf{Tasks}
\end{center}

\begin{itemize}
	\item Send the article
	\item compile a list of new words
	\item make up 6 questions on the contents
	\item write the main thought of this all
	\item make a video talk with a presentation
\end{itemize}








Some considered quarks as a mathematical device allowing to 
predict the properties of hadrons because they violated the 
Pauli principle and were always confined. But color degrees of 
freedom solved these issues. Another large evidence for the 
existence of quarks came from deep inelastic scattering, when 
it showed point-like objects inside the nucleon.

Since its first success, the quark model has been extended with 
three additional quark flavors. In 1974 the charm quark was 
discovered, with a mass roughly 1500 MeV, in the $J/\psi$ meson. 
In 1977, the bottom quark was discovered, with a mass about 
4500 MeV, in the $\Upsilon$ meson. And finally, in 1995, the 
top quark, having a mass roughly 171,000 MeV, was discovered at 
the Tevatron, without any bound state.

The quarks were grouped into three doublets: (u,d), (c,s), and 
(t,b). In each pair the first quark has a charge $+$2/3 and the 
second quark $-$1/3. Except for the light quarks, the doublets 
are also characterized by exceptionally large mass splittings.

We also know that masses of baryons are heavily affected by sea 
quarks, so we cannot expect perturbative theory to work with 
light baryons. However, the charm and bottom quarks are much 
heavier, and the contribution of sea quarks and gluons should 
not be so large. This leads us to heavy quark effective theory, 
which tries to describe properties of $b$- and $c$-hadrons 
assuming numerous simplifications.

Now we have a theory of quark interactions called quantum 
chromodynamics, which is a non-Abelian gauge theory of 
interaction mediated by exchange of gluons. This theory predicts 
asymptotic freedom of quarks, so at very high energies, like in 
deep inelastic scattering, they act like and are seen as free 
quarks. On the other hand, at low energies, quarks interact 
very strongly, and gluons also interact, unlike photons in QED. 
This causes another problem for the understanding of quark 
interactions. One more major property of quark interactions is 
that if two quarks, or a quark and an antiquark, are separated, 
a gluon flux tube forms between them and causes them to produce 
quark-antiquark pairs and hence break the tube. As a result, 
quarks don't exist alone and are always surrounded by other 
quarks. This property is called quark confinement.

Now let's consider experimental results indicating in favor of 
the quark structure of baryons
%}}}
