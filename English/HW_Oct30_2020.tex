\documentclass[a4paper, 12pt]{article}

\usepackage[utf8]{inputenc}
\usepackage[T2A]{fontenc}
\usepackage[russian, english]{babel}

\usepackage[
	vmargin=1in,
	hmargin=1in
]{geometry}
\linespread{1.3}
\usepackage{xcolor}
\definecolor{allrefs}{HTML}{1010aa}
\usepackage[
	colorlinks=true, 
	allcolors=allrefs
]{hyperref}
\usepackage{indentfirst}
\usepackage{graphicx}
\usepackage[multidot]{grffile}
\usepackage[labelsep=period]{caption}
\usepackage{enumitem}
\setlist{nolistsep}
\usepackage{mathtools}

%\usepackage{times}
\usepackage{multicol}
\usepackage{lipsum}
\setlength{\columnsep}{.25in}

\def\task#1{\begin{center}\it #1\end{center}}
\def\ans#1{\textit{#1}}

\newif\ifcols
\colsfalse

\begin{document}

\noindent
Kerim Guseynov, 113M group
\hfill
Oct 30

\task{Unit 6, p. 68--69, translation and three questions}
% Translation {{{
\begin{otherlanguage}{russian}
	\begin{large}
		\textbf{Космическое микроволновое фоновое излучение}
	\end{large}
	\vskip-.5\baselineskip
	\par\noindent\hrulefill
	\vskip-.3\baselineskip
	\noindent \textbf{Нобелевская премия по физике 2006
	\\Пресс релиз
	\\3 октября 2006}

	Шведская королевская академия наук решила присвоить Нобелевскую премию по 
	физике в 2006 году совместно Джону С. Матеру, НАСА, Космический центр полетов 
	имени Годдарда, Гринбельт, Мэриленд, США, и Джорджу Ф. Смуту, Университет 
	Калифорнии, Беркли, Калифорния, США, ``за открытие чернотельную форму 
	и анизатропию космического микроволнового фонового излучения''.

	\textbf{Изображения новорожденной Вселенной}

	В этом году награда по физике присуждена за  работу, которая оглядывается 
	назад ко временам, когда Вселенная была младенцем, и пытается достичь 
	некоторого понимания происхождения галактик и звезд. Она основана на 
	измерениях, сделанных с помощью спутника COBE, запущенного НАСА в 1989 году.

	Результаты COBE еще сильнее подкрепили Теорию большого взрыва для рождения 
	Вселенной, поскольку только в этом сценарии предсказывается обнаруженный COBE 
	вид космического микроволнового излучения. Эти измерения также отметили 
	основание космологии как точной науки. Незадолго до этого еще один спутник, 
	WMAP, получил даже более точные изображения фонового излучения. Очень скоро 
	европейский спутник имени Планка будет также запущен для изучения излучения 
	в еще больших деталях.

	Согласно Теории большого взрыва, космическое микроволновое фоновое излучение 
	-- это реликт самой ранней фазы Вселенной. Непосредственно после самог\'{о} 
	большого взрыва Вселенная похожа на светящееся тело, излучающее радиацию, 
	в которой распределение по длинам волн определяется одной лишь температурой. 
	Форма спектра этого типа излучения имеет особый объект, называемый абсолютно 
	черным телом. Когда оно излучалось, температура Вселенной была почти 3000 
	градусов по Цельсию. С тех пор, согласно Теории большого взрыва, излучение 
	постепенно охлаждалось по мере того, как Вселенная расширялась. Фоновая 
	радиация, которую мы можем измерять сейчас, соответствует температуре в всего 
	лишь 2.7 градуса выше абсолютного нуля. Лауреаты смогли рассчитать эту 
	температуру благодаря спектру черного тела, обнаруженному измерениями COBE.

	Кроме того, целью COBE было также искать маленькие вариации температуры 
	в разных направлениях (термин анизатропия означает именно это). Чрезвычайно 
	маленькие изменения температур такой величины, как у космического фонового 
	излучения --- в диапазоне статысячной градуса --- предоставляют важные улики 
	о том, как галактики зарождались. Вариации температуры показывают, как 
	вещество во Вселенной начало собираться. Это было необходимо для развития 
	галактик, звезд и в конце концов жизни. Без данного механизма материя приняла 
	бы совершенно другую форму, распространяясь равномерно по всей Вселенной.

	COBE был запущен с помощью собственной ракеты 18 ноября 1989. Первые 
	результаты были получены после девяти минут наблюдений: COBE зарегистрировал 
	идеальный спектр абсолютно черного тела. Когда кривая была затем показана на 
	конференции по астрономии, участники аплодировали стоя.

	Успех COBE был итогом необыкновенной командной работы, включающей более 
	тысячи исследователей, инженеров и других участников. Джон Матер 
	координировал весь процесс и также нес основную ответственность за 
	эксперимент, открывший чернотельную форму спектра микроволнового фонового 
	излучения, измеренного COBE. Джордж Смут был главным ответственным за 
	измерения малых вариаций температуры излучения.
\end{otherlanguage} %}}}
% Questions {{{

\textbf{Questions}
\begin{enumerate}
	\item Why is it important to know the anisotropy of the cosmic background 
		radiation? \ans{It helps us understand why the Universe has the 
		distribution of matter it has.}
	\item What was the Universe like in the early stage of its life? \ans{It was 
		like a glowing body, emitting radiation with a spectrum similar to the 
		spectrum of the absolutely black body.}
	\item What kind of work is it to conduct an experiment involving a satellite? 
		\ans{It is an enormous team work requiring more than a thousand people and 
		several people in charge of the progress of the experiment.}
\end{enumerate}
%}}}

\task{Ex. 1 p. 69}
% {{{
\textit{Излучение черного тела} -- blackbody radiation;
\textit{попытки понять происхождение галактик и звезд} -- attempts to gain some understanding of the origin of galaxies and stars;
\textit{быть основанным на измерениях} -- to be based on measurements;
\textit{предоставить значительную поддержку теории большого взрыва} -- to provide increased support for the Big Bang Theory;
\textit{возникновение космологии как точной науки} -- the inception of cosmology as a precise science;
\textit{дать еще более четкие изображения} -- to yield even clearer images;
\textit{реликт наиболее ранней стадии существования вселенной} -- a relic of the earliest phase of the Universe;
\textit{суметь измерить температуру} -- to be able to calculate the temperature;
\textit{предоставить важную подсказку о том, как появились галактики} -- to offer an important clue to how the galaxies came into being;
\textit{принять совершенно иную форму} -- to take a completely different form;
\textit{равномерно распространиться по вселенной} -- to spread evenly throughout the Universe;
\textit{результат удивительной командной работы, включавшей более 1000 исследователей, инженеров и других участников} -- an outcome of prodigious team work involving more than 1000 researchers, engineers, and other participants;
\textit{координировать весь процесс} -- to coordinate the entire process;
\textit{нести основную ответственность за что-либо} -- to have main responsibility for something.
% }}}

\task{P. 70--73 main thoughts of the interview}
% {{{
Professor George F. Smoot was awarded the Nobel Prize in Physics for his 
research on anisotropy of cosmic background radiation, which can tell us about 
the earliest galaxies of the Universe. In terms of time, it is like looking at 
a few hours old embryo. The discovery of anisotropy further confirms the 
accepted theory of cosmology, which is the Big Bang Theory. Professor Smoot 
himself was sleeping when he was called from the Royal Academy of Sciences, his 
time was 3 a.m., and he didn't sleep since then till the call. And he has many 
things on the schedule even apart from meetings related to the Nobel Prize.
% }}}

\task{P. 73--76 main thoughts of the interview}
% {{{
Professor John C. Mather was awarded the Nobel Prize in Physics for the 
discovery of cosmic background radiation, a faint signature left by the early 
Universe. It is important to use satellites because on Earth the atmosphere 
makes the picture a lot different. The spectrum of radiation has a shape of the 
blackbody radiation spectrum, and it tells us that it did come from the big 
bang. In the background radiation, we now should study the polarization to find 
out how the first hot objects were created. And we need to increase precision 
to continue the work.
% }}}

\task{P. 77--78 translation}
% {{{
The Nobel Prize in Physics for 2006 was awarded to American researchers John 
Mather and George Smoot ``for their work allowing us to track the development 
of the Universe and understand the origination process of cosmic space, stars, 
and galaxies'' -- said in the statement of Swedish Royal Academy of Sciences.

At the beginning of the 1960s, American researchers Arno Pensias and Robert 
Winson discovered a weak electromagnetic radiation coming from the space. This 
radiation was aptly called relic by a famous Russian astrophysicist I. S. 
Shklovskiy because its fluctuations (deviations from the average value) 
correspond to the density distribution of matter in the Universe, which existed 
many billions years ago.

The Big Bang scenario implies that at the first phase of the Universe 
development, it was filled with heated to hundreds of thousands degrees ionized 
gases -- hydrogen and helium, which were opaque for radiation. Approximately 
after a half million years the temperature fell to 3000 degrees, gas became 
neutral and transparent. Radiation started to transverse in all directions, 
carrying information about the matter density distribution. During the time 
passed from that moment, the radiation temperature fell to 2.725 K, and the 
average value of fluctuations is about $10^{-5}$ K. The relic radiation 
spectrum matches the absolutely black body spectrum.

It was possible to measure such small temperature differences using the 
satellite COBE launched by NASA in 1989 and a system of Soviet satellites 
Relikt and Relikt-1, which were orbiting in the same years. The analysis of 
data received from the orbital equipment let us prove validity of the Big Bang 
Theory and obtain a map of primary distribution of matter in the early 
Universe, which stars, galaxies and galaxy clusters came from.
% }}}


\end{document}
