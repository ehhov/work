\documentclass[a4paper, 12pt]{article}

% Configuration {{{
\usepackage[utf8]{inputenc}
\usepackage[T2A]{fontenc} % T1 for English
\usepackage[russian, english]{babel}

\usepackage{enumitem}
\setlist{nolistsep}
\usepackage{mathtools}
\usepackage{xcolor}
\definecolor{dimblue}{HTML}{1010aa}
\usepackage[
	colorlinks=true, 
	allcolors=dimblue
]{hyperref}
\usepackage[
	vmargin=1in,
	hmargin=1in
]{geometry}
\linespread{1.3}
\usepackage{indentfirst}
\usepackage{graphicx}
\usepackage[multidot]{grffile}
\usepackage[labelsep=period]{caption}

%\usepackage{times} % for English
\setlength{\parskip}{10pt}
\setlength{\parindent}{0pt}
% }}}

\begin{document}

% Heading {{{
\begin{center}
	\begin{large}
		\textbf{Home Reading} \linebreak
		\textbf{Nucleon Resonances and Quark Structure}
	\end{large}
	\\\textit{\small Pages 18--29 of the PDF file, numbered 1152--1163 in the header, sections 5--7 inclusive}

	Kerim Guseynov \\
	\textit{Group 113M} \\
	November 6, 2020
\end{center}
% }}}

% Vocabulary {{{
\begin{center}
	\textbf{Vocabulary}
\end{center}
\vskip-.5\baselineskip
No word was unknown to me. The closest to being unknown was convoluted 
(section 6.2.2), which I know what means in mathematics (and here) --- 
свёрнутый --- but didn't know what means in general context --- очень 
сложный, запутанный.
% }}}

% Questions {{{
\begin{center}
	\textbf{Questions}
\end{center}
\begin{enumerate}
	\item How can we obtain information about the distribution of 
		$s$-quark in the proton?

		To obtain such information, one can observe charge-changing 
		reactions with protons and muon neutrinos and search for a pair of 
		oppositely charged muons in the final state.

	\item How is the information about PDFs extracted from experimental data?

		High energy deep inelastic scattering experiments provide us with 
		data using which one can calculate an $F_2$ structure function for 
		the proton. $F_2$ in its turn can be expressed through PDFs.

	\item What carries the proton momentum?

		The proton momentum is carried by particles inside it, but not 
		mainly valence quarks as supposed by a naive picture. It turns out 
		that valence quarks carry about 35\%, sea quarks carry 15\%, and 
		gluons carry almost 50\% of the proton momentum.

	\item Are PDFs the same in the nucleus?

		PDFs don't stay intact in the nucleus because of several reasons. 
		First, the nucleus consists of many nucleons, some of which end up 
		being on the surface and some in the center. The latter ones have 
		less access to what happens outside the nucleus. Second, nucleon 
		interactions are carried by pions, which also consist of quarks and 
		hence affect their distributions. And third, nucleons in the 
		nucleus have their own momentum distributions which distort PDFs.

	\item What is deeply virtual Compton scattering?

		It is a reaction where a charged lepton is interacting with 
		a nucleon and exchanges a virtual photon with a high $Q^2$ 
		producing a real photon in the final state.

	\item What are the new and proposed facilities to study the nucleon 
		structure?

		New facilities are very expensive to build so most suggestions 
		relate to upgrading the current experimental setups. The upgrades 
		include higher energy electron beams at Jefferson Lab, a new 
		electron ring at RHIC, and a new light ion ring at JLab again.
\end{enumerate}
%}}}

% Main thoughts {{{
\begin{center}
	\textbf{Main thoughts}
\end{center}
\vskip-.5\baselineskip
Over the past three decades, we were able to develop the idea of parton 
distribution functions and calculate them to some extent. While we 
reached an uncontroversial picture describing the proton and its 
valence quarks, some problems still exist. The first such problem was 
the proton momentum. However, the gluon contribution was estimated, and 
scientists managed to incorporate momentum in the quark model. The 
second problem is the proton spin. According to polarized 
electron-proton scattering data, only a third of the proton spin is 
carried by quarks spins. The current gluon and angular momentum data 
are not enough to solve the issue, and the problem remains open. To 
investigate it further, new facilities are going to be build and old 
ones are going to be upgraded.
%}}}
\end{document}
% Presentation speech {{{
\begin{center}
	\textbf{Presentation speech}
\end{center}
\vskip-.5\baselineskip
As I said last time, there is a way to describe the structure of 
the proton. It is parton distribution functions. They express 
the probability for a parton to carry a specific fraction of the 
proton momentum. Subsequent high energy deep inelastic 
scattering experiments provided enough data to calculate 
structure functions and compare them against theoretical ones to 
verify and improve PDFs. We succeeded in calculating parton 
distributions over a wide range of Bjorken $x$ values. And 
evolution equations let us extend the measured distributions to 
larger and smaller $Q^2$ values to compare the results of 
different experiments.

One of the most precise methods to obtain structure functions on 
a hadron collider is the Drell-Yan processes. In these 
processes, a quark from one hadron annihilates with an antiquark 
of the same flavor from the other hadron, and a lepton pair is 
produced. It is very sensitive to antiquark distributions and 
can also provide information about heavy quarks in the colliding 
hadrons.

Another effective method is neutrino charge-changing reactions. 
When a muon neutrino is incident on a proton, it produces a muon 
and a $W^+$ boson which then reacts with an $s$ quark and 
produces a heavy quark and an antimuon. So in the final state, 
there are a high energy muon pair and a heavy hadron. This 
reaction is obviously sensitive to the $s$ quark distribution.

In a similar way was calculated every single component of the 
parton distributions. The result is presented here. One can 
clearly see valence quarks, sea quarks, and gluons. The picture 
looks logical even in a naive model since the down quark is 
approximately half the up quark over the whole $x$ region. So 
this seems like we have a correct representation of the structure 
of the proton. But we should probably discuss in more detail what 
we can tell about the proton using this information.

Since PDFs are momentum distributions, the first thing to look at 
is momentum. One can integrate the distributions over $x$ and 
obtain total momentum contributions for each quark flavor. The 
result was mentioned last time but is still a bit surprising. 
The valence quarks carry 30\% of the proton momentum, sea quarks 
carry 15\%, and gluons carry 50\%. This differs from a naive 
approach, but the experimental evidence is unambiguous.

Then, there is the nucleus. The neutron has a different set of 
PDFs but it is correlated with the one for the proton and is 
hence known. Nuclei consist of protons and neutrons, but do 
partons inside them behave the same as in nucleons? The answer 
is no for several reasons. First, the nucleus consists of many 
nucleons, some of which end up being on the surface and some in 
the center. And the latter ones have less access to what happens 
outside the nucleus. Second, nucleon interactions are carried by 
pions which also consist of quarks and hence affect quark 
distributions. And third, nucleons in the nucleus have their own 
momentum distributions which distort the bare nucleon PDFs.

Now let's consider polarized PDFs which also include information 
about the parton spin direction. In a naive model, the three 
valence quarks have two spins up and one down and account for 
100\% of the proton spin. In reality, one can imagine that the 
proton spin is made up of quarks spins $\Delta\Sigma$, gluons 
spins $\Delta G$, and their angular momenta $L\,q,g$. Polarized 
PDFs for quarks let us calculate their spin contribution 
$\Delta\Sigma$. And it appears that they account for only 
a third of the proton spin. On the other hand, this number is 
similar to what we saw for momentum, so we should look at other 
components.

The difficult part here is that it's very hard to extract the 
gluon contribution to inelastic scattering reactions. To do so, 
one has to detect a hadron and a lepton from the same event at 
the same time. This requires some modifications for most 
existing detectors. The few detectors capable of conducting such 
experiments now show controversial results and can't even tell 
us which sign the gluon contribution is.

Another component of the proton spin is angular momentum. It has 
to deal with the spatial distribution of quarks and gluons in 
the proton. But up to now we were only talking about momentum 
distributions and were not concerned with spatial ones. So this 
component requires a completely new approach. It was realized 
long ago, and some steps have been made. The largest was the 
introduction of so-called generalized parton distributions, 
which also deal with transverse momentum and hence a bit of 
spatial distribution. However, we are far from being able to 
calculate or plot them.

As you can see, there is a lot of work to be done in this field. 
So new accelerators and upgrades of old ones are planned. At 
Jefferson Laboratory, the accelerator upgrade would provide 
a higher energy electron beam. And a new light ion accelerator 
ring would allow for a broader range of scattering experiments 
to be conducted at the laboratory. An alternative to that light 
ion accelerator can be constructed at RHIC at Brookhaven. This 
accelerator already has an ion collider and can be complemented 
with an electron accelerator to provide similar functionality. 
The difference would be that heavier ions could be accelerated, 
but the electron beam would be somewhat weaker.

After upgrading these installations we will be able to further 
investigate the structure of the proton and other hadrons and 
solve some of the mysteries we can't explain now.
%}}}
\end{document}
