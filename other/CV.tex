\documentclass[a4paper, 12pt]{article}

\usepackage[utf8]{inputenc}
\usepackage[T1]{fontenc}
\usepackage[english]{babel}

\usepackage[
	colorlinks=true,
	allcolors=blue
]{hyperref}
\usepackage{enumitem}

\usepackage[
	%showframe,
	vmargin=1in,
	hmargin=.6in
]{geometry}
\linespread{1.3}

%\usepackage{times}

\setlength{\parindent}{0pt}


%%%%%%%%%%%%%%%
\newlength{\indlen}
\setlength{\indlen}{.5in}
\newlength{\contw}
\setlength{\contw}{\dimexpr(\linewidth-\indlen-1pt)}
\def\field#1{{ \bf #1 } \\*[.05in] }
\newcommand{\content}[3][]{
	\hspace*{#1\indlen}
	\parbox[t]{\contw}{
	\begin{tabular}{#2}
		#3
	\end{tabular}
	}
	\\[.5\baselineskip]
}
\newcommand{\contentnotab}[2][]{
	\hspace*{#1\indlen}
	\parbox[t]{\contw}{
		#2
	}
	\\[.5\baselineskip]
}
\def\mailto#1{\href{mailto:#1}{#1}}


%%%%%%%%%%%%%%%%
\begin{document}

{ \bf\large
	Curriculum Vitae
	\hfill
	Abdul-Kerim Guseinov
}
\vskip\baselineskip

\field{Personal data}
\content{ll}{
	Born & Nov 13, 1998 \\
	Location & Moscow, Russia \\ 
	Nationality & Russian \\ 
	Mobile phone & +7 915 273 3700 \\
	Email & \mailto{guseynovkerim@gmail.com} \\
}

\field{Education}
\content[0]{rl}{
	Sep 2016 -- Jun 2020 & \textbf{Lomonosov Moscow State University} \\
	& Faculty of Physics, department of General Nuclear Physics \\
	& Awarded for only excellent grades achieved throughout the course of study \\
	-- Jun 2016 & \textbf{Lyceum N1586 in Moscow, Russia} \\
	& Graduated with distinction (only excellent grades in Grade Certificate)
}

\field{Research and other educational activities}\\[-2\baselineskip]
\begin{itemize}[nolistsep]
\item Since 2019, I am a member of a HEP research group at my University which works with the LHCb collab.
\item In 2019 I was awarded a grant to take part in the Second Trans-Siberian School on High Energy Physics, organized by Tomsk Polytechnic University and CERN.
\item On behalf of the LHCb collab., I have presented the preliminary results on $\Lambda_b^0$ multihadron decays at a conference at my University. 
\item The talk is going to be published in an e-magazine Memoirs of the Faculty of Physics, Lomonosov Moscow State University. \url{http://uzmu.phys.msu.ru/}
\item Currently I am working on systematic uncertainties of the mesured $\Lambda_b^0$ baryon decays' branching ratios.
\end{itemize}
\hspace*{0pt}

\field{Computing skills}
\content[0.09]{ll}{
	Languages: & C, C++, Python  \\
	\hfill C, C++ & Completed a course at my University \\ 
	\hfill Python & Studied on my own \\ 
	Working experience: & Linux, Bash shell, ROOT, PyROOT \\
}


\end{document}

I am interested in physics of heavy b,c quarks as well as in high pT physics including measurements of top quark properties. Currently I am involved in physics data analysis, but I would be very excited to extend my field of study with research programs on modern detectors development for HEP and on the front-end DAQ systems, since these would be quite a new experience for me. 
