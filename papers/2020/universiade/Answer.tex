\documentclass[a4paper, 12pt]{article}

\usepackage[utf8]{inputenc}
\usepackage[T2A]{fontenc}
\usepackage[english, russian]{babel}

\usepackage{tune}
\linespread{1.3}

\def\BR{\mathcal{B}}
\def\Lb{\varLambda_b^0}
\def\Lc{\varLambda_c^+}
\def\Dp{D^+}
\def\Dsp{D^{*+}}
\def\Km{K^-}
\def\pim{\pi^-}
\def\pip{\pi^+}
\def\piz{\pi^0}


\begin{document}

\textit{
	Есть ли оценки или соображения, как соотносятся канал 
	$\Lb\to\Dp p\pim\pim$ 
	и ранее изучавшийся канал 
	$\Lb\to\Lc\pip\pim\pim$?
}

\hspace{\baselineskip}

Экспериментальные результаты для довольно похожего распада 
$\Lb\to D^0 p\pim$, $D^0 \to \Km\pip$ 
в нормировке на канал 
$\Lb\to\Lc\pim$, $\Lc\to p\Km\pip$
уже были опубликованы 
(\href{https://arxiv.org/abs/1311.4823}{arXiv:1311.4823}). 
Согласно им, 
$$ \frac{\BR(\Lb\to D^0p\pim)}{\BR(\Lb\to\Lc\pim)}\times
\frac{\BR(D^0\to\Km\pip)}{\BR(\Lc\to p\Km\pip)} = 0.0806\pm\ldots.$$

В настоящей работе очарованные барионы идентифицируются в распадах 
$\Dp\to \Km\pip\pip$, $\Lc\to p\Km\pip$. Учитывая разницу в вероятностях рассматриваемых распадов $D$-мезонов и $\Lb\to\Lc\pim(\pip\pim)$ (\href{http://pdg.lbl.gov}{PDG}), для оценки изучаемой вероятности распада $\Lb$ необходимо внести поправку
$$
\begin{aligned}
\frac{\BR(\Lb\to\Dp p \pim\pim)}{\BR(\Lb\to\Lc\pip\pim\pim)} 
\times 
\frac{\BR(\Dp\to\Km\pip\pip)}{\BR(\Lc\to p\Km\pip)} 
\approx \text{\hspace*{15em}}
\\ \approx
0.0806\times 
\frac{\BR(\Dp\to\Km\pip\pip)}{\BR(D^0\to\Km\pip)}
\times
\frac{\BR(\Lc\to\Lc\pim)}{\BR(\Lb\to\Lc\pip\pim\pim)}
\approx 0.1218.
\end{aligned}
$$


\end{document}
