\clearpage
\nonumbersection{Выводы}
\label{sec:conclusion}

Распады $\Lb\to\Dp\p\pim\pim$ и $\Lb\to\Dstarp\p\pim\pim$ наблюдались 
с использованием данных, собранных детектором LHCb в протон-протонных 
соударениях и соответствующих интегральным светимостям $1$ 
и $2$~фб$^{-1}$ при энергиях системы центра масс 7 и 8~ТэВ, 
соответственно. Обе исследованные моды относятся к малоизученному классу 
распадов прелестных барионов, где $c$\nobreakdash-кварк, образующийся 
в результате перехода $b \to c$, адронизуется в конечное состояние, 
отличное от бариона, то есть очарованность и барионный заряд оказываются 
в разных адронах. Такие многоадронные распады проявляют богатую 
резонансную структуру.

Используя распад $\Lb\to\Lc\pip\pim\pim$ в качестве нормировки, 
были измерены отношения вероятностей распадов
\[ \frac{\BR{\Lb\to\Dp\p\pim\pim}}{\BR{\Lb\to\Lc\pip\pim\pim}}
  \!\times\!\frac{\BR{\Dp\to\Km\pip\pip}}{\BR{\Lc\to\p\Km\pip}}
  = (5.35 \pm 0.21 \pm 0.16) \!\times\! 10^{-2}, \]
и
\[ \frac{\BR{\Lb\to\Dstarp\p\pim\pim}}{\BR{\Lb\to\Dp\p\pim\pim}}
  \!\times\!\BR{\Dstarp\to\Dp\piz/\Dp\gamma}
  = (61.3 \pm 4.3 \pm 4.0) \!\times\! 10^{-2}, \]
где первая погрешность статистическая, а вторая -- систематическая.
%
Используя известные значения вероятностей распадов $\Dp\to\Km\pip\pip$, 
$\Lc\to\p\Km\pip$~\cite{PDG}, для отношения вероятностей каналов $\Lb$ 
получается выражение
\[ \frac{\BR{\Lb\to\Dp\p\pim\pim}}{\BR{\Lb\to\Lc\pip\pim\pim}}
  = (3.58 \pm 0.14 \pm 0.11 \pm 0.19) \times 10^{-2}, \]
где последняя ошибка обусловлена погрешностями вероятностей распадов 
очарованных адронов.

Используя известные значения вероятностей каналов $\Dstarp\to\Dp\piz$ 
и $\Dp\gamma$~\cite{PDG}, для резонансного распада через $\Dstarp$ мезон 
относительная вероятность
\[ \frac{\BR{\Lb\to\Dstarp\p\pim\pim}}{\BR{\Lb\to\Dp\p\pim\pim}}
  = 1.90 \pm 0.19, \]
где все погрешности объединены. Для многоадронных распадов с большим 
высвобождением энергии отношение вероятностей распадов через $\Dstarp$ 
и $\Dp$ должно быть аналогичным отношению частот рождения их 
в высокоэнергичных адронных или электрон-позитронных столкновениях. 
Наивный учет спинов частиц предсказывает, что это отношение равно 3.
%
Для оценки реального отношения сечений образования $\Dp$ и $\Dstarp$ 
напрямую при столкновении протонов на БАК были использованы результаты 
работ~\cite{lhcb-charm-prod-5tev, lhcb-charm-prod-7tev, 
lhcb-charm-prod-13tev},
\[
  \frac{\sigma^\text{direct}_{pp\to\Dstarp X}}
       {\sigma^\text{direct}_{pp\to\Dp X}} \approx
  \frac{\sigma_{pp\to\Dstarp X}}{
    \sigma_{pp\to\Dp X} - \sigma_{pp\to\Dstarp X}
    \times \BR{\Dstarp\to\Dp\piz/\gamma}
  }
  = 1.5 \pm 0.1.
\]
Это значение меньше отношения вероятностей распада $\Lb$, но находится 
в пределах двух стандартных отклонений. Отношение сечений рождения 
мезонов $\Dp$ и $\Dstarp$ в $e^+e^-$ аннигиляции, $1.86\pm0.16$, взято 
из работы~\cite{ee-charm-prod}, где оно оценивалось на основе измерений 
CLEO~\cite{ee-charm-prod-cleo}, ARGUS~\cite{ee-charm-prod-argus}, 
ALEPH~\cite{ee-charm-prod-aleph} и VENUS~\cite{ee-charm-prod-venus}. 
Наблюдаемые сходства указывают на возможное соответствие между прямым 
образованием и фрагментацией очарованных мезонов и их рождением 
в многочастичных распадах прелестных адронов.

\begingroup \par \sloppy
В итоге, в работе были изучены распады $\Lb\to\Doptstarp\p\pim\pim$ 
в нормировке на канал $\Lb\to\Lc\pip\pim\pim$. Получены 
экспериментальные спектры инвариантных масс, построены их модели 
и проведена аппроксимация. Результат был исследован на предмет искажений 
и откорректирован. Были оценены систематические погрешности, 
обусловленные моделью вклада $\Lb\to\Dstarp\p\pim\pim$ и прочими 
источниками. Произведено первое наблюдение распадов 
$\Lb\to\Dp\p\pim\pim$ и $\Lb\to\Dstarp\p\pim\pim$ и измерены их 
относительные вероятности. Оба этих распада и нормировочный канал 
$\Lb\to\Lc\pip\pim\pim$ проявляют богатую резонансную структуру.
%
Измеренные в работе распады $\Lb\to\Doptstarp\p\pim\pim$ в будущем могут 
служить опорой при изучении аналогичных редких распадов как, например, 
$\Xi_b^0\to\Dp\p\Km\pim$ и $\Xi_b^0\to\Dstarp\p\Km\pim$.
%
По результатам исследования распадов $\Lb\to\Doptstarp\p\pim\pim$ была 
опубликована статья~\cite{lb2dppipi-paper}.
\par \endgroup
