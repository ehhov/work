\clearpage
\nonumbersection{Введение}

На сегодняшний день наиболее успешно микромир описывает Стандартная 
модель, однако она не может быть окончательной теорией, поскольку не 
объясняет всех экспериментально наблюдаемых явлений.  Поиск расхождений 
эксперимента со Стандартной моделью может задать направление при 
создании и анализе ее расширений.  Изучение тяжелых адронов и редких 
распадов предоставляет уникальные возможности при поиске новой физики 
благодаря высокой чувствительности к ней.  Со стороны эксперимента, 
Большой адронный коллайдер обладает наибольшей производительностью среди 
всех ускорителей в истории, а расположенный на нем детектор LHCb 
предоставляет возможность изучения физики тяжелых $b$- и $c$-кварков 
в широком диапазоне исследований.  Например, одной из главных целей было 
наблюдение редкого распада $ B_s \to \mu\mu $. Он был измерен с высокой 
точностью, и отклонений от Стандартной модели, предсказывающей 
чрезвычайно малое значение, не обнаружилось.
%изучения редких распадов благодаря системе идентификации частиц 
%и большим сечением рождения тяжелых кварков, тем самым сочетая в себе 
%все необходимое для поиска новой физики. \check{не бред ли это?}

Данная работа является одной из первых, где наблюдается многочастичный 
распад прелестного бариона $\Lb$. Более того, особенностью изучаемого 
в работе распада является то, что в конечном состоянии $c$-кварк 
и барионное число переносятся разными адронами.  Интерес 
к многочастичным распадам обусловлен также возможностью наблюдения 
резонансных адронных состояний, которые либо не наблюдались, либо были 
измерены с недостаточной точностью.  Целью работы является наблюдение 
распада $\Lb\to\Dp\p\pim\pim$ и измерение его вероятности в нормировке 
на известную вероятность распада 
$\Lb\to\Lc\pim\pip\pip$~\cite{lb2lc3pi-lhcb,lb2lc3pi-cdf}, а также 
наблюдение распада $\Lb\to\Dstarp\p\pim\pim$ и измерение его вероятности 
относительно основного распада $\Lb\to\Dp\p\pim\pim$.  Для этого 
очарованные адроны регистрируются в модах $\Dstarp\to\Dp\piz/\Dp\gamma$, 
$\Dp\to\Km\pip\pip$, $\Lc\to\p\Km\pip$, что приводит к существенному 
подавлению систематических погрешностей, поскольку наборы частиц 
в конечных состояниях распадов совпадают. Кроме того, вероятности 
используемых распадов очарованных барионов не измерены с достаточной 
точностью и вносили бы преобладающую погрешность в результат. Поэтому 
они включены в него в виде множителей.  Таким образом, измеряются 
величины
%
\[\begin{aligned}
  R =&\, \frac{\BR{\Lb\to\Dp\p\pim\pim}}{\BR{\Lb\to\Lc\pim\pip\pim}} \times
	\frac{\BR{\Dp\to\Km\pip\pip}}{\BR{\Lc\to\p\Km\pip}} ,\\
R^* =&\, \frac{\BR{\Lb\to\Dstarp\p\pim\pim}}{\BR{\Lb\to\Dp\p\pim\pim}} \times
	\left( \BR{\Dstarp\to\Dp\piz} + \BR{\Dstarp\to\Dp\gamma} \right) .\\
\end{aligned}\]
%
Результат анализа уже опубликован в~статье~\cite{lb2dppipi-paper}.

Анализ основан на данных, собранных детектором LHCb в 2011--2012 годах, 
соответствующих интегральной светимости~3~фб$^{-1}$, и производится 
с помощью программных пакетов ROOT, RooFit, Ostap.

Далее в работе канал $\Lb\to\Dp\p\pim\pim$ будет также называться 
основным, $\Lb\to\Dstarp\p\pim\pim$ -- резонансным, 
а $\Lb\to\Lc\pim\pip\pim$ -- нормировочным.

%\check{Меня беспокоит, что если меня нагло спросят ``а зачем вообще 
%нужно измерять такие многочастичные распады с открытым чармом 
%и переходом барионного числа протону?'', я не смогу ответить точно}
