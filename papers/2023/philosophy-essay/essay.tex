\documentclass[a4paper, 12pt]{article}

% Configuration {{{
\usepackage[utf8]{inputenc}
\usepackage[T2A]{fontenc} % T1 for English
\usepackage[english, russian]{babel}

\usepackage{enumitem}
\setlist{nolistsep}
\usepackage{mathtools}
\usepackage{amssymb}
\usepackage{xcolor}
\definecolor{dimblue}{HTML}{1010aa}
\usepackage[
  colorlinks=true,
  allcolors=black  % not dimblue
]{hyperref}
\usepackage[
  right=1.8cm,
  left=3.2cm,
  vmargin=2.5cm
]{geometry}
\linespread{1.3}
\usepackage{indentfirst}
\usepackage{graphicx}
\usepackage{tikz}
\usepackage[multidot]{grffile}
\usepackage[labelsep=period]{caption}
\usepackage{multirow}

%\usepackage{times} % for English

\usepackage{cite}
\usepackage[russian]{csquotes}
\usepackage{definitions}
% }}}

\begin{document}

% Title page and Contents {{{
\thispagestyle{empty}
\begin{center}
  \sc
  Московский государственный университет им. М.\,В.~Ломоносова
  %\vskip -.8\baselineskip \hrulefill \vskip -.4\baselineskip
  \vskip 0pt
  Физический факультет
  %\hspace{1ex}
  \vskip 0pt
  Кафедра общей ядерной физики
\end{center}

\vfill

\begin{center}
  \large

  \vskip -9ex
  Реферат для сдачи кандидатского экзамена по курсу \\
  \enquote{История и философия науки}
  \vskip 3ex
  Перевод статьи \\
  Дж.~Гаррисона \enquote{Квантовая статистика тождественных частиц} \\
  (J.\,C.~Garrison, \enquote{Quantum Statistics of Identical Particles})
\end{center}

\vskip 5ex

\begin{flushright}
  Выполнил аспирант 1-го года обучения \\
  Гусейнов Абдул-Керим Демирович \\
  \vskip 1ex
  Научный руководитель: \\
  д. ф-м. н. Бережной Александр Викторович \\
  \vskip 1ex
  Преподаватель: \\
  д. филос. н. Хмелевская Светлана Анатольевна \\
\end{flushright}

\vfill

\begin{center}
  Москва~--- 2023
\end{center}

\clearpage
\tableofcontents
\clearpage
% }}}

\paperinit{Перевод}
%{{{
\label{sec:paper}

\papersection{Введение}
%{{{
\label{sec:paper:1}

Существенное эмпирическое свидетельство, что все системы тождественных 
частиц подчиняются либо Бозе, либо Ферми статистике~-- одно из наиболее 
поражающих свойств квантовой физики. Уже давно стоит смежная 
теоретическая задача того, что эти квантово-статистические свойства 
систем множества частиц невозможно вывести из аксиом квантовой механики. 
Их необходимо накладывать как еще одно предположение. Изначально это 
изъявлялось следующим образом~\cite{ref1}. % TODO ref 1

Постулат симметризации (ПС): векторы состояния, описывающие несколько 
тождественных частиц, являются либо симметричными (для бозонов), либо 
антисимметричными (для фермионов) при перестановках частиц.

Знакомая всем квантовая механика (ниже~-- ПСКМ) определяется добавлением 
постулата симметризации к аксиомам нерелятивистской квантовой механики. 
Для $N$ тождественных частиц это означает, что гильбертово пространство 
состояний содержит либо симметричные векторы для бозонов ($\hilbN_B$), 
либо антисимметричные~-- для фермионов ($\hilbN_F$).

ПСКМ версия квантовой теории нашла феноменальный успех, но даже успешные 
теории следует периодически подвергать экспериментальным проверкам. 
Любое правило, например, принцип Паули, которое накладывает строгие 
условия на предсказания экспериментальных результатов, следует проверять 
вновь по мере улучшения экспериментальных методов. Эта эмпирическая 
мотивация, совместно с отсутствием какого-либо убедительного 
теоретического объяснения постулата симметризации, стимулирует 
экспериментальные поиски возможных нарушений ПС~\cite{ref3, ref4, ref5, 
ref6, ref7, ref8, ref9, ref10}. % TODO refs 3--10
С~учетом неиссякающего интереса к этим вопросам, обзор более общего 
теоретического подхода может подтолкнуть интуицию в нужном направлении. 
Для ограничения теоретических возможностей, открываемых таким 
обобщением, рассмотрение ниже ограничивается минимальным расширением 
ПСКМ, которое совмещает идею неразличимости тождественных частиц 
с аксиомами квантовой механики.

Далее в тексте эрмитов оператор $A$, который представляет измеряемую 
физическую величину называется наблюдаемой, а набор его собственных 
значений обозначается $\ev{A}$. Вообще, могут существовать эрмитовы 
операторы, не представляющие какую-либо физическую величину. Поэтому для 
каждой физической системы необходимо предоставить условие, отбирающее 
только те эрмитовы операторы, которые могут быть наблюдаемыми. Результат 
$(a, \E_a(A))$ измерения $A$ состоит из измеренной величины $a \in 
\ev{A}$ и соответствующего собственного пространства $\E_a(A)$ с базисом 
$\ket{a:\mu}$, где $\mu \in \mathbb{N}, \mu < d_a$, $d_a$~-- степень 
вырождения $a$. Для набора коммутирующих наблюдаемых $\vec{A}$ результат 
объединенного измерения состоит из набора измеренных величин 
и объединенного собственного пространства, представляющего пересечение 
собственных пространств каждой наблюдаемой. Если цель измерения~-- 
оставить систему в чистом состоянии, соответствующем определенному 
собственному значению, его называют подготовкой состояния. Поскольку эта 
процедура включает отбрасывание результатов с другими собственными 
значениями, она также называется фильтрованием.

% TODO possibly remove at all
% В разделе~\ref{sec:paper:2} приведены обзор понятия тождественных 
% частиц и описание систем тождественных частиц. Определение 
% неразличимости тождественных частиц приведено 
% в разделе~\ref{sec:paper:3} и используется для формулировки расширения 
% ПСКМ, которое разрешает нарушение постулата симметризации. Особые 
% квантовые свойства систем неразличимых частиц приведены 
% в разделе~\ref{sec:paper:4}. Эксперименты по поиску нарушения принципа 
% Паули кратко обсуждаются в разделе~\ref{sec:paper:5} вместе с моделью, 
% иллюстрирующей, какие свойства ПНКМ были бы вовлечены в их анализ. 
% Общее обсуждение приведено в разделе~\ref{sec:paper:6}.

%}}}

\papersection{Тождественные частицы}
%{{{
\label{sec:paper:2}

Элементарная частица на данный момент понимается как один из объектов, 
включенных в Стандартную модель. Некоторые связные состояния 
элементарных частиц, такие как нуклоны, атомное ядро, атомы и молекулы, 
можно считать частицами до тех пор, пока они не испытывают 
взаимодействия достаточно сильные для возбуждения внутренних степеней 
свободы. Каждая из этих частиц идентифицируется ограниченным набором 
внутренних свойств, которые являются результатами измерений для 
единственной частицы (масса, заряд, спин и т.д.). Такая идентификация 
считается полной в сильном смысле, что не существует дополнительных 
одночастичных измерений, которые могли бы глубже определить частицы или 
различить между двумя частицами с одинаковыми внутренними свойствами.
%
Две частицы тождественны, если имеют одинаковые внутренние свойства, 
и тогда они принадлежат одному виду частиц. Каждая из них, учитывая 
известные тонкости касательно кварков и глюонов, может существовать 
в вакууме, будучи изолированной от всех других частиц. Это свойство 
отличает их от возбужденных состояний многочастичных систем, которые 
называются квазичастицами. Системы квазичастиц могут проявлять 
экзотические виды статистики~\cite{ref12, ref13}, % TODO refs 12, 13
но приведенные рассуждения касаются только статистик, происходящих из 
неразличимости тождественных частиц.

Аксиомы квантовой теории не предоставляют каких-либо отдельных правил 
для систем тождественных частиц, но предоставляют рамки для изучения их 
особых свойств. Для этой цели достаточно рассмотреть систему, состоящую 
из единственного вида частиц. Пространство одночастичных состояний 
$\hilb^{(1)}$ определяется базисом $\ket{\theta_i}$, где $\theta$~-- 
набор квантовых чисел, конкретный вид которого зависит от вида частицы 
(например, импульс и поляризация фотона). После нумерации индивидуальных 
частиц пространство одночастичных состояний частицы с номером $n$ 
$\hilb^{(1)}_n$ оказывается копией $\hilb^{(1)}$. Поскольку не было 
сделано предположений касательно поведения векторов состояний при 
перестановках частиц, $N$-частичное пространство состояний является 
полным тензорным произведением одночастичных пространств:
$\hilbN = \hilb^{(1)}_1\otimes\cdots\otimes\hilb^{(1)}_N$.
Базисные векторы являются тензорным произведением одночастичных векторов 
$\ket{\theta_{n'}}_n$, означающих, что частица $n$ находится в состоянии 
$n'$.

$N!$ различных способов нумерации частиц связаны между собой 
симметрической группой $S_N$, состоящей из перестановок номеров частиц 
$n\to\calP(n)$. Поскольку номера частиц не имеют физической значимости, 
их перестановка при неизменных номерах состояний является пассивной 
перестановкой, аналогичной повороту осей координат при неизменном 
положении физической системы. Пассивная перестановка $\calP$ действует 
на базисный вектор $\ket{\vec{\theta}}$ сначала изменяя номера частиц, 
$\ket{\theta_{n'}}_n \to \ket{\theta_{n'}}_{\calP(n)}$, а затем 
переставляя элементы тензорного произведения в порядке возрастания 
номера частицы $\calP(n)$. Это эквивалентно применению обратной 
перестановки к индексам состояний
$\ket{\theta_{n'}}_n \to \ket{\theta_{\calP^{-1}(n')}}_{n}$.
Перестановка номеров состояний в $\vec{\theta}$ уже является активной, 
аналогичной повороту физической системы при неизменных осях координат. 
Всякая активная перестановка $\calP$ действует на само пространство 
$\hilbN$ оператором $D(\calP)$, определяемым выражением
$ D(\calP)\ket{\vec{\theta}} = \ket{\tilde{\calP}(\vec{\theta})},
\quad \tilde{\calP}(\vec{\theta}) =
(\theta_{\calP(1')}, \ldots, \theta_{\calP(N')})$.
%
Кроме того, $\calP\to D(\calP)$ является унитарным представлением $S_N$ 
на носителе $\hilbN$. Таким образом, пассивная перестановка $\calP$ 
может быть выражена как активная перестановка в виде
$\ket{\vec{\theta}} \to D(\calP^{-1})\ket{\vec{\theta}}$.

%}}}

\papersection{Неразличимость тождественных частиц}
%{{{
\label{sec:paper:3}

Предположение об отсутствии дополнительных одночастичных измерений, 
которые могли бы различить между двумя тождественными частицами~-- 
источник фундаментальной интуиции, что простое изменение нумерации 
частиц не может изменить результаты какого-либо измерения. Это означает, 
что два вектора состояния $N$-частичной системы, связанные пассивной 
перестановкой, должны приводить к одним и тем же распределениям 
вероятностей измеряемых значений для всех наблюдаемых. Совмещая это 
с соотношением между активными и пассивными перестановками, можно прийти 
к идее неразличимости тождественных частиц.

Постулат неразличимости: два вектора состояния нескольких тождественных 
частиц, отличающиеся только через активную перестановку одночастичных 
состояний, приводят к одинаковым распределениям вероятностей при 
измерениях всех наблюдаемых.

Это утверждение применяется ко всем измерениям над полной системой, не 
к одночастичным измерениям над изолированными частицами. В этом 
существенная разница между неразличимостью и тождественностью. Постулат 
симметризации и постулат неразличимости оба совместимы с аксиомами 
квантовой теории, но ни один нельзя из них вывести. Версия квантовой 
механики, определяемая добавлением принципа неразличимости к аксиомам 
ниже называется ПНКМ. Благодаря использованию слова ``наблюдаемые,'' 
принцип неразличимости сам по себе подразумевает правило, определяющее, 
какие эрмитовы операторы на самом деле можно считать наблюдаемыми.

Для всякого состояния $\ket{\Psi}$ вероятность результата измерения $a$ 
наблюдаемой $A$ равна
$\prob{a|\Psi} = \sum_{\mu=1}^{d_a}\left|\braket{a:\mu}{\Psi}\right|^2 =
\bramidket{\Psi}{\Pi\left(\E_a(A)\right)}{\Psi}$,
где $\Pi$~-- оператор проекции на соответствующее подпространство.
%
Для любой наблюдаемой $A$, каждого ее собственного значения $a$, любого 
чистого состояния $\ket{\Psi}$ и любой активной перестановки $\calP$ 
постулат неразличимости требует, чтобы вероятность обнаружения $a$ при 
измерении $A$ была одинаковой для системы до и после перестановки.
%
Учитывая $\ket{\Psi}\to D(\calP)\ket{\Psi}$ и унитарность $D(\calP)$, 
возникает условие $\Pi(\E_a(A)) = D(\calP)^\dag \Pi(\E_a(A)) D(\calP)$.
Обозначая $A_\calP = D(\calP)^\dag A D(\calP)$ и учитывая, что 
собственные значения оператора $A_\calP$ совпадают с таковыми оператора 
$A$, получаем, что $\Pi(\E_a(A)) = \sum_{\mu=1}^{d_a}
D(\calP)^\dag \ket{a:\mu}\bra{a:\mu}D(\calP) = \Pi(\E_a(A_\calP))$.
%
Как видно, операторы $A$ и $A_\calP$ имеют одинаковые собственные 
значения и собственные пространства, а значит, они равны. Таким образом, 
любая наблюдаемая удовлетворяет $A = A_\calP = D(\calP)^\dag 
A D(\calP)$, что эквивалентно $[A, D(\calP)] = 0$, для всех $\calP\in 
S_N$.
%
Оператор, удовлетворяющий этому условию называется 
перестановочно инвариантным, что порождает следующее правило.

Правило перестановочной инвариантности: наблюдаемые для систем 
тождественным частиц должны быть перестановочно инвариантными.

Это необходимое (но не достаточное) условие того, что эрмитов оператор 
является наблюдаемой. В частности, из утверждения видно, что любая 
физическая наблюдаемая должна действовать на всю систему целиком.

Поведение векторов состояний при перестановках~-- в равной значимости 
интересная характеристика квантовой теории тождественных частиц. 
Подпространство $\subhilb$ пространства $\hilbN$ называется 
перестановочно инвариантным, если $\forall \calP\in S_N$ 
$D(\calP):\subhilb \to \subhilb$. Следовательно, представление $\calP\to 
D(\calP)$ над $\hilbN$ включает представление $S_N$ над каждым 
перестановочно инвариантным подпространством $\hilbN$.
%
Поскольку наблюдаемые~-- перестановочно инвариантные операторы, каждое 
собственное пространство наблюдаемой~-- перестановочно инвариантное 
подпространство, а значит, носитель представления $S_N$.
%
Перестановочно инвариантное подпространство называется неприводимым, 
если оно не содержит нетривиальных перестановочно инвариантных 
подпространств. Приводимое подпространство можно выразить через прямую 
сумму неприводимых подпространств~\cite[Гл.~3--13]{ref14}.
% ^ TODO ref 14, chap 3--13
Неприводимое подпространство является носителем неприводимого 
представления $S_N$, и наоборот~\cite[Гл.~7]{ref14}.
% ^ TODO ref 14, chap 7
Неприводимые представления $S_N$ обозначаются $\calP \to 
\D{\gamma}(\calP)$ (или просто $\D{\gamma}$), где $\D{\gamma}$~-- 
унитарная матрица $d_\gamma \times d_\gamma$, а $d_\gamma$~-- 
размерность представления.
% Индекс $\gamma$ принимает значения из набора $\Gamma_N$, определенного 
% в приложении 1.

%}}}

\papersection{Квантовая механика неразличимых частиц}
%{{{
\label{sec:paper:4}

%{{{
В ПСКМ версии квантовой механики пространство состояний 
$\hilbN_\text{ПС}$ системы $N$ тождественных частиц является либо
$\hilbN_B$, либо $\hilbN_F$. Следовательно, $\forall 
\ket{\Psi} \in \hilbN_\text{ПС}$ выполняется $D(\calP)\ket{\Psi} 
= \pm \ket{\Psi}$, где для бозонов знак равен $+$, а для фермионов~-- 
совпадает с четностью перестановки $\calP$.
%
Таким образом, любой оператор, действующий в гильбертовом пространстве 
состояний, будет коммутировать с $D(\calP)$, и необходимое требование 
к наблюдаемым, выведенное выше, выполняется автоматически.

Совершенно иначе обстоит дело с ПНКМ версией квантовой механики, где 
пространство состояний является полным тензорным произведением 
одночастичных пространств. Вместо того, чтобы накладывать условия 
симметрии на вектора состояний, правило перестановочной инвариантности 
используется для выбора эрмитовых операторов, которые могут служить 
наблюдаемыми. Таким образом, в ПНКМ всегда существуют операторы, 
действующие в $\hilbN$, но не являющиеся наблюдаемыми.
%}}}

\papersubsection{Полные наборы наблюдаемых}
%{{{
\label{sec:paper:4.1}

В квантовой механике для различимых частиц полным набором коммутирующих 
операторов (ПНКО) называют конечный набор взаимно коммутирующих 
эрмитовых операторов, для которых каждое совместное собственное значение 
не вырождено. Каждое объединенное собственное пространство оказывается 
одномерным. В конкретных приложениях ПНКО обычно строится из уместных 
наблюдаемых, например, $H$, $L^2$, $L_z$ для скалярной частицы 
в сферически-симметричном потенциале. Наборы эрмитовых операторов, 
удовлетворяющие этому определению, можно построить множеством способов, 
но всегда полагается, что ПНКО можно составить из наблюдаемых. Это 
является важным вопросом при подготовке состояний на практике.

В ПНКМ всякая попытка применить идею ПНКО к системе неразличимых частиц 
сталкивается с серьезной проблемой отсутствия ПНКО. Чтобы увидеть 
причину этого, предположим, что набор наблюдаемых $\vec{A}$ является 
ПНКО. Тогда каждое собственное пространство $\vec{A}$ долджно быть 
одномерным. С другой стороны, каждое собственное пространство $\vec{A}$ 
является носителем представления $S_N$, а одномерными являются только 
симметричное и антисимметричное представления. % TODO см. приложение 1
Таким образом, уникальный базисный вектор $\ket{\vec{a}}$ для каждого 
собственного пространства должен принадлежать либо $\hilbN_B$, либо 
$\hilbN_F$. Собственные состояния, определяемые по ПНКО, должны 
формировать базис в $\hilbN$. Следовательно, пространство $\hilbN$ 
должно быть прямой суммой $\hilbN_B$ и $\hilbN_F$, что справедливо при 
$N=2$, но ошибочно для $N\geq3$. Поскольку любая четная перестановка 
оставила бы вектора из такой прямой суммы неизменными, для опровержения 
достаточно найти вектор из $\hilbN$, который изменится под действием 
четной перестановки. Например, вектор $\ket{\vec{\theta}}$, у которого 
все компоненты $\theta_i$ различны, не инвариантен относительно 
каких-либо перестановок, в том числе симметричных. То есть при $N\geq3$ 
не существует ПНКО для систем неразличимых частиц, описываемых ПНКМ.
%
С другой стороны, в ПСКМ такого ограничения не возникает, поскольку все 
векторы изначально лежат либо в $\hilbN_B$, либо в $\hilbN_F$.

Физическая необходимость подготовки состояний с помощью измерений 
наблюдаемых означает, что любое идейное замещение ПНКО все еще будет 
включать объединенные измерения некоторого конечного набора 
коммутирующих наблюдаемых $\vec{A}$. Определение объединенного 
собственного пространства $\E_\vec{a}(\vec{A})$ гарантирует, что его 
размерность меньше или равна размерности собственного пространства 
каждой отдельной наблюдаемой $\E_{a_i}({A_i})$. Учитывая это, было бы 
полезно ограничить наблюдаемые в наборе $\vec{A}$, требуя, чтобы хотя бы 
одна из них имела не более чем конечную степень вырождения. В этом 
случае все объединенные собственные пространства являются конечномерными 
носителями приводимых или неприводимых представлений $S_N$.

Неприводимость автоматически выполняется для невырожденных собственных 
значений, поскольку их собственные пространства одномерны. Если 
$\vec{a}$ вырождено, а $\E_\vec{a}(\vec{A})$~-- неприводимо, возникает 
естественный вопрос: возможно ли уменьшить степень вырождения 
$d_\vec{a}$ с помощью добавления к набору $\vec{A}$ нового 
перестановочно инвариантного эрмитового оператора $Z$, коммутирующего со 
всеми прежними операторами? Поскольку $\E_\vec{a}(\vec{A})$ инвариантно 
под действием $Z$, найдутся перестановочно инвариантные собственные 
пространства $\E_z(Z)$, являющиеся подпространствами 
$\E_\vec{a}(\vec{A})$, однако $\E_\vec{a}(\vec{A})$ неприводимо и не 
содержит каких-либо нетривиальных перестановочно-инвариантных 
подпространств. Остается только две возможности: либо $\E_z(Z)$ содержит 
один лишь нулевой элемент, либо $\E_z(Z)$ совпадает 
с $\E_\vec{a}(\vec{A})$. Первый случай бессмыслен, а второй означает, 
что каждый вектор $\E_\vec{a}(\vec{A})$ является собственным вектором 
$Z$ с одним  и тем же собственным значением $z$. В результате, не 
существует измерений наблюдаемых, коммутирующих со всеми наблюдаемыми 
в $\vec{A}$, которые могли бы различить между чистыми состояниями 
неприводимого собственного пространства $\vec{A}$. В этом заключается 
физическая значимость неприводимости в отношении к симметрической 
группе.

Если $\vec{a}$ вырождено, а $\E_\vec{a}(\vec{A})$~-- приводимо, то 
$d_\vec{a}$-кратное вырождение можно частично снять, выражая 
$\E_\vec{a}(\vec{A})$ через прямую сумму неприводимых подпространств 
$\E^{(i)}_\vec{a}(\vec{A})$, которые являются носителями неприводимых 
представлений $\D{\gamma_i}$. Как и прежде, неприводимые собственные 
пространства $\E^{(i)}_\vec{a}(\vec{A})$ невозможно разрешить с помощью 
измерения какой угодно дополнительной коммутирующей наблюдаемой, но 
возможно пронумеровать их, в результате чего расширенный набор 
наблюдаемых будет иметь лишь неприводимые собственные пространства.

Это рассуждение устанавливает существование перестановочно-инвариантного 
эрмитового оператора, обеспечивающего неприводимость каждого 
собственного пространства расширенного набора. Однако не существует 
гарантий, что этот оператор представляет измеряемую величину. Как 
и в случае ПНКО, наличие наблюдаемой, имеющей подобный эффект, должно 
быть допущено. В этом случае, расширенный набор становится примером 
полного набора коммутирующих наблюдаемых (ПНКН), который определяется 
как набор коммутирующих наблюдаемых, для которого собственное 
пространство в результате каждого объединенного измерения является 
носителем неприводимого представления симметрической группы. 
Неприводимые собственные пространства, которые получаются в результате 
измерения ПНКН, играют роль чистых состояний, полученных при измерении 
ПНКО для различимых частиц.

Набор собственных значений $\ev{\vec{A}}$ для ПНКН естественным образом 
раскладывается в поднаборы, пронумерованные индексом $\gamma$ так, что 
$\forall \vec{a} \in \ev{\vec{A}, \gamma}$ $\E_\vec{a}(\vec{A})$ 
является носителем неприводимого представления $\D{\gamma}$.
%
Суперпозиция базисных векторов с общим значением $\gamma$ называется 
чистым состоянием с типом симметрии $\gamma$. Суперпозиция состояний 
с несколькими разными типами симметрии называется состоянием с гибридной 
симметрией. Любой вектор из $\hilbN$ всегда можно представить как 
суперпозицию состояний с разными типами симметрии.

%}}}

\papersubsection{Подготовка состояний}
%{{{
\label{sec:paper:4.2}

В ПСКМ чистое состояние, являющееся собственным вектором полного набора 
наблюдаемых, называется подготавливаемым. Таким образом, подготовка 
состояний~-- важное применение ПНКО. Если физическая система имеет ПНКО 
и находится в одном из собственных состояний, то это состояние можно 
приготовить с помощью измерения этого набора наблюдаемых и фильтрования 
лишь тех измерений, объединенное собственное число которых соответствует 
этому состоянию. Поскольку системы неразличимых частиц в рамках ПНКМ не 
содержат ПНКО, инструкция по подготовке состояния должна быть выработана 
заново.

Измерение набора коммутирующих наблюдаемых $\vec{A}$ оставляет систему 
в одном из собственных пространств $\vec{A}$. Если собственное 
пространство оказывается одномерным, то чистое состояние, 
соответствующее собственному вектору $\ket{\vec{a}}$ является 
подготавливаемым. Для неразличимых частиц одномерные собственные 
пространства являются носителями симметричного или антисимметричного 
представлений $S_N$. Таким образом, все подготавливаемые чистые 
состояния находятся либо в $\hilbN_B$, либо в $\hilbN_F$. Такое резкое 
ограничение множества подготавливаемых чистых состояний говорит, что 
в ПНКМ идею подготовки состояний требуется расширить до смешанных 
состояний, описываемых матрицами плотности. Поскольку наблюдаемые 
перестановочно инвариантны, для любой наблюдаемой $Y$, матрицы плотности 
$\rho$ и $\calP \in S_N$ выполняется
$\tr{\rho Y} = \tr{\rho D(\calP)^\dag Y D(\calP)^\dag} 
= \tr{D(\calP) \rho D(\calP)^\dag Y}.$
Если просуммировать это выражение по всем $\calP$, получается, что 
матрицу плотности $\rho$ можно заменить на усредненную, перестановочно 
инвариантную матрицу плотности
$\rhobar = \frac{1}{N!}\sum_{\calP \in S_N}D(\calP)\rho D(\calP)^\dag.$
В результате можно считать, что все матрицы плотности перестановочно 
инвариантны.

Предположим, что $\vec{A}$~-- ПНКН. Тогда измерение с исходом
$\left( \vec{a}, \E_\vec{a}(\vec{A})\right)$ оставляет систему в смеси 
состояний, соответствующих собственному значению $\vec{a}$ и типу 
симметрии $\gamma$. Тогда матрица плотности коммутирует с ПНКН, а значит 
матрица диагональна и является проекцией на пространство 
$\E_\vec{a}(\vec{A})$. Таким образом, измерение ПНКН оставляет систему 
в состоянии, описываемом уникальной матрицей плотности. Такое состояние 
называется подготавливаемым смешанным состоянием. Это наилучший отбор, 
который удается провести для систем неразличимых частиц, описываемых 
ПНКМ.

%}}}

\papersubsection{Правило суперотбора}
%{{{
\label{sec:paper:4.3}

Комбинация правила перестановочной инвариантности с двумя первыми 
леммами Шура % TODO см приложение 1
является основой для доказательства следующего результата. 
Перестановочно инвариантный оператор $X$ не может осуществлять переход 
между состояниями с разным типом симметрии~\cite{ref1}. % TODO ref 1
Следовательно, любой перестановочно инвариантный оператор $X$ имеет вид
$ \bramidket{\gamma', \vec{a}' : \mu'} {X} {\gamma, \vec{a}: \mu}
= \delta_{\gamma' \gamma} M^\gamma_{\mu'\mu} (X)$, аналогично теореме 
Вигнера-Эккарта.
%
Условие перестановочной инвариантности приобретает вид
$[\D{\gamma}(\calP) M^\gamma(X)] = 0$. Применение третьей леммы Шура 
к этому равенству показывает, что $M^\gamma(X)$ должна быть 
пропорциональна единичной матрице. Поэтому можно записать
$ \bramidket{\gamma, \vec{a}' : \mu'} {X} {\gamma, \vec{a}: \mu}
= \bramidket{\gamma, \vec{a}'}{\middle|X\middle|}{\gamma, \vec{a}}
\delta_{\mu'\mu}$, где обозначение коэффициента пропорциональности 
выбрано по аналогии с теоремой Вигнера-Эккарта.
%
Векторы состояний с типом симметрии $\gamma$ образуют подпространство 
$\hilbN_\gamma$, касательно которых формулируется следующее правило.

Правило суперотбора: для любого перестановочно инвариантного оператора 
$A$ верно $A: \hilbN_\gamma \to \hilbN_\gamma$ и $\hilbN 
= \bigoplus_\gamma \hilbN_\gamma$. Подпространства $\hilbN_\gamma$ 
называются секторами суперотбора. Поскольку наблюдаемые и физические 
операторы перестановочно инвариантны, они не могут осуществлять переходы 
между разными секторами суперотбора.

Правило суперотбора имеет глубокие следствия для временн\'{о}й эволюции. 
Унитарный оператор временной эволюции $U(t) = \exp(-itH/\hbar)$ 
перестановочно инвариантен благодаря перестановочной инвариантности 
гамильтониана. В результате $\bramidket{\Phi}{U(t)}{\Psi} = 0$, если 
$\ket{\Phi}$ и $\ket{\Psi}$ относятся к разным секторам суперотбора. Тип 
симметрии состояния не меняется со временем, а если начальное состояние 
являлось суммой векторов с разными типами симметрии ($\ket{\psi} 
= \sum_\gamma c_\gamma \psi_\gamma$), они эволюционируют независимо.
%
Поскольку нарушения постулата симметризации ожидаются небольшими, можно 
предположить, что существует доминантный член $c_{\gamma_{BF}}$, тип 
симметрии которого соответствует статистике Бозе или Ферми. Сила 
потенциальных нарушений ПС определяется изначальными амплитудами 
$c_\gamma$ для $\gamma\neq\gamma_{BF}$.
%
Следует отметить, что ПСКМ версия квантовой теории может быть полностью 
восстановлена в ПНКМ версии (полагая $c_\gamma = 0$ для $\gamma \neq 
\gamma_{BF}$), поскольку все наблюдаемые в ней удовлетворяют правилу 
перестановочной инвариантности, а переходы между состояниями с разными 
типами симметрии запрещены. Другими словами, постулат симметризации 
эквивалентен настолько же загадочному условию, что единственным сектором 
суперотбора, присутствующим в начальном состоянии, является либо 
$\gamma_F$, либо $\gamma_B$. Менее строгим предположением было бы 
$\left|c_\gamma\right| \ll \left|c_{\gamma_{BF}}\right| $ для 
$\gamma\neq\gamma_{BF}$. Учитывая условие нормировки, можно записать
$\left|c_\gamma\right| \ll 1 $, $\gamma\neq\gamma_{BF}$.

Правила суперотбора, с которыми повсеместно сталкивается квантовая 
теория, связаны с непрерывными симметриями, такими как поворот 
и калибровочные преобразования. Оба этих примера, как часто говорят, 
накладывают ограничения на принцип суперпозиции. Например, запрещенными 
считаются суперпозиции состояний с целым и полуцелым полным угловым 
моментом или суперпозиции состояний с разными полными зарядами.
%
Некоторые считают, что эти состояния не запрещены, а лишь являются 
смешанными для всех наблюдаемых одновременно~\cite[Гл.~III.1]{ref15}.
% ^ TODO ref 15, chap III.1
%
Правило суперотбора имеет такой же эффект на дискретную симметрическую 
группу $S_N$. Среднее значение любого перестановочно инвариантного 
оператора $X$ для общего чистого состояния равно 
$\bramidket{\psi}{X}{\psi} = \sum_\gamma \left|c_\gamma\right|^2 
\bramidket{\psi_\gamma}{X}{\psi_\gamma}$. Интерференция между 
состояниями с разными типами симметрии отсутствует, и нет никакой 
информации о фазах коэффициентов $c_\gamma$. Несмотря на то, что 
$\ket{\psi}$ является чистым состоянием, при измерении наблюдаемых 
величин оно видно лишь как смешанное состояние с матрицей плотности 
$\rho_\psi = \sum_\gamma \ket{\psi_\gamma} \left|c_\gamma\right|^2 
\bra{\psi_\gamma}$.

%}}}

%}}}

\papersection{Поиск нарушения принципа симметризации}
%{{{
\label{sec:paper:5}

Жесткие условия, накладываемые на векторы состояний постулатом 
симметризации, формируют базис традиционного (ПСКМ) описания систем 
тождественных частиц, простирающихся от небольшого количества частиц, 
испытывающих рассеяние, до взаимодействующих многочастичных систем, 
таких как конденсаты Бозе-Эйнштейна, сверхтекучие жидкости, 
сверхпроводники и прочее. Настолько широкая область влияния 
подразумевает равносильно широкую область возможных экспериментов по 
проверке ПС.
%
Развитый в предыдущих разделах формализм можно использовать, например, 
для построения расширенной версии квантовой статистики, которая не 
ограничена одними лишь статистиками Бозе и Эйнштейна. Однако 
эксперименты с многочастичными системами могут не оказаться наиболее 
полезным способом. Сложность заключается в том, что размер нарушений ПС 
привел бы, скорее всего, к едва заметным эффектам, которые было бы 
сложно детектировать. Возможно, именно поэтому эксперименты как с Ферми 
системами~\cite{ref3, ref4, ref5, ref6, ref7}, % TODO refs 3--7
так и с Бозе системами~\cite{ref8, ref9, ref10} % TODO refs 8--10
обычно включают взаимодействия единичных электронов или фотонов 
с атомами или молекулами. В то время как регистрация событий, нарушающих 
ПС, для этих систем также чрезвычайно сложна, эти эксперименты могут 
воспользоваться правилами отбора, накладываемыми Бозе или Ферми 
статистикой. Например, в ПСКМ начальное и конечное состояния в событии 
с рассеянием электрона не могут содержать симметричное состояние двух 
электронов.

Для того, чтобы получить наблюдаемый сигнал слабого нарушения правила, 
накладываемого принципом симметризации, необходим сильный поток 
налетающих частиц. Один из способов достижения этого с электронами~-- 
индуцировать сильный электрический ток через проводник. Эксперименты, 
использующие такой механизм для проверки принципа Паули, были проведены 
и тщательно проанализированы~\cite{ref6}. % TODO ref 6
Идея этих экспериментов в том, что при захвате электрона проводимости 
атомом кристаллической решетки возникает радиоактивный каскад. Однако 
захваты, порождающие рентгеновские лучи, запрещены принципом Паули, 
поскольку соответствующие низкие энергетические уровни полностью заняты. 
Так, испускание рентгеновского фотона в определенном энергетическом 
диапазоне служило бы сигналом нарушения принципа Паули. Модель подобного 
события, очевидно, должна быть шире ПСКМ, а ПНКМ версия квантовой 
механики предоставляет минимальное расширение, подходящее для подобных 
целей.

Для иллюстрации того, каким образом непривычные аспекты ПНКМ включены 
в анализ подобных экспериментов, полезно рассмотреть упрощенную модель. 
Соответствующие вопросы не возникают для одиночных частиц и тривиальны 
для систем двух частиц, которые допускают только симметричное 
и антисимметричное представления. Таким образом, наиболее простые 
информативные модели~-- это системы трех неразличимых частиц. Эти 
рассуждения подразумевают следующую упрощенную модель.
%
Налетающий электрон и электроны атома мишени моделируются тремя 
тождественными невзаимодействующими частицами со спином $1/2$, движение 
которых ограничено одним измерением. Невозмущенная динамика частиц 
описывается независящим от спина одночастичным гамильтонианом $H_0$, 
а взаимодействие с полем излучения~-- $H_\text{изл}$. Одночастичные 
квантовые числа $\theta=(\eps, s)$~-- собственные значения $H_0$ 
и $L_z$, соответственно. В начальном состоянии атом моделируется двумя 
частицами в основном состоянии $H_0$ с суммарной энергией $2\eps_0$, 
а налетающий электрон~-- одной частицей в возбужденном состоянии 
с энергией $\eps_1>\eps_0$. В конечном состоянии все частицы находятся 
в основном состоянии. Для различимых частиц такая ситуация описывалась 
бы начальным вектором $\ket{\theta_\text{нач}} = \ket{(\eps_0, 1/2), 
(\eps_0, -1/2), (\eps_1, 1/2)}$ и конечным вектором 
$\ket{\theta_\text{кон}} = \ket{(\eps_0, 1/2), (\eps_0, -1/2), (\eps_0, 
1/2)}$.

В ПСКМ версии для фермионов вектора $\ket{\theta_\text{нач}}$ 
и $\ket{\theta_\text{кон}}$ нужно антисимметризовать:
$\ket{F, \theta_\text{нач}} = \frac{1}{\sqrt{3!}} \sum_{\calP\in S_3} 
\sigma_\calP D(\calP) \ket{\theta_\text{нач}}$,
$\ket{F, \theta_\text{кон}} = \frac{1}{\sqrt{3!}} \sum_{\calP\in S_3} 
\sigma_\calP D(\calP) \ket{\theta_\text{кон}} = 0$.
Равенство вектора конечного состояния нулю выражает запрет на 
присутствие двух фермионов в одном и том же одночастичном состоянии.

В ПНКМ версии нужно рассмотреть все три неприводимых представления 
$S_3$: $\gamma=F$, $\gamma=B$, $\gamma=I$. Те неприводимые 
представления, которые пересекаются с несимметризованным трехчастичным 
состоянием $\ket{\vec{\kappa}}$, определяются с помощью вычисления 
нормированной проекции $\ket{\vec{\kappa}}$ на пространство носитель 
неприводимого представления $\D{\gamma}$:
$ \ket{\gamma, \vec{\kappa}: \mu} = \Pi_{\gamma\mu}\ket{\vec{\kappa}} 
/ \sqrt{\bramidket{\vec{\kappa}}{\Pi_{\gamma\mu}}{\vec{\kappa}}{}} $,
где $\Pi_{\gamma\mu}$~-- оператор проекции. Поскольку
$\Pi_B\ket{\theta_\text{нач}}\neq0$,
$\Pi_F\ket{\theta_\text{нач}}\neq0$,
$\Pi_{I\mu}\ket{\theta_\text{нач}}\neq0$,
то все три представления пересекаются с $\ket{\theta_\text{нач}}$.
Однако для с конечным $\ket{\theta_\text{кон}}$ пересекаются лишь 
представления $B$ и $I$:
$\Pi_B\ket{\theta_\text{кон}}\neq0$,
$\Pi_F\ket{\theta_\text{кон}}=0$,
$\Pi_{I\mu}\ket{\theta_\text{кон}}\neq0$.
Общий вид начального и конечного состояний, которыми можно описать этот 
эксперимент, записывается как
$$ \ket{\psi_\text{нач}} = c^F_\text{нач} \ket{F, \theta_\text{нач}:1} 
+ c^B_\text{нач} \ket{B, \theta_\text{нач}: 1} + \sum_{\mu=1, 2} 
c^{I\mu}_\text{нач} \ket{I, \theta_\text{нач}: \mu}, $$
$$ \ket{\psi_\text{кон}} = c^B_\text{кон} \ket{B,\theta_\text{кон}:1} 
+ \sum_{\mu=1, 2} c^{I\mu}_\text{кон} \ket{I, \theta_\text{кон}: \mu}.$$
Испускание излучения с частотой $\omega = (\eps_1-\eps_0)/\hbar$ было бы 
регистрируемым событием, нарушающим принцип Паули. Поскольку 
$H_\text{изл}$ перестановочно инвариантен, матричный элемент 
записывается следующим образом:
$ \bramidket{\psi_\text{кон}}{H_\text{изл}}{\psi_\text{нач}} =
c^{B*}_\text{кон} c^B_\text{нач} \times
\bramidket{B, \theta_\text{кон}}{H_\text{изл}}{B, \theta_\text{нач}}
+ \left(\sum_{\mu=1,2} c^{I\mu*}_\text{кон} c^{I\mu}_\text{нач}\right)
\bramidket{I, \theta_\text{кон}} {\middle|H_\text{изл}\middle|}
{I, \theta_\text{нач}}. $
Таким образом, вероятность регистрации нарушения принципа Паули 
определяется начальными амплитудами $c^B_\text{нач}$ 
и $c^{I\mu}_\text{нач}$ Бозе и смешанного представлений $S_3$. Если 
выполняется $\left|c^B_\text{нач}\right|\ll1$ 
и $\left|c^{I\mu}_\text{нач}\right|\ll1$, то скорость подобных переходов 
мала.

%}}}

\papersection{Обсуждение}
%{{{
\label{sec:paper:6}

ПНКМ версия квантовой механики осложняется необходимостью рассмотрения 
всех неприводимых представлений $S_N$. Как показано 
в разделе~\ref{sec:paper:4.2}, это требует изменений в традиционном 
описании подготовки состояний. Наиболее значимая новая особенность 
ПНКМ~-- присутствие правила суперотбора, связанного с дискретной 
симметрией по группе $S_N$. Как показано в разделе~\ref{sec:paper:4.3}, 
оно приводит к важным следствиям касательно временн\'{о}й эволюции 
и интерпретации суперпозиции состояний с разными типами симметрии. 
Данные особенности ПНКМ являются следствиями трех предположений. 
Во-первых, аксиом квантовой теории. Во-вторых, определения тождественных 
частиц. В-третьих, постулата неразличимости. Эти предположения приводят 
к следующим свойствам. Во-первых, перестановочные симметрии сохраняются 
безотносительно взаимодействий. Во-вторых, перестановочные симметрии 
сохраняются при любых расстояниях между частицами. В-третьих, тип 
симметрии какого-либо состояния сохраняется во времени. Первые два 
свойства чисто кинематические, а третье проистекает из закона эволюции.

Анализ упрощенной модели эксперимента с нарушением принципа Паули, 
представленный в разделе~\ref{sec:paper:5}, сосредоточен на следствиях 
указанных трех свойств ПНКМ. Но было предложено и альтернативное 
описание для этого узкого класса экспериментов. В альтернативном подходе 
мишень и налетающие электроны считаются разными системами, а налетающие 
электроны называют свежими. Термин свежие электроны подразумевает 
следующие предположения. Во-первых, свежие электроны еще не 
взаимодействовали с электронами мишени. Во-вторых, не было установлено 
каких-либо условий симметрии между свежими электронами и электронами 
мишени. В-третьих, установка каких угодно условий симметрии между 
свежими электронами и электронами мишени требует взаимодействия между 
ними. Эти предположения находятся в разногласии с тремя свойствами ПНКМ, 
а значит противоречат хотя бы одному из предположений ПНКМ. Это не 
означает, что понятие свежих электронов нефизично, но показывает, что 
оно требует существенно более радикального расширения ПСКМ, чем ПНКМ. 
И этого тоже нельзя бояться, поскольку отвержение чего-либо настолько 
фундаментального как принцип Паули вполне может потребовать радикально 
новую теорию.

Решение сформулировать ПНКМ с использованием нерелятивистской квантовой 
теории было сделано не для простоты, а потому что оно неизбежно. В самом 
деле, невозможно сформулировать релятивистскую квантовую теорию 
с фиксированным числом частиц~\cite[Гл.~1.1]{ref16}.
% ^ TODO ref 16, chap 1.1
Следовательно, любая теория, предполагающая фиксированное число частиц, 
в частности, ПНКМ, должна быть нерелятивистской. Тот факт, что ПНКМ не 
накладывает какой бы то ни было связи между спином и статистикой также 
должен быть связан с использованием нерелятивистской квантовой теории. 
Связь спина и статистики устанавливается только на основе релятивистской 
квантовой теории поля~\cite{ref15, ref17} % TODO ref 15, 17
для доказательства теоремы о связи спина со статистикой:
частицы с целочисленным спином~-- бозоны, а с полуцелым~-- фермионы.
Утверждение данной теоремы, на первый взгляд, предполагает, что возможны 
только Бозе и Ферми статистики, но это обманчиво. Многочисленные 
доказательства теоремы о связи спина со статистикой зависят от 
следующего предположения: квантовые поля, вычисленные в точках, 
разделенных пространственноподобным интервалом, либо коммутируют, либо 
антикоммутируют. В некоторых доказательствах это предположение включено 
в гипотезу теоремы, в других~-- находится в соответствующих аксиомах 
теории поля. В любом случае, оно используется для наложения 
эмпирического ограничения статистикой Бозе или Ферми в релятивистской 
квантовой теории поля. Точно так же постулат симметризации накладывает 
подобное ограничение на нерелятивистскую квантовую механику. Если 
удастся показать, что данное предположение проистекает из аксиом любой 
непротиворечивой релятивистской квантовой теории поля, то это означало 
бы, что возможны только статистики Бозе и Ферми. Если же существует 
непротиворечивая теория поля, в которой это предположение неверно, то 
окажется возможным включить другие типы статистики в теорию поля.

%}}}

%}}}

\clearpage \section*{Рецензия}
\addcontentsline{toc}{section}{\protect\numberline{}Рецензия}
%{{{
\label{sec:review}

Квантовая теория по множеству аспектов радикально отличается от 
классической физики. С учетом того, что классическая физика изначально 
строилась на повседневном опыте, становится очевидной глубина революции, 
произошедшей в физике примерно 100 лет назад. Сравнительно малый возраст 
квантовой теории также означает, что в ней нет единогласного мнения 
о том, что именно из себя представляют квантовые объекты~-- по сути, 
единственные сущности, с которыми квантовая теория имеет дело. 
Отсутствие четкого образа в сознании осложняет как мыслительные 
эксперименты, так и восприятие эмпирических результатов, которые 
отражены в аксиомах и постулатах квантовой теории.

Одно из наиболее тривиальных описаний поведения ансамблей частиц 
заключено в распределении Больцмана. Его построение, в конечном итоге, 
опирается на конкретный способ подсчета состояний системы, реализующих 
одно единственное макросостояние, то есть имеющих одинаковые параметры 
при измерении физическими приборами. Поскольку молекулы являются 
квантовыми объектами, способ подсчета состояний должен учитывать их 
квантовую природу. Несмотря на многочисленные экспериментальные 
подтверждения справедливости распределения Больцмана, оно оказывается 
лишь приближением, а лежащие в его основе идеи не находят подтверждения 
в реальности. Более точные эксперименты показывают, что два состояния 
системы, которые отличаются лишь переменой мест частиц одинакового типа, 
находящихся в одинаковых внутренних состояниях, на самом деле являются 
одним и тем же состоянием. Этот результат находится в существенном 
разногласии с повседневным опытом с классическими системами, где 
перемена мест (и скоростей) двух шаров, пусть и одинаковых, приводит 
к новому состоянию, поскольку системе потребовалось бы время, чтобы 
самостоятельно осуществить переход между ними. Квантовым системам не 
требуется время для осуществления подобного перехода, поскольку от 
перемены мест изначально ничего не меняется.

Описанное разногласие повседневного опыта и квантовой статистики, в паре 
с отсутствием четких представлений о квантовых объектах, приводит 
к большим сложностям при формулировке и интерпретации законов, которым 
подчиняются тождественные квантовые частицы. Экспериментальные 
результаты свидетельствуют о наличии двух разных типов статистики, 
описывающих квантовые частицы: Бозе-Эйнштейна и Ферми-Дирака. Теорема 
о связи спина со статистикой также подразумевает наличие только этих 
двух вариантов, однако она основана на предположении о коммутации 
квантового поля, вычисленного в разных точках пространства. В итоге, 
ограничение лишь статистиками Бозе и Ферми накладывается в определенной 
мере искусственно.

В квантовой механике постулат симметризации ограничивает гильбертово 
пространство векторов состояний квантовых систем. В современной физике 
нет однозначного ответа, почему содержащееся в нем требование 
симметричности или антисимметричности волновой функции системы 
тождественных частиц должно всегда выполняться. Разумным и полезным 
может оказаться рассмотрение более широких теорий и изучение условий, 
при которых они могли вновь бы обратиться в известную всем квантовую 
механику. Подобные рассуждения и исследования также позволяют 
разрабатывать различные эксперименты по проверке постулата 
симметризации. Одной из теорий, расширяющих традиционную квантовую 
механику, и озадачена статья Гаррисона ``Квантовая статистика 
тождественных частиц''~\cite{source}.

В свете описанных, глубоких и остро стоящих, вопросов квантовой 
механики, касающихся систем тождественных частиц, актуальность изученной 
статьи не вызывает сомнений.

Автор рассматривает возможность замены постулата симметризации на 
альтернативный постулат неразличимости. В постулате неразличимости не 
накладываются ограничения на волновую функцию. Вместо этого, 
утверждается лишь что измерения любых физических величин должны 
приводить к одинаковым результатом до и после перестановок частиц. 
Гильбертово пространство состояний при этом не затрагивается, 
а ответственность за соблюдение постулата принимают на себя уже 
физические величины, наблюдаемые.
%
С одной стороны, требование постулата неразличимости кажется более 
естественным по сравнению с постулатом симметризации, поскольку до 
экспериментатора доходит лишь информация от приборов. Более того, 
в самой распространенной, копенгагенской, интерпретации квантовой 
механики только результаты измерений считаются отражающими 
действительность. С другой стороны, перенос ограничений с пространства 
состояний на множество физических наблюдаемых еще сильнее затрудняет 
интерпретацию квантовых объектов и измерений.

Развивая аппарат квантовой теории с постулатом неразличимости, автор 
рассматривает способ подготовки квантовых состояний. В то время как 
постулат симметризации допускает существование полного набора 
наблюдаемых, измерение которых дает однозначную информацию о состоянии 
системы, постулат неразличимости не позволяет построить такой набор 
наблюдаемых и гарантировать невырожденность собственных значений. 
В результате возникает потребность в модернизации понятия 
подготавливаемого состояния. Гаррисон расширяет его, включая 
диагональные матрицы плотности.
%
Далее автор выводит новое правило отбора, возникающее вследствие 
постулата неразличимости, которое гласит, что наблюдаемые не могут 
осуществлять переходы между состояниями с разными типами симметрии. 
В частности, это касается оператора эволюции.
%
Полное возвращение к традиционной квантовой теории осуществляется, таким 
образом, если система тождественных частиц изначально находилась 
в состоянии, целиком лежащем в подпространстве симметричных (для 
бозонов) или антисимметричных (для фермионов) состояний. Более мягким 
условием была бы малая величина проекций вектора начального состояния на 
подпространства с другими типами симметрии.
%
Затем Гаррисон предлагает и подробно описывает модель эксперимента по 
проверке принципа Паули с тремя электронами и демонстрирует, что 
вероятность запрещенного перехода пропорциональна примесям к изначальной 
волновой функции, лежащим в подпространствах носителях Бозе 
(симметричного) и смешанного представлений группы $S_3$.
%
Обсуждая квантовую теорию поля, Гаррисон обращает внимание на отсутствие 
последовательного доказательства необходимости коммутации или 
антикоммутации квантового поля, вычисленного в разных точках. Если 
удастся доказать необходимость данного коммутационного соотношения, 
вопрос о статистике тождественных частиц будет закрыт. Но в настоящий 
момент равные усилия должны быть уделены поиску непротиворечивой 
квантовой теории поля, не накладывающей такого ограничения.

Я согласен с автором статьи и считаю чрезвычайно важным для понимания 
квантовой теории подвергать сомнениям самые глубокие ее принципы 
и основы. Отсутствие теоретического обоснования постулата симметризации 
наталкивает на мысль о возможности альтернативных статистик 
тождественных частиц. В статье, свою очередь, описано минимальное 
расширение квантовой теории, допускающее другие статистики.
%
Отказ от настолько базового принципа как принцип симметризации приводит 
к существенным изменениям в теории, требующим тщательного поиска 
и изучения всех следствий. В частности, перестановочная инвариантность 
операторов запрещает наличие полных наборов наблюдаемых.
%
Однако ценны любые теоретические и экспериментальные результаты, 
приближающие квантовую физику к осознанию причин экзотического поведения 
квантовых систем тождественных частиц.

% Актуальность, согласен ли я, мысли в целом.
% 
% Слова про постулат неразличимости. Казалось бы, более натурально.
% 
% Автор не настаивает на перевороте всей науки, а обсуждает маленькие 
% поправки в том числе.
% 
% Расширяет имеющуюся систему понятий, привносит новые типы симметрии. 
% Но полностью включает имеющиеся теории.
% 
% Автор предлагает конструктивную модель проверки. Конструктивный 
% эксперимент.

%}}}

\clearpage \phantomsection
\addcontentsline{toc}{section}{\protect\numberline{}Список литературы}
\begin{thebibliography}{99}%{{{
  \bibitem{ref1}
    \bibauthor{Messiah,~A.\,M.\,L., Greenberg,~O.\,W.}
    \bibtitle{Symmetrization postulate and its experimental foundation.}
    Phys. Rev.~B \textbf{136}(1B), B248 (1964).

  \bibitem{ref3}
    \bibauthor{Goldhaber,~M., Scharff-Goldhaber,~G.}
    \bibtitle{Identification of beta-rays with atomic electrons.}
    Phys. Rev. \textbf{73}(12), 1472–1473 (1948).

  \bibitem{ref4}
    \bibauthor{Ramberg,~E., Snow,~G.A.}
    \bibtitle{Experimental limit on a small violation of the Pauli 
    principle.}
    Phys. Lett. B \textbf{238}(2–4), 438–441 (1990).

  \bibitem{ref5}
    \bibauthor{Thoma,~M.H., Nolte,~E.}
    \bibtitle{Limits on small violations of the Pauli exclusion 
    principle in the primordial nucleosynthesis.}
    Phys. Lett. B \textbf{291}(4), 484 (1992).

  \bibitem{ref6}
    \bibauthor{Elliott,~S.R., LaRoque,~B.H., Gehman,~V.M., Kidd,~M.F., 
    Chen,~M.}
    \bibtitle{An improved limit on Pauli-exclusion-principle forbidden 
    atomic transitions.}
    Found. Phys. \textbf{42}(8), 1015–1030 (2012).

  \bibitem{ref7}
    \bibauthor{Bellini,~G., \textit{et al.} (Collaboration Borexino).}
    \bibtitle{New experimental limits on the Pauli-forbidden transitions 
    in C-12 nuclei obtained with 485 days Borexino data.}
    Phys. Rev. C \textbf{81}(3), 034317 (2010).

  \bibitem{ref8}
    \bibauthor{DeMille,~D., Budker,~D., Derr,~N., Deveney,~E.}
    \bibtitle{Search for exchange-antisymmetric two-photon states.}
    Phys. Rev. Lett. \textbf{83}(20), 3978–3981 (1999).

  \bibitem{ref9}
    \bibauthor{Brown,~D., Budker,~D., DeMille,~D.P.}
    Presented at the Conference on Spin–Statistics Connection and 
    Commutation Relations, Anacapri, Italy (2000).

  \bibitem{ref10}
    \bibauthor{Bellini,~G., \textit{et al.}}
    \bibtitle{New experimental limits on the Pauli-forbidden transitions 
    in $^{12}$C nuclei obtained with 485 days Borexino data.}
    Phys. Rev. C \textbf{81}(3), 034317 (2010).

  \bibitem{ref12}
    \bibauthor{Haldane,~F.D.M.}
    \bibtitle{Fractional statistics in arbitrary dimensions: 
    a generalization of the Pauli principle.}
    Phys. Rev. Lett. \textbf{67}(8), 937–940 (1991).

  \bibitem{ref13}
    \bibauthor{Leinaas,~J.M., Myrheim,~J.}
    \bibtitle{On the theory of identical particles.}
    Nuovo Cim. B \textbf{37},~1 (1977).

  \bibitem{ref14}
    \bibauthor{Hamermesh,~M.}
    \bibtitle{Group Theory and Its Application to Physical Problems.}
    Dover, New York (1962).

  \bibitem{ref15}
    \bibauthor{Haag,~R.}
    \bibtitle{Local Quantum Physics.}
    Springer, Berlin (1992).

  \bibitem{ref16}
    \bibauthor{Weinberg,~S.}
    \bibtitle{The Quantum Theory of Fields I: Foundations.}
    Cambridge University Press, Cambridge (1995).

  \bibitem{ref17}
    \bibauthor{Curceanu,~C., Gillaspy,~J.D., Hilborn,~R.C.}
    \bibtitle{Resource Letter SS-1: the spin–statistics connection.}
    Am. J. Phys. \textbf{80}(7), 561–577 (2012).

  \bibitem{source}
    \bibauthor{Garrison,~J.\,C.}
    \bibtitle{Quantum Statistics of Identical Particles.}
    Found. Phys. \textbf{52}, 77 (2022).
\end{thebibliography}%}}}

\end{document}
