\documentclass[a4paper, 12pt]{article}
\usepackage{cmap}
\usepackage[T2A]{fontenc}
\usepackage[utf8]{inputenc}
\usepackage[english,russian]{babel}
%\usepackage[T2A]{fontenc}
%\usepackage[utf8]{inputenc}
%\usepackage[english, russian]{babel}
\frenchspacing
\usepackage[onehalfspacing]{setspace}


\setcounter{tocdepth}{4}
\usepackage[unicode]{hyperref}
\hypersetup{colorlinks=false}
\hypersetup{linktocpage}
\frenchspacing
\righthyphenmin=2
\usepackage{indentfirst}
\usepackage{cite}
\usepackage{cmap}







\def\t{\hspace*{1cm}}
\def\bdot{\textbullet\ }

\newcommand{\newpart}[1]{\par\noindent \begin{center}\textsc{#1}\end{center}}

\begin{document}

%\selectlanguage{russian}

{\noindent Яковлев Владимир Анатольевич \hfill \texttt{goroda460@yandex.ru}}
\vspace{.5cm}

\textbf{Сочинения} по 2800-3000 знаков с учетом пробелов. Пишется в ворде. Должны состоять из трех примерно равных частей:
\\\t 1. Краткое изложение материала с лекций.
\\\t 2. Одна цитата из литературы и пара предложений с раскрыванием смысла ее. 
\\\t 3. Собственные рассуждения о вопросе. 
\\Методичка "Философия. Учебное пособие.", в ней темы  и литература. Тем всего 10, но сочинения можно писать только на 8:
\\\t 1. Философия древнего Китая и Индии.
\\\t 2. Античная философия.
\\\t 3. Средневековая философия. 
\\\t 4. Философия эпохи возрождения. 
\\\t 5. XVII--XVIII века.
\\\t 6. Иррациональная философия. 
\\\t 7. Особенности философии в России. 
\\\t 8. Основные направления современной западной философии. 
\\ На зачете может быть максимум 5 вопросов. Если сделал 5 эссе, получаешь 1 вопрос, если 4 -- 2, 3 -- 3. Сделанным эссе считается, когда его принял Яковлев. Эссе можно писать на все темы, но попытки на каждую всего две. 

\textbf{Источники:}
\\\t \bdot "Антология мировой философии" в 4 томах. Москва 1969.
\\\t \bdot Чанышев "Философия древнего мира", "Курс лекций по древней и средневековой философии".
\\\t \bdot Рассел Б. "История западной философии".
\\\t \bdot Реале Дж. Антисори Т. "Западная философия от истоков до наших дней".

\clearpage

\tableofcontents


\clearpage

\section{Введение}
%{\center \large Философия \endcenter}
\hfill \textbf{Feb 9}

Философия, от греческого филио -- любовь и софи -- мудрость, означает любовь к мудрости и стремление к мудрости. Филомония -- наоборот -- любовь к глупости. Философия основывается на определениях.
\\\t Философия -- царица наук: где она, там затем появляются науки. Философия это квинтэссенция эпохи, чтобы понять народ, надо понять их философию. Например, западная философия основывается на принципе состязания. 
\\\t Философия -- это мышление в предельной степени общности понятия. В таком общем смысле понятие называется категорией. Существуют категории: качество и количество, причина и следствие, причинность и последствие, пространство и время, сущность и явление. У Аристотеля было 10 категорий. У Канта 12.
\\\t Философия это деятельность очищенного мышления.

\textit{Мироощущение} это психологическое понятие (оптимизм, пессимизм). \textit{Мировосприятие} это сопоставление и противопоставление себя миру \linebreak(эгоист считает себя главным, альтруист считает себя мелкой песчинкой и живет на благо общего). \textit{Миросозерцание} это расположение человека к окружающему (интроверт задумывается над тем, кто он, зачем он; экстраверт...).

Функции Философии
\\\t \textbf{I} Мировоззрение -- его дает философия. Каждый человек решает, кто он, что он должен делать итд. Философия -- минимизация риска ошибки. Рефлексия и размышление над основанием. Мысленный эксперимент -- философия. 
Методология -- линия между типами мировоззрений и о философском мировоззрении.
Этапы западной философии:
\\\t 1. Появление в 6-7 веке до нашей эры (не уверен, что до) одновременно в трех местах. Тогда основная проблематика была о космосе. Это космический период, космическое мировоззрение. 
\\\t 2. Средневековье. Религиозное мировоззрение, философия стала служанкой теологии, но всегда несла факел. 
\\\t 3. Возрождение 14-16 века, возрождение культуры, антропоцентризм. 
\\\t 4. 17 век -- появление классической науки, наукоцентризм (наука самый могущественный элемент общества). 
\\\t 5. 19 век -- позитивизм, философия стала служанкой науки, но 20 век породил много проблем и показал амбивалентный характер науки (она может повлечь и добро, и зло).
\\\t 6. Антисцеинтизм (против науки) -- наука порождает больше проблем, чем решает. В конце 20 века решили, что науку надо осаживать (замедлять) -- экоцентризм (гармонизация взаимодействия с природой). Нет никакой окончательной науки и теории. 
\\\t \textbf{II} Методологическая философия имеет глобальный характер, отделяет философское мировоззрение от всех остальных и непосредственно от религии. Принцип основывается на так называемых максимах. Учитель инициирует ученика на критику и пускает дальше. Критическое мышление. Наличие сомнений -- показатель правильности мышления. 

Все главные методы науки даются философами: эксперимент, мысленный эксперимент, дедуктивный метод (теория вероятностей), \linebreak аналитико-синтетический метод (Декарт), гипотетико-...(Галилей), экспликация, систематизация, рационализация.

Части философии:
\\\t \textbf{I} Онтология -- учение о бытие. Космологические вопросы, причинность и случайность, структура мира -- есть ли предел делимости, что в основании -- материя или дух -- материальное или идеальное. Объективный идеализм -- в основе дух. Например, Декарт считал, что в основе всего находятся треугольники, а Гегель -- мировой разум. Субъективный идеализм -- все и вся всегда познается только через ощущения. Гносеологический оптимизм -- возможно все. Ограниченный оптимизм -- ??. Скептицизм -- познать невозможно. 
\\\t \textbf{II} Гносеология -- учение о познании. Эпистемология изучает точное знание, научное. Мнений много, но стремится к одному верному. 
\\\t \textbf{III} Аксиология -- учение о ценности. Не все есть либо истина, либо ложь. Аксиологические учения и размышления.
\\\t \textbf{IV} Праксиология -- от слова практика. Как человеку прожить хорошо. 

Натурфилософия -- первая философия с анто?а направлению. Природа.

Метафизика: в основе физики предпосылки, а физика как ядро. 

\vspace{1cm}

\hfill	\textbf{Feb 16}


Условия появления философии:
\\\t 1. Естественный фактор -- благоприятные географические и климатические условия. Люди могут много дней смотреть на небо, на звезды, это наводит на мысль о порядке и наличие причины на все это. 
\\\t 2. Достаточный уровень экономического развития. Только определенная часть общества может посветить себя духовному развитию или научному. 
\\\t 3. Определенный уровень развития мышления. Надо уметь работать с абстракциями. Например, появление в обществе денег.
\\\t 4. Разделение умственного и физического труда. 
\\\t 5. Появление письменных текстов. Тексты накапливались, носили сначала религиозный характер, но в них были аллегории, из которых происходит искусство, символика, из нее потом религия и тотэмы; присутствуют также и обобщения -- почва для появления философии и науки. 
\\\t 6. Социально-политический характер устройства общества. Греки дали две непреходящие ценности цивилизации -- философию и демократию (власть народа, выборы, борьба мнений). У Греков были тираны, но они не приживались, то есть быстро уходили потом. В Спарте, например, все было круто, но нет никакой свободы -- они не дали поэтому никаких теоретических результатов. В Китае тоже были империи, но они боролись, и их свергали. В Индии тоже. Именно эти движения вызвали философию.

\clearpage

\section{Философия древнего востока -- Индия и Китай}

Время появления в Индии и Китае примерно одинаковое. Числа придумали индийцы (хоть и называются арабскими). Несмотря на все достижения востока, наука там не появилась, поэтому мы обсуждать будем в основном запад. Более тесная связь с религией. Мистические элементы, анонимность (нет стремления выделить систему, все подстраиваются под общее). Сейчас все больше внимания уделяется восточной философии, принципу у-вэй. Лейбниц считал, что Китайский язык станет в конце концов языком науки. Бор, когда приехал в Китай, ему рассказали о Даосизме, он был поражен, увидел в этом первые верные представления о принципе дополнительности. Паули, Гейзенберг, Шредингер и т.д. с большим авторитетом относились к восточной антологии. Однако восточная философия осталась аморфной и расплывчатой, поэтому каждый мог найти в ней все, что хотел. 

\subsection{Древняя индия.} Тексты -- веды. Огромные тексты в совокупности больше миллиарда. Складывались веками, в конце концов появилась структура, названная сан хите, 4 блока: риг веды (восхваляющие богов), сама веды (общие религиозные песни), яджур веды (использовали при жертвоприношении), акхарба веды (связан с магией, сглаз, порча, колдовство). 

Блоки информации: Брахманы (по поводу жизнедеятельности высшей касты), Упанишады (сидеть у ног учителя, определял непосредственно, наиболее теоретизированный блок всех вед, на его основе появляется философия, от аморфных текстов к структуре и к школам, которые дискутировали и действовали в определенных направлениях).

Астика и Настика -- направления школ. Астика поддерживала основные Упанишады. Настика -- оппозиция. В ортодоксальных школах выделяют главные школы, которые сейчас существуют в индии: школа логики -- ниядя, школа большешика -- впервые пришла к мысли, что мир состоит из мельчайших частиц, но духовного плана, школа санхер -- первые дуалистические школы (при решении основного вопроса первичности люди приходят к мысли, что материя и дух на равных, одно без другого не существует. материя слепая, не знает, как ей развиваться), школа йоги -- самое известное есть управление своим телом (достигали и достигают огромных успехов), но  есть и йога познания и так далее, школа миманса -- отношение между учителем и учеником: есть гуру учитель и ученик и школа устанавливает между ними отношения, и наиболее развитая школа виданка -- школа объективного идеализма (постулируется некое начало вне нашего мира, которое является главным и направляет все развитие этого мира). Оппозиционные школы: джаимизма (от слова душа) aka школа повелителей кармы (идея, что можно менять карму -- в ортодоксии нельзя) и идея благоговения перед жизнью (не просто нельзя уничтожать, а вообще жесть. нельзя гулять в темноте, чтобы не задавить никакого насекомого. Днем надо повязку на рот, чтобы никого случайно не проглотить, вейником дорожку перед собой разметать итд), школа буддизма -- философская школа, которая превратилась в мировую религию, третья оппозиционная самая радикальная -- чарвака локаята -- атеистическая, никакой загробной жизни, имели принцип гедонизма, плотского удовольствия. 

Принципы -- гедонизма, счастья (самый абстрактный, ибо счастье), регаризма (человек пришел в мир не для жизни, удовольствия и счастья, а чтобы исполнить свой долг). Появилось два лагеря и они сейчас и они дискутируют, это философия. 

\paragraph{Упанишады}
Вернемся к корпусу Упанишад -- главный философский корпус. Ортодоксальные на него опирались, а оппозиционные отталкивались и отвергали. Изначальный образ -- мифологический. Всемирный человек -- Пуруша тысячи глав, тыс. глаз, ног итд. Боги его приносят в жертву, чтобы появилось мироздание. Из его тела создавали космос, ветер итд и наряду с космосом появляется социум, уже сразу структурированный (в индии была кастовая структура (барды), она и сейчас есть, хоть и с ним борется правительство). Из уст Пуруша появляется брахманы -- высшая каста. Брахман бог конструктор, Вишну бог хранитель, Дишиу бог разрушитель. Брахман -- абстрактное начало в отличие от Пуруши, к которому стремится человек в конечном итоге -- выйти из колеса сансары. Из рук Пуруша -- Кшатрии -- воины. Из бедер -- паты -- свободные торговцы. Из стоп -- шудра -- низший слой. Чалдоны вне структуры. Далее в плане антологии появляется абстрактное идеальное начало -- брахман -- и абсолютно субъективное начало -- акман -- в конечном счете акман должен стремиться к брахману. Земля, вода, воздух, огонь -- из них состоит мир, но есть Акаша -- невесомый элемент (на западе Эфир). 

В плане гносеологии два уровня познания -- эмпирический и теоретический. В теор есть Манас -- показывает связь чувственного и материального. Без него глаз не видит, ухо не слышит итд. Свидетельство других людей -- шаббе??. На одном уровне человек пытается слиться с объектом и раствориться в нем, а на втором человек должен достичь стадии ума, когда с вещей слетают оболочки и раскрывается сущность людей. И уже ничего не надо делать, просто созерцать. Самое главное во всем корпусе философии то, что он направлен на аксиологическую часть философии. В моральном плане все стремится к тому, чтобы представить все жизнь и пребывание человека, как приспособление к существующему порядку, но стремление уйти из этого порядка. Но уйти оказывается непросто, ибо считается, что существует колесо жизни и смерти, душа не умирает, но переходит в другую сущность, и происходит жизненный цикл. Нужно пытаться выйти из колеса жизни и смерти. 

Дхармы -- кодекс поведения. Сколько надо иметь слуг, сколько есть, пить и так далее. Не всегда получается жить по этим правилам. Поэтому после смерти Брахман судит его. В зависимости от оценки человека могут повысить или понизить в следующей жизни. Вниз можно понижать до уровня животного или даже растения. Из этого уже трудно подниматься вверх. А вверх можно дойти до стадии Брахмана. И если ты безукоризненно выполнил все части кодекса, то тебе прямой путь к Брахману, и ты вырываешься из колеса этого. 

Философия создает довольно развитую систему, в которой человек должен не роптать. Это все -- ядро Упанишад. Все школы примерно согласны с этим, но не Оппозиционеры. Наиболее мощная оппозиционная школа -- буддизм. 

\paragraph{Буддизм}
Буддизм начинается тоже с некой мифологии. Некому принцу Сиддхаттха из рода Готама было предсказано, что он может покинуть дворец, но он единственный сын, его оберегают и предоставляют райскую жизнь, никаких бед и несчастий. Но однажды в сопровождении колесничного выбрался из дворца. Тогда он впервые увидел калеку, нищего и смерть и узнал, что в жизни бывает несчастье. Чтобы найти истину, он сбежал и долго плутал, сел под фикусовым деревом, и через несколько дней стал просветленным. Он открыл 4 главных истины: всякая жизнь есть страдания; причина страданий есть сама жизнь (желания и погоня за наслаждением); от страданий можно избавиться, достигнув нирваны; чтобы сделать это, надо пройти восьмиричный путь. Требования: правильно мыслить, правильно говорить, вести правильный образ жизни..., но что значит правильно не говорится, для выяснения этого есть учитель. Например, Будда иногда давал противоречивые ответы, чтобы люди сами осознавали ответ. Для чего надо проходить путь? В конце нирвана, в конце она, угасание всех желаний, стремлений, и всего остального, что превращает человека в потребителя. И если у тебя все получится, то при жизни ты достигаешь этой нирваны. Оппозиционность упанишад в том, что там только после смерти и вообще не только одной. И причем каждый может это все сделать, в этом демократичность и религиозность. 


\subsection{Китай.} Китайская философия. Тексты более авторизованные. Есть мощный корпус текстом -- в первую очередь и-цзинь(книга перемен), ши-цзинь (история), у-цзинь (стихи). Начинается тоже со вселенского человека Пань Гу. Тоже погибает во имя образования космоса. Там же появляются субстанции -- земля, вода, воздух, огонь + металл + дерево. Металл как символ твердости, а дерево устойчивость и гибкость. В Китае также происходит разветвление на отдельные школы, главная -- школа даосизма, школа Конфуция, натурфилософская Ин и Ян, школа исправления имен, и т.н. школа лигизма. 

\paragraph{Даосизм}
В антологическом плане главная школа -- Даосизм. С очень абстрактным ключевым понятием -- дао. В нек. источниках есть два дао -- безымянное (его никак нельзя понять, и нельзя выразить) и еще одно (основа и фундамент для порождения всего и имеет структуру, интерпретируется максимально широко, первоначало, первооснова, путь, дорога, знание всего и вся, прародитель всех вещей и так далее). Имеет структуру по принципу дополнительности: Ян твердое мужское всегда тянется в космос и Инь аморфное мягкое женское. Они друг без друга не могут существовать. Их склеивает цинь. Поскольку дао ответственно за все, человек следует земле, земля следует небу, а небо следует Дао. Человек должен приобщиться к дао. Кто знает, тот не говорит, кто говорит, тот знает. Если приобщился к дао, то знаешь все, что происходит под небом. Отношение к власти -- самый лучший это тот, о котором они знают только, что он правитель. Потом тот, которого знают и могут похвалить или поругать. Самый худший тот, над которым смеются. 

\paragraph{Конфуций}
Наиболее важная противостоящая даосам это школа Конфуция. В Китае воспринимается как религия. Конфуций полумифологическая штука. Золотое правило -- не делай другим того, чего не желаешь, чтобы делали тебе. Старайся выбирать золотую середину между крайностями. На первом месте стоит небо, поклонение небу, как важнейшей антологической структуре. Выделяется три вида знания -- врожденные знания, приобретенные, полученное от других через обучение. Расходится с даосами в том, что надо образовывать всех и максимально. Образование должно продолжаться до смерти. В этом смысл человеческой жизни. Связывает образование с добродетелем. Третья важнейшая идея -- устроение самого общества. Общество должно строиться не на страхе и наказании, а на моральных принципах. Принцип Ли -- поклонение церемониям. У них это принцип жизни. Церемонии и правила и ритуалы должны быть везде. Второй принцип Жень -- человеколюбие, возлюби ближнего своего. Принцип Сяо беспрекословное уважение родителей и принцип Си -- уважение к старшему брату, ибо он глава семьи. Человек, находящийся во главе общества -- отец.

\paragraph{Моизм}
Школа моизма. Мо цзы. Противопоставление Конфуцию. У Конфуция поклонение небу выражало идею судьбы, судьбу нельзя было изменить. Мо цзы считал, что человек обладает свободой и он сам повелитель своей судьбы. Ритуал должен быть распространен не на окружение человека, а на всех в целом. Он предвосхитил идею философа Людвига фер баха. Он сказал -- человек человеку бог. 

Последняя школа будет разобрана на следующей лекции. 
\\

\hfill \textbf{Mar 1}

\paragraph{Лигизм} 

У конфуция ученики Мен Цзы и Сынь Цзы. первый -- принцип любви, люди в основе дружелюбны, надо развивать это качество и строить отношения в обществе как говорил Конфуций. А у Конфуция не было определенной сущности при рождении -- добро или зло. Сын Цзы говорил, что люди изначально злые, поэтому, как говорил Конфуций, их надо обучать. На основе этого развили систему правовых отношений Китая, оно основывалось на том, что никто никого не обязан любить, надо только подчиняться закону, закон -- факт. Стремление регламентировать все поступки. За невыполнение назначается соответствующее возмездие. Философская основа в том, что если пытаться регламентировать по законам все и вся, то по какому принципу должно происходить это самое наказание? Китайцы встали на точку зрения -- чем менее значимо нарушение -- тем меньше должна быть кара. Если люди будут бояться совершать малое, они тем более будут бояться крупных. Возникал идеологический опрос. Опасные преступления обнаружались просто, а вот как найти мелкие непонятно. Чтобы делать это, нужно стукачество. Китайцы изобрели стукачество. Выдавалось поощрение за такую штуку. Надо было издать дополнительно закон о недоносительстве и назначалось суровое наказание. Совсем другая относительно Конфуцианства система. Она даже продолжает работать сейчас. Школа Лигизма. 15 мин.


\section{Западная философия}

7-6 века до нашей эры. Философия, которая легла в основу всей Европейской цивилизации. Для возникновения философии необходимы некоторые условия. В Греции именно они и сложились. Греки могли плавать много где, познавать мир, был высокий уровень рецептурно-технологическими знаниями типа архитектуры, искусства, Были развитые товарно-денежные отношения. Рабский труд был, но не являлся основным. Основу составляли свободные люди. Все это в том числе привело к демократии. К тому же была развитая мифология, Гомер, и другие эпосы. Они были основой взглядов на мир -- мировоззрения. Принцип агона -- принцип соревнований и игр. Олимпийские игры, например. Кроме того, Греки постоянно воевали (Персы, Тунические войны). 

Генезис в греческой философии происходил примерно в 7-6 века до нашей эры. Классическая греческая 5-4 века до н э. Эвинистический период 4до нэ - 1 нэ века. Писсемиды -- школы скептиков. В Александрии появляются школы науки -- Евклид, Птолемей, Архимед. Упадок с наступлением христианства в 1 веке н э. И в конце концов Римский Император Юстиниан издал указ о закрытии последней школы -- Академии Платона. Философы побежали в другие страны. Их там тоже не поняли, и Греческая культура и философия пришли к своей финишной черте и закончилась эпоха. После греков наступила эпоха теоцентризма. Религии строились на отрицании или на христ. мнении о Греческой философии. А в эпоху возрождения Европа обратилась к древней философии. 

\paragraph{Фалес}
Первая школа. Школа софистов, натурфилософия. Философов нельзя отличить от ученых. Милевская школа, Дионийская школа, возникали все на периферии государства, не в центре. Фалес Анаксимандр Анаксимен: учитель ученик-учитель ученик. Фалес первый начал задумываться об устройстве мира не из религиозных, а из теоретических соображений. Архэ (с греческого ``начало, принцип'') -- а что находится в основе мира? Фалес пришел к выводу, что в основе всего Вода -- он опирался на мифологию и на Гомера итд. С одной стороны Фалес создал мифологию, а с другой -- концепт. К тому же он уже был ученым, например, он предсказывал затмение солнца, измерял высоту египетских пирамид (на основе теней итд). Он создал школу учеников. Анаксимандр. Он первый создал географическую карту, ввел естественное происхождение людей (не божественная теория), а главное -- он подверг критике своего учителя по принципу архэ. Он считал, что всякая вода имеет дно, а значит, в основе в таком случае дно. А что под дном идей не было. В итоге он пришел к выводу, что вода ни на чем не должна покоиться (как планеты не существуют на чем-то а просто существуют). Он создал сущность, которую просто так не пощупать. Третий -- Анаксимен -- сказал, что архэ можно считать воду, поскольку без нее никакая жизнь не может существовать, и в итоге занялся метеорологией, климатом, вулканами, землетрясениями итд. Именно с этой школы получается критический рационализм: учитель инициирует ученика на критику и движение своих идей, а ученик должен всегда знать, что усваивает знания критически и может выдвигать свои идеи.

\paragraph{Гераклит Эфесский}
Следующее учение -- Гераклит Эфесский. Его называли философом плачущим, он был потомком знатного рода, но все бросил, ушел в пещеру. Когда у него спрашивали, почему он такое выбрал, он отвечал "и здесь живут боги". Он плакал -- оплакивал людей, потому что видел, насколько несчастна их жизнь. Он твердо знал, что в основе мира лежит огонь -- энергия. Начиная от Солнца. Из огня все происходит, в огне все происходит, в огне все погибает. Кроме этого важная идея -- первое представление о том, что мир устроен закономерно, в нем есть порядок. Прекрасное поле для наблюдения -- звезды, планеты. Гераклит в этом смысле пришел к философской идее, что за всем этим стоит метафизическое явление, объяснение мира с помощью закона -- логос. От него не зависит, как звезды будут себя вести. Логос -- идея безличного закона и порядка. Третья важнейшая идея -- идея диалектики. Мир устроен диалектично. В нем всегда есть добро и зло, четные и нечетные числа и другие странные противоположности. Все течет и все изменяется. Нельзя дважды войти в одну реку. Все находится в движении, которое носит противоречивый характер. Исходя из этого получается, что путь вверх и путь вниз это одно и то же, аналогично также жизнь и смерть. Но эти противоречия не в нас, а в самом мире. Пример -- корпускулярно-волновой дуализм. 

\paragraph{Пифагор}
Следующая школа ясно выражена. Более 500 лет существовала. Школа Пифагора. Пифагорийский союз. Полурелигиозная полунаучная школа. Все ученые были глубоко верующими людьми. Разговариваем с животными, реками, слышали музыку небесных тел. Математикам было сказано слушая учителей придумывать новое. Числа играют важнейшую роль в человеческой культуре. Сначала говорили, что все есть число, потом под влиянием критики других школ сказали, что все можно измерить числом, все подобно числу. А третий шаг -- сущность всего есть число. Это присутствует даже сейчас. Некоторые говорят, что математика первична. Они пытаются найти сопряжение правильных фигур с основными физическими явлениями: воздух октаэдр, вода икосаэдр, огонь тетраэдр, земля куб. Математику можно геометризировать наоборот, геометрию выразить в числах. Точка есть единица, двойка как линия, тройка как плоскость, четверка как куб. Сущность мироздания зиждется на основе числовых отношений, была одна вещь, с которой не справились. В греческой культуре все логично, все правильно, а там они обнаружили несоизмеримость. Катета и гипотенузы прямоугольного треугольника. Это так всех шокировало, что проблему аж решили засекретить. Однако один участник проболтался, и в итоге был вынужден сброситься со скалы.

\paragraph{Эмпедокл}
Эмпедокл. Философ, в отличие от предшественников, считал, что есть система элементов архэ. Четыре основных -- земля вода воздух огонь. Оригинальность в том, что он увидел их взаимодействие, они всегда находятся в динамике, получаются конструкции, потом они разъединяются итд. Вселенная тоже пульсирует и пульсация ведет от хаоса к гармонии и обратно. Он не стал это перекладывать на богов (ответственность), но ввел понятия любовь и ненависть -- то, что движет миром. Когда царствует любовь, все приходит к гармонии, но гармония не может длиться вечно, и в пике вдруг появляется ненависть. Это такая диалектика, которую потом Аристотель назовет не услышал как. Вторая идея Эмпедокла -- идея эволюции. Мир не создан живой единым в мгновение. Были трактаты о природе стихами. Согласно ним: Появились головы без затылков и шей, блуждали голые руки. Очи не знавшие лбов блуждали по земле. Появлялись гермафродиты и прочая дичь, которую только можно составить из этих частей, но они не выживали. Может появиться что угодно, но не все выживает. Процессы повторялись, пока не появились нормальные люди. Грубый вариант, не на молекулярном уровне, но идея правильная. В том числе он первый сказал, что скорость света имеет свой предел. Суперинтересный факт: Эмпедокл был философ одиночка, в Сицилии, был интеллектуальным богом. Он бросился в вулкан, чтобы доказать, что он бог. 

\paragraph{Элейская школа} Основатель Элейской -- Ксенофан, Парменид (ученик), Зенон, Мелисс (военачальник). Ксенофан сказал, что мир может быть и замкнутый, но с бесконечным радиусом. Из ничего ничего не происходит. Нет никаких божественных сил, но всегда есть что-то материальное и причинно-следственная связь. Ксенофан первый подверг критике религиозные учения Греков по причине того, что у всех народов боги похожи на сами эти народы. Он говорил концептуально, что Бог всегда антропоморфен. По-другому изобразить нельзя. Сам он представил Бога, как мистический шар. Вторая мысль -- о тождестве бытия и мышления. Бытие есть, а небытия нет. Все, о чем можно сказать, может реально существовать. Мыль, которую дальше развивает Парменид, пустоты никакой не существует. Потому что даже отрицание пустоты возносит ее на пьедестал существования. Настоящее едино неподвижно. Первая модель стационарной вселенной. На основе размышлений Парменида Зенон показал парадоксы. Движение и время это только иллюзии, в фундаменте мира все по-другому. Первый парадокс дихотомия: Движение в принципе не может начаться. Рассмотрим движение из А в Б. Чтобы пройти путь, надо пройти половину пути. Чтобы пройти половину пути, надо пройти половину половины пути. итд. Аналогично движение не может закончиться. Есть Ахил и есть черепаха. Сначала отправилась в путь черепаха, и потом за ней побежал Ахил. Ахил догоняет, приходит на место, где была раньше черепаха, но черепаха уже отползла. И так снова до бесконечности. И сейчас тоже размышляют об этом, что есть бесконечно малое, квантуется ли всякое такое итд. Фактически наука геометризирует время. Делит время на отрезки и говорит, что можно догнать. Но ведь время всегда движется, его никак нельзя фиксировать. Третья -- стрела, которая должна была показать иллюзорность процесса движения. Мы видим полет стрелы из точки А в точку Б. Эмпирически все понятно, а теоретически что такое полет? В каждый момент времени стрела находится в определенном интервале пространства. Но раз стрела занимает какое-то место, то она покоится. Как из этих бесконечных состояний покоя может сложиться движение? Это все показывает несостоятельность теоретического представления. Два уровня познания -- эмпирическое (назвали докса - пер. с греч. ``мнение'') и теоретическое (эпистеме - пер. с греч. ``знание''). Докса -- сколько людей -- столько мнений. Вы так чувствуете, ощущаете и так далее складывает ваше мнение. Эпистеме -- люди приходят к консенсусу и говорят вот это вот истина. 

\paragraph{Анаксагор} -- работал в Афинах. Первый в Афинах философ, хоть и приглашенный Пириклом. Фины достигли высшего положения в Греческом обществе. Поэтому и философия расцвела. Он первый высказал мысль, что Солнце находится дальше Пилопонеса. Небесные тела не боги, а раскаленные непонятно кто. И даже падают иногда. Его за это хотели казнить, за неуважение богов, но изгнали. По философской сути высказал -- все во всем. Все делится до бесконечности и все элементы всегда присутствуют, но есть уровень гомиомерия. Мельчайшие частицы, которые обладают всеми свойствами самого вещества, но мельчайшие. Они могут образовывать уже вещество. Все зависит от пропорций. Элементы носят качественный характер. Но происходит ли это разнообразие само собой? Он ввел понятие нус (пер. с греч. ``сознание''). Мирового разума. Это не бог, а просто какой-то разум. Оно определяет качественное разнообразие мира из частиц этих. 
\\

\hfill \textbf{Mar 9}

\paragraph{Атомизм}
Последняя мистическая школа Атомизм. Самая знаменитая школа, идеи философов (Демокрит, Ливкипп) вошли даже в науку и все так и считали, что атомы -- неделимые частицы вещества. Фейнман говорил, что достижение этой школы таково, что если бы из всех знаний осталась только идея, что мир состоит из атомов и пустоты, то можно было бы быстро наверстать. Ливкипп говорил, что ничего не происходит из ничего. Все имеет свою причину. Но его труды не сохранились, хоть и известно, что они были. Демокрит был учеником Ливкиппа, и соответственно конспектировал в том числе учителя. Однако довольно сложно разделить, где были мысли Ливкиппа, а где уже Демокрита. Демокрит был сын из знатного рода и имел большое наследство, но говорил, что знания существенно важнее любых корон и богатств. Он всю жизнь путешествовал и поэтому промотал все состояние, его привлекли за это к суду, но во время суда он зачитал пассажи и настолько поразил судей, они поняли глубину его теории до такой степени, что стали ставить статуи и оказывали всевозможные почести и в том числе спонсировали его. Демокрит говорил, что небытия (пустоты) столько же, сколько бытия. Он говорил, что если есть атомы, то они должны быть где-то -- это пустота. Эти атомы, в отличие от гомеомерия, у Демокрита были качественно однообразны. Различные вещи формируются за счет форм атомов. Гладко -- сладко, крючки -- остро. Атомы движутся, трясутся, сталкиваются, сцепляются и образуются некоторые вещи. Как буквы алфавита. Разные по форме, но качественно одинаковые, и могут составлять неограниченное число фраз. И причем не нужен никакой источник движений, не нужен Нус. Демокрит говорил, что без материи нет движения, без движения нет материи (ничто никогда не покоится). Эти атомы могут быть разной величины, они создают миры, и миров может быть бесконечное количество. И эти миры возникают случайным образом, но внутри каждого все жестко детерменировано (как только создана начальная конфигурация атомов, их дальнейшие движения подчиняются каким-то определенным законам). Это потом войдет в физику, как Лапласовский детерменизм. 

Случайность приобретается и воспринимается, как гносеологический факт. ``Люди измыслили идол случая, из-за ленности мышления или невозможности предсказания. А на самом деле все детерменировано.'' В качестве примера он приводит притчу о черепахе и лысом греке. Лысый грек в солнечный день вышел из дома и ему на голову упала черепаха и убила его. Случайность есть лишь пересечение линий причин и пространства-времени. Черепаха грелась на солнце. Черепахами питаются орлы: берут, высоко поднимают и бросают вниз, чтобы разбить панцирь. Причем орел умеет различать твердые породы и мягкие и по отражению от башки орел решил, что это твердая штука. Лысость головы грека, отсутствие головного убора и солнце объясняются очевидным образом. Надо сразу дополнить идею детерминизма. Ученик Демокрита Эпикур жил в Риме, но был последователем атомизма, однако ему не нравилась фатальность. Говорил, ``лучше я буду молиться богам, чем буду верить в эти причинно-следственные связи''. Он шел от свободы воли человека. Он говорил, что имеет право выбора. Исходя из идеи свободы человека, он пришел к выводу, что существует свобода и случайность. И тогда он сказал: атомы летят определенным образом и определенно, но могут отклоняться от заданного законом движения произвольным образом в произвольном месте. Тогда случайность включается в само устройство мира. 

В гносеологическом плане важно: Демокрит был первым, кто различил чувственное и теоретическое познание, исходя из первичных и вторичных качеств. Первичные: форма размер и т.д. (свойства, которые объективно принадлежат вещам). Темное познание дается нам в результате взаимодействия наших органов чувств и внешнего мира (сенсорика) -- действительно ли соль соленая и сахар сладкий? Демокрит как раз называл их темными познаниями. Это не полностью адекватная информация. Но нельзя и сказать, что наши органов чувств совсем обманывают нас. Он сказал, что раз и так все состоит из атомов и пустоты, то да, есть соленость, но за этим понятием стоит объективная атомистическая реальность. Для истинных знаний необходим теоретический анализ. Для морали в аксиологическом плане Демокрит пришел к эвтюмии (пер. с греч. ``хорошее настроение''). Эвтюмия это спокойное расположение духа, поскольку все и так предопределено. 

Далее философия делает резкий поворот и приходит к антропологической философии. Две школы: софистики и Сократа. В обеих школах главной проблемой является человек. 

\paragraph{Софисты}Школа софистики. Аристотель характеризует ее как школу мнимой мудрости, хотя изначально слово софист означало учитель, мудрец, философ итд. Но после деятельности школы слово обрело существенно отрицательный смысл. Софисты были первыми профессиональными философами -- они брали деньги за уроки и так далее. Но кого это интересовало? Это был рассвет демократии в древней греции, выбирались судьи, военачальники, администрация и все они на определенный срок. У людей появляется имущество и необходимы адвокаты и так далее. Адвокатов не было, каждый должен был защищаться сам. Выиграть судебный процесс надо было здесь и сейчас. Поэтому в ход стали идти совсем разные приемы, в том числе ненаучные. Ибо важен был лишь результат. Этому и обучали в школе. Они создали риторику -- умение хорошо выступать, выразительно итд. Мнемоника -- умение запоминать все эффективно, чтобы в нужный момент воспроизвести нужный факт. Учили прокручивать разные варианты в голове и связывать. Принципы релятивизма и первый принцип относительности были провозглашены впервые в этой школе. Все участники делятся на старших и младших: НЕ ЗАПОМНИЛ ИМЕНА. Младшие: Алкидан первый усомнился в делении всех людей на рабов и свободных. (Греки считали, что только греки свободны, все остальные рабы) Алкидан говорил, что бог и природа делает всех людей равными. Никто не рождается рабом. Фрасимах говорил, что демократия никогда не бывает чистой демократией. Ибо все законы всегда основаны на соотношении сил и богатств различных групп людей. Все законы отражают прежде всего интересы богатых. Критий высказывает мысль, что религия имеет прежде всего социальную функцию -- держать в узде людей на основе принципа кнута и пряника. С одной стороны она угрожает, а с другой награждает. 

Главные идеи высказывают старшие софисты. Они проблематизировали сущетвование законов в природе -- не можем точно сказать, есть ли законы в природе самой. Они также говорили, что в обществе тем более нет никаких законов, есть лишь борьба. Поэтому на первом месте надо ставить умение отстаивать свое мнение. Максима -- человек есть мера всех вещей существующих в том, что они существуют, несуществующих в том, что они не существуют. Нельзя понять вещь саму, как она есть, потому что все в мире взаимосвязано. Протогор это сказал. Корги сказал: вообще сомнительно, что что-либо существует. Мы познаем все только через ощущения. Но даже если что-то бы и существовало, мы не могли бы это выразить, ибо оно существует саму по себе, а любое выражение осуществляется языком, а языка в природе самой нет. Даже если что-то и можно было познать, мы не могли бы это передать другим. Это связано опять же с языком. В результате этого принцип относительности и релятивизма создает школу Скептицизма. Такого рода рассуждения, которые имеют логику, приводят к агностицизму. Отказ от познания вообще. Познание не имеет смысла. 

\paragraph{Сократ} Против всего этого выступил Сократ (469--399 гг до н э). Его называют самым благородным и самым мудрым. У него две максимы: я знаю, что я ничего не знаю, но другие не знают и этого. Речь идет об ученом незнании. Когда человек что-то познает, у него получается гораздо больше вопросов. Вторая максима: познай самого себя, прежде чем обращаться к познанию природы. Это показывает антропологическую направленность его философии. Жизнь Сократа оборвалась трагически. Его философство состояло в том, что он ходил и беседовал с людьми. Ничего не писал. Все записи делал Платон, его ученик. Сократ вел очень скромный образ жизни. На вопросы о бедности он отвечал: все важное начнется после смерти. Его проповеди и беседы не прошли бесследно, его обвинили в развращении молодых умов. Его судили 500 присяжных. Он защищал себя сам, но использовал иронию и вызывал к себе еще больше огня. Ему предложили покинуть Афины или здохнуть. Он выпил яд. 

Какие у него были идеи:. Ему мы обязаны идеями научной дискуссии. Сократ указывал три пункта: вступая в научный спор, вы должны предполагать, что оппонент не совсем неправ (мб научишься сам); вступая, надо полагать, что вы тоже не совсем неправы; в результате доброжелательного, объективного и очередного выступления вы можете надеяться на то, что ваши точки зрения сблизятся. Сократ говорил, что истина в споре не рождается. Она рождается к коммуникативной практике. Методом цикличности: обдумал спор и потом встретился снова. В частном плане Сократ изобретает метод маевтики. Маевтика -- повивальное искусство. Суть в том, что маевтика должна помочь родиться истине. Не поставить точку, а помочь. Три пункта: ирония и сарказм (Сократ вел странный образ жизни, беседовал, с софистами тоже. Сократ при разговорах с ними прикидывался простачком, спрашивая, что такое мораль, красота итд. И, задавая вопросы, показывал, что не все так просто); диалектика, субъективная диалектика (у Гераклита объективная -- борьба в самой природе, у Сократа вопрос важнее ответа, вопрос показывает интеллектуальную активность. Чем умнее вопрос, тем полнее ответ. Искусство постановки вопроса); собирать фактический материал и обобщать его -- метод индукции, от частного к общему. На выходе получается понятие -- оно обобщает все существенные моменты и дает представление о классе объектов. Продемонстрируем это на одном из диалогов Сократа. Он у военачальников спрашивал, что такое мужество. Ответ надо держать строй, слушаться, не бояться боли, смерти и так далее. Вопрос есть же и другие вещи: Спартанцы и Скифы бегут в ответ на военные действия, распадаются и в ответ бьют со всех сторон. Это и не трусость, и не строй. В море можно богам молиться, а можно пытаться бороть стихию. А мужество смертельно больного человека тоже мужество? А мужество нищего, который хочет покончить с собой, но все равно старается выжить и борется. В конце пришли к выводу, что мужество это стойкость души, сопряженная с разумом. Этот метод маектики противостоит этистике софистов. Эристика это стремление добиться результата. Сократ ищет истину. 

Еще: Сократ моралист, считает, что человек, познавший себя, априори будет моральным человеком. Он первый считает, что моральность и нравственность человека напрямую зависит от ума. Если человек поступает плохо, он просто не знает, что это плохо, ему надо просто объяснить, что это плохо. За это отождествление Ницше считал Сократа своим врагом. Поле Сократа появляются школы, которые назвали Сократическими. Наиболее известные: киников (циники, собачья философия), киринаиков, кираиков. У киников Антисфен и Диоген. Философия ригаризма. Надо вести скромный образ в жизни в пределе. Лучше умереть, чем испытать удовольствие. Киринаики с точностью до наоборот. Для них именно удовольствие на первом месте. Это Аристип (который называл учителя учителем злополучья), Гигесей (проповедник смерти -- если жизнь перестает приносить удовольствие, надо покончить с ней). Киринаики гедонисты. Мегалики говорили, что количественное отношение переходит в качественное. Если вырывать у человека по одному волоску, то он станет лысым. Наиболее важная вещь -- парадокс лжеца. Критянин приходит к царю и говорит, что все Критяне лжецы. В конечном счете перефразировал Рассел это все на основе теории множеств. Может ли множество всех множеств быть подмножеством самого себя. Эти школы кроме моральных проблем поставили методологические проблемы соотношения общего и единичного. Что первичнее -- общее или единичное. 
\\

\hfill \textbf{Mar 16}

\paragraph{Платон}Философия Платона 427-347 гг до н э. Оказал влияние на математическую физику, в 20 веке некоторые физики считали себя платонистами и говорили ``божественный платон''. Однако до этого на него особо не обращали внимание. Платон -- первый философ, от которого осталось богатое литературное наследие, о многих спорят и сейчас. Литературный стиль -- форма диалога, главный персонаж Сократ. И поэтому тоже проблема разделить мысли Сократа и Платона. Диалоги чаще не заканчиваются определенной мыслью, а просто диалоги. Сам Платон выходец из знатного рода, широкоплечий был, увлекался верховой ездой итд, но после встречи с Сократом решил, что это для него важное. Платон записывал за Сократом. Первое достижение -- создание Академии. Он считал математику первичной по отношению ко всему. Также он создатель действительно первой широкой системы объективного идеализма. Он считает, что есть нечто идеальное вне нас и оно первично по отношению ко всему материальному. Отсюда и появляются две ясно выраженные линии в философии -- материализм от демокрита (атомы и пустота и нет ничего идеального) и объективный идеализм. В истории философии есть еще две великие системы объективного идеализма. Готфрида Лейбница 17-18 века, идея существования неких бестелесных монак. И система Гегеля с его представлением о существовании вселенского разума. Вернемся к Платону. Ему представляется идеальным -- вводит знаменитое понятие идеи -- эйдос. Мир эйдосов первичен по отношению ко всему материальному, потом скажут, что идеи правят миром, но это уже будет иметься социальный контекст. А Платон представлял в плане антологии, что мир идей первичен ко всему физическому и причем количество идей ограничено и они не просто как атомы у Демокрита туда суда трясутся итд, а образуют некую иерархию, где есть высшие идеи и низшие идеи. Его идеал -- сопоставить все вещи на земле миру идей. Самая высокая идея в конусе была идея Блага. Все должно быть устремлено к благу. Подкрепляют Благо идеи истины (из-за нее появилась наука), красоты (красота спасет мир, порождает искусство), гармонии, порядка (это уже математика). А дальше вниз до самых простых идей. Выдвинув идею пирамиды идей он не мог полностью осуществить, а дал ученикам. А Аристотель раскритиковал. Если все устремлено к благу, то как охарактеризовать грязь, отходы и прочие такие идеи. В менталогическом плане ошибка -- Платон говорил, что идеи могут соответствовать вещам. А раз есть соответствие идеи вещи, то есть идея этого соответствия. Гегель назвал это дурной бесконечностью.

Мир идей считался ограниченным, и Платон старался объяснить это. Знаменитый миф о Пещере (из диалогов о государстве). Представим пещеру и в ней узников. Они прикованы и не могут не только выйти, но и повернуться к ее выходу. И значит для них весь мир это мир теней, которые появляются, когда что-то происходит у входа в пещеру и они видят на противоположной входу поверхности наблюдяют действия и это для них единственная реальность. Они будут воспринимать это, как единственно возможный мир и единственную реальность. Более того, даже если усилить условия, то есть если кто-то сможет выбраться из пещеры и увидит мир, а потом расскажет другим это, то они не захотят ему верить, ибо он выскочка кажется. А теперь представим, что и мы живем в таком мире теней, но у нас есть солнце, которое дает нам возможность видеть, но все это есть лишь отражение идей (что мы видим). Мир идей можно если не увидеть, то аргументировать. Где-то там есть мир идей. 

Этот мир по Платону устроен так: мир есть отражение идей. И по этому принципу можно отличить Сократа от Платона. Он создает космологическую картину мироздания с помощью понятий, которые сейчас в ходу. Первое представление -- представление единого. Категория единого. Все мироздание начинается с единого, абстракция высшей степени порядка и не зависит ни от чего и в этом смысле оно трансциндентно. Его нельзя увидеть непосредственно, можно лишь догадываться, как о Дао. Из единого выходят димиург, хора и парадигма. О них уже можно говорить что-то конкретное, их называют трансциндентальными, предельными. Парадигма это некая матрица идей, на основе которых создаются наши существующие вещи. Димиург (пер. с греч. ремесленик) -- бог, который создает вещи из идей. Хора это некая мистическая субстанция -- кормилица, восприемница, некая иррациональная материя. А в идеологии теней по сюжету это некое вместилище, в котором происходит действие. Но нельзя полностью отождествить с пространством. А дальше Димиург первое что творит -- мировую душу. Она дает импульс всему мирозданию, мироздание начинает дышать и появляется время. Время это подвижный образ вечности. Душа анимирует все остальное и можно сказать, что космос у Платона живой. Дальше происходит творение планет, солнца, устроение самого космоса (платон представляет как огромный шар с конечным радиусом, земля в центре, космос уютный и гармоничный, потому что Димиург благ и он не мог не создать благой космос). Дальше на земле создаются боги олимпийцы, которые живут на горе Олимп, а уже сами олимпийцы создают людей и полубогов. Живые люди созданы богами и прежде всего боги создают мужчин. В человеке, поскольку он создан богами, главным является душа, и душу платон разделяет на разумную, рациональную и вожделеющую, чувственную. И иногда вторая сторона души -- темная -- одерживает верх, и человек плохо начинает себя вести в жизни. Для Платона пример был Атлантида. Платон ее описывает подробно, будто сам там был. Они сначала хорошо жили, а потом стали жадными итд и их покарали и отправили на дно. По Платону некоторые мужчины могут вырождаться в женщин, и иногда мужчины деградируют даже в других животных и в самом низу -- в рыб. 

Какую роль играет душа? Мрачная ведет к тому, что можно и в рыбу превратиться, а светлая нет. Платон вводит анамнезис. Знание припоминания. Эмпирическое знание присутствует, но мудрости в нем нет никакой. А вся мудрость находится в хоре, а души до помещения в людей живут на небесных телах и звездах и общаются и знают все и сразу. Но после вселения в тело, душа забывает об этом своем знании, которое она черпала, пока была в свободном полете. Победа заключается в вспоминании этого всего. Платон демонстрирует это на примере мальчика раба, к хозяину которого приходит Сократ. Их разговор ``ну да мальчик шустрый, такой, но ничего не знает, ничего не понимает итд''. Сократ начинает беседовать с мальчиком. Мальчик понимает интуитивно что такое углы и прямые итд, а Сократ с помощью метода маевтики приводит его к доказанию теоремы. Настоящее знание уже было в мальчике. Это как раз знание математических истин, и оно автоматом присутствует в человеке. Надо только достать их. 

По Платону мироздание состоит из воды, земли, огонь, воздух. Это видимое. А невидимое представляет правильные геометрические фигуры. Это у Платона от пифагорийцев. Но он идет дальше. Говорит, что все фигуры можно разделить на треугольники и в итоге веси мир разделяется на прямоугольные треугольники. Александр Поляков (недавно получил премию за теорию струн) в одном из интервью у него спросили, как он дошел до такой жизни. Он ответил ``я со студенческих лет запомнил миф о пещере и его рассуждение о том, что мир покоится на математической реальности. То есть мир не такой, каким он представляется с помощью органов чувств.'' Действительно. 

Дальше Платон создает общую теорию творчества. Творчество понимается, как переход из небытия в бытие. Платон фактически признает небытие, но не как у Демокрита, а как преддверие бытия. То есть бытие гораздо мощнее небытия. И Демиург выдавливает бытие из небытия и то есть оно находится в нем. Отсюда идея, что все люди беременны творчеством. Но у разных людей проявляется по разному. Классификация: в основании пирамиды массовость и общественная ценность. Первая ступень физическое или физиологическое творчество (создание себе подобных). Художественное творчество -- мимесис (``подражание'') раз вещи есть подражание идей, то произведение есть тени подражания. Художники импульсивны, экспансивны итд. Далее техническое творчество. Для этого надо уже обладать знаниями. Это уже меньшая группа людей, и их влияние уже гораздо выше, чем у художников итд. Дальше научное творчество -- духовное и теоретическое. Еще меньше людей и еще больше влияние. В то время не было особенного теор знания, но Платон предвидел его появление и необходимость. Нет ничего практичнее, чем хорошая теория. И высшим видом творчества считалось Социальное -- законодательство. 

Какой стимул заниматься творчеством? Главный стимул заниматься любым  видом творчества был стимул красоты. В физическом -- идеал человеческого тела. Например в статуях появляется. Красота и истина. 

Социальная теория Платона. Он первый создал такое, сейчас это называется политологией. В содержательном плане она была первой коммунистической утопией. Мир и общество должны быть справедливым, все должны работать по способностям, получать по потребностям, люди должны быть равны и экономическое равенство должно результировать в гармонии. И также приплетает сюда космос. После этих мыслей Платон видел демократию, которая убила его учителя -- Сократа. И войны со Спартой показали Платону всю крутизну Спарты. И он говорил, что там круто. Демократию Платон приравнивал к охлократии -- власти толпы, которая подчиняется инстинктам. Когда демократия переходит к этому, появляется диктатура и тирания. После тиранов приходит власть военных -- тимократия. Власть силы. Власть может быть также богатых, которые могут сменять постепенно власть военных. И этот цикл постоянно может происходить. Как из него вырваться? Платон разрабатывает теорию идеального государства и в основу кладет антологическую предпосылку, что люди не могут быть одинаковыми во всем, из-за антологических причин. Потому что ценные металлы и прочая дичь приходит в людей. Платон говорит, что там, где рождены люди с большим удельным весом золота, люди самые умные, с серебром чуть меньше, а железо и прочее не может претендовать на успех. Из золота вырастают философы. Они и должны править людьми. Из серебра вырастают воины. Из Железа и меди крестьяне и ремесленники. Похоже на касты в Индии, но с другим происхождением. Коммунизм состоит в том, что ни философам, ни воинам нельзя иметь собственность, они должны делать все для блага общества. Потому что имущество заставляет забыть об обществе в целом. Собственность должна принадлежать ремесленникам и крестьянам, в виде утешения. Но были и исключения, ибо сложно определить чего в ком больше. Платон устраивает для этого тестовые испытания, которые по идее разрешают даже переход между тремя кастами. Есть три различные науки: экспериментальные (физика и астрономия), промежуточные (математика, геометрия, музыка), высшие (диалектика). В различные возрасты происходят тесты по различным наукам. В результате в идеальном государстве все расписано, как в армии.

Платон пытался и претворить это в жизнь. Он считал, что для этого достаточно закрытый остров иметь. Он беседовал с правителями итд, но его не поняли и даже хотели продать в рабство. Подводя итоги хочется сказать, что Платон создал систему, в которой много рациональных вещей: первичность математической реальности и огромное значение, важность понимания идей, как фундаментальных понятий, теория познания, где всегда нужен Сократ, который может вытащить тайные знания итд, элементы равенства людей.



\paragraph{Аристотель} Аристотель ученик Платона, учился в Академии, был одним из первых учеников, надеялся, что Платон оставит для него школу, но Платон выбрал другого. Аристотель обиделся, уехал, вернулся, и создал свою школу -- Лицей. У Платона школа существовала школа 500 лет, а у Аристотеля закончилось все с его смертью. Сам Аристотель не был чистокровным Афинянином и не получил гражданство, он был сыном врача Александра Македонского. Сам Аристотель стал наставником Александра Македонского, который называл Аристотеля вторым отцом. Аристотеля называют инцеклопедистом, он охватил гораздо больше сфер предметной реальности (биология, политология, итд). Все эти мегаданные Аристотель собрал благодаря Саше, который имел ученых исследователей покоренных земель. Аристотель разработал новую модель исследований в отличие от Платона. У Платона наверху диалектика. А у Аристотеля это скорее психология, чем философия и называет топика?? Говорит, как человек приходит к чему-то новому трудно понять до конца, и диалектика слишком абстрактна, чтобы считаться наукой. Аристотель предложил аналитику. Научный метод -- метод аналитический. Согласно этому методу ученый должен строить свою работу, которая должна быть не только отысканием чего-то, но и экспликацией в научном сообществе. Ученый должен выразить в правильном виде. Модель экспликации научных результатов предлагает Аристотель. \\

\hfill \textbf{Mar 30}

Аристотель (384-322 гг до н э). Благодаря Македонскому, Аристотель создал громадную энциклопедию животного мира. Тем не менее то, что мы знает об Аристотеле -- дело случая. Аристотель умер в изгнании, потому что империя Македонского развалилась. Все труды Аристотеля оказались в Александрийской библиотеке и через 2 века архивариус Андроник нашел их. После этого уже они произвели впечатление. Многих философов мы знаем только из трудов Аристотеля. Он излагал сначала чужие идеи, а потом свои. Аристотель первый начал писать, как профессор, как профессионал. Даже у Платона все было написано все в диалогах, где люди просто общались и потом расходились. А Аристотель создал модель изложения научных материалов. У Платона диалектика, как метод достижения нового. Но новое сложно уловить и непонятно, есть ли оно вообще. Люди просто высказывают точки зрения и сближают их иногда. А Аристотель называет это топикой. И в противоположность ей создает аналитику. Философская идея состояла в том, что мы трудным путем приходим к новому, и даже сейчас нельзя сказать как это происходит, как появляются идеи. Аристотель говорит, что этого не знаем, и надо исходить из другого. Если что-то случилось и мы претендуем на место нового. Как в этом убедить других? Это метод изложения научного результата таким способом, чтобы он стал результатом научного общества. Формальные требования, предложенные им, 
\\\t 1. Четко сформулированное дело.
\\\t 2. Что было сформулировано до вас, показать, что вы знаете, что уже было сделано. 
\\\t 3. Какой новый материал привлекается для изучения
\\\t 4. Что является действительно новым, выделить. 
\\\t 5. Вопрос о том, насколько полно решена проблема. Закрыта ли она или есть возможность дальше работать. 

Второе важнейшее достижение, которое и сейчас актуально. Систематизация формальных законов логики. Чтобы люди понимали друг друга, они должны соблюдать эти правила. В научном тексте нужно руководствоваться ими, чтобы ученые и другие вас понимали. Разумеется, Аристотель не изобрел эти законы, но он их эксплицировал и возвел в форму теории. Причем Аристотель панлогист, не просто считает, что главное учение -- теоретическое, а не чувственное, но еще и считает логику есть отражение логики самого мира. Рациональность присутствует в самом мире. Мир устроен рационально. Логика возможно лишь в силу того, что мир устроен рационально. На базе этой логики и может устраиваться наука. Законы логики:
\\\t Закон тождества. А тождественно А. Смысл понятия не меняется. Нельзя подменять понятия. 
\\\t Закон недопустимости противоречия. А и не-А не могут существовать в одном и том же контексте. А и не-А взаимно исключают друг друга. Аристотель правильно различает противоречия на Контранные противоречия и Контрадикторные. Контранные -- кажущиеся. Для человека и рыбы. Белый и черный. Контрадикторные -- настоящие. Белый и небелый. 
\\\t Закон исключенного третьего. Всегда верно либо А, либо не-А. Третьего быть не может. 
\\\t Закон достаточного основания (сформулирован Лейбницом в 18? веке). В плане теоретического: где считается разумным оборвать цепь доказательства. Полностью ли доказано. 

На основе этих законов логики Аристотель формулирует систему силлогизмов. (если хочешь найди сам). Имеется в виду -- о многих вещах, если строится правильный силлогизм, можно не проверять результат эксперимента. Например, все люди смертны, сократ человек -- сократ смертен.

Третье важнейшее достижение. Классификация наук. Он разделил все науки на теоретические, творческие и практические. В теоретические: метафизика (первую философию, которая имеет дело с вечными сущностями, лежащими в основе всего и образующими наше мышление. Аристотель перывй формулирует категории -- 10), математика (тоже имеет дело с вечными, но есть материальный подтекст, а метафизика полностью идеальна), физика (полностью эмпирическая наука). Творческие науки: поэзия, музыка, пластическое искусство. То, что требует вдохновения, а не расчетов. Практические: экономика, этика, мораль, умение вести хозяйственную деятельность (относится к экономике). 

Первая философия -- метафизика. Приставка мета означает после. Могут быть двоякие суждения -- то ли физика важнее, а метафизика должна идти после, то ли наоборот. Сейчас под метафизикой подразумевают самые глубинные и не измеримые экспериментально темы. Принципиальное отличие Аристотеля от Платона. Аристотелю принадлежит фраза: ''Платон мне друг, но истина дороже``. В плане образного различия посмотреть картину Рафаэля Афинская школа. Рафаэль изобразил всех знаменитостей древней Греции, а в центре под аркой две фигуры: пожилая и с длинной бородой и другая помоложе. Это Аристотель и Платон. Платон указывает вверх, а Аристотель вниз. У Платона главное -- мир идей, а Аристотель считает, что наш мир и есть единственно реальный. Аристотель считает, что в основе нашего мира лежат 4 метафизические причины. 
\\\t 1. Формальность. Каждая вещь имеет свою форму. С точки зрения Платона эта форма задается объектом из мира идей. А Аристотель считает, что эти формы находятся в самих вещах. Именно из-за этой принадлежности мир является дискретным. Идеальные формы не реализуются, но близкие к ним осуществляются в мире. Идеальные формы находятся в реальных вещах. 
\\\t 2. Причина движения. Все движется чем-то. Никакого принципа реакции нет, потому что иначе мы бы пришли к первому двигателю. И он будет иметь идеальную форму, которая задает первоначальный импульс материальному миру. Идеальное движет материальное. 
\\\t 3. Энтелихиальная причина. Целесообразная, телеологическая. Подразумевается, что мир устроен целесообразно. Все имеет предназначение. Если нам что-то непонятно, это не значит, что этого не судествует. Целесообразность рассматривается не в индивидуальном смысле, а в родовом. С каждым отдельным яйцом может случиться что угодно. Но цель у всех яиц -- вывести птенца. Душа человека всегда целесообразна. Устремлена к познанию, хотя каждый отдельный человек может быть и не очень хорошим. 

Все эти причины основываются на идеальном, они все сходятся к одному образу. Бог -- форма всех форм, движитель, цель всех целей. Все три линии соединяются в этой точке. Аристотель признает высшее идеальное начало, но это философский идеальный Бог, и нет ничего антропоморфного. Без этого идеального существа не можем ничего объяснить. 
\\\t 4. Бог творит мир не из ничего, а из материи. Материальная причина, которая сосуществует с Богом. Аристотель называл материя-прима. Философская сущность, с одной стороны она природная, но до нее нельзя докопаться. Это нечто материальное, но постигаемое только умом. Из него получаются другие осязаемые материи. Между материальным и формальным осуществляется диалектическая связь. Например, статуя может рассиматриваться и материально (из чего сделана), и формально (что имеется в виду итд). 

Аристотель создает первую научную полноценную картину мира. Связаны все основные компоненты, начиная от космоса и заканчивая ... Природа боится пустоты, пустоты нет. Не как у Демокрита. Аристотель сказал, что все потенциально. Попытался разрешить парадоксы Зедана (нельзя разорвать пространство и время, пространство не существует без тела). Аристотель создал то, что называют качественной физикой. Разделил все мироздание на две части. Подлунный и надлунный. В надлунном все идеально (круговые всякие орбиты итд), а в нашем мире все неидеально. Поэтому само экспериментирование вредная процедура. Вмешиваясь, мы искажаем природу. Надо лишь созерцать. Только в 17 веке появилось четкое понимание эксперимента, как постановка объекта в искусственные условия. 

В области этики Аристотель вводит новации. Аристотель не соглашался с Сократом, что люди делают плохо только когда не понимают, что это плохо. Люди у него выступают как медиаторы крайностей. У Аристотеля мужество между трусость и безрассудство. Все добродетели это золотые серидины. Аристотель ярый сторонник рабства. Рабу полезно быть рабом, они от природы рабы. И даже говорил Македонскому порабощать всех покоренных в рабов. В понимании государства у Платона есть теоретическая попытка, а Аристотель идет от практики. Государство есть естественное усложнение структур человеческого общества, начиная от семей и племенных отношений до городов и союзов городов. Государство притягивает все эти первичные образования. На выходе получается из синтеза этих структур, но как идея лишь притягивает. Правильность государства определяется не теоретическими принципами, а практической пользой. Какова цель правителей. Он разделяет все формы власти на правильные и неправильные. К правильным относит аристократию, монархию и политию. Неправильный тирания, олигархия и демократия (власть толпы). Монархия -- я отец, мой народ мои дети, я должен заботиться, а они меня уважать. Правильность находится в этической области. Зависит от того, на что направлена власть. В области экономики Аристотель противостоит Платону и считает, что в основе должна лежать в частная собственность. Само существование ее вдохновляет человека. И это движет экономику. 

Размышления об искусстве. Искусство это конечно творческая стадия развития человеческого духа, но не высокая, ибо это просто подражание чему-то. Он первый открывает эмоциональную силу искусства и связывает это с трагедией. В момент восприятия люди переживают пик сопереживания -- катарсис. Люди этим самым очищается душой и может продолжать дело умерших. Это сила воспитательного искусства. 

Последний период греческой философии -- эленистический период. После Платона и Аристотеля и вплоть для заката и разгона (529 год н.э.). 338 год греции Македонским. Распад империи на 4 влиятельных царства. Нельзя сказать, что это абсолютный упадок Греции, но тем не менее в философском плане происходит сильный спад. Самые яркие -- школа Эпикура, Стоиков, Скептиков, ........

Школа Эпикура -- сады Эпикура. Период, связанный с генезисом Римской империи. Эпикур продолжает линию Демокрита -- мир это атомы и пустота, но вносит новшество, что атомы это мельчайшие частицы вещества, двигаются под влиянием силы тяжести (впервые понятия силы тяготения) и с равными скоростями. И самая главная новация -- атомы могут отклюняться от траекторий -- клинамен. Вносит идею вероятности в идею мироздания. Идя от свободы воли человека. В космологии развивает идею бесконечности мира и существования многих миров. И даже помещяет между мирами богов. Это есть этика эвдеманизма. Два понятия: Гедонизм, Эвдеманизм (смысл жизни -- тостижения счастья). Но Эпикур не имел в виду распутство и не делал акцент на физеологических потребностях. Он выступал за здоровый образ жизни. Все приносит удовольствия в нужных количествах. Чтобы быть счастливым, надо побороть свои страхи. Например перед катаклизмами природными. Ими нельзя управлять все равно. Страхи перед болезнями, эпидемиями, старостью. И самый страшный страх -- перед смертью. Из-за этого люди начинают верить в Богов и бессмертность души. Он считает, что ничего после смерти нет. После смерти все атомы распадаются итд. Поэтому не надо особо и беспокоиться об этом. У природы нет плохой погоды. Богам нет смысла молиться, им нет дела до мирских страхов. А против смерти он изобретает бессмертный силлогизм. Смерти люди не должны бояться потому что когда мы есть ее не будет, а когда она пришла, нас не будет. Чтобы прожить жизнь счастливо, надо избавиться от бед и избегать социальных катаклизмов. Счастливо прожил тот, кто хорошо спрятался. Не надо лезть никуда, просто живи в согласии с природой.\\

\hfill \textbf{Apr 6}

Аристотель продолжает линию Платона в плане понимания истины. Все стремятся к истине, но что есть истина? Аристотель акцентирует внимание на этом, считая, что познание начинается с чувственных ощущений, которые имплантируются в наше сознание, как от горчего на воске. Но Аристотель панлогист (законы логики могут рассматриваться как законы природы), поэтому он говорит, что чувственное конечно есть, но в нем нет мудрости. Мудрость начинается с аналитики. Истина понимается в рационалитическом плане. Это корреспондентская теория истины. Суждение ``снег бел'' истино тогда и только тогда, когда снег действительно бел. Корреспондировать (соотнести) содержание суждения с реальным положением. Играет роль и чувственное, и наш разум. 

Есть 4 теории истины. Корреспондентная -- самая распрастраненная в естественнонаучных направлениях. Вторая -- когерентная -- прежде всего работает в математике и различных теориях логики. Когерентность это внутренняя непротиворечивость. Третья, идущая от Пуанкаре -- непонятно как называется -- говорил, что истина это умение организовать саму теорию, значит, все должны согласиться. Есть факты, но их надо теоретически организовать. На одном и том же эмпирическом материале можно построить несколько теорий, но надо выбрать только одну. Пуанкаре говорил, что факты это книги в библиотеке, а теории это наши катологи. Результат зависит от того, что нам удобно, от конвенции. Четвертая, самая простая, используется в технонауке -- прагматическая. Истина то, что работает на практике. 

Аристотель был основателем теории истины. 

Школа стоиков протестует против Эпукуриистов. В основном по вопросам этики. У стоиков этика ригаризма: повиновение человека чему-то, долг превыше всего. Ты появился в мире не для счастья, а для исполнения долга. Надо жить энергично, не прятаться, чтобы исполнять долг, но не сетовать на судьбу, если не все получается. Хорошо, конечно, быть здоровым и богатым, но если не повезло, надо принимать, как само собой разумеющееся. Этика пассивного георизма. Свобода -- это познанная необходимость. Все, что человек может, это познать необходимость и смириться. Свобода превращается во внутреннюю. Свобода заключается в том, что человек осознает, что это судьба. Например, собака, привязанная к движущейся колеснице. Ее свобода определяется длиной поводка и скоростью колесницы. Если все благоприятно (очень длинный поводок и медленная колесница), собаке будет казаться, что она свободна. Судьбы ведут тех, кто хочет, а тех, кто не хочет, тащит за уши. 

\paragraph{Стоики}Философия стоиков (названо по месте Стоя). Школа существовала с 3 века до н э до 3 века н э. Различают ранюю стою (), среднюю (поэций посидоний), позднюю (синека (воспитатель Нирона), эпиктет (бывший раб, вольноотпущенный), император Марк Аврелий). Марк говорил, что надо умереть так же спокойно, как созревшая олива падает с дерева. Стоя была самой демократичной школой. К их этическому учению примыкает посыл космополитизма: человек в первую очередь субъект космоса, а не государства итд. Поэтому рабства идейно не существует. Рабом может быть сам человек по своему убеждению. Стоики отрицают атомы и пустоту. Они считают, что мир это некая сплошность, заполненность. В мире существует единство материального и идеального. Это получило название ``пантеизм'' -- одушевленность космоса, космос дышит. Где-то они составляют единое -- пневмо -- одновременно и материя, и логос. Нематериальное нечто. Это пневмо пронизывает все и вся. Синека говорил, что пневмо пронизывает мир, как мед соты. Когда человек умирает, его душа вливается в пневмо. Мир развивается циклично, волнами. За точку отсчета берется год Гераклита, но у Гераклита мир сгорает полностью, и в новом мире могут появиться новые законы итд, а у стоиков возникает новый виток того же самого. Петлевая причинность (причина становится следствием, а следствие причиной). И все повторяется именно безошибочно. Отсюда вытекает, что если пневмо есть синтез материи и духа, то это одновременно и бог, и логос, и материя, и судьба. Отсюда этическая концепция, что невозможна свобода воли как таковая, а как познанная необходимость. Из этих метафизических представлений они развили структуру логики: ввели риторику и нарратив. Попытались объединить идеи Платона и Аристотеля. Риторика Платона, как искусство построения вопроса, а нарратив это аналитика, точное изложение чего-то. Дискуссии и изложения результатов исследований. Весь блок наук -- логика, этика и физика. Логика как скорлупа яйца. Этика это белок в яйце. Физика это желток. Каталиптическое восприятие. Обычно наше восприятие это как толчок к размышлениям, попытка проводить анализ и приходить к заключениям. А каталиптическое восприятие это схватывание сути вещей немедленно, без посредников. Сначала инкубаторский период мысли, а потом прозрение. Как байка про таблицу Менделеева во сне. 

Отношение к проблеме зла в мире. Если бог, природа, судьба, логосы и так далее это одно и то же, и мир движется по этим законам, то откуда в мире зло? Попытки оправдать зло при наличии веры в бога впервые появилось у стоиков. Чисто с позиции разума и логики пытались решить проблему. Классифицировали аргументы: логический -- зло оттеняет добро, если бы не было зла, мы не знали бы, что такое добро. Космический -- мир такой большой и так сложно устроен, что зло является вмонтированным элементом, отдельный элемент огромного космического механизма, но без этого элемента не все работает. Гноссеологический -- люди ограничены в своем знании и понимании, то, что им может казаться злом, в конечном счете может оказаться добром. Этический -- зло дается для проверки устойчивости человеческой психики, или наказание за неподчинение судьбе. Стоики подробно разработали первую теорию психологических аффектов. Страх -- предчувствие зла. Ревность, любовь итд все описывали. 

В христианство, например, вошло подчинение судьбе: ``пути господни не исповедимы'', ``человек предполагает, а бог располагает''. 

\paragraph{Скептики} Основатель Пиронист. Школу оценивают очень по-разному. Рассел говорил, что это идеология лентяев, а Маркс говорил, что это первые ученые среди философов. Пирон, Имаксимон, Агрипа -- самые большие имена. Скептицизм может быть разным, у этих скептиков он граничил с агностицизмом, разрушительный. У Декарта например можно подвергнуть сомнению все, но надо найти дно и отталкиваться от него. Но критикуя всех и вся, они пришли к выводу, что на словах нельзя ничего доказать, а все академики занимались построением теорий, которые не имели реального основания. Скептики дошли до того, что каждому суждению можно поставить противоположное, и вообще всякая вещь может и не быть этой вещью. И главная задача философа не выдвинуть свое, а критиковать оппонента. И вообще лучше помолчать. Для критики они выдвинули систему тропов (логических оборотов). Различали тропы от судящего (субъекта) и судимого (объекта). Наиболее значимые тропы от судящего: человек это один из возможных представителей живых организмов, у него своя сенсорика, мозг, телосложение итд. Нет причин думать, что он самое совершенное существо и именно его познание природы правильно. Сами люди очень различны и амплитуда сенсорики каждого настолько различается, что сколько людей, столько и мнений. Люди пытаются постичь мир с помощью искуственных образований, которых в природе нет (человеческий язык). Более того, люди занимаются математикой, а ее тоже нет. Субъект никогда не занимается чистым познанием, всегда над ним находятся социальные установки -- аксиологические вещи и корысть за занятием познания. Никакой объект не является константой, он всегда находится в пространственной точке, в определенном интервале времени, в определенных условиях и так далее, а уследить за всеми течениями непонятно как. Восьмой троп -- относительности. Мы не можем познавать никакой объект сам по себе, мы его познаем только относительно каких-то других и так далее. Критика скептиков была рациональна, ибо показывала несодержательное познание мира. 

Эклектика изначально считалась хорошей. Они собирали все лучшее из всех областей. 

\paragraph{Неоплатоники}Последняя наша школа неоплатоников. Плотин -- как платон ничего не писал, но за ним ходил ученик. Прокл -- последний значимый в античности философ (4 век н э). Жесткая конкуренция этой школы с идеологией христианства. Для христиан все предыдущие философы -- язычники, а для философов христиане просто необразованные люди. Само название неоплатонизм говорит, что они должны что-то брать от Платона, но и вносить что-то новое свое. За основу берется основная мысль Платона о том, что наиболее важной структурой мироздания является единое. Единое это все и ничто. Единое дивергирует на две части -- единое-трансцендентное и единое-трансцендентальное. Второе они сравнивают с Солнцем. Солнце это свет, тепло, жизнь, постоянное излучение итд. Вводят термин эмонация. Солнце эмонирует. Первая ступень эмонации из единого -- появление ума -- нус. В этом уме и заключаются все идеи. Ум есть и мыслимое, и мыслящее. После ума идет новая структура -- душа (у Платона тоже была). Мир анимируется, появляется время. И процесс эмонации замыкается на материи. Материя -- синоним костности, мрака и зла. Но раз материя существует, то должна быть идея материи. Интиллегибельная материя находится в уме. Космос получается живой -- вытянутая в длину жизнь. В материи их больше всего интересует человек, который состоит из материи (темноге) и духа (светлое). Человек может не только познавать структуру вселенной, но и может преодолевать материальную природу себя. Тело есть темница души. Можно совершить обратный путь и воспарить к единому. Обратный процесс получил название экстатическое расхождение -- от слова экстаз. Все это очень напоминает Буддизм. Плотин достиг единого аж 4 раза. Но не все в этом мире поддается рациональному объяснению, есть элементы мистики. 


\section{Философия средних веков}

\paragraph{Христианство}Христианство одна из мировых религий. Появляется на рубеже двух веков. Существует в самых разнообразных формах. Крупные ветви -- католичество, православие и протестантизм. Происхождение христианства -- одна из сект иудаизма. По иудаизму были книги -- тора, талмут, талх, ветхий завет (идеология в том, что существует богоизбранный народ -- евреи, из-за их изгнаний и вообще такого всего, то должен прийти миссия и всех спасти). Из этого посыла появилась идея, что миссия уже приходил -- Иисус, но его не узнали и он пострадал. Но после распятия он воскрес и этим доказал, что он бог. Так, христиане признали в качестве теоретического обоснования ветхий завет, но создали новый корпус трудов -- новый завет. Наибольшее важная часть -- евангелие Марка, Луки, Иоана и Матфея. А также апокалипсис. В евангелие описывается жизнь Христа, а в апокалипсисе описывается конец мира, время остановится, наступит страшный суд, будут судить всех живших и живущих. Грешников отправят в ад на вечные физические муки, а хороших отправят в рай, и они будут счастливы всегда. Бог ветхого завета страшный и грозный. А в новом завете бог есть добро, свет, истина, любовь. Такая дуальность создает громадное поле для интерпретации. Все может пригодиться для объяснения ситуаций, в которых обитает человек. Христианство создало абсолютно новую систему ценностей. Раньше боги вели себя как люди, а на земле существовала иерархия. А в христианстве провозгласили равенство вообще всех перед богом. С этим не могли согласиться главы постов. Христианство создало баланс страха перед адом и надежды на счастье в раю. Из-за равенства того христиане были ганимыми, с ними очень жестого обращались. Их бросали к львам и распинали. 

Постепенно, в силу демократичности христианство привлекало больше людей, и в 324 году император Константин уравнял все религии (несмотря на то, что Нирон устраивал показательные экзикуции христиан), а Константинополь освещали вообще все религии. А 380х Феодосий возвел христианство в государственную религию. Христиане стали громить всех в месть. Было два лагеря, одни не принимали язычников, другие говорили, что у них можно научиться. \\


\hfill \textbf{Apr 13}

Теоцентризм -- любовь к мудрости через Бога. Философия служанка религии, несет факел, но все же это просто служанка. Главное -- тео, бог и служители его. В христианстве душа уникальна, бог дает ее только один раз, нет реинкарнации, и все души изначально равны. В итоге возник вопрос противоположности христианских взглядов и предыдущих. Тентуртиан -- говорили, что все после евангелие неинтересно. Аригент -- учитывали остальные взгляды. Идеи христианского народа иногда считались первичными к греческим, но в итоге рационализация взяла верх. В греческой культуре творение мира происходило всегда на основе чего-то, а в христианстве сначала непонятно, а потом утвердилось (Августин блаженный), что Бог сотворил мир изничего. Материя прима считалась потенциальностью. Движение у Аристотеля образуется из перводвигателя, а у христиан Бог придал и его. Как и время. В ветхом завете бог управляет всем, ни один волос не упадет с головы человека без воли божьей, нет свободы воли. В новом завете есть воля. Далее возникает патристика -- формирование основных догматов христианства, это происходит с 0 до 6 века до нашей эры. Никто с неба не спустил, все обсуждалось отцами церкви. А с 6 века, когда четко на различных соборах сказали ``только так и никак иначе'' начинается схаластика. Схаласты уже комментировали догматы христианские, разъясняли их публике. Основные догматы: бог един в трех лицах, непорочное зачатье, богочеловек, богородица родила бога в образе человека, потом бога вообще распяли, как так, а потом воскрешение бога. В период становления обсуждались многие вещи -- как воспринимать что бог един в трех лицах? Две линии: Афанасьевцы и Арьевцы. Афанасий говорил, что все три эпостасьи бог-отец, сын и святой дух единосущны. А Арий говорил, что единоподобны. В одних соборах победили Афанасьевцы, в других Арьевцы. Например, у Ньютона были проблемы из-за того, что он был арьевцем. Потом арьевцы стали более многочислены, потом снова афанасьевцы и так далее. В конце победили все же афанасьевцы. Также обсуждалось рисование икон, как можно изобразить Бога? Далее обсуждалось монашество. Угодно ли богу такое пожертвование человека? 

К шестому веку все эти разговоры затухли, но появилось новое противоречие, которое выросло в антогонизм. На месте римской разрушенной империи возникли много варварских государств. После падения рима христианство осталось, потому что варвары были христианизированы. Рим брали только по политическим и экономическим мотивам. Римский Папа остался. А в Константинополе появился Патриарх. Возникло католичество и православие. Филиокве -- самое большое приткновение. Католики начали считать, что бог-святой-дух происходит от бога-отца и бога-сына. А православие осталось на позиции, что только от бога-отца. Папа римский отлучил Константинопольского патриарха от церкви, а патриарх проклял Папу. И дальше все вылилось в крестовые походы и прочие религиозные войны. В 1каком-то году все противоречия были забыти. Если раньше считалось, что каждая ветвь является богоизбранной, то сейчас говорят, что путей к богу много. Экуминическое движение -- стремление прийти к единству. Про афанасьевцев и арьевцев. В единоподобии и единосущности одно различие в букве йот. Отсюда выражение не уступить ни на йоту. 

В начале в том числе было много евангелие. И много выяснялось, какая правильная. Встречались многие противоречия в описанных действиях Христа. Один схаласт пытался разрешить противоречия. В книге ``да и нет'' собрал все противоречия. Разумеется, у него не получилось решить всех их. И в результате он плохо кончил, ибо не надо было примирять вообще. 

Совсем недавно появилось еще одно Евангелие -- от Фомы. Фома был самым рациональным учеником, он требовал все время доказательства. В этой книге рассказывались все события с точки зрения Иуды. В историю он вошел, как образец предательства. А после евангелие от Фомы выяснилось, что не Иуда виноват, а Христос сам его попросил. Ибо Христос хотел доказать всем, что он бог. А повесился Иуда не из-за мук совести, а из-за того, что его правда стали считать предателем Христа. 

Проблема соотношения рационализма и номенализма. Спорить о догмах нельзя, комментировать их можно, но поле комментирование такое большое, что все схаласты разделились на номеналистов и реалистов. Реалисты посчитали, что надо идти от слов из одного евангелие -- в начале было слово, это слово было у Бога и это слово было Бог. В оригинале у греков это было Логос. Это слово первично по отношению ко всем вещам. Получается, сначала рацио -- понятие, а потом все материальное. Например, сначала понятие лошадности, а потом уже сами лошади. Из этого посыла КТО-ТО выдвигает первые рациональные доказательства существования бога. Он идет от понятия -- самые отпетые атеисты и язычники на вопрос как вы понимаете понятие бог ответят это совершенное существо. А вместе с описанием совершенного существа должен существовать атрибут существования. А если нет этого атрибута, то мы можем помыслить что-то еще более совершенное. Значит, предыдущее уже не совершенное. Значит, бог есть. Реализм -- стремление из понятий выводить реальные вещи. Номеналисты говорили, что понятие это сотрясение воздуха, но надо описывать материальные объекты. Концептуализм -- средняя линия. Все зависило от того, как воспринимать само существование. Можно понимать это как существование в голове или как материальное. Коротко можно охарактеризовать еще одно философа -- умеренний рационалист. Фома Аквинский. 

\paragraph{Фома Аквинский}Его учения в 1879 году Папа возвел в каноническое учение, долгое время все образовательные институты принимали именно его. Он отрицательно отнесся к главному доказательству существования бога: таким лобовым способом мы не можем доказать существование. Нужные доказательства -- косвенные, но системные. Например, он брал аргументы у Аристотеля и интерпретировал. Все в мире движется -- существует двигатель. Все в мире причинно -- должна быть первая причина. Все есть необходимое или случайное -- должна быть первая необходимость. В мире существуют разные степени совершенства -- должна быть высшая степень. И наконец целесообразность. Кто и что управляет миром? Кто-то должен это делать. 

Арио политики -- две мысли с описанием бога. Катафатика -- описание бога через суммирование и интегрирование тех определений, которые даны в библии. И другой путь -- бога вообще нельзя определить языком. Бог это таинственный мрак, бог везде и нигде. Вторая мысль -- иерархическое устройство мира. Бог в лице троицы наверху, а дальше вокруг него крутятся другие сущности: сирафимы, херуфимы, крестовы, архангелы, ангелы. Отсюда и иерархия на земле -- епископы, священнослужители, дьяконы, миряне. В животном мире тоже. Лев царь, а остальные поменьше.

Говоря о Фоме Аквинском снова. Система доказательств косвенная, но когда она собрана в один кулак, сложно противопоставить что-либо. Другая его важная идея -- как понимать материю-приму? Он возвращаеся к идее Аристотеля -- чистая потенциальность существует, но чтобы из нее что-то выцепить, нужна воля бога. Также он выделил что можно рационализировать, а что необходимо принимать на веру. Бог не может изменить прошлое -- не может нарушать созданное им. Бог не может даже изменить сумму трех углов треугольника. Такую рациональность можно познавать. 

Францисканцы и доминиканцы -- два больших ордена, они дискутировали по философии. Например, может ли Бог создать камень, который не сможет сам поднять, если он всемогущий. 

Другие Номеналисты -- Думскот, Роджер Бекон. Думскот полемизирует с Фомой и выдвигает тезис, что материя актуальна. Бог создал ее, но дальше она существует актуально, все, что происходит, происходит и по воле бога, и по материальному плану. Также он выдвигает идею об интонациональности нашего сознания. Даже когда мы стараемся ни о чем не думать, все равно о чем-то думаем. Направленность проявляется в творчестве людей. Бекон вводит важную идею о необходимости математики для науки. Вся математика была только на востоке в то время, и вообще восток опережал европу в плане теоретических знаний. Там были построены храмы мудрости, обсервотория, были заложены основы алгебры, основана медицины, позиционное исчисление (числа, которые мы называем арабскими). Коран можно читать двояко. Образно, видить не буквальный смысл, а интерпретировать. На западе эта идея приняла название двух истин. Одна прямая, а другая еще одна. И эта еще одна победила в итоге, несмотря на протесты церкви. Галилей создал свою теорию двух книг: две книги бог написал: ?? и книгу сотворения, которая была написана на языке математики. 

Схаласт Уильям Оккам -- последний важный для нас. Высказал мысль о необходимости ограничивать число понятий, которые объясняют окружающий мир. Не надо умножать сущности без необходимости. Этот методологический прием назвали бритва Оккама. В 19 веке этот методологический прием воспроизведет Эрнст Мах. Будет интерпретировать это, как принцип экономии мышления. И на основе этого критиковал Ньютона. 

\paragraph{Августин Блаженный}Последняя фигура, интересующая нас, Августин Блаженный. Сам он по молодости увлекался неоплатонизмом. Потом был манихеем и потом уже стал наиболее праведным христианином. Его даже звали молотом для язычников. Он высказал важные мысли, которые потом обсуждались и продолжают обсуждаться -- в антологии, гносеологии. Антология -- главное отличие в том, что в христианстве Бог добровольно, силой своей мысли создает мир изничего. Дальше в антологии -- понимание времени. Релятивизирует время. Он расщепил понимание времени на два важных направления. В отношении субъекта время всегда относительно. Жизнь это миг между прошлым и будущим. Человек всегда имеет дело с настоящим. Но настоящее не субстанция. Время не субстанциально. (В отличие от Ньютона). Августин говорит, что прошлое это наша память. Говорит, что это настоящее в прошлом. Мы вспоминаем здесь и сейчас. И будущее это настоящее будущего. Мы в настоящем предполагаем, что что-то будет. В реальности нет ни прошлого, ни настоящего, ни будущего. Есть только настоящее настоящего, настоящее прошлого, настоящее будущего. Августин расщепил время. С одной стороны время фикция, но с другой стороны, когда он обращается к истории, говорит, что время циклично и сравнивает с собакой, которая старается поймать свой хвост. Августин говорит о стреле времени. Бог запустил стрелу. Но когда-нибудь стрела упадет, и будет страшный суд. Время имеет начало и время имеет конец. А тем, кто спрашивает, что было до того, как бог сотворил мир, надо отвечать, что никакого до не было. 

В гносеологии важные мысли: принцип методологического сомнения, который потом будет развивать Декарт. Отрицательно относится к скептиком, у них разрушительный порыв. Августин говорит, что сам факт, что ты сомневаешься, говорит, что есть хотя бы одна истина. Вторая мысль -- отношение к математике. Бог устроил мир числом, мерой и весом. Раз это так, математика имеет априорный характер. 
\\

\hfill \textbf{Apr 20}

В социокультурном плане выдвинул идею двух градов: земной и божий. Он был свидетелем захвата Рима. Он говорил, что Рим это заслужил, и вообще главное в этом мире не град земной, а град божий. Церковь должна быть превыше всего, цари должны быть одобрены церковью и так далее. Важная аксиологическая идея вины человека перед богов на основе притчи о первородном грехе (яблоко с древа познания съели, бог изгнал людей, лишил бессмертия, вручил муки женщинам при родах). Вопрос -- есть ли у человека сила воли вообще? Августин говорил, что бог знал, что Адам съест яблоко, но не мог не дать Адаму свободу воли. Августин в этом смысле вступил в полемику с Пилаге, который был представителем манихейства. Там существует принцип равномощности добра и зла. Они всегда борются, но не может никто победить, не может быть вечное ни зло, ни добро. Августин совершенно другую теорию предложил: зло не субстанциально, добро субстанциально. Зло это отсутствие добра. Как свет и тьма. Второе важное положение -- первородный грех наследуется. Пилаге отрицал это, говорил, что человек свободен в выборе добра и зла. А Августин говорил, что что бы человек ни делал, оно всегда отягощано злом. А добро делается только по воле божьей. И никто не знает до конца, кто является богоизбранным. Но из-за этого встал очень важный вопрос -- зачем нужна церковь? Если все думали, что она нужна, чтобы отмолить грехи, а теперь говорится, что в рай попадают только богоизбранные. В конце концов церковь поправила Августина и сказала, что благодать божья может снизойти на человека только в церкви. Церковь увеличивает шанс стать избранным. Но, например, когда пришли религиозные войны, все церкви раздельные вернулись к этому вопросу. И многие пришли к выводу -- церковь бесполезна, веруй и может быть спасешься. И поэтому они упрощали церковь, убрали все богатое и прочее. Но католичество прошло по другому пути. И вообще они в конце провозгласили Папу обладателем благодати -- святым на земле. У него столько благодати, что он может даровать ее другим. Отсюда пришла идея индульгенции. 

В заключение по средневековью: в позднем средневековье сложилась определенная философия по образованию. Если в древнем мире школы имели очаговый характер, то тут появилась более обширная идея. Образовывались школы при монастырях -- в два этапа тривиум (грамматика, диалектика, риторика) и квадривиум (арифметика, геометрия, астрономия, музыка). Боэций -- последний римлянин -- предложил эту идею. В 11 веке появляются первые университеты, потом они уже быстро появляются. Внутри каждого складывается система образования. Основные факультеты: факультет искусств, медицинский, теологический, и юридический. Последние три были выше, но первый необходим. Потом этот факультет искусств стал называется философским. Там разрешались дискуссии и вольномыслие, но в определенных пределах. Изначально все университеты были при церквях, епископы назначали преподавателей и так далее. Но главное -- эти очаги образовательные были защищены от обскурантизма. Со студентами не разрешалось плохо обращаться. Структура университетов тех стала считаться стандартной. Ректор, деканы, факультеты, профессора, доценты, экзамены, защиты и т.д. 

\section{Эпоха возрождения}

Эпоха с 14 по 16 век. происходит в европе, несмотря на короткий временной период оказывается принципиальным для переворота, который потом реализуется в появлении науки. До этого времени европа сильно отставала от востока, а за эти два века возвысилась весьма. Антропоцентризм заменил теоцентризм. Еще в 13 веке Папа пишет о презрении к миру и низости человека. А в 14 веке пишут о величии человека. Вместо стремления наставить человека на путь истинный за богом появляется идея обожествления человека. Вместо выражения ``помни о смерти'' появляется ``помни о жизни''. 
Появляются идеи объективизма, секуляризма, гуманизма. Метущаяся душа, требующая действия, принцип агона воспылал, но уже не на уровне соревнований, а на уровне нации. Европейцы устраивали экспансии, плавали через океан. В архитектуре появляются шпили, устремленные ввысь. В живописи появляется перспектива, да Винчи использует это, и правило золотого сечения тоже. В музыке появляется нотная грамота -- рационализация. 7 нот, из которых можно создать неограниченное число произведений. Светская литература появляется, например, круги ада. Меняется отношение к женщинам, новая форма поэзии -- стихи к прекрасной даме. Появляется гуманизм. Обращение к душе человека. Секуляризация -- отделение светского от религиозного. 

Причины появления -- экономические. Переход от ремесленничества (продукт делается полностью одним человеком) к мануфактурному производству (разделение труда). Из-за повышения производительности появляется продукт, который может быть использован людьми, которые не производят продукт. И появляется слой людей, которые занимаются умственным творческим трудом -- интеллигенция. Далее кросс-культурные события: войны на перенейском полуострове, получение доступа к сокровищам из Александрийской библиотеки. 1453 год -- падение Константинополя. Который был хранителем древней культуры. И вообще вся интеллигенция побежала в европу -- ближе всего Италия. Первая академия появилась там. Сначала принялись переводить все тексты с греческого на латынь. Их вклад настолько велик, что, по сути, все, что мы знаем о древности, мы получили из этих переводов. В техническом плане европейцы наследовали компас с востока, но там его не использовали, а европейцы начали. Далее порох, который в европе перевернул всю военную стратегию. Рассчитывали полет и так далее. И важнейшее -- книгопечатание. Первой напечатали библию, потом начали уже другое. Из-за книгопечатание сильно снизилась церковная цензура, поэтому мысль перестала быть скованной. Книгопечатания -- третья информационная революция. Первая речь, вторая письменность, а третья печать.

Последняя причина -- реформация. Реформация происходит в католичестве и связана с интерпретацией идей Августина. Вылилось в разделение на лютеранство с Лютером, кальвизинм с Кальвином и англиканство, когда король рассорился с Папой из-за отсутствия развода. Это вылилось в религиозные войны. Но в конце никто никого до конца не уничтожил, север европы  остался за протестантством, а юг вернулся к католичеству. Появилась святая инквизиция, сжигали еретиков. Потом появляются более серьезные идеи -- Мишель Монтель. Возрождение скептицизма, спрашивал ``а что я вообще знаю?'', критиковал официальную католическую доктрину, призывал проведение действительных опытов и экспериментов. Эразм Роттердамский -- полемика по поводу свободы воли, раб ли божий человек. Главное его произведение -- похвала глупости. Высмеивает власть предержащих, говорит, что вся власть полна дураков, обычный человек не должен думать, что власть умнее его и относиться соответствующим образом. Никола Кузанский и Джордано Бруно. 

\paragraph{Никола Кузанский}Никола Кузанский сам епископ, друг детства Папы. Основное произведение об ученом незнании опирается на сократа, который говорил, что знает, что ничего не знает, но другие не знают и этого. Мысли Николы следующие. Дуалистическая концепция мироздания: потнеистическую и механистическую. Называет вселенную мировой машиной -- машина-мунди. Кстати, в европе появляются механические часы, и время преобретает другой фактор. С другой стороны он понтеист и считает космос божественным. Сохраняет католическую дактрину, но говорит, что это шар с бесконечным радиусом. Значит центра нет, особенного места нет, все изотропно, бог везде и нигде, его нельзя описать. Даже святую троицу он изображает, как треугольник с тремя прямыми углами, устремленный к бесконечности. Истина -- многоугольник вписанный в круг. Сколько не вписывай, из многоугольника круг не получится. Но мы приближаемся. Развивает библейскую идею, что бог при мироздании пользовался геометрией, арифметикой и музыкой. Значит математика от бога. У него даже есть глава о могуществе математики при познании природы. И вообще раз вселенная такая, то миров может быть бесконечно много. Но с одной оговоркой -- наш мир наилучший, бог именно здесь поместил человека. Все эти идеи оказались плодотворными для появления науки. Гуманизм привел к появлению науки. 

\paragraph{Джордано Бруно}Джордано Бруно -- беглый монах, воспитывался в религиозном ордене, но сбежал и выбрал жизнь странника, оратора, возмущал всех речами, его считали еретиком. Долго ловили и в конце церковь сделала ловушку и сожгла его. Бруно высказывал почти то же, что и Никола, но у него не было покровителя. Бруно первый уравнял физику неба и физику Земли, а также поверил в теорию Коперника, дополнив картиной бесконечного открытого космоса. Бруно говорил, что все это единое пространство. В понтеистическом плане сказал, что монада есть единица, которая и духовная, и материальная. Монада можно представить как атом. Тот же Эрнст Мах будет говорить о нейтральных элементах, которые обладают психофизическими свойствами. И из-за них нет смысла искать первичное среди духовного и материального. У Бруно была не только идея о бесконечном космосе, но и все миры одинаковые. Доказать ничего не мог, но при допросах у инквизиции говорил, что недостойно было бы божественного могущества создать только один мир. Вдруг он неудачный? Надо создать несколько. Раз бог всемогущий. И из-за этого идея, что человек есть центр мироздания, крушилась. У Бруно в целом натуралистический подход. 

К тому же, именно в эпоху возрождения работает Николай Коперник -- появляется первая система гелиоцентризма, как в противоположность Птолемею. 

\paragraph{Утопии}
К социальному пласту философии относится еще: поскольку были войны и прочая дичь, многие задумывались о строении идеального государства. Появляется новый тип утопий. От них уже отталкиваются Маркс и Энгельс с коммунизмом. Пионером является Томас Мор с его книгой ``Утопия''. Там правитель Утоп на острове Нигдея и случайный мореплаватель, попавший туда, описывает увиденное. И утопия, и антиутопия исходят из существования человека и его приспособления к миру культуры и всего другого. Как легче выжить и преуспеть: сообща или в одиночку? Общая или частная собственность лучше? Как и противостояние Платона и Аристотеля. Европейская культура идет в направлении частной собственности. Это потом называют меркантилизмом. Это первые шаги к обществу массового пртребления. Далее произведение Томаза Кампанелы ``Дети солнца''. Коммунистические порядки, управляет всем метафизик, есть министерство мудрости, военной мощи, любви. Везде ограничен физичекий труд, упразднены деньги. Равенство в распределении продуктов, доступ к образованию. В Германии была первая попытка воплощения общего равенства -- восстание крестьян. Самая примечательная утопия Френсис Бэкон -- Новая Атлантида. В этой новой Атлантиде в основе дом Соломона. В нем показывается прообраз современных академий наук. В первых утопиях акцент делается на производительном труде, а у Бекона делается на роли науки. Прежде всего идеи высказываются учеными, а затем доводятся до ума. Поэтому ученым приоритет, памятники при жизни и прочее. Идеи научных достижений разные -- подводные аппараты, передача звука на расстояния, вывод растений (генетика). Бекон является переходной фигурой от возрождения к новому времени -- наукоцентризму. Все эти утопии говорят, что для счастливой жизни нужно объединяться, это коммунистические утопии. 

Также Бэкон привел к наличию академий. В европе эти академии организовывались сами, а в россии не было кадров, ученых приглашали за большие деньги. 

Этим утопиям противостояла антиутопия Макиавелли, который пошел от Аристотеля с частной собственности. 
\\

\hfill \textbf{Apr 27}

\paragraph{Макиавелли} Никола Макиавелли с книгой "Государь"\ сразу же попал в запрещенный церковью список, но это, разумеется, не помешало читать ее. Считается, что он положил начало юридическому мировоззрению. Отделил его от церкви и придал статус эмпирической науки. Книга была инструкцией к дальнейшему управлению. А философская идея была прямо противоположна утопиям. Он считал, что именно частная собственность двигает прогресс -- "человек скорее переживет смерть своего отца, чем потерю частной собственности." И частная собственность собственно сама и появлялась в Европе, Макиавелли исходил из просто увиденного. Он был Итальянец, страна была в очень плохом положении, и он задумывался, как Рим мог прийти к такому? Он пришел к выводу, что виновата церковь. И в связи с этим выдвинул очень важный принцип -- отделение церкви от государства. Государство это не церковь, она не должна вмешиваться во все экономические дела. Это, разумеется, не понравилось церкви, даже появился термин макиавеллизм. Сейчас это подразумевается под мнением, что государство должно использовать церковь как хочет. Также государство должно быть сильным, правитель тоже, нужно объединять нацию и не брезговать никакими средствами. Власть должна быть лицемерной, и ни в коем случае не связывать себя моральными и религиозными догмами. Сегодня можно говорить одно, а завтра другое. Государь должен быть хитрым как лисица и свирепым как лев. Власть находится вне морали, договоры это просто бумажки, если они перестали отвечать интересам, то их не стоит никак учитывать. Этому потом следовал Наполеон. Под макиавеллизмом понимается сейчас и это тоже. Это положено в основу государства из книги, и также произведение сейчас изучают везде. 

Итог: Европа больше тысячи лет находилась в средневековом застое, темные века, мрачный застой. И вдруг за два века сделала гигантский скачок, все появившиеся ценности легли в фундамент для появления новой формы культуры -- науки. Читайте далее: 17 век -- наукоцентризм. 

\section{XVII -- XVIII века}

Наукоцентризм по определению означает главенство науки. С точки зрения практики она еще ничего не давала, но понимание перспективности этого дела позволило появиться специфическим отрядом интеллигенции -- ученым. Это прежде всего основатели науки -- Галилей, Ньютон, ... Они все разумеется были в том числе философами. Плоды наука начинает приносить только со второй половины 19 века -- появляется цикл Карно. А дальше электродинамика -- первая настоящая теория. А потом из теории уже плоды. Но потребовалось два с лишним века, чтобы наука начала приносить реальную пользу. 

В это время формируется два направления философии -- они с противоположных сторон исследуют вопрос о фундаменте познания. Искали то, на что наука может опереться. Они -- рационализм и эмпиризм. К эмпиризму привел Френсис Бэкон -- он разработал метод научной индукции. Его установка, что опыт и эксперимент -- фундамент познания. Наука там, где мы можем что-то доказать напрямую. Наиболее ярко эмпирический подход выразил Джон Локк с тезисом -- нет ничего в разуме, что до этого не было бы в чувстве. Ничего на веру. 

В то же время появляется противоположное направление с тоже выдающимися учеными -- Рене Декарт, Готфрид Лейбниц. Опыт необходим, но он только помогает доходить до истин, которые находятся в глубине нашего сознания или же на небесах, куда можно подключиться. Это называют априоризмом и трансцендентализмом. Это наша проблема, что мы считаем каким-то образом, но когда мы приходим к какой-то истине, она же от нас не зависит. Лейбниц в ответ на тезис Локка добавил -- "кроме самого разума". В разуме изначально что-то есть. 

В целом философская проблематика стягивается к гносеологии. Наука познает, а как познает это философия. 

Общественно-политические теории. Мы рассмотрим три, связанные с появлением государства. Теории общественного договора. 

\paragraph{Гоббс}
Основатель Томас Гоббс -- написал "Левиафан". Это библейский персонаж -- гигантское морское чудовище, которое пожирает даже своих детей. Этот образ он использовал для понимания структуры государства. По поводу этого произведения было много дискуссий, многие говорили, что спорить нечего, ибо всякая власть от бога, бог дал Адаму, а потом это транслировалось через поколения. Гоббс, несмотря на религиозность, хотел найти естественный путь появления. Такого еще не было до него. Во всех теориях были противоречия. 

Гоббс ввел естественное состояние и гражданское. Естественное -- состояние до государства. Люди жили по принципу тому же, по которому живут животные. Отсюда -- человек человеку волк. Войны за еду и так далее были на истребление. Это были войны всех против всех. Их нельзя было остановить, если не задуматься, к чему это может привести. А могло это привести к уничтожению всех особей, ибо люди готовы биться до конца (не как животные). А когда люди осознали это, они решили заключить договор. Конвенция. Пришли к выводу, что естественное существование придет к смерти, и надо избавиться от этого. Вводится власть, относительно которой осуществляется все: и обвинения, и наказания, и вообще все. Во главе должен стоять самый главный суверен, который отчуждается раз и навсегда. Дальше эта власть передается по наследству. Сам монарх оказывается вне закона, он издает их и осуществляет надзор, но сам он вне подозрений. Аргументация была такая: сам Гоббс наблюдал разные события в англии -- смену династий, и он со всеми властями общался, и видел, что к добру это не приведет. Он сторонник сильной власти, которая остается всегда в кулаке. В таком случае порядка будет больше, потому что меньше возможности коррупции, которая разъедает гос механизм. Гоббс считал, что если власть в кулаке, то монарх может жестоко обходиться со всеми пидорасами. А его не подкупишь, ибо ему ничего не надо, у него уже все государство в собственности. Монарх должен быть отцом государства, а все граждане его дети. 

\paragraph{Джон Локк}Джон Локк разошелся во мнении и создал другую теорию. Государственная власть должна основываться на положениях: избрание этой власти, сменяемость этой власти, принцип подчинения меньшинства большинству, легальное существование оппозиции. Абсолютная власть развращает абсолют. Всякие короли могут отвернуться от народа, а король заботится только о себе. Государство понимается не как договоры, а как шляпа на голове, но не голова. Эти принципы получили название принципы либерализма и легли в основу всех современных устоев. Власть дробится на законодательную, исполнительную и конституционную. И если между частями возникают конфликты, то обращаются к конституционному суду, который и ставит точку. Например, америка вся живет по этим законам и признает это и благодарна Локку. 

\paragraph{Жан-Жак Руссо}Третья модель -- Жан Жака Руссо. Это эпоха просвещения. Центр событий в конце 18 века перемещается во францию, там куча ученых, прежде всего математиков. В общественно-политической жизни это связано в великой французской революцией. Лозунги: равенство, свобода, братство. Идея была построить царство разума на земле, с опорой на науку. Для этого надо было расшатать старый государственный строй. Жан Мелье -- священник, который разочаровался во всем, перестал в бога верить, религию считал средством управления, король погряз в роскоши, надо их всех разморжить. Он был идейным началом революции. Говорил о несовместимости Ньютона и Декарта. Первый человек, который написал всемирную историю, говорил о всемирно историческом прогрессе. Есть флуктуации, а история в итоге одна. Должна быть наука, демократия, культура. 

Дуалисты говорили, что бог существует, но дает только первый толчок, а дальше оставляет мир. Потом появилась группа философов, которые провозгласили себя атеистами. Они выдвинули кучу плодотворных научных идей, некоторые даже приезжали в россию и пытались вразумить Екатерину отменить рабство. Главное достижение -- они издали первую энциклопедию. 28 томов. Она объединила все научные знания на то время. Она разошлась по всей европе и в том числе в россию, она привела к подъему образовательной культуры. Стали считать, что образование есть государственное дело, а не частное. Демократия начинается с образования. Государство должно создать условия для массового, а лучше общего образования. Эта идея была революционной. И вторая идея -- полная секуляризация. Или преподается теория Дарвина, или вы лох. Особняком стоит Жан Жак Руссо в том смысле, что он первый усомнился в тезисе, что можно построить царство разума на земле с помощью науки. Люди были бы еще хуже, если бы рождались учеными. Наука не обязательная и скорее всего является проявлением плохого. Появляются сциентизм и антисциентизм. Антисциентисты считают, что наука не связана с прогрессом, а может привести к тупику. Это направление впервые появляется у Жан Жака Руссо. Он за эту работу даже получил премию. Второй его важный вклад -- еще одна попытка создать идеальную модель государственного строя. В отличие от английских философов, он задумался о неравенстве людей и появлении частной собственности. Он считал, что в естественном состоянии люди были счастливы. Люди производили столько, сколько им надо было для проживания. Остальное время посвящали досугу. Появлялись хорошие игры и прочее. А потом произошел казус. Кто-то взял, оградил участок и сказал "это мое". А остальные ему поверили и пошли по такому же пути. И появляется эгоизм и стремление умножать эту собственность. Сначала за счет увеличения производительности труда, а потом захвата чужих территорий. Железо и зерно испортили мир. И дальше понеслось и поехало. Переход от естественного состояния к частной собственности во-первых случайный, а во-вторых именно там начало войны и осознания необходимости власти, люди отчуждают свои права в пользу власти. Он считает, что власть с самого начала отвернулась от народа и служила богатым. В конце концов чаша терпения не выдерживает, власть свергают. А дальше нужен новый договор. Оказывается два договора -- второй на основе печального опыта проб и ошибок. Власть в таком случае сметают полностью. Революция должна быть беспощадной. Новая власть должна быть постоянно под отчетом, и в любой момент народ может смести власть. Либо собранием (прямым способом), либо референдумом (косвенно). Идея подотчетности власти и сменяемость, а также суверенность самого народа самые важные. Метафизическая часть в том, что избрание новой власти есть показатель единения народа на основе всеобщего боя. Необходим результат. Он становится всеобщей волью. А после этого не стоит, и даже вредно протестовать, поскольку это уже воля всего народа. А осуществлять эту волю должен глава, вождь, фюрер. В этом рассуждении есть опасность из-за узурпации власти. 

\section{Немецкая классическая философия}

Это конец 18 века и 19 век. Все философы -- немцы. Основатель Кант, дальше Фихте, Шеллинг и Гегель. И единственный материалист Людвиг Фербахт. Все это профессиональные философы, являются профессорами университетов, создают свои системы. Они связаны с теорией познания, а также в них разворачивается широко диалектика (например оч широко у Гегеля), идея априоризма и трансцендентализма тоже. Подробно остановимся на философии Канта. 

\paragraph{Кант}
После критики теории познания английскими философами скептиками, которые возражали против Ньютона: "сила"\ только у бога, математика лишь описывает отношения между телами, а где же основания науки. Кант захотел обосновать. Он написал "Критику чистого разума". Обсуждает как возможна математика, как наука, физика и метафизика. Математика существует из-за априорного и трансцендентального. А метафизика загадывает вперед, поэтому наукой не является, но является бойскаутом для науки. Второе произведение "Критика практического разума". Как человеку жить в этом мире? Что является главным? Есть такая вещь, как категорический императив. Две вещи в мире поражают воображение -- звездное небо над ним и моральный закон под ним. Все это априорно, на основе неба можем построить науку, а на основе морали общество. Третья книга "Критика способности суждения". На что мы можем надеяться? Кант говорит, что не только на бога, но и на себя, а еще на общественные институты. Поэтому война неизбежна, а вечный мир возможен, но если и если и если. Критика возникает при попытках разрешить неразрешимые противоречия -- антиномии. Кант говорит, что антиномию можно сформулировать: мир самодостаточен и природа тоже и они могут существовать сами, или же существует внешняя сила? По сути -- есть бог или нет его? Можно ли доказать существование бога? Приходит к выводу, что все доказательства сводятся к антологическому доказательству, с которым он не согласен. Он попытается выделить бога из морали. 

Доказательство из понятия, что бог совершенное существо, Кант назвал антологическим. Доказательство Фомы Аквинского и прочих товарищей он приравнивает к антологическому. Ибо все сводится. 
\\

\hfill \textbf{May 4}

Кант не согласен с антологическому доказательству, потому что сто таллин у меня в кармане и сто таллин у меня в голове это абсолютно разные вещи. Поэтому он хочет вывести бога из морали. Априорные и трансцендентальные -- синонимы. Мораль и порядок в небе такие. Моральный закон называют категорическим императивом и говорит о его априорности. Императив -- закон. Условные императивы -- императивы повседневной жизни. Если хочешь быть хорошим физиком, изучай философию. Все понятно, есть цель, есть средства, надо напрягаться и следовать этому алгоритму, чтобы достичь. Таких условных императивов много: мой руки перед едой, например. А вот категорический императив уже нечто другое. Цель не прагматика, которая находится во вне, а сам человек -- цель. Формулировка -- поступай так, чтобы максима (цель) твоей воли могла стать принципом всеобщего законодательства. И тогда будет идеальное общество. Из-за такой абстрактной формулировки многие называют это бесполезным. Это чем-то похоже на Конфуция, но все же другое. Кант различает виды поступков человека: легальные и моральные. Легальные это те, которые вписываются в условные императивы. А моральные это поступки, где нет никакой для человека выгоды, но он не должен поступать иначе. Жесткий ригаризм. У стоиков была подобная вещь относительно судьбы. 

Пример. Вы идете мимо водоема и видите тонущего человека. Вы бросаетесь и спасаете его. Нельзя сразу сказать, легальный или моральный это поступок. Надо проанализировать мотивы. Если есть закон, который говорит, что вы должны оказывать помощь человеку в экстренной ситуации, а если не делаете, будет наказание. Вы можете увидеть, что тонущий человек -- ваш родственник или друг. Даже если тонущий человек ваш враг, а вы возьмете его в плен или получите медаль. А моральным будет поступок, если вы спонтанно сразу бросаетесь и видите в тонущем только просто человека, не помышляя ни о каких наградах и прочем, может быть вы даже сами плохо плаваете, но все равно пытаетесь. И тогда вы подчиняетесь категорическому императиву. 

Канта подвергли критике за абстракцию и отсутствие жизненности. Получалось, что если вы любите своих близких, то все поступки относительно них не будут моральными. И в конце жизни Кант сказал, что моральные поступки надо совершать не с холодным сердцем. И еще ввел казуистические поступки -- в них нельзя сказать, надо ли следовать категорическим императивом. Например, эвтаназия или вопрос сообщать ли смертельно больному о его проблеме. Кант понимает это и говорит, что есть сложные случаи, когда трудно следовать императиву, но надо стремиться. Если вам нечего есть, вы можете украсть хлеб, но вы все равно должны признавать, что вы мор. Но любой вор знает, что он вор и поступает плохо.

Соответственно, из этой морали Кант выводит бога. Вор знает, что он вор, но все равно занимается этим. Его может покарать общество, а может и нет. А чтобы следить за исполнением категорического императива нужен уже бог, он ничего не пропускает. Кант так переворачивает последовательность бога и морали в христианстве. И с этим церковь не могла согласиться. 

Также Кант пришел к выводу о возможности вечного мира. Это было время перед Наполеоном и прочим, а Кант пришел к этому. И мир можно достичь заключением союзов и коалиций. Нападение на одно государство считается нападением на все сразу. По этому принципу существует НАТО, существовал Варшавский договор и так далее. 

Шеллинга и Фихте пропускаем, ибо многое связано с наукой и проблемами обоснования науки. 

\paragraph{Гегель}
Поговорим о Гегеле. Это самая грандиозная система объективного идеализма. Его даже называют абсолютным идеализмом. Три великих системы: Платон с миром идей, Лейбниц с духовными сущностями и Гегель. Гегель в основе всего и вся кладет понятие абсолютного разума, мирового разума. Есть нечто абсолютно идеальное, оно первично и оно развивается. Гегель таким образом представляет еще и диалектику. Сейчас под диалектикой понимают именно Гегеля. Он вводит в оборот более ста категорий. Они лежат в фундаменте всей культуры и естествознания. Абсолютная идея, как точка отсчета, а дальше она развивается в системе цикла: тезис, антитезис, синтез. В философии идеи эта триада фундаментальна. Тезис -- идея развивается сама в себе в сфере чистой логики. Основное произведение Гегеля "логика"{}, Там проходит чистое бытие как тезис, ничто как антитезис, синтез дает категорию становление. После развития идеи антитезис -- инобытие. Природа не первична, а есть отчуждение этой идеи. Природа это то, где дух порезвился. Антитезис природа. Синтез идея возвращается к самой себе через сферу культуры. Человеческое есть возвращение этой идеи через тот цикл. Более конкретные механизмы этого Гегель назвал законами диалектики. Он хорошо знает законы логики и формальной логики, но, в отличие от Канта, Гегель называет формальную логику логикой рассудка. Люди, чтобы общаться, должны следовать этому. Это собственно и есть мысль Аристотеля. Гегель говорит о высшей логике -- логике разума. Это и есть диалектическая логика. Есть школьная арифметика -- четыре действия, без которых прожить нельзя. А есть высшая математика, без которой многие живут, но она все же дает много. Но математика понятная вещь, а диалектика не очень. Более того, диалектика в обычной жизни может запутать. 

Законы диалектики: 
\\\t 1. Закон единства и взаимодействия противоположностей. Все в мире имеет противоположность. Пифагорийцы уже это подмечали. Наиболее наглядно проявляется в животном мире: хищник и жертва. Антиподы, но одно без другого не может быть вообще. Гегель говорит, что мир в обществе лишь передышка между войнами. Как ветер необходим, чтобы озеро не превратилось в болото, война нужна, чтобы общество не сгнило. Иначе не будет развития. 
\\\t 2. Закон перевода количественных изменений в качественные. И может быть обратный. Идея в том, что развитие постепенное эволюционное происходит до определенного момента, а после этого следует скачок, а после этого переходит в качество. Примеры нагревание воды. Нагреваем постепенно, а потом бац и пар. И с охлаждением так же. 
\\\t 3. Закон отрицания отрицания. Его можно иллюстрировать на примере таблицы Менделеева. Или же вы сеете семечко в почву, из него появляется стебель, стебель таким образом отрицает семечко и разрушает его, а потом через некоторое время появляются новые зерна, которые отрицают стебель. Новые зерна изоморфны первым, но они другие. Развитие по спирали. Тезис -- антитезис -- синтез. Третий закон по Гегелю есть универсальный механизм развития, который демонстрирует сложность механизма развития. Гегель первый уловил, что развитие носит такой спиралевидный, но в то же время прогрессивный характер. Не проходит уже присказка, что все новое есть хорошо забытое старое. 

Эти законы Гегель считает максимально абстрактными, их можно интерпретировать на всем подряд, а осознать уже сложнее. Например, нельзя не зная формальную логику овладеть диалектикой, но диалектика не сводится к формальной. 

Посмотрим, как Гегель использует эту методологию по отношению к культуре. Когда дух возвращается к самому себе через культуру. Коротко можно сказать, что он различает субъективный дух (психология, антропология и феноменология -- то, что человек ощущает в себе), объективный дух (который уже существует в различных формах общественного сознания -- право, мораль, нравственность). Право фиксируется в системах различных законов, но в конечном счете это духовные отношения, и еще важно, что поддерживаются государством. Государство обладает силой и является носителем права. Гегель считает государство особой духовной благодатью и еще, что государство тогда государство, когда оно показывает силу и мощь. Насчет морали Гегель не признает категорических императивов и говорит, что мораль есть набор чувств, которые испытывает человек при жизни в обществе. Человек может знать, что морально, а что нет, но может не соблюдать. Никаких сдерживающих скреп, как у Канта, у Гегеля нет. Нравственность у Гегеля уже есть корпоративная вещь. И она выше морали. Например, если ты ученый, то плагиат и подтасовка результатов никак не сочетаются с этим. Так, нравственность носит корпоративный характер и обязывает к подчинению. И на уровне семьи, и производственных коллективов. Третья форма развития духа -- абсолютный дух. Религия, искусство и философия. Все эти формы есть тезис, антитезис и синтез, но в каждом есть еще разделы. Например, искусство бывает классическое древнегреческое, романтическое. Религия это политеизм, монотеистические религии, а высшая форма христианство. Но и эта форма уступает философии. Философия самая высшая форма культуры. Если раньше считали, что философия отрицает саму себя (как говорили скептики, например), а Гегель говорит, что философия принципиально плюралистична. На философию надо смотреть не как на отдельные фрагменты, а как на большое количество камушков, которые все необходимы для формирования картины. Философия -- мышление в предельно абстрактных категориях и эпоха, выраженная в мысли -- квинтесенция эпохи. Гегель понимает, что он велик и в его системе пришло осознание плюралистичности философии. Гегель не закрывает философию на своей системе, пришло лишь осознание того факта, а развитие вообще не завершено. Кроме того, он считает, что философия это царица наук. Она в основе всего. Гегель не выделяет науку, как форму общественного сознания, она лишь конкретизация философии. 

Дальше Гегель переходит к философии истории. Как понимать с точки зрения философии всю историю? Периоды война мир, мир война неизбежны, это толчок к развитию. Есть эволюционные типы развития и революционные. Например, великая Французская революция, которая резко изменилась. В развитии истории Гегель выделяет три периода. Деспотический период -- преобладающей формой правления является тирания. Там, где свободен один, не свободен никто. Деспот только думает, что он свободен, а сам оказывает несвободен. Рабовладельческий строй, где свободны некоторые. И строй, который Гегель называет образцом прусской демократии, где правление осуществляется на уровне законов, которые он считает поступью божественной идеи на земле. Дальше народы -- исторические народы (крупные известные цивилизации Египет, Китай), неисторические народы (славяне, но выделяет Россию, ибо видит потенциал), внеисторические (африка). Внутри каждого народа есть своя иерархия движущих сил. Великие личности и обычные люди. Великие личности -- те, после которых мир становится другим, и обратного движения не может быть. Например, Цезарь, Ганнибал. Их жизнь коротка и напряжена. Абсолютный дух ими движет. Тут Гегель приходит к выводу, что история развивается вроде бы по воле людей, но на выходе получается всегда не то, что человек собирался начать. Равнодействующая мирового развития складывается по параллелограмму сил. За всем стоит мировой разум. Гегель называет это хитростью мирового разума. Мировой разум не заставляет никого, но на выходе получается его воля. Глобальный вывод в плане истории -- все действительное разумно, все разумное действительно. На наличие плохого в мире Гегель отвечает, что существование и действительность разные вещи. Действительность это существование с необходимостью. 


\def\fierbah{Фейербах}

\paragraph{\fierbah}
Последний из классической немецкой философии -- Людвиг \fierbah. Называют заблудшим кем-то, ибо он единственный материалист. Наиболее важное произведение "сущность христианства". \fierbah\ говорит, что чувство играет важную роль в познании мира, во дворцах думают совсем иначе, чем в хижинах. Главное его достижение -- первая серьезная и подробная критика христианства. Тайна любой теологии заключается в антропологии. Смысл в том, что человек всегда состоит из разных сторон и положительной, и отрицательной, но всегда хочет надеяться на что-то, но он  отчуждает от себя все хорошие качества и возводит их в абсолют. И получается всеблагий, вседобрый, всемогущий и т.д. Поэтому религии необходима возможность создания абстракций. Но на деле это комбинация качеств человека. А потом человек начинает этому поклоняться. \fierbah\ говорит, что все развитие истории есть развитие религии. Значит, люди без религии не могут, ибо смертны. Если мы разоблачили христианство, как религию, то надо создать новую. \fierbah\ пытается создать религию от человека. Не бог есть любовь, а любовь есть бог. Человек человеку бог. \fierbah\ понимает, что не так просто достичь этого идеала. Он говорит, что человек всегда печется о своем благополучии, и поэтому должен полюбить себя. Эгоизм не всегда плох, а часто помогает выжить и развиваться. Дальше любовь к близким. И самое трудное, идеал, любовь к человеку вообще. Если каждый будет следовать этому, то мы получим счастливое общество. 

\paragraph{Марксизм}Материализм \fierbah а лег в основу марксистской концепции с одной важной поправкой. \fierbah\ критикует Гегеля и диалектику и считает это выдуманным, а Маркс говорит, что \fierbah\ вместе с грязной водой выбросил ребенка. Маркс прямо пишет, что у Гегеля диалектика стоит на голове, но ее надо поставить на ноги, найти в самой природе. Это принципиально отличает Марксизм и от Гегеля, и от \fierbah а. В целом скажем, что марксизм очень сложная конструкция. Части: политическая экономия, философия, теория научного коммунизма. Политическая экономия с грандиозным произведением "Капитал". Там Маркс показывает сущность капиталистической эксплуатации: вводится понятие прибавочной стоимости. Рабочий, работая на капиталиста, увеличивает стоимость и дает прибавочную вещь капиталисту. Вторая часть -- философия. Основывается на \fierbah а и Гегеля. Третья часть -- теория научного коммунизма или социализма. Существует всемирно исторический процесс, в котором все народы придут к высшей формации, коммунизму. 

Философия делится на диалектический материализм (поиск диалектики в природе, занимался Энгельс), исторический материализм (занимался Маркс). Материальный мир первичен, сознание вторично. Общественное бытие первично, общественное сознание вторично. Прежде чем заниматься искусством и прочим, людям нужно пить, есть, одеваться, умываться и так далее. Форма, по которой создаются эти материальные вещи, определяют общественное сознание. Производительные силы (орудие труда (приспособление), средство труда (материалы), сами люди) и производственные отношения (производственно-технологические отношения (как делать продукт), производственно-экономические отношения (по поводу собственности)). Рабовладельческие отношения, феодальные, капиталистические, социалистические (средства принадлежат всему обществу). Теперь динамика -- происходит по законам диалектики. Производительные силы постоянно эволюционируют, ибо это выгодно, а производственные отношения изменяются скачками. Закон перехода количественных отношений в качественные. Скачок совершается через революцию. Первый закон указывает на источник развития -- борьба классов. Второй закон показывает механизм развития. Третий закон показывает, что развитие происходит по спирали, там, где существовала общественная собственность, потом появилась частная собственность, обратно потом вернется общественная собственность, но уже другая. 

Обобщая, необходимо сказать, что это относится к категории общественно-экономической формации. Там находится базис и плюс появляется надстройка. 
\\

\hfill \textbf{May 11}

Последняя лекция. 
Зачеты с 23 по 29 мая. Аудитория 5-27, начало 15.20 во все дни, кроме 28 числа -- тогда аудитория 5-26 тоже в 15.20. 

23 группы 01, 03;
24 -- 02, 04, 05;
25 -- 06, 07, 08, 11;
27 -- 13, 14, 35;
28 -- 40, 42;
29 -- 41, 43;

В связи с тем, что пропадает одно занятие, на зачете спрашивается только пройденное. И еще можно написать два эссе по оставшимся темам -- по марксизму (в историческом варианте, исторический материализм) и иррационализм. Но попытка уже только одна. Дедлайн 21 число 12.00. 

\section{Иррациональная философия}
Середина и вторая половина 19 века. Марксизм это еще развитие рационального способа мышления, четкие установки, все можно понять, описать, а дальше эта линия превращается в еще более связанную непосредственно с наукой. Получилась линия позитивизма -- наука стоит во главе культуры, а философия обслуживает науку. Как в средневековье считали, что философия служанка религии, так и тут. Философию считали лишь систематизацией науки. Этот позитивизм существует и сейчас вторая волна связана с Махом, третья волна в 20 веке. Но в это же время появляется противоположная линия -- иррационализм. Там проповедуют антисцеитистские установки. Наука для духовной жизни человека не имеет значения, "что меняется в этике человечества, если оно начинает верить в Коперника, а не в Птолемея"{}, "как ученый может познать свой духовный мир". Ницше, Шопенгауэр, Освальд Шпенгер, Анри Бергсон, Зигмунд Фрейд. 

К самому понятию иррационализм. Это философское направление -- критический подход, мышление в абстрактных понятиях. Не надо думать, что иррационализм это мистика и прочая чушь, это тоже философские рассуждения, не такие систематические, как в рационализме, но вполне можно уловить основные мысли и привести в систему. Иррационализмом называют потому, что во главу ушла ставят не рацио, а совсем другие части сознания и духа. У Кьеркегора это страх. У Шопенгауэра это воля к жизни, все производно от этого. У Ницше воля к власти, власть движет всем и вся в мире, начиная с ада животного мира, заканчивая за власть в человеческом обществе. У Шпенглера это ансамбль мировых душ, каждая из которых ответственна за появление и развитие определенной культуры. Где есть душа, появляется нечто новое и оригинальное. Но в конце все равно все затухает и умирает, но появляется нечто новое. У Берксона это абстрактный порыв. У Фрейда сексуальное начало. 

\paragraph{Серен Кьеркегор} Сёрен Кьекегор -- датский философ, прожил трагическую короткую жизнь, при жизни признан не был, но потом опубликовали его 28 томов и увидели, что это оригинальная мысль. Он считается одним из основоположников экзистенциализма. Он исходит из того, что в основе существования -- экзистенции -- лежат некоторые константы, а главная из них -- константа страха. Он дает классификацию страхов: тревога, боязнь темноты, ..., ужас, отчаяние, тошнота. Главным является страх перед смертью. Каждый понимает, что он смертен, и этот страх определяет всю линию поведения человека. В то же время страх это с одной стороны показатель духовности человека, нормальные люди действительно должны бояться многих вещей, это уходит и в биологию (инстинкт самосохранения), и с другой стороны понимание страха это условие человеческой свободы. Свобода состоит не в познанной необходимости, а понимается, как некая константа человеческого сознания, основанная на страхе перед небытием. Мироздание это жесткая причинно-следственная связь, а свобода это некая дырка в мироздание. Свобода не имеет ничего общего с материальным миром, она первична по отношению к мирозданию. Человек духовно свободен от всего. Сначала надо испугаться, а потом страх преодолеть. У человека есть выбор всегда, или паниковать, или бороться. Но не испугавшись и не осознав, что ты боишься, нельзя это преодолеть. Человек обретает реальный выбор, когда понимает, что он смертен.

В этой связи Кьеркегор анализирует три состояния существования. Первая стадия -- испугавшись смерти, человек может повести себя по принципу гедонизма. Надо взять от жизни все, что хочешь. Пример -- Дон Жуан. Пускаясь в удовольствия, человек забывает о смерти. Это низшее состояние человеческого духа. Вторая стадия -- этическая. Примером является Сократ, который всю жизнь посвятил тому, чтобы образовывать людей, чтобы они поступали хорошо. Для этого и работает культура, пишутся книги, создаются скульптуры и прочее. Чтобы увековечить добро и обессмертить себя. Работай для всех, чтобы побороть бренность бытия. Но в этой стадии приходит осознание, что и такое не бессмертно. Третья стадия -- религиозная. Наиболее прочная и вечная, человек находит опору в боге, и бог обещает вечную жизнь. Религиозная вера это единственная опора и высшая стадия. Но эта вера носит парадоксальный характер. Бога описывают и делают понятным, иманентным. А в ветхом завете бог трансцендентен, и Кьеркегор с этим согласен. Когда многодетная мать считает, что лучше умереть, чем жить, она все равно не может это сделать. И бог ей воздаст после смерти. Наиболее противоречивую вещь философ берет из ветхого завета -- авраам и девка. Они хотели детей, но получился ребенок только в старости. И после этого к Аврааму приходит кто-то во сне, говорит, что он ангел и посланник божий, и говорит, чтобы Авраам убил на скале в жертву богу своего сына. И все вроде бы выстраивается так, чтобы Авраам усомнился в боге. Но Авраам поднимается на скалу и заносит нож, а потом там оказывается агнец, его убивают, и все хорошо. Вера запрещает сомнения и рассуждение. Любое рациональное рассуждение ведет к неверию в этой истории. Кьеркегор приходит к выводу, что Авраам -- рыцарь веры. И вообще вера иррациональна. Или веришь, или рассуждаешь. Критикуя диалектику Гегеля, он предлагает качественную диалектику, где нет никакого синтеза: или или. 

Другие стадии рыцарства тоже есть, имеют определенный смысл, но не сравнимы с этим высшим статусом рыцаря. Это жертвоприношение во имя общего, когда все всем понятно. Во многих культурах приносили жертвы, жертва трагический герой, но порождает катарсис, жертвуют одним, чтобы существовал род, племя, и не может быть попятного действия. Все идет по алгоритму, трагедия, но высокая трагедия, ибо все понимают, что так необходимо. Другой вид героизма -- романтический. На поверхности все тоже красиво, но бесполезно. Яркий пример -- Дон Кихот. Он совершает подвиги ради женщины. А рыцарь веры наподобие Авраама -- уникальная вещь. Человек оказывается один на один с богом. Человек выбирает экзистенциально. 

\paragraph{Шопенгауэр}Артур Шопенгауэр -- философ, который выдвинул идею, что основным двигателем развития является некая воля. Его ключевое произведение -- "Мир, как воля и представление". Идея в том, что можно рассуждать об эволюции, но когда в конечном счете встает вопрос о движущей силе самой эволюции и хочется идти дальше, надо находить отправную точку. Ее выбрали абстрактной волей, мировой волей. В природе магнетизм, притяжение и отталкивание, электричество, гальванизм, борьба за существование в животном мире и в человеческом обществе. Все это он называет волей к жизни. Когда раскрывается первый глаз, мир обретает свое существование. А почему нельзя без глаза? Воля заставляет первому глазу открыться и животному осознать, что мир существует, а дальше эволюция и борьба за существование в животном мире. Шопенгауэр говорит, что в обществе мировая воля распределяется на субъективную. Борьба за ресурсы определяется мировой волей. Печальный вывод Шопенгауэра -- все к худшему в этом худшем из миров. Мир это не дело рук бога, а скорее дьявола. Шопенгауэра называют провозвестником всех катаклизмов с природой и войнами в 20 веке. Все становится только хуже. 

\paragraph{Ницше} Фридрих Ницше (1844--1900) -- долгое время перед смертью был помещен в дом умалишенных, хотя и там продолжал писать. Уже в 20 лет ему присвоили звание профессора, оставил яркие произведения, например "Так говорил Заратустра"{}, "Воля к власти". Последнюю использовали фашисты для практического использования идей. У Шопенгауэра воля есть некая отрицательная сущность, которая ведет к коллапсу. Ницше говорил, что, читая Шопенгауэра, чувствовал, что это написано прям для него. Читая Достоевского "Преступление и наказание"{}, рассматривал вопрос -- такой же я, как все, или могу перешагнуть через труп, или через трупы. Как Наполеон и прочие. У Ницше воля к власти -- положительная вещь. У Ницше не просто воля к жизни, а прям к власти, и положительная потому что в результате борьбы между людьми за власть может родиться новый человек, которого называют белокурой бестией, и этот человек будет относиться к остальным, как мы к обезьянам. Это восприятие и переняли фашисты. Гитлер много раз был в доме Ницше и считал, что немцы как раз представляют эту сверхрасу. Если есть высшая и низшая расы, то надо руководствоваться моральными принципами своих, но не среди всех. У Ницше оказывается философия вышибал, слабых надо уничтожать. И к самим немцам он относился неоднозначно. Знаменитый немецкий дух это плохой воздух из испорченного кишечника. Но этого не заметили. У Ницше есть рассуждения об Апполонийском и Дианисийском началах. Он рассуждал: греческая культура была основана на культуре бога Дианиса, сила лежала в основе морали, но пришел Сократ, которого Ницше называет личным врагом, ибо он сказал, что сила ни при чем, надо вести всех людей к добру и свету, стало Апполонское начало. А дополнило это дело христианство. Слабые и вообще левые ребята объединились в стаю и победили сильных. Но не все пропало. В аристократических началах Ницше увидел возможность появления нового человека. Кроме этой картинки, которая может быть интерпретирована, как философия людоедства, агрессивности, войн (Ницше прямо говорил о мировых войнах). Это была философия фашистов. Есть и другая интерпретация. Наиболее ярко проявляется в притче трех состояний человеческого духа -- в ``Как говорил Заратустра''. Заратустра это персидский чел, который учил всех по притчам. Он говорил, что любой человек может пройти три состояния духа, но может оставаться на каком-то из низших. Первое состояние -- верблюд. Принцип -- ты должен, сколько на него кладут, столько и надо нести. Ну что поделать, родился верблюдом, значит и должен нести. Верблюд просто несет, а потом падает и умирает. Вторая стадия -- если верблюд взбрыкнет, то превратится во льва. Принцип -- я хочу и я могу, ничто мне не указ, я сам царь, никакая мораль и законы мне не важны. Многие предпочитают взять такую форму существования, разрушительную. Лев может все, но это все только разрушить. Наиболее оптимальная стадия -- ребенок. Ребенок оптимист, созидатель, конструктор и прочее. Получается, что в человеке сосуществует творец и тварность. В каждом человеке могут сосуществовать все три. Каждый может бороться со своей тварностью, чтобы стать творцом. Воля к власти над самим собой. 






\end{document}