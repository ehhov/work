\documentclass[a4paper, 12pt]{article}

\usepackage[utf8]{inputenc}
\usepackage[T2A]{fontenc}
\usepackage[english, russian]{babel}

\usepackage{enumitem}
\setlist{nolistsep}
\usepackage{mathtools}
\usepackage{xcolor}
\definecolor{dimblue}{HTML}{1010aa}
\usepackage[
	colorlinks=true, 
	allcolors=dimblue
]{hyperref}
\usepackage[
	vmargin=1in,
	hmargin=1in
]{geometry}
\linespread{1.3}
\usepackage{indentfirst}
\usepackage{graphicx}
\usepackage[multidot]{grffile}
\usepackage[labelsep=period]{caption}


\begin{document}

\noindent
Гусейнов Керим Демирович
\hfill
11 ноября 2020

\begin{center}
	\textbf{5. Спектр солнечной энергии}
\end{center}

Наша планета является открытой системой, взаимодействующей с космосом. При обсуждении климата наиболее интересен обмен энергией, то есть поступающая из космоса энергия и излучаемая Землей. Среди всех внешних источников энергии несомненно самый большой вклад имеет Солнце.

Приходящая с Солнца энергия есть электромагнитное излучение, и оно хорошо описывается спектром абсолютно черного тела.
$$ B_\lambda = \frac{2\hbar c^2}{\lambda^5 \left(e^{\frac{hc}{\lambda kT}}\right) - 1},
\qquad
B(T) = \int B_\lambda(T) \mathrm{d}\lambda = \sigma T^4,
\qquad
\sigma = \mathrm{const}.
$$
Измерения показывают, что солнечный спектр соответствует температуре излучающего вещества 6000 К. Интенсивность излучения Солнца характеризуют отдельной величиной, называемой солнечной постоянной. По определению солнечная постоянная $I_0$ описывает энергию, приходящуюся на единицу площади, перпендикулярную радиусу. Сама эта величина зависит от расстояния планеты до Солнца: солнечные постоянные на расстояниях $r_1$ и $r_2$ связаны соотношением
$$I_1 = I_0 \frac{r_0^2}{r_1^2}.$$

Поскольку Земля имеет форму, близкую к шару, на единицу поверхности Земли приходится вчетверо меньшая мощность
$$ I_0 \frac{\pi r ^ 2}{4 \pi r ^2} = \frac{1}{4} I_0. $$
Кроме того, часть приходящего с Солнца излучения может отразиться от атмосферы, а может поглотиться в ней. Отражение обычно описывают параметром $\alpha$, называемым планетарным альбедо, и в результате
$$I_{\text{пов. Земли, }\alpha} = \frac{1 - \alpha}{4} I_0.$$
Поглощение в атмосфере невозможно описать таким образом, поскольку оно меняет форму спектра, уменьшая приходящую на поверхность энергию по-разному для разных длин волн. Это показано на рисунке~\ref{fig1} (на следующей странице). Наибольший вклад имеют содержащиеся в атмосфере кислород, вода и углекислый газ.

\begin{figure}[b!]
	\includegraphics[width=\linewidth]{fig.pdf}
	\caption{Спектр солнца на верхней границе атмосферы и у поверхности Земли. Пунктиром обозначен спектр абсолютно черного тела при температуре 6000 К.}
	\label{fig1}
\end{figure}

\end{document}
