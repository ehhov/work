\documentclass[a4paper, 12pt]{article}

% Configuration {{{
\usepackage[utf8]{inputenc}
\usepackage[T2A]{fontenc} % T1 for English
\usepackage[english, russian]{babel}

\usepackage{enumitem}
\setlist{nolistsep}
\usepackage{mathtools}
\usepackage{xcolor}
\definecolor{dimblue}{HTML}{1010aa}
\usepackage[
	colorlinks=true, 
	allcolors=dimblue
]{hyperref}
\usepackage[
	vmargin=1in,
	hmargin=1in
]{geometry}
\linespread{1.3}
\usepackage{indentfirst}
\usepackage{graphicx}
\usepackage{tikz}
\usepackage[multidot]{grffile}
\usepackage[labelsep=period]{caption}
\usepackage{subcaption}
\usepackage{multirow}

%\usepackage{times} % for English

\def\d{\mathrm{d}}

\def\task#1#2{
	\vskip2\baselineskip
	\phantomsection
	\addcontentsline{toc}{section}{#1}
	\begin{center}
		\textbf{#1}
	\end{center}

	\begin{center}\begin{minipage}{.8\linewidth}\small
		#2
	\end{minipage}\end{center}

	\vskip\baselineskip
}
% }}}

\begin{document}

\noindent Керим Гусейнов \hfill группа 213М

\tableofcontents

% hw 1 {{{
\task{Задание 1}{
	Вы прослушали и сдали множество учебных курсов.
	Нарушались ли преподавателями принципы дидактики
	при преподавании какого-либо курса? Какие именно
	принципы? Как именно нарушались?
}

Я бы хотел рассказать об одном из преподавателей общей физики. Я думаю, 
Вы уже читали истории о ней, поскольку это очень яркий пример неудачного 
сочетания особенностей преподавателя.

Курсы по общей физике ведутся по примерно одним планам всеми 
преподавателями, а семинары проводятся с опорой на методички. Однако 
даже это не запрещает возможность нарушить принцип научности: 
неоднократно преподаватель настаивала на неверных утверждениях. 
Например, на практикуме она задавала вопрос о том, что происходит 
с ускорением свободного падения при погружении под землю, а правильным 
ответом считала увеличение пропорционально $1/r$, забывая о том, что 
часть исходной массы оказывается выше наблюдателя и перестает оказывать 
влияние. Указание на это не изменяло что-либо. Аналогичные очевидные 
ошибки часто появлялись в процессе вывода формул, например, для 
колебания струн. Ответ на доске, в свою очередь, всегда был верным, 
поскольку в самый последний момент решения переписывался из методички. 
Нередко создавалось впечатление, что преподаватель сам не уверен в том, 
что пытается объяснить.

Принцип доступности нарушался по двум параметрам. Во-первых, даже если 
изначально идти от простого к сложному, принцип нарушается при 
недостаточном усвоении каждого этапа курса, а достигнуть достаточного 
усвоения довольно сложно при наличии ошибок в решениях задач. Во-вторых, 
преподаватель не отвечала на вопросы во время и после семинаров, 
поскольку неполные и чрезвычайно трудные к осознанию ответы 
провоцировали лишь больше вопросов, после чего звучали слова ``если 
обсуждать то, что разобрано в методичке, мы ничего не успеем.''

Принцип целенаправленности не нарушался, поскольку семинары опирались на 
методичку, в которой каждая задача направлена на достижение какой-либо 
определенной цели, которые, как кирпичи, складываются в общую картину. 
С другой стороны, далеко не все из этих целей-кирпичей были достигнуты 
ввиду некачественного разбора задач.

Принцип систематичности и последовательности не нарушался по той же 
причине. Оговорка в этом случае имеет даже меньший вес, поскольку верные 
ответы к каждой задаче, если иногда воспринимать их как данные факты, 
подкрепляли события следующих задач.

Принцип наглядности соблюдался даже активнее двух предыдущих, поскольку 
каждая задача иллюстрировалась исчерпывающими рисунками, что позволяло 
в большей степени ощутить смысл формул и даже помогало при 
самостоятельном разборе задач, не удостоенных достаточного времени на 
семинаре.

Принцип связи обучения с повседневной жизнью тоже не нарушался, 
поскольку за многие годы темы курсов общей физики оттачиваются до 
примерного состояния.

Принцип сознательности и активности, в целом, тоже не нарушался, 
поскольку форма семинаров побуждала многих усерднее работать 
самостоятельно. Некоторые студенты, скорее всего, не принимали бы такое 
активное участие в семинарах, если бы в них не встречались неточности. 
С другой стороны, большинству студентов было сложнее принимать участие 
вообще ввиду нетрадиционного формата.

Принцип прочности знаний соблюдался лишь частично. Проверка домашних 
заданий происходила редко и обычно не заканчивалась лучшим пониманием 
задач, если те были решены неверно. Контрольные, в свою очередь, были 
организованы хорошо за одним исключением, когда преподаватель попросила 
нас прийти к первой паре вместо второй, чтобы успеть разобрать небольшой 
кусок материала перед контрольной, а в итоге опоздала на полторы пары 
и все равно выдала нам контрольную, хоть и на 45 минут вместо 
запланированных 90, не успев, соответственно, перед этим покрыть тот 
небольшой кусок материала.

Принцип воспитания и развития, пожалуй, нарушался больше всех остальных. 
Нельзя сказать, что преподаватель служил примером для подражания. Она 
была нашим семинаристом только в течение одного семестра, и под конец 
семестра существенная часть взаимодействия преподаватель--студенты имела 
личностный характер и включала многочисленные обсуждения, не относящиеся 
к предмету вообще. Разумеется, вина студентов в этом далеко не мала, но 
за многие годы работы преподаватель имеет больше возможностей отточить 
свои навыки, а студенты в начале семестра не были настроены негативно.


% Киров вызывал к доске
% Володин вызывал к доске
% Галлямова....

%%%%%%%%%%%%%%%%%%%%%%%%
%%% FROM THE LECTURE %%%
%%%%%%%%%%%%%%%%%%%%%%%%
% 1) Научность.
% 2) Доступность.
% 3) Целенаправленность.
% 4) Систематичность и последовательность.
% 5) Наглядность.
% 6) Связь обучения с повседневной жизнью.
% 7) Сознательность и активность.
% 8) Прочность знаний.
% 9) Воспитание и развитие. 
% 
% 1 -- Научность
% 
% Сущность принципа: содержание обучения должно
% соответствовать реальным фактам и отражать
% современные научные данные.
% 
% Требования принципа: формирование у учащихся
% системы теоретических знаний, достоверность
% изучаемых фактов, подтвержденность действий и
% выводов педагога наукой. 
% 
% 2 -- Доступность
% 
% Сущность принципа: обучение должно соответствовать
% индивидуальным особенностям учащихся и имеющимся у
% них знаниям. Обучение не должно быть ни очень легким,
% ни чрезмерно сложным. Сам Ян Коменский писал, что
% обучение должно идти от простого к сложному, от
% известного к неизвестному, от близкого к далекому.
% 
% Требования принципа: нужно учитывать образовательный
% уровень, познавательные возможности, профессиональную
% подготовку, характер, опыт, возрастные особенности,
% потребности и интересы учащихся. 
% 
% 3) Целенаправленность.
% 
% Сущность принципа: необходимо осознанно создавать
% организационные, методические и содержательные
% основы педагогического процесса, направляя его к
% достижению поставленных образовательных целей.
% 
% Требования принципа: содержание образовательного
% процесса должно соответствовать содержанию
% воспитания, а обучение должно соответствовать
% учебному плану, рассчитанному на достижение
% определенных результатов.
% 
% 4) Систематичность и последовательность.
% 
% Сущность принципа: необходимы особый порядок и система
% преподавания, основанные на логике и хронологии. Подача
% информации должна планироваться, информацию нужно
% разбивать на разделы, модули, темы, выделять идейные
% центры и главные понятия, расставлять акценты.
% 
% Требования принципа: учебный материал нужно
% преподносить в строгой логической последовательности,
% обеспечивая одновременное применение полученных знаний
% на практике с целью их закрепления.
% 
% 5) Наглядность.
% 
% Сущность принципа: при обучении нужно в первую очередь
% опираться на зрительные органы, и лишь затем – на остальные
% органы чувств. Поэтому крайне необходимо применять
% средства повышения наглядности. Следует помнить, что
% максимальной информативностью обладает именно зрение,
% т.к. оно даёт человеку 80% знаний.
% 
% Требования принципа: демонстрирование нужно проводить в
% определенном порядке с определенными целями, разные
% виды наглядности должны сочетаться друг с другом,
% наблюдаемое должно подвергаться анализу учащимися и
% педагогом, наблюдаемое должно соответствовать культурным
% и психологическим требованиям. 
% 
% 6) Связь обучения с повседневной жизнью.
% 
% Сущность принципа: процесс обучения нужно сопровождать
% постоянным сомнением и проверять теорию с помощью
% практических критериев. Данный принцип называют иначе
% принципом связи теории с практикой.
% 
% Требования принципа: учебно-воспитательный процесс
% должен иметь явно выраженную профессиональную
% направленность, в ходе процесса обучения нужно отвечать
% на вопросы – когда, где и как в жизни можно применять
% полученные знания. 
% 
% 7) Сознательность и активность.
% 
% Сущность принципа: поскольку в педагогическом процессе принимают
% участие две стороны – педагог и учащийся, то обе эти стороны должны
% быть активными, понимать свои цели. Педагог является субъектом
% образования, а учащийся – объектом.
% 
% Активность учащегося состоит в усвоении содержания обучения, в
% самостоятельной организации своей работы и проверке её результатов.
% Активность педагога состоит в мотивации обучения, формировании
% познавательных склонностей учащегося, использовании разных методов
% обучения.
% 
% Требования принципа: учебный процесс должен быть двусторонним,
% педагог должен использовать активные формы обучения, побуждать
% учащихся к самостоятельности и творчеству, развивать у них научное
% мышление и навыки применения полученных знаний для решения
% практических задач. 
% 
% 8) Прочность знаний.
% 
% Сущность принципа: необходимо стремиться закрепить
% содержание обучения в сознании учащихся. Для этого
% нужно стимулировать стремление к познанию,
% систематически повторять материал, регулярно
% контролировать результаты обучения.
% 
% Требования принципа: знания должны повторяться и
% закрепляться, умения и навыки должны применяться на
% практике, должен обеспечиваться систематический
% контроль, сочетаемый с индивидуальным подходом к
% каждому учащемуся.
% 
% 9) Воспитание и развитие. 
% 
% Сущность принципа: педагогический процесс должен быть
% направлен на воспитание и развитие в учащихся не только
% профессиональных качеств и навыков, но и на развитие
% учащегося как адекватной, здоровой, пристойной, состоятельной
% и живой личности. То есть образование и воспитание должны
% «идти рядом». Этот принцип называют также принципом
% воспитывающего и развивающего обучения.
% 
% Требования принципа: нужно помнить, что основные цели
% обучения – развивающая, воспитывающая и познавательная –
% должны достигаться параллельно.
% 
% ------
% В процессе обучения нужно формировать у учащихся научное
% мировоззрение, творческое мышление, инициативность и
% самостоятельность, способность делать выводы, сопоставлять,
% сравнивать, выделять основное, обобщать, анализировать.
% Нужно также воспитывать дисциплинированность, навыки
% культурного поведения, интеллигентность, гуманность,
% гражданскую ответственность и патриотизм.
%  }}}

% hw 2 {{{
\task{Задание 2}{
	Сформулируйте по одному вопросу каждого типа из какого-либо спецкурса по 
	Вашей специальности или из-какого-то одного математического курса (мат. 
	анализ, ТФКП, линейная алгебра и т.д.). Напишите, какие еще, по вашему 
	мнению, типы вопросов можно использовать при работе со студентами на 
	семинаре? в практикуме? на экзамене?
}

Вопросы по общему курсу дифференциальных уравнений.
\begin{enumerate}
	\item \textit{Что это такое? (Дайте определение ...)}

		Что такое характеристическое уравнение линейного дифференциального уравнения с постоянными коэффициентами?

	\item\textit{Сформулируйте ...}

		Сформулируйте теорему Коши существования и единственности решения дифференциального уравнения первого порядка.

	\item\textit{Напишите формулу (уравнение) ...}

		Напишите формулу для общего решения неоднородного линейного дифференциального уравнения.

	\item \textit{Нарисуйте график ...}

		Изобразите фазовый портрет дифференциального оператора вблизи устойчивой точки покоя.

	\item \textit{Приведите пример...}

		Приведите пример автономного дифференциального уравнения второго порядка.

	\item \textit{Изобразите схему опыта...}

		Опишите метод последовательных приближений.

	\item \textit{Как соотносятся...}

		Как соотносятся функция Грина дифференциального оператора и решение неоднородного уравнения с этим оператором?

	\item \textit{Сколько?}

		Сколько элементов содержится в фундаментальной системе решений дифференциального уравнения $n$-го порядка?

	\item \textit{Почему?}

		Почему для краевой задачи, в отличие от задачи Коши, не существует теоремы существования и единственности решения?

	\item \textit{Найдите ошибку в утверждении...}

		Рассмотрим систему дифференциальных уравнений
		$$\frac{d x_i}{d t} = \sum_{j=1}^{n} A_{i,j} x_j,\ \ i=\overline{1,n}$$
		и решения $\lambda_k$, $k=\overline{1,n}$ уравнения
		$$\mathrm{det}(A-\lambda\,1_{n\times n}) = 0.$$

		Найдите и исправьте ошибку в следующем утверждении:

		Решение системы $x_i = 0$, $i = \overline{0, n}$ называется точкой покоя типа фокус, если существует хотя бы два числа $i, j$ от $1$ до $n$ такие, что $\mathrm{Re} \lambda_i \cdot \mathrm{Re} \lambda_j < 0$.
\end{enumerate}
% }}}

% hw 3 {{{
\task{Задание 3}{
	Придумать одну задачу (с решением) из
	любого курса общей физики или из спецкурса по
	Вашему выбору, которая допускает различные
	решения в зависимости от выбранных абстрактных
	моделей. 
}
\label{task3}

\begin{figure}[b]% {{{
	\centering
	\begin{tikzpicture}[x=1mm,y=1mm]% simple {{{
		\draw [-stealth] (-10-30, 10) -- (-10-30, 0);
		\draw            (-10-27, 0) node{$z$};
		\draw [-stealth] (-10-30, 10) -- (-10-20, 10);
		\draw            (-10-21, 7) node{$x$};

		\draw [line width = .05mm] (-15, 10) -- (15, 10);
		\draw [line width = .05mm] (-15, 10) -- (-13, 12);
		\draw [line width = .05mm] (-13, 10) -- (-11, 12);
		\draw [line width = .05mm] (-11, 10) -- (-9, 12);
		\draw [line width = .05mm] (-9, 10) -- (-7, 12);
		\draw [line width = .05mm] (-7, 10) -- (-5, 12);
		\draw [line width = .05mm] (-5, 10) -- (-3, 12);
		\draw [line width = .05mm] (-3, 10) -- (-1, 12);
		\draw [line width = .05mm] (-1, 10) -- (1, 12);
		\draw [line width = .05mm] (1, 10) -- (3, 12);
		\draw [line width = .05mm] (3, 10) -- (5, 12);
		\draw [line width = .05mm] (5, 10) -- (7, 12);
		\draw [line width = .05mm] (7, 10) -- (9, 12);
		\draw [line width = .05mm] (9, 10) -- (11, 12);
		\draw [line width = .05mm] (11, 10) -- (13, 12);
		\draw [line width = .05mm] (13, 10) -- (15, 12);
		\draw [line width = .5mm]  (-10, -10) -- (-10, 10);
		\draw [line width = .5mm]  (10, -10) -- (10, 10);
		\draw [line width = 1mm] (-10.25, -10) -- (10.25, -10);
		\draw [line width = .1mm, -stealth, dashed] (0, -6) -- (0, -9.5);
		\draw (0, -14) node{$L_0, \kappa = G\cdot S$};
		\draw (-13, 3) node{$l_0$};
		\draw (-13, -2) node{$k$};
		\draw (13, 3) node{$l_0$};
		\draw (13, -2) node{$k$};

		\draw[line width=.05mm](40  +-15,     10)   --  (40 +15,      10);
		\draw[line width=.05mm](40  +-15,     10)   --  (40 + -13,    12);
		\draw[line width=.05mm](40  +-13,     10)   --  (40 + -11,    12);
		\draw[line width=.05mm](40  +-11,     10)   --  (40 + -9,     12);
		\draw[line width=.05mm](40  +-9,      10)   --  (40 + -7,     12);
		\draw[line width=.05mm](40  +-7,      10)   --  (40 + -5,     12);
		\draw[line width=.05mm](40  +-5,      10)   --  (40 + -3,     12);
		\draw[line width=.05mm](40  +-3,      10)   --  (40 + -1,     12);
		\draw[line width=.05mm](40  +-1,      10)   --  (40 + 1,      12);
		\draw[line width=.05mm](40  +1,       10)   --  (40 + 3,      12);
		\draw[line width=.05mm](40  +3,       10)   --  (40 + 5,      12);
		\draw[line width=.05mm](40  +5,       10)   --  (40 + 7,      12);
		\draw[line width=.05mm](40  +7,       10)   --  (40 + 9,      12);
		\draw[line width=.05mm](40  +9,       10)   --  (40 + 11,     12);
		\draw[line width=.05mm](40  +11,      10)   --  (40 + 13,     12);
		\draw[line width=.05mm](40  +13,      10)   --  (40 + 15,     12);
		\draw[line width=.5mm] (40  +-10,     -13)  --  (40 + -10,    10);
		\draw[line width=.5mm] (40  +10,      -13)  --  (40 + 10,     10);
		\draw[line width=1mm]  (40  +-10.25,  -13)  --  (40 + 10.25,  -13);
		\draw [line width = .1mm, -stealth, dashed] (40, -9) -- (40, -12.5);

		\draw[line width=.05mm](80  +-15,     10)   --  (80 +15,      10);
		\draw[line width=.05mm](80  +-15,     10)   --  (80 + -13,    12);
		\draw[line width=.05mm](80  +-13,     10)   --  (80 + -11,    12);
		\draw[line width=.05mm](80  +-11,     10)   --  (80 + -9,     12);
		\draw[line width=.05mm](80  +-9,      10)   --  (80 + -7,     12);
		\draw[line width=.05mm](80  +-7,      10)   --  (80 + -5,     12);
		\draw[line width=.05mm](80  +-5,      10)   --  (80 + -3,     12);
		\draw[line width=.05mm](80  +-3,      10)   --  (80 + -1,     12);
		\draw[line width=.05mm](80  +-1,      10)   --  (80 + 1,      12);
		\draw[line width=.05mm](80  +1,       10)   --  (80 + 3,      12);
		\draw[line width=.05mm](80  +3,       10)   --  (80 + 5,      12);
		\draw[line width=.05mm](80  +5,       10)   --  (80 + 7,      12);
		\draw[line width=.05mm](80  +7,       10)   --  (80 + 9,      12);
		\draw[line width=.05mm](80  +9,       10)   --  (80 + 11,     12);
		\draw[line width=.05mm](80  +11,      10)   --  (80 + 13,     12);
		\draw[line width=.05mm](80  +13,      10)   --  (80 + 15,     12);
		\draw[line width=.5mm] (80  +-10,     -13)  --  (80 + -10,    10);
		\draw[line width=.5mm] (80  +10,      -13)  --  (80 + 10,     10);
		\draw[line width=1mm]  (80  +-10.25,  -13)  parabola bend (80, -15)  (80 + 10.25,  -13);
		\draw [line width = .1mm, -stealth, dashed] (80, -11) -- (80, 
		-14.5);
	\end{tikzpicture}% }}}
	\caption{Иллюстрация к \hyperref[task3]{заданию 3}. Слева вертикальные 
	нити нерастяжимы, а горизонтальная балка абсолютно твердая. В центре 
	вертикальные нити растяжимы, а горизонтальная балка абсолютно твердая. 
	Справа вертикальные нити растяжимы, а горизонтальная балка имеет 
	конечное значение модуля сдвига.}
	\label{fig:abstract}
\end{figure}% }}}

Описанная мной задача вряд ли может использоваться для подготовки 
студентов ввиду вопиющей простоты, но она все равно иллюстрирует 
возможность применения разных абстрактных моделей к одной системе. 
В задаче требуется найти положение точки в центре горизонтальной балки.

\textit{Вариант 1.} Нити нерастяжимы, а балка абсолютно твердая. 
В данном случае положение целиком определяется заданными длинами нитей: 
точка находится на $l_0$ ниже потолка.

\textit{Вариант 2.} Нити растяжимы и имеют коэффициент упругости $k$; 
масса нитей $m$, а балки -- $M$. Проще всего задачу решить на основе 
закона сохранения энергии.
$$ E = 2\,\frac{k}{2}\,(l - l_0)^2 - 2mg\frac{l}{2} - Mgl, $$
$$ \frac{\partial E}{\partial l} = 2\,k\,(l - l_0) - mg - Mg = 0, $$
$$ l = l_0 + \frac{(m+M)g}{2k}. $$
В итоге, центр балки находится на расстоянии $l = l_0 + g(m+M)/(2k)$ 
ниже потолка.

\textit{Вариант 3.} Нити растяжимы с коэффициентом упругости $k$, для 
балки произведение модуля сдвига на площадь сечения $G \cdot 
S = \kappa$; массы нитей $m$, а балки -- $M$. Вновь запишем выражение 
для энергии.
$$ E = 2\,\frac{k}{2}\,(l - l_0)^2 - 2mg\frac{l}{2} - Mg \frac{\int z\,\mathrm{d} L}{L} + \frac{\kappa}{2} \int \left(\tg\theta\right)^2\,\mathrm{d} L, $$
где $\mathrm{d}L$ -- элемент длины балки, а $\tg\theta$ -- угол смещения 
элемента балки по сравнению с предыдущим. $\tg\theta$ таким образом 
оказывается равным ${\mathrm{d}\over\mathrm{d}x} \, ({\mathrm{d} z \over 
\mathrm{d} x}) = \frac{\mathrm{d}^2 z}{\mathrm{d} x ^2}$. Элемент длины 
балки также можно выразить через зависимость $z(x)$: $\mathrm{d}L 
= \sqrt{1 + \left(\frac{\mathrm{d} z}{\mathrm{d} x}\right)^2} \mathrm{d} 
x$. Таким образом, функционал энергии приобретает вид
$$ E[z(x)] = 2\,\frac{k}{2}\,(l - l_0)^2 - 2mg\frac{l}{2} - Mg \frac{\int z\sqrt{1 + (z')^2}\,\mathrm{d} x}{\int \sqrt{1 + (z')^2}\,\mathrm{d} x} + \frac{\kappa}{2} \int (z'')^2\,\sqrt{1 + (z')^2}\,\mathrm{d} x, $$
$$ \text{при условиях} \qquad z(-L_0/2) = l, \qquad z(L_0/2) = l. $$
$$\begin{aligned}
	\delta E[z(x)]
	=& - Mg \frac{\int \delta \left(z\sqrt{1 + (z')^2}\right)\,\mathrm{d} x}{\int \sqrt{1 + (z')^2}\,\mathrm{d} x} + Mg \frac{\int z\sqrt{1 + (z')^2}\,\mathrm{d} x}{\left(\int \sqrt{1 + (z')^2}\,\mathrm{d} x\right)^2} \int \delta \sqrt{1 + (z')^2}\,\mathrm{d} x +\\
	&+ \frac{\kappa}{2} \int \delta \left((z'')^2\,\sqrt{1 + (z')^2}\,\right) \mathrm{d} x = \\
	=& - Mg \frac{\int \delta z\ \sqrt{1 + (z')^2}\,\mathrm{d} x}{\int \sqrt{1 + (z')^2}\,\mathrm{d} x}
	- Mg \frac{\int z \frac{z'\delta z'}{\sqrt{1 + (z')^2}}\,\mathrm{d} x}{\int \sqrt{1 + (z')^2}\,\mathrm{d} x} + Mg \frac{\int z\sqrt{1 + (z')^2}\,\mathrm{d} x}{\left(\int \sqrt{1 + (z')^2}\,\mathrm{d} x\right)^2} \times \\
	&\times \int \frac{z'\delta z'}{\sqrt{1 + (z')^2}}\,\mathrm{d} x + \frac{\kappa}{2} \int z'' \delta z''\ \sqrt{1 + (z')^2}\,\mathrm{d} x + \frac{\kappa}{2} \int (z'')^2 \frac{z'\delta z'}{\sqrt{1 + (z')^2}}\,\mathrm{d} x.
\end{aligned}$$
Заметим далее, что, с учетом очевидных ограничений на $\delta z$
$$
\delta z(-L_0/2) = 0
,\quad
\delta z(L_0/2) = 0
,\quad
\delta z'(-L_0/2) = 0
,\quad
\delta z'(L_0/2) = 0
,$$ 
для произвольной функции $f(x)$ справедливы выражения
$$ \int f(x) \delta z' \d x = -\int f'(x) \delta z \d x,$$
$$ \int f(x) \delta z'' \d x = \int f''(x) \delta z \d x.$$
Тогда вариацию $E$ можно записать в виде
$$\begin{aligned}
	\delta E[z(x)]
	=& - Mg \frac{\int \delta z\ \sqrt{1 + (z')^2}\,\mathrm{d} x}{\int \sqrt{1 + (z')^2}\,\mathrm{d} x}
	+ Mg \frac{\int \left(\frac{z z'}{\sqrt{1 + (z')^2}}\right)'\delta z\,\mathrm{d} x}{\int \sqrt{1 + (z')^2}\,\mathrm{d} x} - Mg \frac{\int z\sqrt{1 + (z')^2}\,\mathrm{d} x}{\left(\int \sqrt{1 + (z')^2}\,\mathrm{d} x\right)^2} \times \\
	&\times \int \left(\frac{z'}{\sqrt{1 + (z')^2}}\right)'\delta z\,\mathrm{d} x + \frac{\kappa}{2} \int \left(z'' \sqrt{1 + (z')^2}\right)''\delta z\,\mathrm{d} x - \\
	& -\frac{\kappa}{2} \int  \left(\frac{(z'')^2 z'}{\sqrt{1 + (z')^2}}\right)'\delta z\,\mathrm{d} x.
\end{aligned}$$
Введем обозначения $L = \int \d L$ -- длина балки, $z_0 = \int z\d 
L / \int \d L$ -- среднее значение $z$ на балке. Уравнение 
Эйлера-Лагранжа приобретает вид
$$\begin{aligned}
	0 =& - \frac{Mg}{L} {\sqrt{1 + (z')^2}}
	+ \frac{Mg}{L} {\left(\frac{z z'}{\sqrt{1 + (z')^2}}\right)' - \frac{Mgz_0}{L} \left(\frac{z'}{\sqrt{1 + (z')^2}}\right)' + \\
	&+ \frac{\kappa}{2} \left(z'' \sqrt{1 + (z')^2}\right)'' - \frac{\kappa}{2} \left(\frac{(z'')^2 z'}{\sqrt{1 + (z')^2}}\right)'.
\end{aligned}$$
Подставляя $z(x) = -ax^2 + b$, где $b = l + aL_0^2/4$, находим выражение 
для $l$, совпадающее с предыдущим, и выражение для $a$. Значение $b$ 
и окажется ответом.

% Мой длительный ответ.
% }}}

% hw 4 {{{
\task{Задание 4}{
	В последнее время все мы вынужденно столкнулись с необходимостью 
	проведения занятий в дистанционном формате. Несмотря на то, что до 
	пандемии специалисты по дистанционным образовательным технологиям 
	утверждали, что дистанционное занятие ничуть не хуже очного, 
	в реальности все оказалось не так просто. Вопрос. Напишите, какие 
	методические находки и ошибки лекторов Вы отметили бы по Вашему опыту 
	посещения лекций в дистанционном формате?
}

Многие преподаватели крайне отрицательно относятся к идее дистанционных 
лекций, но даже такие стараются что-либо предпринять, когда возникает 
необходимость. Два лектора не проводили дистанционных лекций вообще 
весной 2020 года, а лишь отправили свои конспекты в конце семестра. 
Естественно, это крайне негативно сказалось на усвоении материала. 
Немного лучше поступил один преподаватель, который отправлял 
docx-конспекты каждую неделю, но это все равно было далеко от пользы 
точно таких же лекций по Zoom.

Среди хороших практик я могу назвать рукописные листочки, повторяющие 
все, что было бы написано на доске. Так поступил лишь один 
преподаватель, а его способ презентации листочков ослаблял удобство 
этого формата до такой степени, что большинству моих одногруппников 
лекции не показались удачными вообще, но, по-моему, рукописный текст 
(даже написанный заранее) оказывается единственным способом вести 
лекции достаточно медленно, чтобы слушатели успевали осознавать 
материал.

% Лекция 6
% Мой длительный ответ.
% }}}

% hw 5 {{{
\task{Задание 5}{
	Придумайте аналогичный пример проверки элемента знаний из курса общей 
	физики, математики или спецкурса по выбору (составьте 4--5 вопросов по 
	предложенному образцу).
}

Проверка элемента знаний ``ряд Лорана'' курса ТФКП.

\begin{enumerate}
	\item Запишите общий вид ряда Лорана.
	\item Является ли представление функции в виде ряда Лорана 
		единственным?
	\item Напишите выражение для коэффициентов ряда Лорана произвольной 
		функции~$f(z)$ в произвольной точке $z_0$.
	\item Напишите ряд Лорана для функции $f(z) = e^z / z$ в точке $z_0 = 0$.
	\item Что можно сказать о поведении модуля функции вблизи точки $z_0$, 
		если ее ряд Лорана в этой точке имеет бесконечность слагаемых 
		с отрицательной степенью? (например, ни один $c_n$ не равен нулю для 
		сколь угодно больших отрицательных~$n$)
\end{enumerate}

% Лекция 8, слайд 28
% Мой длительный ответ.
% }}}

% hw 6 {{{
\task{Задание 6}{
	Вспомните и напишите, встречались ли Вы во время Вашего обучения на 
	физическом факультете (или в другом вузе) с БРС, которая была, по 
	вашему мнению, устроена несправедливо? Если это так, то в чем состояла 
	несправедливость?
}

Я сталкивался с несправедливостью при оценке работы студентов в семестре 
(по отношению к одногруппникам, а не ко мне) только со стороны 
преподавателей, не ведущих БРС. В основном, эти преподаватели вели 
математические дисциплины или общую физику, где БРС требуется высшими 
инстанциями. Преподаватели просто заполняли бланки на свое усмотрение 
без какой-либо системы.

% Лекция 9, слайд 9
% Мой длительный ответ.
% }}}

\end{document}
