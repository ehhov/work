\documentclass[a4paper, 11pt]{article}

\usepackage[utf8]{inputenc}
\usepackage[T2A]{fontenc}
\usepackage[english, russian]{babel}

\usepackage[
	vmargin=.7in,
	hmargin=.7in
]{geometry}
\linespread{1.3}


\begin{document}

\noindent {Гусейнов Керим, 413 гр.} \hfill {\today}

\noindent\textbf{Статья 140 УК РФ. Отказ в предоставлении гражданину информации}

Неправомерный отказ должностного лица в предоставлении собранных в установленном порядке документов и материалов, непосредственно затрагивающих права и свободы гражданина, либо предоставление гражданину неполной или заведомо ложной информации, если эти деяния причинили вред правам и законным интересам граждан, -

наказываются штрафом в размере до двухсот тысяч рублей или в размере заработной платы или иного дохода осужденного за период до восемнадцати месяцев либо лишением права занимать определенные должности или заниматься определенной деятельностью на срок от двух до пяти лет.

\par\noindent Гипотеза: \textit{гражданин запросил у должностного лица материалы, затрагивающие его права и свободы }
\par\noindent Диспозиция: \textit{неправомерный отказ в предоставлении или неполное предоставление информации, затрагивающей права и свободы гражданина }
\par\noindent Санкция: \textit{штраф / лишение права занимать определенные должности }

\vskip10pt

\noindent\textbf{Статья 168 УК РФ. Уничтожение или повреждение имущества по неосторожности}

Уничтожение или повреждение чужого имущества в крупном размере, совершенные путем неосторожного обращения с огнем или иными источниками повышенной опасности, -

наказываются штрафом в размере до ста двадцати тысяч рублей или в размере заработной платы или иного дохода осужденного за период до одного года, либо обязательными работами на срок до четырехсот восьмидесяти часов, либо исправительными работами на срок до двух лет, либо ограничением свободы на срок до одного года, либо принудительными работами на срок до одного года, либо лишением свободы на тот же срок.

\par\noindent Гипотеза: \textit{действия совершает любой гражданин}
\par\noindent Диспозиция: \textit{повреждение чужого имущества в крупном размере (стоимостью выше 250 т.р.), произошедшее по причине неосторожного обращения с источником повышенной опасности }
\par\noindent Санкция: \textit{штраф / обязательные работы / исправительные работы / принудительные работы / лишение свободы }

\vskip10pt

\noindent\textbf{Статья 177 УК РФ. Злостное уклонение от погашения кредиторской задолженности}

Злостное уклонение руководителя организации или гражданина от погашения кредиторской задолженности в крупном размере или от оплаты ценных бумаг после вступления в законную силу соответствующего судебного акта -

наказывается штрафом в размере до двухсот тысяч рублей или в размере заработной платы или иного дохода осужденного за период до восемнадцати месяцев, либо обязательными работами на срок до четырехсот восьмидесяти часов, либо принудительными работами на срок до двух лет, либо арестом на срок до шести месяцев, либо лишением свободы на срок до двух лет.

\par\noindent Гипотеза: \textit{наличие у гражданина или руководителя организации кредиторской задолженности в крупном размере (2\,250 т.р.) или обязанности оплатить ценные бумаги поле вступления в силу соответствующего судебного акта}
\par\noindent Диспозиция: \textit{злостное уклонение от уплаты этих долгов }
\par\noindent Санкция: \textit{штраф / обязательные работы / принудительные работы / арест / лишение свободы }

\end{document}


