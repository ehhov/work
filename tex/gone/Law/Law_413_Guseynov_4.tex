\documentclass[a4paper, 12pt]{article}

\usepackage[utf8]{inputenc}
\usepackage[T2A]{fontenc}
\usepackage[english, russian]{babel}

\usepackage[
	vmargin=1in,
	hmargin=1in
]{geometry}
\linespread{1.3}
\usepackage{xcolor}
\definecolor{allrefs}{HTML}{1010aa}
\usepackage[colorlinks=true, allcolors=allrefs]{hyperref}
\usepackage{enumitem}
\setlist{nolistsep}


\begin{document}
\noindent
Гусейнов Керим, 413 гр.
\hfill 
\today

\begin{center}\bf
Президент в России и во Франции
\end{center}

Президент Франции назначает премьер-министра, они вместе назначают и увольняют других членов правительства. 
Президент России назначает председателя правительства с согласия госдумы, президент назначает и увольняет заместителей председателя правительства и федеральных министров по предложению председателя правительства. 

В России президент предлагает госдуме кандидатов председателя ЦБ, предлагает совету федерации кандидатов судей конституционного суда и верховного суда, ген. прокурора и его заместителей, сам назначает прокуроров субъектов РФ и других федеральных судей, кроме названных выше, формирует совет безопасности. 

Во Франции после передачи правительству закона президент может потребовать от парламента дополнительно обсудить этот закон, парламент не может отказаться. 

Во Франции после консультации с премьер-министром и представителями палат президент может распустить национальное собрание, но не чаще раза в год. 
В России президент принимает решение об отставке правительства, распускает госдуму. 

Президенты обеих стран назначают послов, осуществляют право помилования, назначают на высшие военные должности, являются главами вооруженных сил. 

Для обвинения президента во Франции необходимо, чтобы обе палаты приняли одно решение при публичном голосовании, в котором требуется абсолютное большинство. 
В России необходимо, чтобы инициатива обвинения была выдвинута третью госдумы и при заключения специальной комиссии, образованной госдумой, а обвинительный результат принимается при достижении двух третей голосов. 

При отсутствии президента временно исполняющим его обязанности во Франции является председатель сената (который выбирается сенатом), а в России -- председатель правительства. 

Президент Франции без подписи премьер министра осуществляет только назначение / увольнение премьер-министра, отправление закона на референдум, роспуск национального собрания, необходимые меры в критических ситуациях ниже. 
Президенту России нужна подпись председателя правительства только для назначения / увольнения заместителей председателя правительства и федеральных министров. 

Таким образом, у президента во Франции больше возможностей выбирать работающих в правительстве людей, но процесс его обвинения существенно проще. 
В России президент также назначает на многие должности самостоятельно, но на меньшее количество, чем во Франции, однако у российского президента больше прав для роспуска правительства / госдумы и добиться его обвинения существенно сложнее. 

\hrulefill

Формулировка во французской конституции (статья 16). 
Под серьезной и непосредственной угрозой для институтов республики, независимости нации, целостности ее территории или выполнения ее международных обязательств и когда нормальное функционирование гос. властей прекращено, президент официально консультируется с премьер-министром и конституционным советом и после может делать что угодно. Об этом он информирует нацию. Главная цель -- предоставление конституционным властям возможности вновь функционировать. Средства обговариваются с конституционным советом. 

Формулировка в конституции РФ (статья 87 и федеральный закон о военном положении).
В случае агрессии или непосредственной угрозы агрессии, президент вводит в России или в каких-то субъектах военное положение, сразу сообщая совету федерации и госдуме. Режим военного положения нацелен на отражение или предотвращение агрессии против России. Допустимые меры определены федеральным законом. Аналогично президент вводит чрезвычайное положение, так же сообщая об этом совету федерации и госдуме. 

Предпринимаемые президентом России меры строго заранее обговорены в федеральном законе, а президент Франции, при согласии конституционного совета, ничем не ограничен. Так что президенту Франции даны большие полномочия. Однако мне кажется, что российская формулировка включает больший спектр возможных угроз, то есть у президента больше возможностей ввести военное положение. Кроме того, российский президент обязан сообщать о введении военного положения только совету федерации и госдуме, а не нации. Это не влияет на права президентов, но должно быть замечено. 





\newpage
\begin{center}\bf
Выборность и сменяемость власти
\end{center}

Слова выборность и сменяемость власти можно воспринимать по-разному. 
Один из самых простых путей -- прямой, формальный, исходя из написанных в бюллетенях выборах граждан, без учета других обстоятельств. 
Мысль Гамильтона о том, что президент подлежит переизбранию столько раз, сколько сочтёт необходимым оказать ему доверие народ, больше всего похожа на прямую интерпретацию. 
Однако человечески честным вариантом мне кажется не настолько прямой. 
Люди могут добиваться нужной им галочки в бюллетени разными способами, не всех из которых честны. 
Если выбор граждан не основан на их сознательном здравом рассуждении, он не имеет значения. 
Конечно, для предотвращения получения нечестных голосов можно написать другие законы, не затрагивающие непосредственно ограничения на сроки президентства или процедуру выборов. 
Однако выявление и формализация каждого нечестного пути, а также разработка эффективных законов против них это очень сложные задачи. 
Возможно, даже невыполнимые. 
И определение реального веса голосов для выяснения честно победившего кандидата, естественно, не представляется возможным. 
Таким образом, наиболее резкой, останавливающей силой будет именно ограничение числа президентских сроков, можно даже сказать, что эта мера необходима. 

С другой стороны, далеко не все добирающиеся до поста президента люди ищут личную выгоду.
Если целью президента является благополучие страны, а средства эффективны и разумны, нет причин ограничивать его время пребывания на должности. 
Хороший пример негативного влияния лимита президентских сроков -- Барак Обама, который за отданные ему два срока существенно улучшил многие стороны жизни в США, но не имел возможности продолжать свою работу далее. 

Однако намерение Гамильтона и Мэдисона избирать президента пожизненно, по-моему, привели бы к катастрофическим последствиям. 
Когда срок президентства не слишком велик, люди могут относительно часто выбирать наиболее эффективного, по их мнению, лидера. 
Если президент плохо работал в отданный ему срок, осознанный выбор граждан не падет на него еще раз. 
Но если президент сохраняет свою должность пожизненно, помех делать что ему вздумается станет гораздо меньше. 
Кроме того, пожизненность президентских сроков наверняка привела бы к существенной угрозе их жизни. 

Таким образом, видно, что в прямой интерпретации выборность и сменяемость власти могут противоречить друг другу, а поскольку только такая интерпретация может быть фактически измерена на реальных выборах, то они могут противоречить и в реальном мире. 
Необходимо искать баланс между ними, регулируя время каждого президентского срока и допущенное число сроков. 
Необходимо также учесть, что, как мне кажется, более вероятно встретить нечестного человека, пытающегося удержать за собой должность президента, чем честного, который старается улучшить страну и своими результатами заслуживает доверие граждан. 
То есть сменяемость власти требует большей защиты, чем выборность. 

Касательно нашей конституции, новая формулировка статьи 81 части 3 существенно лучше соответствует принципу сменяемости власти, а часть 3.1 -- наоборот. 
Ни один из этих пунктов не затрагивает выборность власти. 





\newpage
\begin{center}\bf
Направление законов в конституционный суд
\end{center}

Раньше, если президент отклонял федеральный закон, а госдума и совет федерации поддерживали его в той же формулировке, президент был обязан принять его. 
Теперь президент может в такой ситуации отправить его на рассмотрение в конституционном суде, и если суд ответит, что закон не является конституционным, его отклонят. 

Касательного конституционного суда, президент предлагает совету федерации кандидатуры председателя конституционного суда, заместителя его и судей конституционного суда, а также вносит предложения об их увольнении. Раньше президент предлагал только кандидатов судей конституционного суда, но и сам конституционный суд содержал только судей. Изменение однако произошло в их числе: было 19 судей, а стало 11. 

Сам конституционный суд разрешает дела о соответствии законов и прочего конституции РФ по запросам президента, совета федерации, госдумы, правительства, верховного суда и органов законодательной и исполнительной власти субъектов РФ. 

Очевидно, нельзя сказать, что новая формулировка стати 107 части 3 ослабляет президента.
Все судьи конституционного суда~-- предоставленные президентом люди. 
Это, естественно, не означает, что любое обращение президента в конституционный суд будет удовлетворено, но это, столь же естественно, дает положительный вклад в размеры области влияния президента. 
И то, что отклонение законов рассматривается в организации, находящейся в области повышенного влияния президента, усиливает его способность к отклонению законов. 






\newpage
\begin{center}\bf
Значимость государственной думы
\end{center}

Старая формулировка: \\
\textbf{1.} Председатель правительства назначается президентом с согласия госдумы. \\
\textbf{4.} После трехкратного отклонения представленных кандидатур председателя правительства госдумой президент назначает председателя правительства сам, распускает госдуму и назначает новые выборы. \\
\textbf{Статья 83 а:} Президент назначает председателя правительства с согласия госдумы. 

Новая формулировка: \\
\textbf{1.} Председатель правительства назначается президентом после утверждения его кандидатуры госдумой. \\
\textbf{4.} После трехкратного отклонения представленных кандидатур председателя правительства госдумой президент назначает председателя правительства сам и может распустить госдуму и назначить новые выборы. \\
\textbf{Статья 83 а:} Президент назначает председателя правительства, кандидатура которого утверждена госдумой по представлению президента.

Более понятную информацию предоставляет формулировка статьи 83 пункта а. 
В старой формулировке госдума давала согласие на назначение председателя правительства, представленного президентом. 
В новой формулировке госдума утверждает кандидатуру председателя правительства, представленную президентом, а после президент назначает утвержденного кандидата. 

В новой формулировке явно выписаны три шага: президент представляет кандидата, госдума утверждает его, президент назначает председателя правительства. Но и в старой были точно такие же три шага: президент представлял кандидата, госдума соглашалась, президент его назначал. Не изменилось совершенно ничего, относительно выбора председателя правительства значимость госдумы не повысилась, но и не понизилась. 




\end{document}


\hrulefill
Статьи 5 -- 19

Президент Франции.

Избирается на 5 лет. 
Он назначает премьер министра, с которым вместе может назначать и убирать других членов правительства. 
После передачи правительству окончательного закона президент может в течение 14 дней сказать им пообсуждать закон еще раз, они обязаны пообсуждать. 
По предложению правительства президент вправе передать на референдум многие вопросы. 
После консультаций с премьер-министром и представителями палат, президент может распустить национальное собрание, не чаще раза в год. 
Президент назначает на гражданские и военные гос должности. 
Послы и чрезвычайные посланники тоже идут от президента. 
Президент является главой	армии, он председательствует во всех комитетах и советах по обороне. Президент осуществляет право помилования. Необходима подпись премьер-министра везде, кроме назначения/увольнения премьер-министра, отправления закона на референдум, роспуска национального собрания, осуществления мер в критических ситуациях ниже.
Президент обвиняется при вынесении одного решения обеими полатами при публичном голосовании, необходимо абсолютное большинство. 

Под серьезной и непосредственной угрозой и когда нормальное функционирование гос властей прекращено, президент официально консультируется с премьер министром и конституционным советом, и после может делать что угодно. Об этом он информирует нацию. Главная цель -- предоставление конституционным властям возможности вновь функционировать. Средства обговариваются с конституционным советом. 


Президент России. 

Избирается на 6 лет. 
Президент с согласия гос думы назначает председателя правительства. 
Президент принимает решение об отставке правительства РФ. 
Дает гос думе кандидата председателя ЦБ и предлагает гос думе обдумать отставку его. 
Назначает и увольняет заместителей председателя правительства и федеральных менистров по предложению председателя правительства. 
Предлагает совету федерации кандидатуры судей Конституционного суда и верховного суда. Для других федеральных судов назначает сам.
Предлагает совету федерации кандидатуры ген прокурора и его заместителей, предлагает их увольнение. Сам назначает и увольняет прокуроров субъектов и других, кроме городских и районных. 
Формирует и возглавляет совет безопасности. 
Заведует военной политикой. 
Назначает и увольняет высших в армии. 
Назначает и увольняет после консультаций представителей страны в других странах.
Назначает выборы гос думы и распускает ее. 
Назначает референдум. 
Вносит законопроекты в гос думу. 
Верховный главнокомандующий армии. 
Президент решает вопросы гражданства и предоставления политического убежища. 
Награждает за заслуги. 
Осуществляет помилование. 
Президент неприкосновенен. 
Президент может быть обвинен только по инициативе 1/3 гос думы, после заключения специальной комиссии из гос думы, и обвинение принимается от 2/3 голосов в каждой палате. 

В случае агрессии или непосредственной угрозы агрессии, президент вводит в России или в каких-то субъектах военное положение, сразу сообщая совету федерации и гос думе. 
Режим военного положения нацелен на отражение или предотвращение агрессии против России.
Допустимые меры определены законом. 
Президент вводит чрезвычайное положение так же, и так же сообщая совету федерации и гос думе. 

