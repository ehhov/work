\documentclass[a4paper, 12pt]{article}

% Configuration {{{
\usepackage[utf8]{inputenc}
\usepackage[T2A]{fontenc} % T1 for English
\usepackage[english, russian]{babel}

\usepackage{enumitem}
\setlist{nolistsep}
\usepackage{mathtools}
\usepackage{xcolor}
\definecolor{dimblue}{HTML}{1010aa}
\usepackage[
	colorlinks=true, 
	allcolors=dimblue
]{hyperref}
\usepackage[
	vmargin=1in,
	hmargin=1in
]{geometry}
\linespread{1.3}
\usepackage{indentfirst}
\usepackage{graphicx}
\usepackage[multidot]{grffile}
\usepackage[labelsep=period]{caption}
\usepackage{subcaption}

%\usepackage{times} % for English
% }}}

\begin{document}

% Title Page & Table of Contents {{{
\null
\vfill

\begin{center}
	\begin{Large}
		\textbf{Философские вопросы естествознания}
	\end{Large}

	\vspace{\baselineskip}

	Яковлев Владимир Анатольевич

	\href{mailto:goroda460@yandex.ru}{goroda460@yandex.ru}
\end{center}

\vfill

Литература
\begin{enumerate}
	\item Яковлев, Гришунин, Философские вопросы естествознания.
		\\ Подробно расписаны темы курса, даны часы, учебно-методическое обеспечение там же указано.
\end{enumerate}

\clearpage

\tableofcontents

\clearpage
% }}}

\section{Введение}
% {{{

\hfill\textbf{Sep 6}

Всего восемь тем, но разное количество часов на каждую. Натурфилософские 
программы античности, генезис науки. Схоластика и наука в средневековой 
Европе и восток. Возрождение. Немецкая. Антисцеинтистские. Марксизм. 
Философия русского космизма. Вопросы в контексте философии науки. Лекции 
идут онлайн, а семинары могут как так, так и так, пока не запрещено. 
Структура семинаров естественно от дистанта не зависит, потому что 
приборы и лаборатории не нужны. Доклады, презентации, коллоквиумы, 
обсуждение вопросов с лекций. Главное конечно результат оценочный это 
экзамен, формат его остается пока неопределенным, в прошлом году было 
жестко, не было очных экзаменов, оценки ставились по результатам работы 
в семестре. Сразу следует завоевывать баллы, очки и подобное. Курс 
называется философские вопросы естествознания, и понятно, что он имеет 
базу курса философии с того старого курса. Будет в ходе нашего курса 
вспоминать наиболее важные вещи из того курса, не просто чтобы 
продолжать логику развития идей, но и имея в виду определенный 
прагматический эффект. В аспирантуре уже идет курс по именно философии.

% }}}

\section{Натурфилософские программы античности и генезис науки}
% {{{

Философия появляется в 7--6 веках до нашей эры в двух ареалах, восток 
и древняя Греция. Это потом назовут осевым временем эпохи. Были 
философские вещи и в Египте и Вавилоне, но в абстракции именно там. 
Тогда мы делали акцент на древнегреческой философии. Потому что несмотря 
на все важные фрагменты восточной философии, наука появляется именно 
в греческой. Главное все равно, что в восточной философии у-вэй, принцип 
минимизации воздействий, но в принцип соревновательности на деле не 
выходит. Кто приобщился к Дао, уже больше ничего не надо. В Индии было 
мнение, что богиня скрыла всю информацию от них покрывалом, а кто открыл 
уже все понял. Тем не менее, на востоке как раз научная революция 
и Ньютон и остальное, что мы и называем теперь наукой. Однако философия 
из древних источников послужили фундаментом. Основным принципом в Греции 
был принцип агона, соревнований. Та россыпь идей, предложенная там, 
и приводит к науке, называемой натурфилософией. Отличие между ними 
формируется лишь в 17 веке, когда к науке применяется математика, 
а философия остается на качественном уровне. В философии как помнится, 
существует три раздела, онтология (о бытие), гносеология (о познании), 
аксиология (о ценностях), праксиология (о практической жизни) иногда 
добавляют. Научные идеи по мере появления можно классифицировать по этим 
трем первым, увидим как зарождаются и находят подтверждение 
в средневековых отклонениях и в конце концов появляются в науке.

\paragraph{Онтология}
Вспомним для начала онтологию. Как устроен мир. Проблема начинает 
интересовать Греков, из чего устроен мир, принцип архэ, что же лежит 
в основе мироздания. Философская физика так сказать то же самое 
спрашивает у себя в общих чертах. Первыми вспомним Фалеса, архэ у него 
это вода, дальше он предсказал солнечные затмения 500-году до нашей эры. 
У учеников его архэ это апиерон похожий на непонятно что, нечто парящее, 
откуда все идет и куда возвращается. У Анаксимена это воздух. Дальше 
возникают идеи, что люди вышли из больших рыб из океана, пытались 
осмыслить землетрясения, метеорологией. Далее вспоминаем Гераклита 
с тремя великолепными идеями, во-первых что архэ это огонь, откуда 
и куда всё вообще, потом Бор или Гейзенберг говорил, что если огонь на 
энергию поменять, то в общем правда. Затем Гераклит сказал с 18ХХ лет 
вселенная возвращается в огонь. Далее его идея диалектики объективной, 
мир устроен противоречивым образом, все возникает, раздваивается, борьба 
и взаимодействие противоположных начал, нельзя войти дважды в одну реку 
его слова. Мир устроен диалектически, то, о чем будут говорить серьезно 
уже в 19 веке. Третья важная его идея логоса, не мне, а логосу внимают. 
Логос это закон. Гераклит говорит не библейскими словами. Говорят, что 
в восточной философии понятие закона как научного так и не появилось. 
Закон, который не зависит вообще от людей, которому нельзя не 
подчиниться, такого закона понятия нет в восточной философии, а у нас мы 
благодарны Гераклиту. Дальше идет элейская школа знаменитая, важнейшее 
понятие для современной философии -- бытие. Бытие есть, а небытия нет, 
наука имеет дело с бытием, что бытие вечно, неизменно, идея элеата, что 
бытие как сплошность, никакого разрыва. Эмпедокл дальше -- первая 
серьезная теория эволюции, трансформация видов, монстры, гермафродиты, 
потом уже жизнеспособные формы. Кроме того он сказал, что скорость света 
конечна. Его архэ -- земля, вода, воздух, огонь. То гармония, космос, то 
по принципу вражды все распадается, космос живой и постоянно 
развивается. Дальше Анаксагор, первый философ, которого приглашают 
в Афины за его мысли. Все эти звезды на небе оказывается не боги, 
а физические тела, которые даже падают иногда. За это его конечно 
выгнали с Афин, но хоть не убили. Пилопонес сказал что Солнце самый 
большой объект тут вот, у него была идея, что все делится до 
бесконечности, а там гомеомерии, все решается количеством их, прообраз 
молекул. Ну и наибольшее влияние оказала школа атомизма. В отличие от 
анаксагора атомы уже неделимы, разнообразие мира объясняется формами 
атомов. Горькие, сладкие итд. У них атомы и пустота, атомы носятся по 
орбитам в пустоте, за счет разнообразия форм создают тела, которые уже 
могут распадаться. Фейнман вот сказал, что это очень важно и помогло бы 
восстановить. Причинность у них противоположность случайности. Говорит, 
что люди измыслили идол случая, мы просто не знаем всего, Демокрит 
говорит вот это про черепаху падающую на грека. Гносеологический статус 
случайности. В самом мире никаких случайностей нет, просто мы не все 
знаем. Третья важная идея в том, что наш мир совершенно не единственный, 
может быть много или даже бесконечно, Бруно потом скажет, что это может 
быть и не самый лучший мир. У Ньютона все идет по Демокриту можно 
сказать, но вот через пару веков появляется Эпикур, он тоже преверженец 
атомизма, но говорил, что лучше бы молился богам, чем принимать жесткий 
детерминизм. Но ему не хотелось уходить от идеи, что мир состоит из 
атомов. Он хотел ввести свободу воли, ввел клинамен. У него это несло 
характер случайность, и это не есть незнание, а есть в самом мироздание, 
появляется проблема онтологического знания случайности. Дальше идет 
великий Платон, который разработал само понятие идеи, все есть идея 
прежде всего, а только потом может становиться вещью. Существует царство 
идей, а мир это лишь отражение тех идей. Мир это не то, что нам кажется, 
на самом деле он совершенно другой. Квантовый мир это реальный, а наш 
мир лишь отражение того. Платон такое прям не знал, но вот высказался, 
и это подталкивало многих ученых на поиски того самого. Пифагорейцы 
у них архэ это число, сущность мироздания это число, правильные 
геометрические фигуры, идею развивает Платон, мир это правильные 
геометрические фигуры: куб, октаэдр, тетраэдр, икосаэдр и так далее. 
Эксемей говорит, что фигуры можно разбить на правильные и неправильные 
треугольники, и мир получается состоит из треугольников. Это такая 
Платонистская программа развития физики. В наше время знаменитый Рождер 
Пинроуз пишет, что вещи как уравнения Максвелла и множества Мандельброта 
существуют онтологически, а мы потом лишь догадываемся. Мир одушивлен 
получается, время это подвижный образ вечности, Земля в центре 
мироздания, все эти идеи важны для последующего развития его учеником 
Аристотелем, который и создает первую всеобъемлющую физическую 
и космологическую картину мира. Дальше очень очень долго мир развиваться 
еще будет на основе его этого. Мир у нас такой земной, где нельзя ничего 
точного сделать, а есть мир небесный, где эфир, там можно все, там 
точное все, Земля в центре мироздания и даже если бы хотела из него 
выбраться, то все равно бы оказалась там же, потому что тяжелые тела 
стремятся к центру, земля и вода поэтому образуют оболочку земли, 
а воздух вот выше. Движение получается все насильственное, поэтому нужно 
чем-то запустить космос. Есть материя прима, лежащая в основании этих 
четырех элементов. Есть еще причина формы, самая важная, позволяет 
различать вещи, есть форма шара в каждом шаре, хоть на земле нет и ни 
одного идеального шара. Платон и Аристотель на фоне школы, Платон 
указывает наверх, что весь мир там, а у Аристотеля все внутри тел, он 
указывает вниз. Мир такой, что существуют программы мироздания, 
энтелехиальной, телеология, только целесообразное существует. Сначала 
она отбрасывалась наукой, а потом после теории большого взрыва 
вернулось. По типу а как так константы наши все сложились. Аристотель 
создает и космологическую картину мира, и физическую, где вот сначала 
классическая наука от нее отвернулась, а современная наука снова пришла 
к идее, что природа боится пустоты. Сейчас мы говорим конечно, что 
никакой пустоты и нету, или например что в прима-материи есть нечто 
потенциальное бытие, тоже важная идея, которая в современной физике 
более чем рассматривается, все эти виртуальные частицы, ложные вакуумы 
и остальное. Материя не просто существует, а есть такие вот 
потенциальные вещи, которые могут появляться и исчезать. Отделение 
органической природы от неорганической. Проблему отделения их он уже 
осознает, не как у Фалеса магнит имеет душу. Аристотелевское 
представление о том, что наш мир единственный, замыкается определенным 
количеством сфер, а дальше них только господь бог, который и ответствен 
за три причины, материальную, формальную и целевую, существования мира. 
Это же легло в картину мира Клавдия Птолемея и существовала вплоть до 
Коперника и Галилея. Довольно неплохая система, хоть и понятно 
неправильное. После Аристотеля уже упадок в целом, хоть и есть такие 
проблески типа Эпикура с клинаменом, стоиков с пневмо, огнедишушей 
смесью, проникающей во все тела, подчиняется таким классическим циклам. 
Тут все идет по кругу, жесткие причинно следственные связи, не сгорает 
все как у Гераклита. Собака привязана к пояснице и ее свобода зависит от 
длины поводка и скорости колесницы. У нас грубо говоря это все 
конфайнмент примерно напоминает. 

\paragraph{Гносеология}
Метод критического рационализма появляется в философии и потом 
передается науке. Без этого ни философия, ни тем более наука не 
существовали бы. Ученик слушает учителя, имеет право его критиковать 
и предлагать свое. Учитель должен подталкивать на размышления. Ученик 
это не сосуд, который надо наполнить, а факел, который надо зажечь. 
В отличие от религии это. Это появляется у Греков в такой открытой форме 
и это по сути тот же агон принцип соревновательности. В более конкретном 
случае у Элейцев это апории Зенона, попробуйте это доказать. Очень 
важная вещь. У Пелейцев важнейшая гнос. идея в том, что есть обыденные 
знания, а есть научные. Обыденные это наше мнение, а научные это 
эпистеме, которая может существенно отличаться от обыденного знания. 
В современной науке конечно это правда тоже, требуется много усилий для 
осмысления теор науки. Дальше у Демокрита разделение чувственного 
знания, где первичные качества и вторичные качества. Познания начинаются 
с органов чувств, но есть существенные отличия. Количественные вещи типа 
формы размера протяженности они совпадают, а вкус, цвет, запах это уже 
диспозиция и ощущения возникают в результате взаимодействия органов 
чувств с субстанциями. Очень важно для науки выделить чувственную 
ступень познания и более высокий уровень а именно первичные качества 
человека. Дальше у Платона можно снова идею Диалектики, Платон как 
помним ученик Сократа, Сократ воюет с софистами, у кого только 
практический результат рассуждений. У Сократа маевтика, искусство задать 
вопрос, и вообще все в конце должно прийти к концу к итогу, который 
может в общем-то отличаться от какого-то очевидного. Платон развивает 
этот метод -- дискуссия, где предмет рассматривается с противоположных 
и разных сторон, которые помогают прийти к такому полноценному описанию 
предмета. Платон создает первую классификацию наук, где на первое место 
ставится диалектика как искусство, но есть место и физике, и астрономии.

Самый большой вклад в теорию познания вносит конечно снова Аристотель. 
Он вместо диалектики выдвигает аналитику. Не важно как идеи пришли, 
важно как ты можешь изложить их другим. Для этого нужно обладать 
определенными навыками, что он делает в органоне, знаменитое 
произведение, строить определенные модели изложения своих мыслей. Он 
изложил модель, по которой в итоге и строятся все научные статьи. Это 
самый общий подход аналитический, а вот сам он основывается на 
Аристотелевой логике. Законы тождества, исключенного третьего, 
недопустимости противоречия. Он уже создает развитую классификацию наук, 
философия, математика, физика, творческие науки искусство поэзия музыка, 
практические экономика ... Говорят, что Аристотель панлогист, законы 
логики в самом мире. Спекуляции не могут привести к консенсусу, потому 
что нужны опыт и эксперимент. Там были какие-то, но вот тогда еще 
считали, что мыслитель не должен практичными вещами заниматься, максимум 
математикой.

\paragraph{Аксиология}
Цель познания -- приобщение к гармонии мира -- Аристотель. Анаксагор 
занимался тоже чтобы созерцать звездное небо и восхищаться. Это конечно 
важная ветвь науки, постигать гармонию мира, но классические науки 
начались уже когда Бэкон заявил, что для результата наука все же. Другие 
же говорят, что такие философские вещи это лишь игра ума. Закон Архимеда 
конечно существовал, но корабли строились не по нему. Созерцательность 
древнегреческой науки не позволила стать полноценной наукой, и поэтому 
наверно она и стала отброшена на 1000 лет и вспомнили только. 
Рецептурно-технологическое знание лежит в основе научного, но не оттуда 
проистекает наука. Это знание по сути повседневный опыт. На основе него 
люди овладели огнем, научились строить корабли, здания, разводить скот 
и вообще остальное, включая даже плавку металлов. Незнание сущности 
происходящих процессов не запрещает проводить. Рецептура, технология от 
отца к сыну, смотри на меня, делай как я, если можешь делай лучше, но не 
спрашивай почему оно. А наукой называют момент, когда объект осознается 
в теоретической форме. Это одна сторона генезиса. Вторая сторона это 
социально-политические условия. У Греков была полисная демократия, 
власть избиралась, свобода мысли полностью присутствовала в греческой 
культуре, уровень экономики был уже немал, рабы конечно были, но 
в основном за труд платили.

Вот и на основе всего можно сформировать программы натурфилософии. 
Первые это поиски архэ, основы мироздания, начинается с четкого 
разделения материализма и идеализма, сущность мира идеальна или в основе 
материализм. В современном появляется еще информационное, Джон Уиллер, 
разрабатывает его, все состоит в конечном счете из информации, природа 
математических объектов грубо говоря остается. Дальше программа 
структурности мироздания, либо атомы и пустота, дискретность, либо 
сплошность, как говорили природа не терпит пустоты. Третья важная 
проблема причинности, обсуждается и сейчас, или мир устроен однозначным 
образом, жестко детерминестически, все предопределено а мы лишь не 
знаем, либо мир устроен вероятностным образом, и вероятность это 
сущностная характеристика самого мироздания, известная дискуссия Бора 
и Эйнштейна это отражение того, что у Греков уже было в общем-то. Может 
бог и не играет, но мы можем видеть только с вероятностной точки зрения. 
Космологическая программа, или мир единственный возможный, с центром 
в земле прям, или есть и другие миры, как Демокрит говорит, или как 
Аристарх Самосский начнет позже говорить. И последнее программа 
происхождения и сущности жизни, либо жизнь повсеместно, мир одушевлен, 
но только в разной степени, или же существуют какие-то принципиальные 
барьеры, перейти которые мы и сейчас не можем, но природа вот как-то 
перешла. Переходит ли она постоянно эти барьеры тяжко сказать. На самом 
деле можно еще присовокупить движение пространство форма потенциальность 
категориальный аппарат, без которого обойтись не может современная 
наука, все возникло тогда. Про движение борьба присуще ли движение само 
этим или есть какой-то движитель, который не может быть материальным, 
иначе вопрос а что движет им.

Все эти программы возникли за короткий промежуток времени, в небольшой 
области на Земле, но в итоге полноценно попадает в науку.

\hfill \textbf{Sep 13}

Античная философия созерцательная, приобщение к вселенной. Тем не менее 
сформировались некоторые теоретические программы, актуальные и сегодня. 
Например, поиски архэ, истока всего. Здесь главная проблема была 
материализм и идеализм, в современном математические структуры. Дальше 
программа структурности, дискретность или континуальность.

% }}}

\section{Средневековье}
% {{{

Вопросы, поставленные в средневековое время. Хронологически это занимало 
аж 13--14 веков. Упор был на теологию, а философия -- служанка религии. 
В общем-то все переродилось в религиозные дела, и в целом критический 
рационализм не пропал, потому что появляются разные ордена, направления 
христианские, спорили всегда. Дальше появляются номинализм, 
концептуализм, ... И возникают какие-то методы. Христианство появляется 
как ответвление иудаизма, где не были евреи сверхнародом, а все равны. 
Появляется также страх перед богом, перед адом. Однако были выражения 
умножая знания, умножаешь скорбь и прочее, но были и другое, что 
образованный по сравнению с необразованным как свет и тьма. 
В христианстве ветхий завет, где бог карает всех подряд, но в новом 
завете бог любовь, сам жертвует, богочеловек. Равенство тут принимает 
фундаментальную ценность, равенство людей, равенство перед богом, только 
бог решает, что в конечном счете. Дальше отношение к греческой культуре, 
поздний платонизм в раннем христианстве "против язычников", платонисты 
пишут против христиан. В общем образуется две линии, одна что в библии 
вся мудрость Тертулиана, Клемент Александрийский говорил же, что 
в греческой культуре много что можно позаимствовать.

Теперь становление основных принципов, постулатов, патристика -- отцы 
церкви, и схоластика (конец 5--6 веков) комментируют, рассуждают, 
мыслят, интерпретируют, герменевтический подход, что по сути потом 
пришло в науку. Христианство как нам известно в конце концов 
раскололось, поскольку произошли события 410 год Рим пал, Константин 
столицу перенес в Константинополь ну и потом. Вот и в итоге они вообще 
раскололись. Главная проблема -- филиокве, происхождение духа божьего, 
католики говорят, что дух происходит от бога-отца и бога-сына, 
а православие говорит, что только от отца. 1967 год произошло 
примирение, патриарх встречался с папой. Ну и третья есть 
протестантство, ну и помимо этого куча сект, которые иногда даже 
запрещают в государствах, ну вот такое христианство многообразное. 
Сейчас считается, что служить богу можно по-разному, но неплохо бы 
прийти к общему. Вот на этом фоне начинают развиваться новые мысли, 
новые веяния, научное познание. Здесь из отцов церкви выделяли Августина 
Блаженного, вспомним и сейчас основные идеи его. Это принцип 
методологического сомнения, он опережает даже Декарта в этом, для 
Декарта все начинается с этого принципа, но Августин приходит раньше. 
Критическое мышление таково, что можно усомниться во всем, кроме 
сомнения, потому что я могу сомневаться во всем, и это истина. Но 
поскольку есть вот такая истина, и вообще человек по подобию бога, то 
можно познавать мир, но вместе с богом и с любовью к богу. Критический 
рационализм начинает воплощаться в критике самого мышления человека, 
может ли человек достичь какой-то истины путем лишь размышлений, 
оказывается может. Второе -- если такой подход возможен, то возможны 
и другие истины, существующие независимо от человека, а человек лишь 
открывает. Августин приводит цитату из Библии Соломон говорит, что мерой 
числом и весом, ими бог творил мир. Следующее важное положение -- 
онтологическая проблематика, творение самого мира, в античной философии 
у того же Платона есть идея творения, есть субстанция Хора, не зависащая 
от бога Димиурга, и на основе Хоры бог творит вселенную. У Аристотеля 
была материя прима существовала независимо от бога. В самой библии нет 
ясного отчетливого положения, из чего бог творил мир, с помощью чего да, 
те три вещи, а из чего? Августин заявляет, что бог творил мир из ничего, 
в этом его креативная сила, это возможно только для бога. Это 
принципиально, можно найти немало рассуждений и у наших великих физиков, 
которые говорили, что в конце по-другому быть и не может. Бог, творя 
мир, запускает время как механизм развития, и помещает его 
в пространство. Все остальное является вторичным по отношению к этому 
ничего. У нас есть и многомерное пространство, и время появляется только 
на определенном этапе, у Платона как помним подвижный образ вечности. 
Бог в библии кстати останавливает солнце, но не время. Образуется 
представление о времени как стреле, это не собака за хвостом итд, но раз 
стрела, есть начало, должен быть и конец, страшный суд, время 
остановится, произойдет суд, грешники в ад, праведники в рай, совсем 
другие события. Идея стрелы времени была брошена. В физике время 
существует лишь в термодинамике, в других местах возможно его обратить. 
Время появляется на момент создания. Как ни удивительно, у Августина 
есть и еще одно представление времени, релятивистское, для отдельного 
субъекта. С точки зрения него время всегда время настоящее, нет стрелы, 
не можем измерит что-либо в прошлом или будущем, все начинается здесь 
и сейчас. Существует только настоящее настоящего, прошлое мы лишь 
вспоминаем что было в прошлом, но мы вспоминаем здесь и сейчас, так что 
прошлое это опять настоящее прошлого, наша память. А будущее опять мы не 
там, мы здесь, можем строить планы, но мы все равно тут, и настоящее 
будущего, будущее настоящего, существующее в нашем воображении. Августин 
отрицает субстанциальность времени, оно зависит от самого субъекта, 
а остальное становится релятивным. У Ньютона, а потом у Эйнштейна тоже 
такие дела  про время.

Дальше философские и теологические направления скажем все-таки. Реализм, 
номинализм, концептуализм. Речь идет фактически о том, как образуются 
общие понятия. Откуда берутся общие понятия, абстракции. Современная 
наука оперирует абстракциями высокого порядка, время там, есть ли на 
самом деле или это выдумка человека. Этот вопрос начался вот 
в средневековой философии, вспомним немного. Реализм это Гельем Изшампу 
Ансельм Кентерберийский! Исходят из Библии, вначале было слово у Бога 
бог, но если вначале было слово, то оно имеет онтологическую основу, 
мысли бога, которые выразились в словах например да будет свет, первичны 
ко всему существующему, а значит, где-то там онтологическая значимость. 
Все реально наблюдаемое и ощущаемое имеет вторичный характер. 
Кентерберийский предлагает онтологическое доказательство бога (затем 
названо так Кантом). Бог это совершенное существо, а раз оно 
совершенное, то ему априори должен быть и атрибут существования, потому 
что если не приписан, то не совершенное существо. У нас в наше время из 
концептуального сначала понятия, а потом поиски на основе понятия 
существования, природа общих понятий. Вторая линия -- номинализм -- 
реально существуют лишь те объекты, которые мы воспринимаем органами 
чувств, а все эти абстрактные понятия -- лишь сотрясения воздуха. Грубый 
пример -- что первичнее понятие лошадности или реальные лошади. 
Концептуализм получается серединная линия, Пьер Абиляр с произведением 
"Да и нет", где он собрал противоречия из Библии и попытался примирить, 
ничего не вышло у него, но зато вышло вот что. Есть три вида бытия, 
единичных объектов (физическая реальность), бытие, которым владеет 
господь бог (замысел бога достичь нельзя), а третье -- бытие нашей 
мысли, питаются из эмперии, но могут внушаться самим богом. Итак, три 
философский направления о том, какова природа философских понятий.

Дальше вспомним еще одного схоласта, Фома Аквинский, внес важный вклад 
в понимание основ научного познания, хотя конечно под эгидой бога. Бог 
конечно везде присутствовал, но тем не менее рациональность этого 
направления христианства не вызывает сомнений. Главное у него то, что он 
не согласился с доказательством существования исходя из сущности, бога 
можно доказывать только косвенно. Он по сути берет ту же систему 
доказательств, что и Аристотель, но причесывает Аристотеля под 
христианство, если согласно физике Аристотеля все движется чем-то, любой 
объект должен иметь первоначальный толчок от другого объекта, то либо мы 
приходим к дурной бесконечности, либо вносим перводвигатель, причем он 
не может быть материальным. Доказательство необходимости, необходимость 
и случайность, должна быть высшая необходимость. Дальше принцип 
совершенства, все в мире носит такое качество, самое совершенное -- бог. 
Целесообразность, мир устроен целесообразно, иерархически, оно все было 
бы невозможно без господа бога, устанавливающего цель всего мироздания, 
устанавливающая цель остального. У Аристотеля называлось фактически 
интелехией, интелехиальная причина. Фома Аквинский также принимает идею 
первоматерии, но согласно с Августином материю-прима создал сам бог, 
и чтобы из нее что-то появлялось нужна добрая воля бога. Надо различать 
материю как чистую потенциальность, а вот разрешение на появление из нее 
чего-то и дальше усложнений, перевод из потенциальности в актуальность 
происходит лишь по воле бога. Само по себе не может, материя сама чистая 
потенциальность, пассивна, а вот толчок от бога. Это онтология. В плане 
гносеологии тоже следует по стопам Августина, что бог наделяет человека 
разумом, а истины могут носить или характер откровения или характер 
рациональности. Нельзя рационально познать идею создания мира, нельзя 
рационально познать как бога можно распять, понять идею три-единства 
мира отец сын дух, понять непорочное зачатие. В это можно и необходимо 
верить, человек должен уверовать. Но есть истины, которые человек может 
с точки зрения разума прийти. Бог хотя и бог, но он уже и не может 
изменить что-то. Не может изменить прошлое, не может отобрать душу, не 
может даже изменить сумму углов треугольника. В своих произведениях, 
сумы, Фома Аквинский все это подробно обосновывает, и через век-полтора 
посчитали святым, а его учение стало ортодоксальным учением по 
преподаванию начала 20 века. Он конечно тоже как Пьер Абиляр умеренный 
концептуалист, потому что и то, и другое считает крайностями.

Вспомним в позднем средневековье теперь тех, кто стал основоположником 
мира. Это Роджер Бэкон, кто первый провозглашает необходимость 
проведения экспериментов и роль математики, чтобы человек был уверен 
в познании. И Уилльям Окком, бритва Оккома, не надо умножать сущности 
без необходимости, не стремиться доказывать большее меньшим. Ньютона 
дальше будут ругать, что силу притяжения тоже никак не это самое не 
объяснить. Силу материи тоже можно рассматривать как актуальность, на 
равных с самим богом, ибо он ее создал, идея такова, что сама материя 
создана богом, и поэтому обладает внутренней активностью. Дуализм 
появляется такой, что бог создает мир, законы в этом мире, и не руководит 
и вмешивается, но законы уж. Чудеса возможны, но уж очень редки, 
и фактически бог не вмешивается, это несистемно. Для бога чудес быть не 
может. Итак, это схоластическое направление, которое приближает 
схоластику самоактивности материи, движение, возможность человеческого 
разума познавать мир, поскольку бог дал, человек может что-то сам 
познавать. Роль математики возвеличивается и понимание, что в науке 
должны проводиться эксперименты, хоть Бэкон не проводил, но идея была. 
Интеллектуальный фон тоже эволюционирует и таким образом средневековье 
подготовило тот отряд людей, который потом назвали интеллигенцией. Все 
названные нами люди конечно не всех мы назвали, появляется образование. 
В средневековье появляется структурированная система образования. Шаг 
назад мы видим в античности такие очаговые дела, Милецкая школа, 
Пифагорейский союз чуть более структурированное дело, существовало около 
1000 лет, Пифагора самого обожествляли. Конечно академия Платона, 
структура появляется в 4 веке до нашей эры и существует можно сказать 
до конца существования, хоть ее потом император распускает. 
Александрийская школа под покровительством Птолемея, где был Архимед 
и Евклид например, но широкой сети не было. В средневековье появляются 
городские школы, школы при монастырях, важно, поскольку фактически 
в монастырях храмах тоже очаги культуры, там учили математике, 
письменности, физику Аристотеля, а в 9 веке появляются первые 
университеты, в Болонии, Париже, Италии, и в этих университетах уже 
складывается та система, которую мы видим сейчас. Факультет медицинский. 
юридический, свободных искусств будущий философский и конечно 
теологический. В отношении факультетов нужно сказать, что тот 
философский считался подготовительным, но именно там разрешались 
дискуссии, этот факультет прежде всего и сохранял дух критического 
рационализма, без которого наука конечно немыслима. Сами по себе вот эти 
ордена христианские находились в дискуссиях, и несмотря на войны эти, 
образованные люди могли перемещаться и вести дискуссии такие, появился 
язык, отличающий образованного человека от обычного, латынь. Надо было 
его использовать в этих дискуссиях конечно. Эти университеты помимо 
деления на факультеты имели структуры внутри факультетов, бакалавры, 
магистры, доценты, профессора, экзамены, оценки, защита работ публичная 
для получения той или иной степени, все это сложилось в то время, и без 
этого не могла бы существовать наука, потому что наука сама себе готовит 
кадры; для этого нужны вот такие вот структуры, появившиеся как раз 
в средневековой Европе. Также должны сказать о том, какого рода проблемы 
обсуждались в поздней средневековой европе, которые стали потом 
и научными проблемами. В 1277 году Итьен Тампье, ректор Парижского 
универа, выступил и опубликовал почти 300 тезисов, чтобы показать 
возможность другого рода рассуждений о физике мира, чем у Аритотеля, 
который тогда был тогда авторитетом, но считался все еще язычником где 
там везде. Дюге сказал, что классическая наука началась с Тампье потому 
что он первый пошел против физики Аристотеля. Две его наиболее важных 
посыла, утверждения. Круговое движение у Аристотеля главное, а Пампье 
говорит не идеальна. Система планет была запущена сначала прямолинейно, 
а потом пришел к кругам. Дальше космос сказали может быть не ограничен 
названными планетами, не как у Аристотеля. Это были конечно еретизмом, 
но он ректор, и поэтому церковь не пошла сильно против. Теория импетуса 
Генрих Генский и еще два чела обсуждают и развивают систему силлогизмов 
Аристотеля, кинематику, природу движения. У Аристотеля все движется 
чем-то, но движение понимается как, это среда, в которой движется тела, 
и это среда, которая не терпит пустоты. Первоначальный импульс задается 
среде, а тело движется по образу как пловец в воде, когда импульс среды. 
Первоначальный импульс дается телу уже в теории импетуса, и если бы не 
было среды, то тело бы двигалось равномерно и бесконечно. Тоже важная 
физическая проблема, обсуждавшаяся тогда теми учеными. Проблема теперь 
пустоты, как понимать пустоту, это вообще ничто или все же нечто, 
и может ли бог создать такую пустоту, где не будет даже его самого, 
полную пустоту, активно обсуждается. Или пустота это то, в чем 
существуют тела, или пустота это промежуток между самими телами. Бог 
находится за пределами пустоты, может ее создать, но в нее не попадает. 
Обсуждаются и даже такие проблемы, которые у нас даже не считаются 
проблемами, но были, там типа сколько ангелов может поместиться на 
острие иглы, или там какой пол у этих ангелов. И вот последний момент 
это то, что важные научные идеи появляются в это время на востоке. 
В Богдаде появляется дом мудрости, Алгоритмус, решение уравнений третьей 
степени, знакомство с системой Птолемея через арабские переводы, шли 
войны, но и коммуникации, переводы с арабского на латынь, Авецена тоже 
восток, Омар Хайям еще и математик был. Обсерватории в Европе запрещены, 
а там были. Ну и дальше самое главное наверно это теория двойственной 
истины. На востоке есть Коран, и две мысли что читать можно буквально, 
а можно иносказательно, интерпретировать, и некоторая свобода мысли. 
Теория двойственной истины, сибир гобанский отстаивает. Библию тоже 
можно понимать и так, и как книгу сотворения, книга природы, где там 
свой язык. Книга природы написана на языке математики там потом скажет 
челик. Ну и на востоке тоже появляются цифры арабские, на деле 
индийские, вот это позиционное исчисление. В Европу пришло оно с большим 
трудом в 13 веке, а на востоке в 9 уже все супер.

\hfill \textbf{Sep 20}

Переход от космоцентризма к теоцентризму. Вспомнили ъристианства, вера 
иррационализма, в конечном счете побеждает рациональный подход. Схоласты 
обсуждают проблемы, дискутируют, все это имеет историческую подоплеку, 
развивается. Главные проблемы номинализм и реализм. Первичные или 
единичные вещи, понятия это сотрясение воздуха. Ансельм Кентерберийский 
пытается доказать существование бога как совершенное существо, Окком, 
Блэк, Пьер Абиляр говорил, что понятия реально существуют, но только 
в голове человека. Потом Фома Аквинский выделил истины веры и истины 
разума, в одни можно только верить, а другие можно доказывать, поскольку 
человека создали по образу и подобию своему. Кроме того его идея была 
в потенциальной материи, реализуемой по воле бога. Дунс Скотт говорит, 
что материя уже актуальна. Потом был Николай Резм, Шан Буридан, Генрих 
Генский, 1277 год парижский епископ публикует свои. В позднем 
средневековье Европа существенно отставала от востока, где решались 
уравнения третьей степени, были обсерватории, открывались элементы 
(сера). Многие труды греков были известны арабам, а европейцы потом 
переводили это дело на латынь. Позиционное исчисление уже использовалось 
на востоке, а в Европе все еще латинский.

% }}}

\section{Философские вопросы естествознания в эпоху возрождения и в новоевропейскую эпоху 17--18 веков}
% {{{

Тема в принципе нам знакома, ренессанс, обращение к своим корням, 
древнегреческой и римской культуре, средневековье клеймится как пустыня 
во времени, мрачное средневековье. Но это конечно не так, средневековые 
мысленные наработки послужили основой для возрождения. За 10 веков не 
было особых достижений, но вот прошло еще два века, и невероятный 
прогресс. Довольно удивительное явление в целом.

Прежде всего в сфере экономики, от ремесляничества происходит переход 
к мануфактурному производству, разделение труда, производство 
разбивается на множество этапов. Увеличение производительности труда, 
а как мы знаем еще из Греции, это дает возможность какой-то группе людей 
заниматься умственным трудом, искусством. То, что складывалось в позднем 
средневековье, переходит и раскрывается вот тут в возрождении. Группа 
людей, называемая интеллигенцией. Дальше можно назвать причину важную, 
технические новации. И использование уже природных сил, например ветра, 
воды, и непосредственно использование новаций из востока это компас, 
порох, книгопечатание. Об этих новшествах говорят все, в том числе 
Френсис Бэкон. Компас позволяет плавать в открытом море, даже не зная 
почему стрелка показывает так, а не иначе, все равно стали плавать, 
колонизировать. Порох тоже предельно изменил способ ведения войн, копья, 
рыцарство и прочее стали ненужной вещью, пушки стали большой частью, 
а там рассчитывать в общем надо. Призывает теоретизировать по поводу 
конкретных вещей. Ну и книгопечатание, информационная революция третья. 
Первая -- язык, вторая -- письменность, третья -- массовая письменность, 
книгопечатание. Раньше переписывали библию, потом вот она стала первой 
книгой, напечатанной вот так. Был конечно список плохих книг, но только 
пост-фактум сделать. Церковная цензура сильно ослабла, и это стало 
необходимым элементом для более активного общения, написания трактатов. 
Дальше на самом деле очень важно еще такие кросс-культурные 
взаимодействия. Войны шли не только внутри Европы, но и с тюрками, 
с арабами. В христианстве возник раскол, Рим стал католическим, Византия 
православной, потом ее захватили. Многие образованные греки мигрировали 
в Европу, Итальянские богатые города, Венеция, Флоренция, Пиза, с собой 
принесли немало книг. Древнегреческий в то же время стал языком науки, 
появляется целая наука, известная сейчас как филология. Все, что мы 
знаем о древней Греции и Риме, всем мы должны этой первой академии. Это 
было тоже точкой бытовых вещей, конец шелкового пути из Китая, поэтому 
еще богаче, и могли позволить умственный труд. Важно дальше подчеркнуть, 
что все эти новации, появление часов на ратушах диктовало новый ритм 
жизни, время деньги, бог -- банкир, раздающий людям время. Время 
становится не такой абстрактной категорией, а вот так вот. Это 
предпосылки.

Дальше видим появление эффектов, происходит резонанс, во главе угла 
становится проблема человека. Если раньше писали о ничтожестве 
человеческой жизни, презрении, то потом пишут о величии человека -- 
людей таких стали называть гуманистами. Перед этим было надо отметить 
был новый поворот в сфере искусства. В средневековье была плоскостная 
живопись, архитектура такая уж, а появились шпили, устремленные 
в космос, в литературе появляются новые жанры (Божественная комедия), 
живопись эта вся да Винчи, Рафаель. Закон перспективы открывается тоже 
не в физике, а в этих полотнах, преимущественно да Винчи. Первый шаг 
делается в искусстве. Мишель Монтень возрождает античный скептицизм, 
Эразм Роттердамский пишет еще хвалу глупости или хз как. Подход с точки 
зрения эксперимента, изучение природы, теоретики уже философы и ученые, 
это конечно Никола Кузанский, епископ, его основной труд -- об ученом 
незнании. Тоже намек на Сократа. Труд этот стал основой дальнейшего 
развития естествознания. Основные идеи его: использование математических 
символов, опирается на две догмы христианские, одна исходит из двух 
подходов описания природы катофатика и апофатика. Стремление описать 
Бога позитивно, основываясь на хороших вещах из библии, а апофатика 
говорит, что бог -- таинственный мрак, мы можем только догадываться 
о его существовании, а самый близкий путь через абстракцию математику. 
Арифметика, музыка и геометрия использовались богом при создании мира 
(числом мерой и весом....) и они типа существовали всегда. Нужно 
рассматривать бога как актуальную бесконечность, только по отношению 
к богу. А природа -- потенциальная бесконечность. Мир становится 
открытым. Космос можем представить как бесконечный шар с бесконечным 
радиусом, сразу непонятно замкнут он или нет. Вступает мистический 
настрой, бог есть абсолютный максимум и абсолютный минимум. Бог везде 
и нигде. А вселенная оказывается однородной, и возможны даже другие 
миры, но как христианин уж все-таки говорит, что другие могут быть, но 
наш лучше. Многие догматы все равно колеблются. Сами символы 
христианские приобретают математический смысл. Троицу он представляет 
как бесконечный треугольник такой. То есть две мысли: космос становится 
потенциально бесконечным, а математика решает ключевую роль для познания 
как бога, так и реальности. Это первая фундаментальная вещь, которая 
дает основание для появления более конкретных вещей.

А именно в позднем этом уже возникновение Николая Коперника. Используя 
математику он создает модель космоса, противоположную Птолемею. По 
Птолемея работе совершаются большие географические открытия. Коперник 
уже говорит, что каждая деталь теории Птолемея кажется очень хороша, как 
детали греческих скульптур, но если мы будем пытаться создавать 
и комбинировать скульптуру из деталей других, пусть и очень красивых, мы 
получим монстра. И теория Пт. уж была таким монстром. Коперник еще не 
идет до конца, не связывает модель с физикой неба, он говорит, что это 
математическая модель для облегчения расчета, посвящает это папе, 
говорит, это чтобы более точно устанавливать церковные праздники, пасху 
например, накопилось много погрешностей в календаре Юлианском. В конце 
концов уже после смерти Коперника церковь приняла какие-то моменты, 
перешла к Григореанскому календарю католичество, а православие осталось. 
Но мысль от геоцентризма к гелиоцентризму была принципиальна. Да, космос 
занят и ограничен, на границе находится сфера неподвижных звезд, но тем 
не менее другая картина. Оказывается можно помыслить теоретически совсем 
не то, что мы видим глазами. Совсем другая онтологическая картина. 
Аргументы Коперника не только в плане простоты, но и эстетические 
аргументы, биологические -- становится понятно, что от Солнца зависит 
жизнь на Земле. Ну и сам расчет, Коперник создал математическую модель. 
Но и просчеты были. Не до конца, предложив совсем другую картину мира, 
он оставил два философских постулата. Он оставил в покое представление, 
что вращение планет должно происходить по круговым орбитам, круг для 
него остался сакральной вещью, как у Аристотеля, а во-вторых он оставил 
в покое равномерное движение. Такая модель не сходилась конечно 
с экспериментом. У Птолимея в свою очередь было больше точности 
предсказаний. Теория Коперника тоже стала вынуждена.

Джордано Бруно был первым, кто заявил, что у Коперника эта модель 
физическая. В сборниках всякой инквизиции есть даже больше информации. 
Бруно говорил даже, что мир бесконечен физически, не как у Кузанца, 
могут существовать другие миры, может быть и лучше нашего. Бруно 
абсолютно уравнял наш мир с возможными другими мирами. Физическая 
бесконечность, у Земли ординарный статус в бесконечности. Вторая вещь 
это пантеизм, который можно сказать у того Кузанца тоже был, но Бруно 
идет дальше, заявляет, что бог во всех вещах в природе. Мир состоит из 
монад, единиц, одновременно симбиоз божественного и материального. Бруно 
уравнял физику неба и физику Земли, что тоже шло против обычного тогда, 
пришедшего от Аристотеля еще. Бруно говорит, что мир един, и если есть 
там, то может быть и в вещах, эфир является составной частью каждой 
вещи. Главные мысли -- о бесконечном космосе, возможном равенстве 
космических миров не хуже земного.

Дальше можно сказать о Тихо де Браге, профессиональный астроном, первая 
мощная обсерватория на острове, который ему дал в бессрочную аренду 
Датский король. 10 тыщ звезд. У него идея была, что можно представить 
картину мира так, что Земля в центре, Солнце вокруг нее, а остальные 
планеты вокруг Солнца. Еще не было таких уж конструкций для наблюдения 
точного, но его наработка эмпирическая позволила двигаться дальше.

Позволила -- Иоганн Кеплер. Он формулирует основные законы в книге ..... 
... Он изначально тоже руководствовался некоторыми ложными приколами, он 
сначала постулировал круговое движение планет, но поскольку самого 
эмпирического материала было больше, и он пользовался более телескопами, 
то конечно гармония мира изначальная не укладывалась в реальность, он 
очень расстраивался и переживал, но признал в итоге гармонию, 
позволившую описать, и сейчас его законы никто не опровергает. Первый 
закон -- орбита планет имеет эллиптический характер, солнце в одном из 
фокусов эллипса, второй закон то, что планеты неравномерно движутся, при 
приближении ускоряются, причем они так вот колеблятся, и третий закон 
в том, что движение планет, прохождение вокруг солнца, зависит от 
расстояния напрямую. Такая уж гармония, Кеплер говорил, мы не должны 
идти от выдуманных конструкций, мы должны идти от реального. Бэкон потом 
тоже говорит так. Это и есть гармония, установленная господом богом, 
а то самое выдумали уж мы, наш разум не совершенное орудие. Это первые 
ученые, которые уже на основе философских рассуждений, и Бруно, 
и Кузанского, создают то, что начинается классической наукой. Дальше 
будем говорить непосредственно и о Галлилее, и Паскале, и Ньютоне.


Дальше надо сказать о фигуре такой важной -- Френсис Бекон. Говорят, он 
-- основатель естествознания. Раньше о нем тоже говорили, но не то, 
а про утопию, которая была основана на идее, что в основе развития 
и достижения благосостояния общества должна быть именно наука. В Новой 
Атлантиде был дом Соломона, где работают ученые, в почете в обществе, 
являются элитой общества, заняты созиданием, открытием вещей. 
Распространение звука по проводам, построение глубоководных аппаратов, 
скрещивание животных, омоложение людей, использование солнечной энергии. 
В этом состояло его кредо, наука является фундаментом развития общества, 
ученые должны быть организованы, а государство должно им помогать 
и создавать условия. Это и было необходимым при создании первых уже 
реальных академий: Лондонское королевское общество и французская 
академия наук, они все ссылаются на Бэкона. Бэкон стал теоретиком 
и предвидел кооперацию университетов, новые программы университетов. Не 
просто кооперация, а союз академии и государства. А потом появились 
эрлинская, петербургская наша. Главное, тем не менее, у него другое. Он 
-- теоретик принципиально нового естествознания, теоретик нового метода, 
который получил название эмпирико-индуктивный метод. Новое понимание 
опыта, научный опыт, научный эксперимент, новый тип классификации наук, 
и самой науки и роли ученых. Огромный класс проблем, о которых мы сейчас 
не скажем ничего. Итак, начнем с пееросмысления понятия опыта. Опыт как 
эксперимент, как активное участие самого естествоиспытателя в создании 
постановок, в интерпретации результата. Принципиально важным является 
то, что во времена греческой науки опыт понимался как созерцание, 
наблюдение за естественным ходом вещей. Предпосылка -- мир гармоничен, 
а человек наблюдая приобщается к этой гармонии. Но делать ничего не 
надо, иначе исказишь картину, вмешавшись в природу. А Бэкон понимает 
совершенно по-другому. Это активное включение естествоиспытатель. 
Пытатель природы. Чтобы человек сказал правду, надо поставить его 
в такие условия, чтобы он заговорил. Естествоиспытатель должен задавать 
вопросы к природе, через различного рода приборы, установки. Природу 
надо ставить в стесненные условия. Бэкон первый понял то, что природа 
должна испытываться и быть испытанной, и человек именно тот, кто должен 
это делать. Коренное и правильное понимание, что же такое опыт 
и эксперимент. Соответственно с этим новым пониманием опыта 
разрабатывается и методология. Достижение правильных результатов. 
Экспериментально-индукционный метод. У Аристотеля это дедуктика, Бэкон 
не отрицает этого, но говорит, что недостаточно. Идти надо не от мысли, 
а от опыта, от эмпирической базы. Вот эта установка стала одним из 
главных, но не единственным в это время. Этот период называют войной 
методов. И эти методы и сейчас живы, а остальное -- модификации. 
Экспериментально-индуктивный метод Бэкона. Аналитико-синтетический метод 
Декарта (главные истины существуют в нашем сознании, в уме, надо 
правильно ими воспользоваться). Гипотетико-дедуктивный метод Галилея 
(придать форму мат модели, делать следствия, проверять 
экспериментально). Четвертый это Блэз Паскаль -- аксиоматический метод 
(бесспорно в математике). Но это потом, а сейчас про 
экспериментально-дедуктивный. Прежде чем предлагать принципиально новое, 
надо освободить человеческий разум и соответственно разум ученого, 
интеллектуала, чистых ученых тогда было раз два и все, тот же Кеплер 
занимался астрологией, подрабатывал и говорил, что астрология незаконная 
дочь астрологи, но кормит свою мать. Бэкон значит считает, что 
человеческий разум на данный момент обременен идолами, предрассудками, 
суевериями. Прежде надо очистить человеческий разум от них, они царят 
над человеком целые столетия. Это деструктивная задача, но она 
необходима. Первый идол -- рода -- состоит в том, что человек считает, 
что с помощью своих органов чувств он и получает истинную информацию 
о мире, раз бог уж так создал, все это соответствует реальности. Бэкон 
говорит, что все наши органы чувств ограничены. А что же тогда? Надо 
расширять амплитуду возможную для познания с помощью приборов. 
Открываются новые частицы, космические объекты, ну и все практически. 
Бэкону же принадлежит фраза, что наука развивается по мере развития 
научно-технической базы. Второй идол -- пещеры -- у каждого человека 
есть еще и своя пещера, связанная не с биологией, а социальным 
окружением, воспитанием, культурой, семьей, внушение идола начинается 
с детства, и очень трудно избавиться. Бороться можно только 
коммуникацией с другими людьми, которые по-другому образованы и прочее, 
все эти форумы, съезды, защиты, диссертации, все это попытки снять вот 
такое. Третий идол -- рынка, площади, базара -- где люди сталкиваются по 
поводу продажи или покупки чего-то, часто конфликтные ситуации, 
и говорят на обыденном языке. Этот идол состоит в принципиальной 
ущербности обыденного языка для науки. Одни и те же понятия могут 
означать совсем разное для разных людей. Отсюда бесплодность всех 
разговоров на рынке итд, именно то, о чем Сократ говорил. Наука должна 
стремиться к одному языку, не математическому он говорил к сожалению. 
И последний идол -- театра -- главным в театре является режиссер, он 
замышляет постановку, может сам придумать, может текст взять, но 
у разных режиссеров совершенно по-разному. Идея в том, что есть 
некоторые фигуры, являющиеся главными на поле интеллектуальной 
деятельности, в науке, то, что называют авторитетом. Главным авторитетом 
был Аристотель, вот почему критика прям против него. Он поставил 
спектакль науки, а он оказался не востребованным. В методологическом 
плане в науке есть исторический авторитет, тот же Аристотель много чего 
сделал, но это было в свое время, а если не исторический, а нормативный 
авторитет, то есть на веки вечные, а это получается религия. Если 
воспринимать какого-либо ученого как нормативный авторитет, становится 
ужасно. И исходя из этих представлений, очищение разума, а дальше 
эмпирико-индуктивный метод заключается в сборе материала и обобщению по 
принципу индукции. Причем стоит большее внимание обращать на 
отрицательные примеры. Можно рассматривать признаки присутствия теплоты, 
от Солнца, огня на земле, вулканов, внутри нашего тела, вот присутствие. 
Вот свет есть у луны, а тепла нет. Светлячки тоже, свет есть, а тепла 
нет. Динамика, свойство тела повышать температуру, побольше дров значит 
тепла больше. Ну и приходит к заключению важному, что теплота это 
движение мельчайших тел, их столкновение, опережая термодинамику. Ну 
и последнее, на основе этого говорит, что наука имеет два уровня 
развития, теоретическая наука и практическая наука. Он называет это 
светоносные и плодоносные. Он говорил, что нет ничего практичнее, чем 
хорошая теория. Нужно сначала развить теорию, а потом строить реальное 
применение. Ученый должен быть не муравьем, который собирает но не 
понимает, но не пауком, из себя достающим, а пчелой вот такой, которая 
собирает вещи, и потом перерабатывает в мед.

\hfill \textbf{Sep 27}

Метафизические предпосылки идеальности и совершенства кругового 
и равномерного движения, что не соответствовало эмпирии, 
в предсказательности можно было сомневаться вплоть до Кеплера. Бруно же 
придал физический статус этому, сказал, что могут быть миры не хуже 
нашего, и сурово поплатился. Наиболее важная фигура в это время -- 
Френсис Бэкон. Переосмысление самого понятия опыта, придание ему той 
формы, как сейчас, в опыте надо ставить соотв ситуации, приборы 
использовать, не просто созерцать, а активно участвовать. Идея чистки 
человеческого разума от накопившихся идолов, предрассудков: рода, 
пещеры, площади, театра. В позитиве он дает новый метод 
эмпирико-индуктивный метод, который и становится основным 
в естествознании, наравне с двумя еще, названными там. 
Аналитико-синтетический метод один из них, предложен Декартом, еще 
скажем. Бэкон также дает новую классификацию наук, великое 
восстановление наук, его произведение. Аристотелевская существовала, 
а Бэкон предлагает классиф исходя из способностей человека: память 
воображение и разум. Память -- исторические (естественные) науки, 
становление самого мира природы + история человечества и общества. 
Воображение -- поэзия и искусство, тоже в понимании Бэкона науки в то 
время, Да Винчи вот использовал золотое сечение, перспектива, в музыке 
тоже появились 7 нот вот этих, можно запоминать, воспроизводить, 
творить. Архитектура тоже включалось туда. Разум -- прежде всего 
философия, но может быть философия самой теологии может существовать, 
Бэкон конечно как и остальные был верующий, только легкие глотки 
философии приводят к атеизму, а если подробно ознакомиться, то нельзя 
без Бога объяснить ни появление, ни устройство. Теология из 
метафизических причин четырех оставляет две для теологии -- формальная 
и целесообразная (телехеальная), а для философии движущая 
и материальная. Естествознание таким образом освобождалось от вопросов 
а зачем и откуда все это. Ну и последнее что было то, что он организатор 
науки, и утопия -- Новая Атлантида -- послужила мотивацией создания 
академических вещей типа Лондонского королевского общества и Парижской 
академии наук.

\subsection{Эмпиризм и рационализм}

Эмпиризм --- Френсис Бэкон, Пьер Гассенди, Томас Гоббс, Джон Локк. 
Некоторых мы разбирали как политических вот вещей и общественных, 
а сейчас естествознание. Основная идея направления -- в конечном счете 
мы все знания получаем из наших органов чувств. Органы чувств являются 
последней инстанцией во взаимодействии с миром. Знаменитая фраза Джона 
Локка -- нет ничего в разуме, чего до этого не было бы в органах чувств. 
Наш разум при рождении это некая чистая доска, и только при 
взаимодействии с миром в результате опытов на основе органов чувств 
и дает нам знания. Источник здесь.

Рационализм --- Рене Декарт, Бенедикт Спиноза, Годфрид Лейбниц. Органы 
чувств -- ненадежная опора для истины, а все истины априори содержатся 
в нашем разуме. Опираться можно не на органы чувств, а только на наш 
разум. В этом отношении Лейбниц добавляет важное замечание к тезису 
Локка -- кроме самого разума. Разум нам в ощущениях никаким образом не 
дан. С точки зрения разума с точки зрения рационалистов и строится вся 
наука.

Учение, о которых мы также будем говорить, будут использовать и ту, и ту 
установку.

Переходим сейчас к Рене Декарту, основоположнику философии европейской, 
как говорит Гегель и еще кто-то говорит, что он ведет исследование как 
ученый, не просто учит и не занимается дидактикой, а слушателям хочет 
поучаствовать их участие хочет. И названия там метафизические 
размышления, всякие такие. Он первый задает такое направление 
рационализма. Мы основные идеи его воспроизведем. Но сначала скажем вот 
что, раз он основатель рационалистического дела, он жизнь прожил 
короткую, но успел очень много, и как ученый -- декартовы координаты, 
аналитическая геометрия, алгебра, принцип инерции, закон соударяющихся 
тел, преломление света на границе сред, космогоническая гипотеза вихрей, 
введение в физику понятия корпускула, принципа континуальности, принцип 
сохранения движения (первый импульс дается богом, но потом сохраняется), 
представление исследования по принципу черного ящика и прочее..... Сам 
он дворянин из знатного дворянского рода, но все средства тратит на 
существование как ученого, переселяется в Нидерланды, где чуть более 
свободно было, а во Франции цензура не пускала, тем не менее идеи 
проникают, и физика Декарта опережала долгое время физику Ньютона. 
Королева Швеции приглашала к себе для чтения лекций, но Швеция конечно 
не Европа континентальная, хуже погода, заболел воспалением легких 
и умер.

Основные идеи его. Основная -- идея единства научного знания. Единство 
всех наук в конечном счете, общность методологии для этих наук, что 
отличает науку от ненауки, признание той же идеи Бэкона, что наука не 
просто удовлетворение любопытства ученого, а может приносить пользу 
всему обществу. Его впечатляющий образ он сам пишет, могучее дерево, 
корни которого метафизика, философия, ствол это физика, а все ветви 
с плодами это науки: медицина, механика, этика были среди них главными. 
познания тоже были так, ученый не может познавать что-то одно, ученому 
должно быть легче ориентироваться во всех науках сразу, изучать 
комплексно, чтобы сравнивать и идти дальше. Но как уже было сказано, 
Декарт -- рационалист, и считает, что в самом нашем сознании есть 
врожденные идеи и принципы или аксиомы. Идеи -- протяженность, 
пространство, время, движение, ну и конечно идея бога. А аксиомы вот 
логико-математического плана как постулаты Евклида, законы логики 
Аристотеля, законы логики Аристотеля, закон транзитивности, Декарт 
говорит, спит человек или бодрствует, 2*2 все равно 4. А если человек 
сосредоточится на своем разуме, будет рефлексировать, то познает их. 
Истины носят диспозиционный характер. Надо углубиться, размышлять. 
Декарт признается, что все эти истины к нему пришли в армии он грелся 
у печи, он углубился в размышления, и пришел к этим истинам. Но истины 
не зависят ни от какого опыта, они бы все равно существовали в разуме, 
но для обнаружения конечно нужен вот такой труд. Ну вот это о природе 
математических истин, и на основе этих истин Декарт и считает, что можно 
строить науку, а органы чувств ненадежны, обманывают нас во многом, тем 
не менее сами истины человеку надо отыскивать в своем мышлении. И чтобы 
отыскать Декарт предлагает рациональный метод, он так и входит 
в философию науки как метод Декарта. У Бэкона был эмпирико-индуктивный, 
у Декарта аналитико-синтетический метод.

Если все истины от нашего разума, их можно установить путем обращения 
к самому разуму с помощью интеллектуальной интуиции. Понятие интуиции не 
было новым, но в средневековье связывалось с озарением, искра божья, 
надо в церкви для этого молиться долго, может бог осенит, укажет путь 
тебе, но знание ни при чем, бог указывать путь должен там, и человек 
должен начать действовать, мистическая интуиция. А тут интеллектуальная 
интуиция. Декарт прямо говорит, что интеллектуальная интуиция это не 
вера в наши органы чувств, а ясные, отчетливые, четкие понятия нашего 
ума. Такие очевидные и ясные, что будучи эксплицированными будут ясны 
всем. Прийти к истинам этим можно только путем анализа. Даются на уровне 
чего-то там, и надо вести анализ до тех пор, пока не упремся в такие 
ясные очевидные истины. Это анализ, аналитический подход, мы 
эксплицировали истины. А дальше синтетический подход, мы идем наверх 
всего объекта. Чтобы контролировать ход и вниз к истинам и обратный ход, 
нужен еще один подход. Энумирация -- чтобы не пропустить чего-нибудь, на 
каждом шаге размышлять и проверять а так ли на самом деле. Построение 
тех же теорий идет не сразу. Эти четыре пункта Декарт предлагает как 
основу рационалистического метода.

Но сам Декарт задает вопрос себе, где доказательство, что этот метод 
действительно ведет к истине? Принцип методологического сомнения. 
Метод-то метод, но чтобы его воспроизводить, ученый должен 
предварительно посомневаться, принцип сомнения как таковой в философии 
всегда присутствовал, в Греции там, но он у скептиков например носил 
разрушительный характер, все знание не знание, ничто доказать нельзя, 
лучше следовать обыденному здравому смыслу, а в теоретическом плане 
лучше помолчать, мир объявлялся непознаваемым. Но вот у Августина 
Блаженного этот принцип уже был конструктивным, посыл, чтобы убедиться, 
что бог не обманывает нас, и можно положиться на человеческий разум, 
поскольку он создан самим господом богом. Декарт по факту опирается на 
этот принцип Августина, но расширяет. Говорит, усомниться можно во всем, 
и даже в самом господе боге, а вдруг дьявол внушает 2*2=4. Но нельзя 
усомниться в сомнении, что сомневаюсь я, и сомнение для меня абсолютно 
истинно, никто за меня не сомневается, это я сам. Предпосылка 
о приоритетности доступа человека к собственному сознанию. У Сократа еще 
было познай самого себя. Мне легче познать самого себя потому что я не 
имею доступ к этому. В собственном сознании я уверен, что это я, 
и истины находятся в моем сознании, и соответственно первая истина 
получается. Сократ тут приходит к тезису своему cogito ergo sum -- 
мыслю, следовательно, существую. Воспроизведение аргументация Ансельма 
Кентерберийского. Это панлогизм, который Декарт демонстрирует, за 
который одни его критикуют, а другие опираются. Сомнение есть один из 
вариантов мышления. А дальше если истина получена, то вводится понятие 
бога как врожденная идея, нельзя из сущности выводить существование 
конечно по Фоме Аквинскому, но Бог особый, его так можно. Опираясь на 
этот принцип Декарт приходит к конкретным четырем положениям.

Следующая важная проблема в руках Декарта, он ее ставит, метафизическая 
проблема сознания. Декарт в этом смысле является основоположником 
направления философии, которое называется дуализмом. То есть есть два 
основных направления -- материализм и идеализм, а вот Декарт считает, 
что в отношении по крайней мере в отношении субъекта познания материя 
и дух равносильны. Конечно сам человек создан богом, а бог не 
материален, и в таком плане бог первичен, но в отношении человека бог 
допустил такую дуальность, и это теперь дело науки, наука должна 
познать. Но как совместить эти противоположные стороны, это 
психофизическая или психофизиологическая, майнд боги, с бумом 
когнитивных наук это особенно активно пытаются решить сейчас. Существуют 
институты мозга, и в нашей стране, и на западе, вся эта роботехника, 
искусственный интеллект, все решают эти проблемы. Как же в человеке 
сочетаются дух и материя. Как нечто духовное может вообще влиять на эту 
физику? Причинно-следственные связи должны быть циклом, и ничего 
внешнего не должно быть, и тогда сознание лишь эпифеномен, сознание нам 
лишь кажется типа, а мы так лишь только думаем. Почему если я хочу 
поднять руку и поднимаю его, это действительно происходит от моей воли, 
или это запрогроммировано уже, а мне лишь кажется что так происходит. 
И сейчас такие теории есть, задаются вопросом почему вообще такие 
процессы не идут в темноте, и вся наша сенсорика фактически для нашего 
успокоения, и ничего на деле не решает. Для науки важно все-таки понять, 
есть ли нечто такое, что не поддается непосредственно физическому. 
Разделяет материальное и идеальное по принципу протяженное или нет. Если 
протяженное, можно измерить, зафиксировать, определить параметры, то 
материальное, а вот идеальные вещи нельзя так, типа память и прочее. 
Декарт поставил такую проблему, но конечно не решил. В то время были 
открыты уже круги кровообращения, и он предложил, что в мозге существует 
шишковидная железа, где встречаются духовное и материальное, но механизм 
непонятен, и сейчас тоже.... Декарт все-таки считал, что этот дуализм 
в конечном счете предопределен богом в отличие от животных. Здесь важно 
очень сказать, что Декарт проводит жесткую черту между человеком и всеми 
животными. Он считает, что животные -- автоматы. Такого рода определений 
как механицизм, одним из авторов которого является Декарт, что идет еще 
от Николы Кузанского и машины мунди. Все эти рассуждения Аристотеля 
о растительной душе, духовной душе Декарт отклоняет, а в человеке есть 
сознание и возникает такая проблема. Трудно выяснить конечно истоки 
сознательной деятельности, вот такого всего. Дальше эти идеи все-таки 
попытаются развивать его последователи и противники, развиваются 
окказионализм, и направление синхронно заведенных часов.

Окказионализм это Мель Бранж, говорит, что за каждым сознательным 
действием человека стоит господь бог, и поэтому здесь действительно 
некая случайность и вероятность, но позволить чему-либо случиться может 
только бог, свобода воли становится иллюзией для человека.

Теория синхронных часов, Гейликс, предложила такой ход интересный, 
параллелизм материального и идеального. Так же как причинно-следственные 
связи бывают в материальном, так и могут быть в идеальном. Они 
принципиально неустранимы, и никогда не пересекаются, но совпадают по 
времени. Мое желание поднять руку пришло от огромного количества событий 
в идеальном мире и реальность поднятия руки вот так вот совпали. Два 
параллельных ряда, как свисток электрички и начало движения ее. Итак это 
метафизическая проблема, состыковка духовного и материального, которые 
ставит Декарт.

Ну и последнее по Декарту это его физика. Основные достижения уже были 
названы в большом количестве, а в физике важные положения -- первая 
формулировка принципа инерции, закона соударения тел, закон преломления 
света и космологическая идея строения космоса. Еще один термин надо 
ввести. Деизм. Мир состоит из корпускул, мельчайшие частицы вещества, 
которые плотно прилегают друг к другу и находятся в вихреобразном 
движении. Бог первоначально придает или запускает импульс в мироздание, 
бог создает корпускулы, наделяет первым импульсом, а дальше все 
происходит по механике, и образуется мир, который мы наблюдаем. Это 
первая демонстрация мысленного эксперимента по построению нашего мира. 
Декарт ведь верующий человек, знает библию, и знает как в ветхом завете 
бог создает мир, последовательно, быстро, за 5 или 6 дней, 
в совершенстве, на уровне целостных структур, земля, небо животный мир, 
растения. Все это создается сразу и окончательно. Он не может 
противоречить. Он пишет, что признает, что бог создал мир сразу 
и в совершенстве, но чтобы понять устройство этого мира, можем провести 
мысленный эксперимент. Предположим, что мир состоит из мельчайших 
корпускул, а бог не создает мир в таком варианте конечном, а дает вот 
такой первый импульс, а дальше мир развивается сам по физическим 
законам. И дальше мир создается все равно вот таким каким есть, но мы 
можем познавать. И Декарт кроме того заявляет, что Земля остается 
неподвижной. У нас конечно понятно, что галактики спиралевидные, все эти 
корпускулы не просто летят по механике, а вот так конструктивно. Надо 
сказать дальше, забегая немного вперед, что такого рода представление 
о строении мироздания, оно стало таким после Аристотеля гораздо более 
дополненным, на уровне мысленного эксперимента. Да бог устроил мир, но 
не как непосредственно в Библии, это для необразованных, а образованным 
легче так, что мир создал вот частицы, дал первый импульс, и мир 
устроился вот по законам. Когда появилась первая мировоззренческая 
картина на основе физики Ньютона, возник спор между Ньютонианциами 
и Картезианцами. Какую же роль все-таки играет бог в построении 
мироздания? С одной стороны если деизм, то бог создал корпускулы 
и законы, дал импульс и больше не вмешивается в мир, он все сделал, мир 
дальше сам, тогда спрашивается а как же чудеса, благодать божья и все 
такое. А ньютонианцев упрекали в том, что получается несовершенная 
картина мира, а если брать закон тяготения, то по этому закону все 
планеты должны попадать на солнце, Бог значит должен вечно подправлять 
творения, перезапускать систему, координаты и импульсы атомов приходится 
подправлять, ну как же господь бог и вдруг несовершенное творение. Ну 
и сама сила тяжести имеет мистическую природу. В общем, физика Декарта 
и Ньютона долго конкурировали, и Декартовая долго была первой потому что 
более понятная. Но в конечном счете спор решился тем, что Ньютоновые 
вещи попросту предсказывали и объясняли гораздо больше вещей, не упуская 
ничего из физики Декарта. Появилось размежевание, настоящая физика, 
настоящая наука, появляется не только тогда, когда что-то объясняет, но 
и предсказывает. Физика Декарта осталась на качественной основе, 
и сохранила метафизический статус. Однако с появлением электродинамики 
все вновь вспомнили о физике Декарта, континуальности и прочее.

Дальше коллега Декарта -- Пьер Гассенди. Профессор математики, 
священнослужитель, написал труд система философии в 6 томах. 
Полемизировал с Декартом в частности о своих идеях важных. Прежде всего 
он поставил на твердую почву понятие авторитета в науке. Он защитил 
Аристотеля после того как Бэкон просто обвинил, что он затормозил все 
и вся. Пьер показал, что Аристотель был важной персоной в науке, 
а последующие схоласты сделали из Аристотеля лоскутное одеяло, 
и Аристотель был потерян в целом виде. Дальше критика самого Декарта, из 
сущности нельзя выводить существования. Мыслю следовательно существую не 
так просто, существуют лишь мысли. Декарт сначала прислушивался потом 
забил. Дальше Гассенди вводит понятие молекулы, атомы это сравнить можно 
с буквами, а молекулы это слова, из которых буквы состоят. Он вводит 
понятие атомов и молекул как различных уровней организации материи, 
и атомы созданы богом, а правила их образования в молекулы созданы богом 
тоже, но и то, и другое являются физически существующими. В отличие от 
Декарта Гассенди снова воссоздает атомистические учения от Лукреция 
Карра о природе вещей. Гассенди воссоздает атомизм как физическую 
теорию. Бойль дальше возьмет это понятие молекулы, и появится наука 
химия. Ньютон тоже берет понятие атома, пространства, движения итд.

\hfill \textbf{Oct 4}

Декарт -- корпускулы, континуальность.

Пьер Гассенди защитил Аристотеля в плане исторического авторитета, но 
нормативный авторитет конечно да. Критика дальше самого Декарта, 
введение понятия молекулы. Дальше Гассенди возрождает атомизм, расчищая 
путь физике Ньютона. Сам он был священник, говорил, что атомы сами 
созданы богом, церковь довольна.

Теперь дальше, в это время работают уже профессиональные ученые, Декарт 
пишет свой труд про мир тот, во Франции все книги его запрещены были, 
сам жил в Голландии, тем не менее не публиковал, чтобы не достала 
инквизиция. Посмотрим на вещи. Принцип гносеологического оптимизма, 
нет ничего не познаваемого, чем больше человек познает, тем ближе 
к Богу. И одновременно гносеологический пессимизм, что познавать нет 
смысла, ибо мир бесконечен, как в космосе, так и в микромире, и поэтому 
задача людей не столько познавать мир, сколько заботиться о спасении 
своей души. Это исходные позиции философские позиции, морально-этические 
позиции, на основе которых появляется наука.

\subsection{Галилей}

Подробнее о Галилее. Высоко образованный человек, музыка, читал даже 
божественную комедию. В каких областях науки заявил о себе: механика, 
динамика, принцип инерции, Ньютон прямо воспроизводит в своих делах этот 
принцип и ссылается на Галилея. В средневековье был такой подход принцип 
импетуса, но была крупная недоработка, что состояние покоя было 
основным, а любое движение требовало толчка. Дальше принцип инерции, 
принцип относительности Галилея (речь про корабль и кабину), принцип 
свободного падения тел (можно сказать и закон... уравнивание фактически 
тяжелого и легкого, что в общем расходилось со здравым смыслом. Важно 
подчеркнуть, что все эти три принципы это результат мысленного 
эксперимента. Галилей впервые серьезно вносит в физику мысленный 
эксперимент, как сейчас в ОТО и прочее. Принцип инерции -- невозможно 
провести опыт, доказывающий принцип инерции, что без сил тело движется 
бесконечно. Принцип относительности -- то же самое, Галилей описывает 
чисто мысленно, быть в каюте корабля, задернуть шторы, аквариум 
с рыбками, бабочки летают, невозможно определить движется или нет. Ну 
и главный принцип -- закон падения тел -- есть много историй, как он 
бросал шары с Пизанской башни, но скорее всего такого не было, хотя бы 
потому что часов особо и не было, что быстрее падает невозможно 
замерять. Но из произведений Галилея известно, что чтобы доказать нашу 
правоту, что все тела падают с ускорением, можем связать тяжелый шар 
и такого же размера, но полый, и бросить с высоты. С одной стороны, по 
физике Аристотеля общая масса шаров должна увеличиться, и должны они 
падать быстрее, чем один шар, но с другой стороны, поскольку они 
связаны, на расстоянии, то легкий шар падает медленнее, и должен 
затормаживать тяжелый. Получается противоречие такое. Отсюда сначала 
спекулятивно говорит, что скорость должна совпадать, а потом вывел 
математически. Подтвердилось это уже потом, когда изобрели насос и стали 
показывать шар и перышко. Вот здесь и проявляется то, что мы называем 
гипотетико-дедуктивным методом. Но не меньшее достижение его это конечно 
достижение в утверждении физической значимости теории Коперника. У него 
это была математическая модель для облегчения, Бруно спекулятивно так 
вот подходит, а Галилей уже научно пытается обосновать. На опыте... 
И для этого Галилей берет телескоп и усовершенствует его, пытается 
создать научный прибор, и это уже совсем другой прибор, телескоп, он сам 
был мастер в этом деле, и добился принципиально усовершенствования, в 30 
раз лучше. Что он увидел? Млечный путь -- огромное скопление звезд. Он 
увидел спутники Юпитера, которые вращаются вокруг него, что усилило 
предположение, что малое вращается вокруг большого. Он увидел фазы 
Венеры, что показывало, что планеты меняют положение относительно 
солнца. Он дальше увидел неровности на луне, которая считалась эфирным 
телом таким. И конечно дальше он увидел пятна на Солнце, что говорило, 
что вот так вот в Солнце тоже процессы, оказывается. Это конечно не 
доказывало физичность Коперника напрямую, но вот сильно намекало. 
Папская академия наук проверяла эту вещь, но вообще даже усомнилась 
в методе через трубу эту. Но Галилей пишет диалог о двух главных систем 
мира, в таком наполовину литературном жанре, между собой ведут диалог 
три персонажа, два из них похоже очень образованных, а один Симпличео. 
Двое долго убеждают, что система Коперника верна, а Симпличео все 
с точки зрения здравого смысла отвергает. А в итоге эти двое 
интеллектуалов в конце соглашаются с Симпличео, что Земля покоится, все 
вращается вокруг нее, но не прошло и года, и уже свершился суд над 
Галилеем. Галилея заставили раскаяться, запретили заниматься 
астрономией, заключили под домашний арест, и в конечном счете он уже не 
мог работать дальше, но тем не менее выпустил в Голландии книгу 
математических доказательств своей теории на языке математики. Прошло 
много лет, и только в 1984 году Папа Иоанн II признал неправомерность 
суда вот этого. Было сказано, что люди тогда были виноваты, но не 
церковь конечно же. Гносеологический оптимизм, но трагедия самой 
личности Галилея. Важно сказать еще одну вещь, что Галилей конечно 
принципиально меняет и философскую установку науки. Если до этого наука 
опиралась на физику Аристотеля, качественную физику, где эксперимент 
созерцательный, то Галилей переходит на позицию Платона, что физика 
и наука начинается там, где мы можем что-то измерять. Галилей это 
доказывает практически. Он переписывается с Кеплером, у которого были 
проблемы сложные, но как только он отказался от Аристотеля и физики его, 
сразу получились вот эти вещи. Традиции пифагорейцев и Платона 
становятся важнейшими, кладутся в фундамент появления науки. Исходя из 
этого, он рассуждает в методологическом плане, о теории двойственной 
истины, завершает эту эпопею двойственной истины, с востока до Европы. 
Галилей завершает это. Он разводит религию и науку. Причем он не атеист 
далеко, он говорил, что наука должна приносить плоды и подавать их 
религии. Но книга откровений это Библия, написанная на языке людей, 
может быть переписана на любой язык человеческий, ее интерпретация это 
интерпретация церкви, а другая книга это мир сотворения, и она написана 
на языке математики. Это прежде всего язык геометрии, геометрические 
фигуры. Знания это есть припоминания, у Галилея тоже, считает, что книга 
написана уже, мы только открываем ее, читаем самые первые строки, 
перелистываем страницу, идем дальше, познаем логико-математические 
истины и остальное. Финроуз нобелевский лауреат говорит тоже, что мы не 
конструируем истины, существует логико-математическое поле вот такое 
вообще, и мы с помощью наших особых антенн постигаем. Вот это 
философско-методологическая идея. Метод мы назвали 
гипотетико-дедуктивный. И теперь сам процесс познания. Галилей разделяет 
два типа истин, знаний: экстенсивные и интенсивные. Экстенсивные знания 
это знания обо всем и доступные только богу, бог знает все и сразу. 
Интенсивные знания это те, которыми владеет человек, оно процессуально, 
можем знать меньше, можем больше, но знания освоенные, они уже 
окончательные истины. Эти истины одинаково знает бог и человек. Эта же 
мысль была у Фомы Аквинского, который говорил, что бог вот уже не может 
изменить сумму углов треугольника и прочее. И таким образом познаем 
и приближаемся к богу, вот такая эта мысль. И отсюда Галилей закладывает 
фундаментальные представления о характере развития науки, кумулятивный 
подход. Создается фундамент научного знания, и он первый бросил камень 
в туда, а дальше строится здание знаний, по подобию как строят кирпичную 
стену. Но внизу все уже нерушимо. Истина всегда одна, если мы познали 
объект, другого быть не может. Вот такое представление, оно существовало 
вплоть до начала 20-го века с квантами и теорией относительности, 
фундамент пошатнулся.

\subsection{Блез Паскаль}

Считается одним из основателей классической теории вероятностей (вместе 
с Ферма), закон гидравлики, потом давление, впервые показал, что 
Атмосфера земли тоже обладает давлением, тоже тело, первый калькулятор 
построил. Вместе с Декартом считается одним из разработчиков 
интегрально-дифференциального исчисления. В произведении мысли тем не 
менее, он разочаровывается в науке, прерывает свою научную деятельность, 
уходит в монастырь, довольно быстро тоже умирает. Предположения у него 
важные: понял, что познание бесконечно, потому что бесконечен мир. 
Галилей показал уже, что никакой сферы неподвижных звезд нет, галактик 
бесконечное количество тоже. Говорил, что его ужасает безмолвие пустоты 
этой. В то же время изобретают микроскоп, в котором видно оказывается, 
что жизнь и меньше существует, и . Актуальная бесконечность, она не 
убудет сколько ни познавай. Величие человека у Галилея состоит 
в познании, а у Паскаля -- в осознании своей ничтожности в космосе, что 
человек как мыслящий тростник на ветру, цивилизацию может прихлопнуть 
самая незначительная катастрофа в космосе. Величие состоит только в том, 
что он понимает ничтожество на фоне этой бесконечности, может объять 
в своем сознании эти все космические просторы, и ужаснуться. Не случайно 
такое направление, как экзистенциализм, считает Паскаля своим 
основателем. Предназначение человека не космические деяния, а земная 
жизнь, ибо остальное выливается в страх. Панацея получается обратиться 
к богу. Чтобы побороть страх перед необъятностью космоса, Паскаль 
предлагает разделить все истины на истины сердца и разума. Все эти 
истины научные носят какой-то характер. А экзистенциальные вещи 
в сердце. У сердца свои законы, которые отличны от законов разума. 
Проблема бога решается в первую очередь сердцем, поверить в бога. 
С помощью истин разума можно лишь привлечь какие-то рассуждения для 
поддержания истины. Пари Паскаля -- мы не можем непосредственно 
установить связь с богом, да есть такая литургия, молитвы, храмы 
и прочее, но будет ли бог разговаривать с человеком, внушит ли, то ли 
это дьявол, то ли бог. А разум может обосновать необходимость в веру 
в бога. Между нами и богом бесконечное расстояния, космический хаос, бог 
где-то там, мы должны с помощью рассуждений обосновывать. Поэтому 
сыграем в определенного рода игру. Предположим, что бога нет, тогда нашу 
короткую жизнь мы можем прожить не оглядываясь на ограничения, которые 
накладывает религия. Все эти ритуальные действия становятся пустыми, 
если бога нет, ты тратишь часть своей жизни на то, чего не существует. 
Но вдруг бог есть, вдруг написанное в писании действительно существует. 
Ты проиграл вечность. Все хорошие ушли в рай, живут вечную супер жизнь, 
не соизмерима твоя короткая жизнь и вечность плохая. Ну пусть теперь бог 
есть, нужно значит часть жизни тратить на то, чтобы доказать ему свою 
веру, послушание и прочее, и ты умираешь, а бога нет, ты вроде бы 
проиграл, но проиграл ты ничтожно малый свой отрезок земного 
существования. Речь идет о вероятности и соотнесении известного предела 
человеческой жизни и вечности. Есть она, есть ли бог с его вечностью, 
нет ли его. Такого рода рассуждения относятся сейчас к стартапам..... 
Риски всякие, но если выигрываешь, то много. Паскаль как раз занимался 
азартными играми с точки зрения математики тоже, вот он и предложил вот 
такой подход почти что математический, что выгоднее полезнее делать, 
верить или не верить в бога. Метод у Паскаля аксиоматико-дедуктивный 
метод.

Подводя итоги скажем, что происходит не только столкновение методов, но 
и философских вещей, какое место занимает наука для отдельного человека, 
нужно ли, ну и так далее. Не только у философов видим, но и вообще.

\subsection{Бенедикт Барух Спиноза}

А теперь вернемся все-таки к философии профессиональной, линия 
рационализма. Бенедикт Барух Спиноза. Говорят, самый благородный философ 
того времени. Сам выходец из еврейской общины, перебравшейся к Голландию 
оттуда где были гонения. Резко потом разрывает с общины, пытаются 
наставить на путь истинный, но он отказывается, и его отлучают от 
общины, от церкви, анафему производят, и это была трагедия, поскольку 
соплеменникам не просто запрещали встречаться, беседовать, но и табу на 
публикации его, возможность материальной поддержки. Всю жизнь был 
вынужден для пропитания заниматься трудом, не сильно физический, но 
утомительный и для здоровья жесткий, шлифовка оптических линз. Очки 
тогда появились, подзорные трубы, телескопы. Спиноза был одним из тех, 
кто точил линзы. Умер вот от этого тоже, туберкулез, чахотка она же. При 
жизни ничего не опубликовал почти, богословский трактат о философии 
Декарта. И только после его смерти вышла его Этика, а полное собрание 
сочинений только в 19 веке. Но само название произведения, этика, 
говорило о том, что Спиноза связывал тоже прежде всего науку и нравы, 
мораль, воедино. Тесно связывал с научным знанием. Надо еще конечно 
сказать, что в этом отношении он противоположность Декарту, который был 
состоятельным, мог позволить себе на досуге, у Спинозы это была позиция 
героического энтузиазма. Бруно еще это сделал в отличие от стоиков и их 
пассивный героизм (принять судьбу). Наука для Спинозы была стержнем 
жизни, наука прежде всего. Само произведение этика это натур философское 
произведение, опять-таки в своей конструкции прослеживается мысль 
рациональная, что математика лежит в основе науки, и даже философия 
подстраивается под Евклида. В книге его есть аксиомы прямо выписанные, 
теоремы, ход доказательств прописывается, выводы получены так, что уже 
неопровержимый характер носят, даже аффекты человеческие, добродетели, 
по мнению Спинозы можно так выписать, математическими конструкциями. Он 
короче продолжает линию Декарта, но не согласен во многом. Говорит, что 
есть неопровержимые истины, постигаемые интеллектуальной интуицией, но 
существенно поправляет, что вот эти очевидные и ясные вещи ненаучно, 
собственные личные такие вещи, не хватает общения. Не просто что-то 
внутреннее, а необходима еще и дефиниция. Надо дать дефиницию, обсуждать 
ее, устанавливает для научного сообщества на уровне интуиции. Кроме того 
Спиноза различает разные виды познания, действительно первая ступень это 
чувственное познание, с помощью органов чувств, никуда не денешься 
в любом случае, и это познание конечно не бесполезно, и это знание не 
должно дико расходиться с научным, иначе мы бы не могли просто 
существовать. Второй вид знаний это знания по наслышке, на уровне 
коммуникационной деятельности между людьми, это знание может быть уже на 
уровне здравого смысла, обыденные вещи, но может быть и на уровне 
научного, научной коммуникации, мы не все проверяем, но многое усваиваем 
из лекций, книг и прочее. Третий вид знаний -- как раз на основе 
интеллектуальной интуиции -- логико-математические истины, которые мы 
постигаем, но от нас не зависят. Приходит к важному философскому 
обобщению, снова расходится с Декартом. Критикует за дуализм, что Декарт 
дуалист. Несмотря на то, что бог сочетает материальное и идеальное 
в человеке, но все равно. Спиноза же монист. Это единое, природа, 
субстанция, бог. Природа -- причина самой себя. В мире нет творения, 
а только порождение. Есть природа творящая и природа творимая, все это 
одно и то же, не надо трех слов, просто в разных ситуациях мы ставим 
акценты или на активности и творчестве, поскольку бог в природе, или на 
результатах, творение это уже то, что перед нами к качестве реальных 
объектов природы. Субстанция. Еще одно определение Спинозы -- пантеизм. 
Сталкивались уже у стоиков, Бруно, Кузанского. Это некая космическая 
религиозность, бог везде и нигде, разлит в природе, но в то же время 
творческое что-то. Эйнштейн тоже говорил, что в бога -- космическую 
рациональность он верит. Мир каузально замкнут физически, в нем царит 
рациональность, и соответственно нет никакой случайности в этом мире. 
Спиноза считает, что он близок к фатализму, все фактически 
предопределено, и для человека свобода это только познанная 
необходимость. Смерть человеческая, бояться ее человек не должен, смерть 
вплетена в естественный ход событий, только жалкие духом боятся смерти, 
сильные духом думают только о жизни, не плакать не смеяться а понимать, 
ход вещей понимать. Таким образом монизм как неразрывное единство 
сущности бога и природы. Спиноза в то же время признает различие 
материального и идеального как атрибутов бога, природы и сущности. Раз 
бог, то у него может быть огромное число атрибутов, но мы сейчас знаем 
о двух точно, материальный и идеальный. Так он низводит субстанции 
Декарта к статусу атрибутов единой субстанции. И третий его ход -- 
отдельные вещи. Отдельные физические вещи -- модусы. Мир это 
совокупность всех модусов, но сами модусы -- лишь низшее звено этой 
пирамиды. Модеистический подход и пантеистический подход. Понятно теперь 
почему Спиноза разошелся с ветхим ортодоксальным заветом. Отсюда же 
мысль, что случайностей быть не может, и сильное предположение, что 
порядок и связь идей такие же, как порядок и связь вещей. Раз бог един, 
то рациональность есть во всем. Человек может начинать с вещей, а может 
начинать с идей. Эта мысль тоже, раз неоднократно повторял Эйнштейн, что 
он строил ОТО как чисто математику, а потом нашелся выход куда надо. 
Догадки, математические гипотезы, а потом на основе их экспериментальные 
вещи. Эти идеи Спинозы вошли конечно в арсенал философии науки, к ним 
обращаются и сейчас, а спинозизм стал синонимом вольнодумства или даже 
атеизма. Хоть сам Спиноза отвергал религию, но говорил о боге.

\hfill\textbf{Oct 11}

\subsection{Готфрид Вильгельм Лейбниц}

Способствовал организации Питербургской академии наук в 1725 году 
выпущен указ о создании. В Европе организация науки шла постепенно 
и естественным путем, в эпоху возрождения появление интеллектуалов, 
кружков таких вот, потом академических сообществ. В России же все было 
по приказу, создать академию наук, был небольшой универ, ученых не было, 
их приглашали из-за границы. Лейбниц вот сыграл большую роль, предложил 
форму организации по образцу из Европы. Как ученый он известен, что 
поправил уравнение Декарта о сохранении импульсов при столкновении двух 
тел. Создатель математической логики, как рационалист верил, что мат 
логика есть будущее науки в любом из вариантов развития науки. Говорил, 
что настанет день, когда философы перестанут спорить, а скажут, как 
математики, сядем за стол и посчитаем. Он панпсихист или органицист, не 
считает, что животные это автоматы, признает психическую деятельность 
у животных. Панпсихизм даже на уровне неорганической материи. Он 
считает, что фактически бог устраивал мир по правилам логики 
и математики, но психическая деятельность неравноценно распределяется 
даже среди живых организмов и даже в самом человеке. Есть рациональное 
ядро, свет, прожектор человеческой души, а есть более затемненные 
теневые стороны сознания, фактически философы науки считают, что Лейбниц 
впервые заговорил о неосознанных процессах человеческой психики. Они 
есть, но мы их не осознаем. Неосознанные процессы или подсознание он 
сравнивает с шумом прибоя морской волны. Прибой мы слышим, это суммарное 
действие всех микро частиц воды, отдельных волн, которые мы собственно 
не слышим. Это по сути речь о пороговых эффектах. Мы не можем 
воспринимать один фотон даже хотя он действует на нас. Организм вот 
подвергается на периферии сознания есть такое подсознание. Здесь можно 
подчеркнуть принципиальную разницу между подсознанием = периферия и тем, 
которое потом будет у Зигмунда Фрейда. Для контраста нужно заметить. 
У Фрейда наоборот именно подсознание в конечном счете определяет всю 
деятельность человека. Главным у Лейбница остается рацио. Дальше можно 
отметить у Лейбница представление о законах логики Аристотеля как 
образце, показателе рациональных истин в нашем сознании, он дополняет 
эти три закона логики четвертым законом, который имплицитно был 
и у Аристотеля, но Лейбниц уделяет особое внимание. Закон достаточного 
основания или закон достаточного обоснования. Так же как Спиноза для 
Лейбница это есть законы мироздания, мы можем к ним приобщиться, Лейбниц 
впервые ставит вопрос в плане онтологии, почему мир такой, какой он 
есть, почему мир вообще существует, если не существовать даже легче. Мир 
существует и именно в том виде, в каком мы знаем. Современная астрология 
этот закон интерпретирует как антропный принцип. Совокупность физических 
констант такая, что дает возможность появления нас самих. Если чуть 
изменить, то некому было бы спрашивать, почему мир вообще существует. 
Этот принцип в онтологическом плане приходит как раз из закона 
достаточного обоснования. А в логическом плане или гносеологическом этот 
закон также очень важен, до каких пор мы можем проводить анализ, чтобы 
прийти к этим логико-математическим истинам, то что Декарт называл 
интеллектуальной интуицией. В онтологическом плане Лейбниц считает, что 
каждая вещь может появиться только потому что другое уже существует, так 
и в гносеологическом плане все новое может утвердиться потому что может 
быть доказано на фоне того, что может быть признано. Если вписывается 
в систему признанного знания, то доказательство получается проведено. 
Лейбниц приходит к мысли, что сама по себе идея врожденного знания 
непростая, нельзя понять, что мысли какие-то прям написаны у нас 
в сознании. Лецбниц говорит, что есть диспозиции в нашем сознании, 
ничего готового нет, а чтобы понять их найти надо проделать непростую 
работу. Апперцепция -- ап перцепция -- ап восприятие -- ретроспекция -- 
близко к рефлексии -- по отношению к своему сознанию, к поиску истин. Он 
сравнивает этот процесс с работой скульптора, который увидел в глубине 
мрамора прожилки, и мысль что следуя прожилкам можно создать скульптуру. 
Отсечь все лишнее. Чтобы добраться до истин разума надо серьезно 
потрудиться, отсекать лишнее не так-то просто. Таким образом Лейбниц 
вводит понятие вечной истины, добываемые путем апперцепции, они все 
добываются логико-математическими методами. И это основные истины науки. 
А есть еще простые истины фактов. Посмотрим за окно -- сегодня светит 
солнышко, вчера был дождь. Но это другое, просто констатируем те или 
иные явления, но не добираемся до сути. Если спрашивать почему солнышко 
или вчера был дождь, то в конце концов придем к тем вечным истинам. 
Начинаем с фактов самых простых и очевидных, а в конечном счете ищем вот 
эти вечные истины. Вот это рационализм в духе Декарта и Спинозы, все 
истины рациональны и эта рациональность проистекает из рациональности 
самого мироздания, то, что по сути закон достаточного основания является 
как онтологическим так и гносеологическом. Но можно еще дальше идти. 
Попытаться метафизически обосновать структуру мироздания. Почему эта 
структура носит рациональный характер, а не хаотический, когда мы не 
можем ничего вывести. Для этого Лейбниц обосновывает все 
гносеологические и методологические вещи с помощью онтологии, которые он 
объясняет монадологией. У Лейбница это не просто единица, а некая 
духовная сущность, типа атома Демокрита, но духовная, и сумма духовных 
сущностей есть все мироздание. Эти сущности созданы богом, задает 
программу им тоже бог. В то же время они обладают своими специфическими 
качествами, о которых мы уже поговорим после перерыва..... Итак, все 
мироздание это объединение идеальных мельчайших частиц вот этих монад. 
Диалектические принципы к пониманию организации и действия монад. 
С одной стороны, монады созданы богом в конечном счете, и только он 
может превратить их вникуда. Но монада обладает творческим импульсом 
и силой, бог наделяет этим. Второе, монады организованы иерархически 
в виде некоторых пирамид. Монадой всех монад является бог, а в основании 
лежат так называемые спящие монады, неорганическая природа, но насколько 
она неорганическая Лейбниц конечно сомневается, поскольку непонятно где 
существует барьер между неорганической и органической. А дальше они 
постепенно просыпаются, одухотворены, обладают виетальной силой, 
и главной силой является сознание. Наполняемость духовной сущностью 
различна. Третья такая дуальность -- монады с одной стороны автономны, 
замкнуты, без окон и без дверей, не могут подсмотреть что творится 
с другой монадой, но с другой стороны каждая монада отражает в себе все 
мироздание. Лейбниц тут воскрешает или переформулирует идею фрактала 
в мироздании. Ее поднял еще Анаксагор в понятии гомеомерия, мельчайшая 
структура, где отражается все во всем. А раз в каждой монаде отражается 
все мироздание, то они в конечном счете синхронизированы. Если у Декарта 
была синхронизация материального и идеального, то у Лейбница вот монады. 
Поскольку их построил господь бог, то в конечном счете это приведет 
к гармонизации мироздания. Да, мир создан богом в этих монадах, есть 
зло, но зло не имеет онтологического статуса (как темнота), 
и синхронизация приведет к лучшему из миров. Все к лучшему в этом лучшем 
из миров. Монады работают, зло уменьшается, они все синхронизированы, 
все вот так вот. Над этим тезисом можно конечно много, в итоге 
Шопенгауэр придет к совершенно противоположному выводу на основе теории 
воли к жизни скажет, что все к худшему в этом худшем из миров. Еще 
скажем о чем, Лейбниц вступает в полемику с Ньютоном по проблеме 
структуры мира и ключевым понятием физики пространства, времени 
и движения. Он один из первых критикует силу тяготения, говоря, что это 
смешно, оккультная сила из средневековья, дальше он отвергает 
субстанциальное восприятие пространства времени и движения, и говорит, 
что это лишь атрибуты монад. Пространство, время и движение это не 
фоновые декорации, на фоне которых разыгрываются события, 
а непосредственно связаны с этими духовными сущностями. Релятивизм. 
В конце это заканчивается на Эйнштейне понятное дело. Останутся ли 
пространство и время, если убрать все. У Эйнштейна да остается такой 
ящик, а у Лейбница конечно все исчезает. Исходная точка расхождения. 
Последний штрих это так называемый позорный спор о приоритете в открытии 
дифференциально-интегрального исчисления. Спор между Ньютоном 
и Лейбницем, переписки, Ньютон жалуется об ущемлении своих прав 
в Лондонское королевское общество, как президент его создает комиссию, 
которая конечно приводит к выводу, что Ньютон первый. Но опубликовал 
первый Лейбниц. И символика современная принадлежит конечно Лейбницу, 
а не Ньютону. Дальше в науке конечно немало таких споров будет, но вот 
это так.

Декарт, Спиноза, Лейбниц говорили, что фундамент науки надо искать 
в рациональности, ибо само мироздание рационально, основные истины это 
логико-математические. У кого-то они написаны, у кого-то надо стараться.

\subsection{Дальше}

Теперь к другому направлению -- эмпирико-индуктивный метод, эмпирист, 
сенсуалист. Иногда отождествляют, но не по праву, все эмпирики конечно 
сенсуалисты, что нет ничего в разуме, чего не было в органах чувств 
(кроме самого разума -- Лейбниц). Но эмпирики вводят предпосылку, что 
наши чувства это проводники к внешнему миру. Мы не просто имеем сигналы 
на наших органах чувств, это проводники, и правдивость обосновывается 
тем, что мы как-то живем ведь. Но это сложно проверить потому что 
существование тоже органы чувств... А сенсуалисты Беркли и Юн согласны, 
что органы чувств это органы, где мы имеем сигналы, но не являются ли 
они как раз барьером к постижению миру, не обманывают ли они нас. Они не 
доходят до так называемого солипсизма, когда ты вообще можешь оказаться 
один на всем свете, все остальные люди под вопросом, раз тоже даны 
в ощущениях. Главное в том, что органы чувств не считаются проводниками, 
а могут и как барьер, искажение реальной картины. В свете современной 
науки с квантами, большим взрывом, относительностью, какие тут вообще 
органы чувств. Но те философы, о которых сейчас будем говорить, как 
Бэкон, считают, что органы чувств передают в мозг инфу, и ошибаются не 
органы, а мозг при обработке. Совершенная фигура обращения планет это 
круг говорилось, Кеплера это долго тормозило. Палка в воде кажется 
сломленной, а раз мы видим типа она сломана, это недостаток обработки, 
а не органов чувств. Это принципиальное расхождение между эмпиризмом 
и рационализмом. Крайностью считается, что разум человека при рождении 
-- чистый лист бумаги. Дальше по мере взросления заполняется чистый лист 
при рождении. Локк говорил, что могут быть какие-то диспозиции, которые 
проявляются вещи, но их можно обнаружить при экспериментах.

\subsection{Томас Гоббс и Джон Локк}

Слышали у них про государство и общественный договор. Человек человеку 
волк, война на истребление, чтобы не истребить друг друга все такое. 
У Гоббса построение монархического государства без коррупции, а у Локка 
наличие оппозиции, либерализм, В эмпиризме уже меньше думают 
о мироздании, больше концентрируются на познании. Гоббс приходит 
к мысли, что понятие материи как метафизическое очень расплывчато. 
Материя, говорит, совокупность всех природных систем, всех природных 
объектов. Материя становится не материей прима, подпоркой, основой, 
а лишь собирательным понятием, которое обобщает все реальные системы. 
Кроме этого у Гоббса возникновение языков, не было пра-языка, к чему 
ведет развитие языков, сольются ли, где граница между естественным 
и искусственным, та же математика в науке искусственный язык. Когда-то 
в библии был один язык, как-то общались, люди умножались, но язык 
оставался один, но люди возгордились, начали строить вавилонскую башню, 
посмотреть что там наверху, а бог взял и создал огромное количество 
языков, люди перестали понимать друг друга, и все. Второе уже дает 
Платон, говорит, что за каждым словом, пропозицией, стоит определенная 
идея, а через нее отражается и мир. Имена не случайны, они проистекают 
из идей, которые находятся в царстве, а реальные события под названием 
этой идеи существуют в действительности, полумистическая теория. А Гоббс 
ставит впервые на более-менее научную основу говорит, что язык по сути 
тоже конвенция.... Каждый может на основе своих голосовых связок 
изобретать любое название для любого объекта. Первые слова обозначения 
это метки, и они могут закрепляться за знаками. В первых языках знаки 
были похожи на реальные символы. Знак метка остались давно, на лавке, 
где торгуют вином, может висеть знак бочонка или гроздь винограда, 
отражает суть. Рыцарские гербы тоже знаки для интерпретации. С этого 
начинается язык. Дальше проблема коммуникации между людьми. Что 
поднимает важную проблему, что язык обладает довольно большим 
количеством функций, самая простейшая это соматическая, реакция на 
раздражители внешней среды, многое здесь сходно с животным миром: крик 
от боли, радости. Дальше функция сигнальная, знак об опасности, тоже 
у стадных животных есть, когда сторожевой кто-то кричит, и стадо 
срывается. Дальше функция коммуникативно-описательная или объяснительная 
функция. Когда есть необходимость что-то объяснить и показать на словах 
и на знаках, которые эти слова обозначают, чтобы потом самому не 
участвовать. Наглядный пример это карта, можно по ней ориентироваться 
даже не бывав никогда. Ну и если путешественник первооткрыватель 
рисовать карту, и дальше не участвовать, а пусть другие сами. 
Описательная функция у некоторых животных существует, у пчел например, 
у муравьев похожее есть. У человека наиболее важная и отличительная это 
аргументативная функция. Для познания нового надо участвовать 
в дискуссии, доводы за и против, все это невозможно без довольно 
насыщенного, полного, объемного языка. Словообразование и прочее тут 
важны. Все это, две последние тоже, показывают конвенциональный 
характер. Все языки -- искусственные образования. Их образовали люди на 
основе конвенции. Их образовали люди просто давая ясные определения 
используемым вещам. Гоббс вступает в дискуссию с Декартом, что нужны 
дефиниции. Спиноза тоже говорил так, но у Гоббса сильнее.

\subsection{Джон Локк}

Это он сказал в Опытах о человеческом разуме, что вот этот чистый лист, 
нет ничего в разуме чего не было в чувстве. Разделяет на первичные 
и вторичные качества. Первичные это сенсорика, а вторичные это 
рефлексия, введение внутреннего опыта как время протяженность число итд. 
И те, и другие получаются из опыта. Все из опыта. Но опыт может быть 
внешним через органы чувств и внутренним, рефлекции, когда погружаемся 
в себя и обнаруживаем эти понятия или идеи. На основе опыта появляются 
идеи. Тогда наше все познание предстает в виде трех направлений: 
интуитивное, на уровне органов чувств, треугольник ограничен тремя 
прямыми и белое не черное, демонстрационное, сенситивное. Идеи могут 
быть простыми и сложными. Простая идея на уровне органов чувств это 
цвет, а сложная например друг. Простые идеи на уровне рефлексии это идея 
протяженности пространства времени движения. Сложные на уровне рефлексии 
это субстанция, модус, отношение. Все из опыта, опыт на две группы 
делится, получаем идеи и там и там простые и сложные, и проблема истины 
будет таким образом заключаться в правильной комбинации этих идей, как 
полученных из внешнего опыта, так и из внутреннего. Истина это 
комбинаторика этих идей, которые сами проводим, и при ошибках всегда 
можем сказать, что неправильно провели комбинаторный анализ. Все идеи 
бесспорны, а комбинирование при обращении к тому или иному пласту 
реальности это задача нашего разума. А против самих рационалистических 
убеждений Локк применяет метод наглядной агитации эмпирический метод. Ну 
у детей нет врожденных инстинктов. Не чувствуют противоречий, не 
понимают законов логики. Так же аборигены тоже не владеют даже во 
взрослом состоянии. Люди с дефектами психики тоже. Даже идея бога везде 
разная. Такого рода эмпирическое сравнение, психология, показывает, что 
рационалисты правы и знание всегда упирается в конечном счете в органы 
чувств.

\hfill\textbf{Oct 18}

Исаак Ньютон основатель классической физики, философские основания 
теории, сам он признавал, что вся теория основана на философских 
постулатах, атомизм, абсолюты пространства, времени, движения, сила 
тяготения, идея бога чтобы солнечная система была устойчива, критика 
и полемика с Картезианцами, поскольку у Ньютона дальнодействия, у них 
близко. В конце концов победа Ньютона несмотря на необъяснимость 
постулатов, можно было выводить законы Кеплера, потом даже отыскание 
новых небесных тел.

Дальше Лейбниц, его полемика с Ньютонианцами по поводу пространства 
и времени, Лейбниц предложил отношение к ним как способ организации 
положения всех монад, некие духовные сущности лежат в основе всего 
и вся, степень духовности разные, спящие, высшие, бог даже, в то же 
время монады автономны, активны, действуют по единой программе, отсюда 
оптимизм все к лучшему в этом лучшем из миров, принцип передачи импульса 
от точки к точке. Рационализм, признание априорных истин, подчеркивание 
роли закона достаточного основания или обоснования, онтологический 
характер, в мире все устроено так, что для появления нужно чтобы было 
сопряжено. И тут постановка вопроса почему мир существует, ведь легче 
с точки зрения физики не существовать. Истины вечные и истины фактов. 
В основании всего и вся лежат истины вечные, разума, а истины фактов 
должны выводиться из вечных.

Дальше это эмпирики и материалисты, Гоббс и Локк, теория языка в том 
числе, договорный характер искусственный, не божественный. Различные 
функции языка от Гоббса, самая важная аргументативная, отделяет речь 
человеческую от других звуковых сигналов, существующих и у животных. 
Локк с критикой рационализма, нет ничего в разуме, чего не было бы 
в органах чувств, все происходит из опыта, но опыт может быть как 
внутренний, так и внешний. Идея опять разделение всех качеств на 
первичные и вторичные, идеи простые и сложные, как на уровне чувств, так 
и на уровне рефлексии. В конечном счете проблема истины это проблема 
объединения или комбинаторного искусства. Органы чувств не ошибаются, 
а ошибается ум, неправильно соединяя.

Теперь новое сегодня два философа, Джордж Беркли и Дэвид Юнг. Они 
проблематизировали все предшествующие теории познания, а как дальше 
скажет Кант, Юнг пробудил его от догматической спячки. До этого считали, 
что знание проистекает естественным путем, а они поставили вопрос как 
происходит познание, можем ли доверять. Проблематизировали сам процесс 
познания. Лежат в основе вещи, которые сами не проверяемы, ученые должны 
брать их на веру. Тут как онтологические, так и гносеологические. Самый 
главный и первый вопрос это с чем вообще ученые и люди имеют дело, 
с проблемой ощущений. Мы имеем только ощущения, и логически опровергнуть 
нельзя, всегда имеем только ощущения, и дальше верим, что за ощущениями 
стоит объективный мир, что познаем не только опыт ощущений, а познаем те 
вещи, которые вроде бы ощущаем. В это ученые должны поверить. Дальше 
ученые должны поверить, что в мире существует единообразие, и те 
регулярности и законы в одном месте должны повторяться везде. Третий 
важный элемент, что мир довольно устойчивый, Юнг и Беркли подчеркивали, 
что устойчивость мира должна постулироваться. В этой устойчивости 
регулярности носят повсеместный характер и мы можем их познавать. Самое 
удивительное в мире, что мы можем его познавать, кто-то еще другой 
сказал.

\subsection{Джордж Беркли}

Сам он священнослужитель, миссионер, выезжал в Америку, написал много, 
разговор между филосом и д... Долгое время воспринимались идеи даже 
с насмешками, тот же Байрон, но в 19 веке идеи снова всплыли, и за 
Беркли если раньше клеймили как философа близкого к солипсизму, но в 19 
веке появился позитивизм, и лозунгом стало наблюдаю следовательно 
существую (для науки существую), все остальное метафизика или религия, 
вспомнили снова о Беркли. Стали его называть первым феноменологом, 
признает не сущности, причинно-следственные связи, а связь феноменов, 
связь между явлениями. Остальные разговоры о сущности и первопричинах 
для науки несущественно. Беркли подходит так, что выдвигает идею 
репрезентации. Мы не можем познавать некие абстракции в силу того, что 
они никаким образом не проявляют себя в опыте, всегда имеем дело 
с отдельными представителями, берем их как показатель определенного 
класса явлений, возрождает традицию номенализма, о которой говорили 
в средневековье. Общие идеи только сотрясение воздуха, а реальные вещи 
одиночные только существуют. Мы не можем представить треугольник вообще, 
всегда какой-то именно определенный. Понятие человек то же самое, что 
значит понятие человека. Мы всегда должны говорить худой, высокий, 
национальность, раса. Его еще называют имматериалистом. Материя с его 
точки зрения пустое понятие. Материя не несет никакой научной нагрузки. 
Ньютон говоря о материи говорит по сути ни о чем, все эти атомы, 
пустота, лишены смысла абстракции. Никакой материи не существует, 
никаких сил нет, никаких причинно-следственных связей нет, существуют 
связи в природе. Для Беркли реальной силой обладает лишь господь бог, 
а все эти законы физики не более чем математические гипотезы, более или 
менее удачно представляемые. В этом плане первый критик физики Ньютона. 
Во многом основываясь на его идеях Мах потом начнет критиковать. Беркли 
конечно считает что прежде всего виновато понятие Материи. Пример 
нередко приводит. Возьмем яблоко, что для нас яблоко. Яблоко имеет 
определенную форму, цвет, вкус, запах. Мы можем включить все наши органы 
чувств и на основе всех них имеем представление о яблоке. Яблоко это 
комплекс наших ощущений. Мир это комплекс наших ощущений, существовать 
значит быть воспринимаемым. Проводя это близко очень к солипсизму 
получается, поэтому он говорит, что если кто-то, единичный дух, не 
воспринимает нечто, вот он видел дерево во дворе, а потом ушел домой, 
дерево не пропадает, потому что есть другие люди, на него смотрящие. Бог 
так или иначе воспринимает мир. Существование мира обусловлено 
восприниманием богом. Как у Беркли, если с материалистических позиций, 
как это без субъекта нет объекта, вот раскапывает говорит существовало 
вот такое, есть геология тоже говорит, есть биология, говорящая, что 
человека когда-то не было, есть космология. Берклианцы, современные 
есть, Шопенгауэр, говорят, что переносим туда виртуального субъекта, 
переносим себя. Мир начинает свое существование как только открывается 
первый глаз. Говорят даже, что вселенная была запрограммирована так, 
чтобы появился наблюдатель. Это тот стержень, на который нанизываются 
все эти ощущения, на духовную субстанцию. Конечно эта теория довольно 
диковенная, существует лишь сам субъект, в общем-то да, но мы должны 
думать как ученые, а говорить как толпа. Разведение этого обыденного 
сознания, обыденного опыта и научного, мы видели даже в Элейской школе 
с эпистеме и доксе, Беркли возвел в мировоззренческий принцип.

\subsection{Дэвид Юнг}

Светский человек, хоть и писал даже по поводу религии, где сказал, что 
любая религия есть психологический опыт, любая религия имеет корни. 
В отличие от Беркли идет еще дальше, упраздняет духовную субстанцию, что 
это было еще с Сократа познай самого себя, но наше сознание это поток, 
мы не контролируем, это некий пучок ощущений, впечатлений и идей. Мы их 
можем тоже познавать с помощью внутреннего опыта, опыт остается главным, 
но понимание опыта -- опыт ассоциации, ассоциативный опыт, не 
логико-рациональное выстроение, да все наше познание основано на 
ощущениях, но из них можно выделить впечатления и идеи. Впечатления 
здесь и сейчас, яркие, живые, энергичные. Идеи -- воспроизводим 
результат ощущений по памяти, и репродуктивное впечатление, вспоминаем 
о чем-то, закрываем глаза и воспроизводим объект от здесь и сейчас, 
а можем и фантазировать, но в конечном счете это сочетание различных 
комбинаторико кого-то ощущений. Механизм связей между ощущениями 
ассоциативный. Связываем ощущения по определенным принципам. Один -- 
сходства или контрастности. Черное белое, если портрет, значит 
впечатление сходства с реальным прототипом, похожесть человека, это вот 
один тип ассоциативных связей, второй тип это ассоциации по пространству 
и времени. Его пример там если говорят собор Сеон де Ми, то вот 
с Парижем, если колокола, то идти на службу или еще что. Оба типа дают 
возможность ориентироваться в этом мире. А вот третий тип ассоциаций 
идет в науку и называется причинно-следственная связь. В этом 
принципиальное отличие науки от обыденного знания. Юнг проблематизирует 
и эту ассоциативную связь. Возьмем бильярдный стол, какой-то Адам 
приходит, он не видел ничего больше, и вот он наблюдает за игрой 
в бильярд. Игрок с помощью кия бьет по шару, он перемещается, бьет еще 
один шар и инициирует движение. Действительно ли Адам наблюдает 
причинно-следственную связь? НЕТ! Это пространственно-временная 
смежность, движется один потом другой, а причинно-следственной связи 
нет. Адам понятия не имеет, что есть причина. Нужно набраться опыта, 
тогда уже может заявить, что движется потому что попал. Ну вот 
пронаблюдал один раз, другой, третий, но когда делать обобщение? Оно 
ведь всегда неполное. И вот Юнг говорит, что ассоциативная связь потом 
отождествляется. И все так в науке, все методы в конечном счете 
упираются в индукцию, а она не полна. Не имеет смысла говорить 
причинно-следственной связи. На деле это психологическая, привычка 
такая. Это очень серьезный аргумент, Кант говорит даже, что это 
пробудило его от догматической спячки.

И Беркли, и Юнг построили свои вещи на критике по Гоббсу, где первичные 
и вторичные, а эти двое сказали, что у нас нет доступа к первичным, все 
в конечном счете зависит от нашего восприятия, нет прямых качеств.

\subsection{Французское просвещение}

А дальше о ком мы говорим из Французов говорят, что все качества объекта 
фактически первичные, и хотя логически нельзя опровергнуть тезис Беркли, 
но то, что ощущения дают верную в основном картину мира объясняется тем, 
что мы живые существа, и мы не могли бы выжить, если б получали 
искаженную информацию о мире. Это из наблюдений за животными, что если 
страдает сенсорика, они быстро удаляются из среды. Сначала в целом 
о французском просвещении. Эти философы работают накануне великой фр 
революции, 18 век, подготовили своими трудами фр революцию, которая во 
многом изменили мир. Кабинская республика, диктатура, Наполеон. Их 
называют философами просветителями, раз они поставили вопрос о гос 
образовании, построение царства разума на земле, невозможно без науки 
и техники, написали первую светскую энциклопедию, 28 томов, разошлась по 
всему свету. Пытались образовывать правителей, Дедро с Екатериной 2, 
просил про крепостное право отменить. Делят на два лагеря, деисты 
и материалисты. Деист Вальтер, Шарль М Кондельяк, Руссо, хз кто еще. 
Философским идеям Вальтера прибавляется определенная доля организацонной 
деятельности, литературная деятельность, политическая, хотя являлся фр 
академией наук ссылали. Во многом способствовал распр идей Ньютона, 
философские пиьма Вальтера, сравнение Декарта и Ньютона, как же так 
физика претендует на рациональное знание, тем не менее на континенте 
материя сплошная, на туманном альбионе атомы и пустота, на континенте 
земля вытянутая как кабачок, в Англии как тыква, на континенте вихри 
корпускул, В Англии силы. Мадам де Шатли перевела мат начала натур 
философии на франц, а сам Вальтер написал обзор самой системы Ньютона. 
В философском плане Вальтер как остальные говорит что мир создан Богом, 
но деистический вариант там создается и дается импульс, а дальше Бог не 
властен. Наиб новые мысли Монтескье и Кондельяк. Монтескье 
географический детермизинм. Пытается показать, что устройство 
государства, политика, во многом обусловлены объективными окружающими 
факторами среды. Выделяет три наиболее важных фактора: климат, рельеф 
или ландшафт и почва. Климат может быть такой как в Европе умеренный, 
а может как в Африке. Говорит где слишком жарко люди ленивые, мало 
работоспособны, слабая воля, легче захватить власть какому-то человеку. 
Наоборот в Европе более бодрящий климат тенденция к свободе. Ландшафт во 
многом определяет полит устройство, где есть пересеченная местность 
культуры обособляются и растут вверх, развивается культурная 
деятельность людей, а где степи, культура расползается, строительство 
и систематизация культуры проблематичны, сильны всегда настроения 
сепаратизма, здесь нужна монархическая или даже деспотическая власть. 
Почва тоже имеет различные вещи, где очень плодородное, культура не 
очень процветает, еды хватает, на полит устройство не обращают внимание. 
Где не плодородно, приходится ремесленничеством заниматься. В пример 
приводит Голландию, где были линзы, корабли, кружева голландские. Тоже 
могут быть крайности и нужно считаться. Каждое из положений 
в отдельности уязвимо, климат не менялся сильно, а гос устройство 
менялось часто. Но вместе заставляет задуматься, что география влияет на 
полит устрой. Сейчас даже есть геополитика, построенная на идеях 
Монтескье, каждое государство проводит политику на основе своей 
географии. Сейчас сильный пример по поводу освоения Арктики, либо она 
принадлежит государствам, кто с ними граничит, либо принадлежит всем. 
Сейчас там свободное прохождение по морскому пути. Кондельяк теперь 
выделим его из лагеря дуалистов. Известно его попытка доказательства 
важности наших ощущений для опровержения положения Беркли, что мир есть 
комплекс наших ощущений. Беркли проводит мысленный эксперимент 
с оживлением статуи, что нужно сделать для превращения, наделить 
ощущениями. По очереди включает сенсорные датчики на статуе постепенно, 
но каждый из них недостаточен для достоверности в мире, и решающим 
оказываются тактильные ощущения. Каждое отдельное ощущение недостаточно 
для удостоверения в мире, а все вместе плюс ощущения других объектов 
получаем представление о внешнем мире.

Деди Дедро, Клод Гельвеций, Гальбах и Жульян Нометри -- атеисты. Впервые 
в философии атеисты. Их идеи были наиболее прогрессивные и для науки, 
и для философов, на них опирались и Фейербах, Маркс, Энгельс, Ленин. 
Коротко какие идеи были. Прежде всего идеи об окружающем мире. Проблему 
существования решали ощущения вот эти что мы живем как-то, дальше как 
устроен мир. Они считали, что в основе мироздания лежит материю, но 
понимают ее как совокупность всех природных тел как Гоббс. В фундаменте 
всех природных тел лежат мельчайшие частицы, но понятие атома не считали 
необходимым, потому что не нравились атом и пустота, а вот молекулы да. 
Мир состоит из молекул, и эти молекулы обладают изначальной 
чувствительностью, в основе лежит что-то. И камень чувствует. Но дальше 
там по усложнению появляется сенсорика и органы чувств. Материальные 
системы существуют в пространстве и времени, а Ньютон описал силы 
природы, а не бога, и силы природы свидетельствуют об активности 
природы, творческом потенциале. В самом фундаменте лежит активность 
природе. Итак, мир материален, и принцип единства мира впервые высказан 
этими философами, что мир един в своей материальности, все мироздание 
и есть эта материя, и богу не отводится места. Материя существует 
в пр-ве и времени именно как Ньютон задает. Движение, они обходятся без 
деистического принципа, материя и движение неразделимы, если мы не видим 
движение, не значит, что не существует в других формах. В целом 
философском плане движение представляется как изменение вообще, и это 
изменение происходит в разных формах. Есть механическая, химическая, 
биологическая, общества даже. Есть фундаментальные формы изначальные, 
а можно и сказать, что высшие формы так или иначе надстраиваются, но не 
сводятся к ним, это тоже очень важная мысль. Итак, постоянное движение, 
из одной формы может в другую, но не только это, и ставится вопрос не 
только о движении, но и о развитии. Мир не только в движ, но 
и в разитии, и оно может наблюдаться в самых различных формах 
и состояниях, если системы материальные есть уровня космического а есть 
какого-то. Развитие материи как в космологическом плане, так и Земном. 
Впервые ставится вопрос появления нашей вселеннной (естественно 
в известном тогда смысле). Идея космогенеза -- мир не создан богом, 
должен был как-то появиться, даже наша земля должна иметь естественное 
происхождение. Идеи космогенеза и геогенеза, как земля появляется 
и развивается, что мы видим в плане рельефа считают не всегда так было, 
и были эмпирические факты, что в Альпах ракушки. Появились 
естествоиспытатели нептунисты и вулканологи. Определяющую роль сыграл 
мировой океан или вулканы, соответственно. В конечном счете обе идеи 
приняты сегодня наукой. Вот это вторая идея появления нашего айкумена, 
места жительства на земле, тоже не вдруг и не сразу. И третий момент 
развития это появление жизни, откуда взялась жизнь. Опять если убираем 
бога, надо объяснять. Две линии: или жизнь существовала всегда (и тогда 
является атрибутом материи) и была занесена на землю, а вторая линия, 
что жизнь появляется на определенном этапе из неогранической природы, но 
раз так, то что она может появляться всегда и везде при определенных 
условиях или это какой-то сингулярный, но только потом доказали уже что 
в наше время органика только из органика появляется. До сих пор не 
показали как из неорганики появляется органика. Следующий важный вопрос 
это причинно-следственные связи. Здесь тоже фр материалисты считают, что 
они существуют в самом объективном нашем мире, и это не ассоциации 
какие-то, а наоборот очень жесткого типа причинно-сл связи. Зная импульс 
и координату тела, можем предсказывать что будет с ним и сказать что 
было. Гольбах тут. Написал системы философии и был объявлен личным 
врагом Папы, показал, что все предопределено, в каждом облаке пыли или 
волне, если б могли рассчитать короче. Лапласовский детерменизм потом 
назвали, все знают что это такое. Лаплас придумал уже свое существо 
демон Лапласа и все, о чем оворил Гальбах. Если бы демон знал все 
импульсы всех частиц, развернул бы всю эту карту. Человек не свободен ни 
одну секунду своей жизни. И дальше важный момент единство не только 
материального, но и живого, Ламетри медик(?), пишет человек машина 
и человек растение, пишет Декарту, что у человека и животных органы на 
одних и тех же принципах, сенсорика, а потом мышление это функция мозга, 
в конце концов результат развития самой материи, и душа как часы, если 
сломаны, если тело сломано, то и души не будет. Идея в том, что если 
считать, что животные это автоматы, то человек такой же автомат. Но если 
человек мыслящее существо, то нельзя отказывать в определенных 
когнитивных процессах и животным. В конечном счете процесс познания 
представляется в плане как источник это органы чувств, не обманывают, 
обманывают органы чувств, наука считается высшим этажом познавательной 
деятельности, многие вещи обнаружили вот эти французы. Эволюция была 
грубой такой вот хаотичной, но отбор. Ну и гениальные идеи о связи 
электрических и магнитных явлений, если в мире все связано, он вот 
высказал. И последнее отметим идея Гельвеция гипотеза о когнитивном 
равенстве всех людей и необходимости образования. До этого существовала 
идея, что люди наиболее способные достойны образования, по крови, 
дворяне, выделяются среди людей, а вот он сказал, что мозг слишком 
сложная вещь, и не можем заранее сказать кто хорош а кто нет, 
образовывать надо всех. Жан Жак Руссо первый антисцеинтист, усомнился 
в том, что научно-тех прогресс и прогресс человечества в целом синонимы. 
Он сказал, что развитие науки может привести цивилизацию в тупик, а сами 
науки могут быть связаны с пороками, физика с любопытством, этика 
с спесью, геометрия со скупостью.... Руссо положил начало направлению, 
что любое научное знание несет амбивалентный характер. Это вот основные 
идеи эпохи просвещения.

\hfill\textbf{Oct 25}

Акцент на человеческом разуме, что мир познаваем, сам разум есть 
естественная эволюция природы, эта идея тоже присутствует, а наука -- 
высший этаж приспособительной способности разума к окружающей среде. Соц 
полит идея, что с помощью науки сцеинтистская идея, что только так можно 
построить царство разума на земле. Руссо единственный противостоял, 
говорил, что прогресс и научное знание не связаны, знание может носить 
амбивалентный характер, что собственно и подтвердилось потом.

\subsection{Немецкая классическая философия}

Основатель школы Иммануил Кант, затем Фихте, затем Шеллинг, Георг 
Вильгельм Гегель и Людвиг Фейербах. Это люди уже профессионалы, до этого 
все философы любители самоучки, а здесь уже это школа, многие люди 
кончали теологические факультеты, но все равно проходили универ 
образование и даже сами преподавали. Кант, Гегель, Фихте становились 
ректорами. Фейербах не попал туда потому что изгнали из-за материализма, 
но он тоже преподавал. Системы у них академические, 
образцово-показательные, классическая немецкая философия не спроста, 
разносторонние подходы, это система. Все они исследуют деятельность 
субъекта познания, условия, горизонты научного познания, основания его. 
Используют почти все метод диалектики как метод мышления, разрешения 
антиномий, стремление показать как развивается человеческое познание. 
Вот это собственно общее между ними, а так они очень разные.

\subsubsection{Кант}

Основатель школы всей. Сын ремесленника. Произведения первые его, 
критика чистого разума, сначала не произвели впечатление, но потом 
проникли в универы, он написал три диссертации. В философской биографии 
различают два периода: некритический и критический. Некритический это 
естествознание, сейсмология (наблюдал землетрясение разрушительное), 
космос бесконечность огромное количество галактик. Считается автором 
небулярной теорией солнечной системы. Наша галактика и солн сист 
образовалась из некой космической пыли, которая по законам механики 
Ньютона и приобрела вид как сейчас. Лет через 50 эту идею в более 
широком физ материале описал и развил Лаплас, и она получила название 
теории холодного происхождения вселенной. Механика не может объяснить 
даже гусеницу, хотя казалось бы. Тем не менее в докритическом он 
оптимист, наука может объяснить не все, то очень многое, даже лозунг 
был, дайте мне материю, и я покажу, как устроен мир. У него произведение 
всеобщая история и теория неба. Не был доволен Ньютоном из-за того, что 
теория не получилась такой строгой полной, центробежная сила была от 
бога. Кант не был первым, Лейбниц и Гофыва?фывфывафвыа?? хз. Сам Кант, 
когда прочел Юма, сказал, что Юм пробудил от догматической спячки, 
насколько мы можем вообще доверять нашему знанию, раз интуиция везде, 
законы могут в любой момент быть подвергнуты сомнению. Уже в зрелом 
возрасте Кант пишет три свои знаменитые работы, три критики. В этом 
ключе прямо задает тенденцию для последующих работ, и очень многие 
философы затем следовали этому примеру, давали такое название своим 
работам -- критика. Там он отвечает, как ему кажется, на определенные 
важнейшие вопросы познания и этики -- критика чистого разума -- что 
я могу вообще знать, и могу ли вообще что-то знать твердо и устойчиво. 
Ответ положительный, да, мы можем знать, но с определенными 
ограничениями. Дальше критика способности суждения нашего -- здесь 
исследуются вопросы искусства, нравственности, этики. И последнее -- 
критика практического разума -- говорили о ней в предыдущем курсе, что 
я должен делать в этом мире? Категорический императив Канта. Человек 
должен опираться на разум свой, на бога и на гражданские институты. Идея 
гражданских институтов, необходимых для прогресса общества как и наука. 
Каждая из этих критик постановка вопроса что я могу знать, что мне 
делать, на что надеяться, и в конце концов что такое человек. Последний 
исследуется в его антропологии. Человек -- очень сложное противоречивое 
существо, проблема свободы воли человека и в то же время детерминизма 
окружающего мира, нужно решать.

Критика чистого разума -- наиболее важный гносеологический вопрос. 
Нормативно законы, которые формулирует наука, должны быть всеобщие 
и необходимые, иначе это не закон, иначе просто идем по принципу 
индукции, набираем стат материал. Настоящий закон должен быть всеобщим 
и необходимым. Анализ типов утвеждений в трех частях. Трансцендентальная 
эстетика. Слово эстетика не в нашем смысле, не искусство, у Канта это 
наука о чувственности, трансцендентальное чувствование, сам субъект 
представляется трансцендентальным. Дальше трансцендентальная аналитика. 
И третье -- трансцендентальная диалектика. Здесь Кант исследует 
антиномии человеческого разума, в которые можно впадать, если ставить 
слишком широкие сразу вопросы. Нам известны априории Зенона, которые 
потом можно разрешить оказалось, стрела и покоится и нет. 
В трансцендентальной эстетике решается вопрос как возможна математика, 
возможна ли математика как наука, или тоже упирается. Ответ 
положительный. В аналитике решается вопрос возможна ли чистая и вот теор 
физика, где законы носят всеобщий и необходимый характер. Ответ тоже 
положительный. Диалектика дальше -- как возможна метафизика как наука, 
а тут ответ отрицательный. Логика видимости, никакие вопросы она 
окончательно решить не может в отличие от математики и физики. Все три 
части исследуют какого-то типа суждения. Суждение или пропозиция научны 
тогда и только тогда, когда статус всеобщности и необходимости, но чтобы 
дойти до такого статуса, нужно посмотреть детально, какие суждения 
вообще существуют. Выделяет два типа суждений, точнее две категории 
суждений. Первый тип суждений -- аналитические и синтетические суждения. 
Аналитические суждения -- когда фактически предикат суждения вытекает из 
объекта суждения. Можно развернуть, и будет тавтология. Все тела 
протяжены, вот такое суждение. Оно аналитическое потому что если 
зададимся вопросом а что такое тело, то придем так или иначе к тому, что 
это нечто, занимающее место в пространстве. Суждение, что все тела 
протяжены аналитическое. Приращения знания нет. Синтетические знания 
обладают этим качеством, например, все тела имеют тяжесть. Это 
синтетическое потому что из понятия тела не следует, что у него есть 
тяжесть. Эпикур первый ввел понятие тяжести, что атомы не просто летят 
по орбитам, а обладают тяжестью, летят сверху вниз, у Демокрита не было 
такого конечно. Итак, науку интересуют синтетические суждения, дающие 
вклад. Но синтетические коррелируют с еще. Апостериорные и априорные, 
существующие до опыта. Эмпирические это происходящие их опыта на основе 
наблюдений (сегодня ясная погода, вчера шел дождь), априорные 
соответственно те, что у нас в конечном счете получают звание доопытных, 
хотя к ним мы приходим все равно в результате опыта. Трактуется как 
трансцендентальный. Различия между трансцендентные и трансцендентальные. 
Трансцендентальные это априорные условия возможности самого опыта. 
Трансцендентные принципиально выходит за пределы опыта, о чем можем 
гадать, но в опыте нам не дано. Та же идея бога трансцендентна с точки 
зрения Канта, в органах чувств быть дан не может. Но как увидим дальше 
Кант считает, что и любая вещь в сущности трансцендентна. Вещь сама по 
себе или вещь в себе, ноумен. А то, что нам дано в познании, это всегда 
феномен. Ноумен вещь в себе трансцендентна принципиально непознаваема, 
а наука всегда феномен. Позиция Беркли чуть видна тоже. А теперь 
вернемся к анализу. Цель науки с одной стороны в суждениях, суждения 
должны быть синтетическими, прирост знания, а с другой стороны 
трансцендентальность или априорность. Ключевые требования к суждениям, 
которые претендуют на статус научных суждений, подразумевая закон науки. 
Первое что там все такое это математика. В его время -- арифметика 
и геометрия, где есть вычисления и где есть фигуры и линии. Как возможна 
арифметика, 4 действия арифметических, которые чел усваивает постепенно, 
на основе опыта, учат его складывать, абстракция какая-то сначала, потом 
вообще символы, позиционное исчисление, но это уже символы. А последняя 
стадия -- понимание категории количества. Все действия в конечном итоге 
обобщаются количеством. Мы приходим к этому через опыт, но когда освоили 
арифметику понимаем, что от нас не зависит. Это наша была проблема -- 
освоить действия -- но вне зависимости от наших знаний 4 действия 
существуют. Это показатель того, что они априорны. Но в принципе это 
можно отнести и ко всем мат операциям. Ощущение времени -- чувство 
времени, внутреннее чувство, что-то тикает внутри нас, самые 
быстродействующие компы работают по времени, а само время не зависит от 
вычислений, все это происходит на фоне времени, само время является 
опорной конструкцией, на основе которой и возможны все наши 
вычислительные операции. Кант очень приближается к Ньютону, у кого было 
тоже абсолютное время. Суждение априорное и синтетическое. Мы можем 
приходить через эмпирический опыт к тому, что кратчайшая линия -- 
прямая, но когда мы на основе опыта приходим к этому, понимаем, что от 
нас это суждение не зависит. Пространство это внешнее чувство, ощущение, 
на основе которого мы общаемся с внешним миром. Любая линия любая фигура 
находится в пространстве. Важнейшие науки вычислительные и геометрия. 
Возможны на основе того, что сущ априорные пространство и время и все 
действия, которые мы производим, вычислительные и геометрические, дают 
нам приращение знания. Кант говорит, что если раньше философы прежде 
всего ориентировались на сам объект, положение в пространстве и времени, 
то на самом деле мы должны выяснить наши умственные способности 
к синтезу синтетического и априорного. Отсюда его смелый вывод, что мы 
не столько черпаем законы из природы, сколько предписываем законы ей. 
Вторая часть трансцендентальная аналитика, как возможна физика как 
наука. Ответ снова положительный, но уровень ну ваще круче. Физика как 
наука возможна на основе синтеза чувственности и категорий. Категории 
тоже носят априорный или трансцендентальный характер. Эти категории 
упираются в конечном итоге в количество, качество, отношения 
(вероятностные, жесткие, бесконечные разнообразия связей). Каждое 
разбивает на еще мелкие, и на основе синтеза получаются общие категории, 
которые присутствуют вот так вот необходимо. Откуда они взялись? Можем 
прийти через опыт, но когда придем, они от опыта не зависят, наоборот 
опыт накладывается на них. Они присутствуют в неявном виде, даже за 
законами Ньютона стоят эти категории. Можно идти по линии примеров 
и категориальная сетка. Рассудок говорит, что нельзя мыслить ничего 
в научном плане без категориального аппарата, а сами категории от опыта 
не зависят, но приходим к ним через опыт. Второй уровень синтеза наук 
и ответ на вопрос как возможна физика. Трансцендентальное единство 
апперцепции. Человек познает сам процесс познания, сливаются 
и чувственность, и разумность, и самосознание. Кант сравнивает этот 
процесс с плавильной печью, где сплавляются в единое и чувственность, 
и рациональность, и разум как высшая ипостась. Это таким образом вторая 
часть критики чистого разума как возможно теор естествознание. И третья 
часть трансцендентальная диалектика. Логика видимости. Все вроде бы идет 
рационально, но приращения знания не наблюдается. В чем же дело. Кант 
считает, что принципиальным недостатком такого рода познания является, 
что субъект хочет познать сущность объекта. А сущность принципиально 
непознаваема, потому что любой объект дан через чувства. Кант 
проблематизирует этот вопрос, мы не знаем. Вещь в себе подает нам 
какие-то сигналы, но что это за все это мы не знаем. Рассматриваем любую 
вещь как сигналы на уровне чувственности и с помощью категорий каких 
можем. Происходит лишь коллапс. Логика видимости. Кант отсюда вывод 
делает, что метафизика не наука. Она под науку попадает только тем 
образом, что дает пропозиции, суждения, но претендует на абстракцию, что 
нет связи с миром. Антиномии. Первое -- мир имеет начало во времени 
и в пространстве или не имеет. Если предположим, что мир имеет начало во 
времени, то всегда вправе поставить вопрос, что было до того, тогда 
возникает вопрос как от пустого времени возник переход к длящемуся 
времени. Если мир не имеет начала во времени, значит время бесконечно, 
актуально бесконечно, тогда как это настоящее существует? Случайность 
тоже. Человек уже давно является феноменом. Бога в конечном счете 
выводит из морального принципа. Такого рода вопросы выходят за пределы 
науки, а метафизика пытается их решить, то метафизика не является наукой 
как математика и физика. Если бы Кант на этом остановился, то не сказал 
бы главное. Нельзя отбрасывать метафизику, такого рода рассуждения. 
Метафизика необходима для науки как дыхание для человека. Человеку 
свойственно выходить за пределы опыта. То что сегодня является 
метафизикой, завтра может на опыте. Как атомы.

\hfill\textbf{Nov 8}

\subsubsection{Фихте}

\subsubsection{Фридрих Шеллинг}

Противоположен Канту и тем более Фихте. Проявил он себя очень рано, уже 
в 17 лет стал магистром философии, в 21 профессором, но основные 
произведения написал до 30 лет, дальше мистический период. Нас 
интересует Идеи философии природы и Система трансцендентального 
идеализма. Тоже можно сказать прямо противоположные теории, но согласно 
Шеллингу в конечном счете дают ту философию, которую он назвал 
философией тождества. Тождество но учитывая и различия. Чтобы понять 
воспроизведем мысль, где он сравнивает свою систему с другими известными 
системами. Первое -- Декарт. У него была две субстанции, материальная 
и идеальная. Шеллинг говорит, что субстанция единая, но она изначально 
обладает духовным началом. Этот подход называется пантеизм -- у природы 
есть всегда потенциал духа, пусть даже потенциал. Шеллинг говорит тоже 
о потенциале, неорганическая природа просто заснувшая, но может 
пробудиться. Если две субстанции, проблема духовности непонятно как 
разрешаются. А тут они тождественны в своем основании. Дух 
непосредственно всегда присутствует в природе, пусть даже усшувший 
потенциальный. Спиноза теперь -- монизм. Шеллинг говорит, Спиноза слил 
в единое бога природу и субстанцию. Шеллинг говорит, что различает 
природное начало и то, где активно действует уже хз что. 
Трансцендентальные структуры Канта. С точки зрения Шеллинга 
трансцендентальные структуры присутствуют в самой природе. И тот же 
самый Фихте, у него субъект сам конструирует объект, а Шеллинг говорит, 
что объект существует вне зависимости от субъекта, напрямую не зависит 
от субъекта, субъект проявляет активную деятельность, но не 
конструирует.

Так, теперь наука уже развивается, прежде всего физика, прежде всего 
электричество. Тут и Гальвани, гальванизм, животное электричество. Тут 
и Вольта с батареей, химическое электричество. Электричество природное 
на уровне гроз было давно известно. Статическое электричество с законом 
Кулона. Проблема была в том, что считали, что это разные виды 
электричества. В биологии тоже разное, связанное с всяким. В химии 
кислородные технологии горения. В самой природе существует творческий 
креативный потенциал, который в познании феноменов выражается в принципе 
полярности. В природе изначально существует некая полярность, можно 
сказать бинарность противоположных начал, которые ответственны за то, 
что происходит развитие в самой природе. Силы в самой природе и на 
поверхности они проявляются в различных феноменах. Как притяжение 
и отталкивание, два противоположных элемента, на уровне неорганической 
материи, но без них ничего дальше не происходит, вот они позволяют. 
Электричество вот тоже, плюсы и минусы. Проявляет себя на плоскости, как 
говорит Шеллинг, в плоскостном варианте, а магнит тоже северный и южный 
полюс, противоположности. В линейном масштабе проявляют себя магнитные 
явления. Вот и сравнивая это все он даже до Эрстеда попытался обосновать 
что это одно природы. И наконец механицизм, электричество, магнетизм, 
химизм -- объемные проявления активности. Все химические реакции идут 
в пространстве и времени. Хим реакции тоже носят противоположный 
характер, диссоциации, ассоциации, катализ итд итп. Дальше Шеллинг 
говорит следует принципиальный скачок в живую материю, где тоже видим 
различные проявления уровней активности. В биологических системах уже. 
Рост, умирание, все реакции в своей совокупности носят противоположный 
характер, деление отмирание... Шеллинг все это выписывает 
и прослеживает, что чем выше уровень материи, тем больше таких 
разноплановых реакций, что сводится к принципу полярности. Это 
получается ключевой принцип для понимания развития всего и всех. А сам 
принцип полярности включает в себя т.н. точку безразличия, когда 
непонятно куда будет склоняться чаша весов. В реакциях, магнитах, итд, 
такая точка есть, а потом выводится из равновесия. Эти процессы идут 
в плане развития, но могут идти и по деградации, Шеллинг говорит 
называет их эволюцией и инволюцией. Восходящее Шеллинг называет 
эволюцией, от самых простейших типа отталкивания притяжения до самых 
сложных типа существования человека, но говорит, что может быть 
и нисходящая ветвь, инволюция, свертывание, снова в эту точку 
безразличия. Диалектическая ступень, где есть оба направления. Этой 
точке безразличия имплицитно или диспозиционно заложена программа 
развития. На любом уровне можем выделить точку безразличия, а потом 
можем смотреть как будет развиваться. У человека это гены.

\subsubsection{Георг Вильгельм Гегель}

Вершина идеализма немецкого вообще. Объективный идеализм. У Гегеля есть 
тоже идея неких точечных живых организмов. Вся система наиболее ярко 
прослеживается в произведении Энциклопедия философских наук, где есть 
логика, так называемая Малая логика, Философия природы, Философия духа. 
Вот три тома. В том курсе был разговор о логике и философии духа. Смысл 
объективного идеализма в том, что некое движущее идеальное начало 
выносится вовне, у Гегеля это абсолютная идея, мировой разум и другие 
названия, существует вовне и развивается. В трех ипостясях развивается. 
Дух в себе. Надежда сознание -- причина следствие -- сущность сознание. 
Дальше дух для себя -- отчуждение духа в природу. Природа не обладает 
духовным началом, лишь разворачивается в пространстве, но не развивается 
во времени. Дух порезвился в природе, а природа это окаменевший дух, 
отчуждение духа. И третья стадия -- возвращение духа к самому себе 
в истории. Субъективный дух, объективный дух, абсолютный дух, как мораль 
этика право и для абсолютного мифология религия искусство философия. 
Философия высшая это осознания духа, дух осознает себя через систему 
Гегеля. Сам метод Гегеля -- метод триады -- тезис, антитезис, синтез. 
В этой логике есть и три раздела это бытие, сущность и понятие. Где 
Гегель выстраивает свою систему категорий, объясняющих и бытие, 
и сущность бытия, и сознание. Сознание как важнейший элемент сознания 
и как принцип познавательной деятельности вообще. Развивается и мысль 
о формировании формальной логики и диалектической логики тоже. 
Формальная логика -- логика рассудка, которой человек должен овладеть, 
чтобы его вообще понимали, чтобы он выражался и строил свои мысли 
рационально, а логика разума -- высший этаж логики -- диалектическая 
логика. Диалектической не обязательно должны овладевать люди, разум без 
рассудка ничто, а рассудок без разума нечто. Суть диалектической логики. 
Формальная логика известна, а диалкетическая. Это логика движения 
осмысления противоречие. Гегель прямо говорит, что противоречие движет 
миром, и смешно думать что мы не можем осмыслить противоречие. Понимание 
что такое противоречие приобретает решающее значение. Бытие сущность 
и понятие Гегель и раскрывает законы диалектики.

\textit{Первый} -- закон единства и взаимодействия противоположностей. 
Закон с точки зрения Гегеля показывает сущность или причину развития, 
причина любого развития -- единство и взаимодействие противоположностей. 
На разных уровнях разные формы, может быть и пассивные как 
в неорганической природе, кк говорил Шеллинг, но наиболее ярко 
в органике, а еще круче в обществе как будет строить Маркс, единство 
и борьба за источники, классовая, экологическая. Конкуренция, в общем. 
\textit{Второй закон} -- закон перехода количественных изменений 
в качественные. Можно проследить на уровне категорий. Первая категория 
становления, качество есть то, без чего объект не может существовать, но 
чтобы до конца понять качество, нужно ввести количество. Количество это 
инобытие качества. То, что поддается измерению. Любая физическая 
характеристика -- количественная. Где начинается количественная 
характеристика там наука, но качество тоже необходимо. При определенных 
нарушениях параметров количественных изменений происходит качественное 
изменение. Момент характеризуется категорией меры. Нарушение меры, при 
выходе за пределы определенных количественных измерений, переход 
в качественное. Это как агрегатные состояния получается.... Вода лед 
пар... А всего лишь только нагреваем. Это вот самый простой пример для 
иллюстрации. По мысли Гегеля такие примеры можно находить везде такое, 
даже в обществе, как нашел Маркс потом. Это механизмы развития. О том 
как происходит развитие Гегель всегда отвечает -- путем перехода 
количественных изменений в качественные. \textit{И третий закон} -- 
направление развития. От низшего к высшему, но не прямолинейно и тем 
более не в форме замкнутого какого-то цикла, закон отрицания отрицания. 
Это отрицание диалектическое. Пример Гегеля есть голое -- было зерно, 
потом нет, смололи, из муки обратно не сделаешь. А диалектическое -- 
посеяли зерно -- росток отрицает зерно -- колос отрицает росток -- зерно 
отрицает колос ... Не такое же зерно, но подобное. Во всем по спирали 
отрицание отрицания, все новое воспроизводит старое, но на новом витке. 
Существенным законом отличаются.

Для Гегеля эти законы -- законы самого духа. Дух так развивается. На 
уровне логики, на уровне категорий Гегель показывает. В природе Дух 
просто порождает что мы называем природой, а никакого развития в самой 
природе не существует. Эту часть философии Гегеля как раз называют 
какой-то такой. Вся философия природы. У Фихте субъект сам конструирует 
пространство и время, у Канта они заданы, априорные структуры в сознании 
человека. У Шеллинга пространство и время объективно существуют. Гегель 
критикует представление Ньютона об абсолютах. Гегель говорит что звук 
это механическая душевность. Белый свет сложная субстанция как уже было 
известно тогда после Ньютона и призмы, для Гегеля неприемлемо, он 
говорит, что свет это простейшая мысль в природе, а призма искажает. 
Гете тоже так считал. Магнетизм это проявление умозаключений в природе. 
Элементы какие-то, что вода это H2O он говорил нельзя так ни за что, 
чтобы жидкость состояла из двух газов. Некоторые говорят, что он 
предугадал таблицу Менделеева, что хим элементы могут быть упорядочены 
по свойствам. Неприятие всех теорий жидкости логистона теплорода. 
Говорил, что животные все созданы были как есть сейчас.

Дальше будем говорить, что сами законы диалектики, присущие самому духу, 
абсолютному разуму, Марксизм перенесет диалектику как говорит Энгельс 
наука о наиболее общих законах природы, общества и мышления.

\subsubsection{Людвиг Фейербах}

Последний философ немецкой классической философии. Главный пафос его -- 
борьба с Гегеля идеализмом, логика сама по себе как у него является 
порождением природы, человека надо изучать как мыслительной так 
и телесной деятельности, на основе критики христианства, сущности 
христианства, приходит к выводу о необходимости создания материализма 
который называют антропологическим материализмом. Фейербах стоит 
в основе антропологии. Мы не можем действовать методом Сократа познай 
самого себя, нужно представить человека как объект, использовать 
комплекс наук и тогда постепенно изучать, познавать, чем отличается, чем 
схож с остальным мирозданием.

\hfill\textbf{Nov 15}

\subsubsection{Диалектический материализм}

Диалектический материализм имеет главные разделы: онтологический 
и теория познаний.

Сам диалектический материализм -- наука о наиб общ законах развития 
природы, общества и мышления. Дает Энгельс и использует и развивает Лей. 
Ключевое понятие -- диалектика и законы. Диал понимается 
материалистически, усматривается в самой природе. Диал как наука 
отражает реальные процессы, происходящие в природе самой. Синтезирует 
в себе идеи Фр. материализма и диалектики Гегеля. У Фейербаха тоже 
материализм, но опять же как классики говорили, с грязной водой выбросил 
еще и младенца, потому что отрицал диалектику Гегеля как принцип 
познания. У Гегеля диалектика стоит на голове, надо ее поставить на 
ноги. Второе ключевое понятие -- законы. Диал матер тоже различает эти 
законы как законы наиб общ философского типа, общего общенаучные законы 
(как закон сохр энергии) и частные законы (как опред закономерности при 
опред условиях и опред рамки имеют, как Бойля-Мариотта). Законы 
диалектики -- высший этап развития познания, обобщ все типы законов, сущ 
и известные до этого момента, и претендует на то, что законы диалектики 
носят абсолютный или универсальный характер. В задачу диал матер входит 
и то, что поставил Демокрит и Платон. Атомы и пустота 
материалистическое, а у Платона мир идей и наше отражение его... Диал 
матер встает на сторону материалистическую, но само понятие материи уже 
требует в 19 веке объяснения, но потом скажем. От Демокрита до середины 
19 века под материей понималось вещество. Во Фр. предлагали 
отождествлять понятие материи с природными хз чем упустил. 
Дополнительным являлось открытие единой природы живых существ -- клетка. 
Потом доказали что энергия не исчезает не появляется а лишь переходит, 
и Энгельс говорит потом ``Мир един в своей материальности. Нет ничего 
кроме материи в разных формах...'' Такой материалистический подход был 
связан с субстанцией материи. Конечная точка -- атомы и молекулы. А во 
второй половине 19 века физика начинает открывать новое: Рентген, 
радиоактивность от Кюри, в 1897 году Томсон открывает электрон. Нет 
вещества есть только формулы... На фоне вот этих вещей Ленин 
в Материализм и империоктирицизм вводит понятие материи как философской 
категории для обозначения объективной реальности, данной нам 
в ощущениях, которые копируют эту реальность фотографируют с какой-то 
степенью точности, но материя существует вне зависимо от самого 
субъекта. Материя теперь категория... Там же он критиковал и Беркли, 
и Маха, но принципиально новым является ассимиляция открытий в физике 
и того, что говорил Энгельс, что с каждым открытием в естествознании 
диалектический материализм может изменяться... Вот это основной такой 
принцип понимание того, что есть материя, как понятие развивается, 
начиная в античности и кончая концом 19 века. И теперь понятие закона. 
Диалектика это о законах развития.

Другие ключевые понятия онтологии -- структурные организации материи 
и ключевые понятия как пространство время движение и развитие. Любое 
изменение это движение. Это теперь как и во фр атрибут материи, нет 
движения без материи и нет материи без движения. Но разные формы 
движения. Механическая, физическая, химическая, биологическая, 
социальная. Механическая форма это движение макротел в пространстве 
и времени, физическая это молекулярная и атомарная... химическая это 
сложномолекулярная форма движения (тут Энгельс уже опирается на таблицу 
Менделеева и хвалит, что там законы диалектики видны...), биологическая 
это уже ну вот понятно и социально это в обществе человеческом. Принципы 
такие вот. 1 -- принцип иерархичности, механическая, над ней 
настраиваются и в конце соц. 2 -- более высокие формы движения 
ассимилируют в себе все предшествующие 

\hfill \textbf{Nov 22}

\subsection{Шопенгауэр}

Воля к жизни. Жизнь начинается ровно тогда, когда открывается первый 
глаз. Воля эта имеет агрессивный характер, заставляет бороться за место 
под солнцем. Шопенгауэр приходит к пониманию, что общество ступило на 
путь потребительства, и ресурсов всегда будет не хватать. Все к худшему 
в этом худшем из миров. Наука только усугубляет это.

\subsection{Ницше}

Воля к власти. Наука -- один из каналов достижения. Но эта воля со 
знаком плюс, в конце концов воля к власти приводит к появлению 
принципиально нового вида человека, который будет относиться к нам, как 
мы к обезьянам. Сверхчеловек -- белокурая бестия.

\subsection{Фрейд}

С одной стороны ученый, а с другой на основе его ученых достижений 
медицинских он создает философскую теорию, пытающуюся объяснить если не 
все, то многое в истории, культуре, человеке вообще. Сознание -- 
островок в океане бессознания. Лейбниц открывает бессознательного, но 
там это периферия сознания, а само сознание рационально. Бессознательное 
какая-то бахрома, на пороге чувствительности нашей сенсорики.

% }}}

\section{Иррационализм}
% {{{

\subsection{Кьеркегор}
\subsection{Шопергауэр}

\subsection{Берксон}

У животных инстинкт, у человека -- разум. И то, и то красиво, но вот 
разное...

Вступил потом в полемику с Эйнштейном, что время вот не как наука 
понимает... Потому что стрела времени особо не была тогда нигде... 
Только в термодинамике, пожалуй. Время как кинематографический метод. 
При определенной скорости прокручивания создается иллюзия движения. 
Берксон говорит, что наука как раз использует такой метод и тем самым 
омертвляет реальное творческое время.

Развивая дальше, можно сказать, что и в современной физике положение 
остается точно таким же как во времена их дискуссии Берксона 
и Эйнштейна. За одним исключением. Нобелевский лауреат Пригожин 
присоединяется к Берксону и говорит, что наука не может понять 
необратимость времени. Время у него это время должно быть абсолютной 
величиной, существовала всегда и должна существовать тоже так же всегда.

Берксон в этом смысле иррационалист, предлагает и сверхсознание, 
и творческий порыв, и понимание времени как креативного начала, как 
говорит, 1000 фотографий Парижа не есть настоящий Париж.
% }}}

\section{Позитивизм}
% {{{
Философия должна заниматься именно проблемами самой науки, науки 
развитой, имеющей дисциплинарную структуру, доказавшую свою 
эффективность. Можно сказать, что кредо этого направления состоит в том, 
что наука -- сама себе философия. Никаких дополнительных теорий 
философского плана строить не надо, поскольку наука уже вышла (не без 
помощи философии) на самостоятельное поле, уже стала взрослой, и ученые 
могут сами осмысливать как результаты деятельности своей, так 
и перспективы. Это вплоть до уже 30-х годов, но зарождаются еще 
в середине 19 века. Систематизация, обобщение, прояснение тех идей, что 
высказывают ученые. Но не надо изобретать ничего нового. Философия 
должна быть слугой науки. В средневековье философия была слугой 
теологии. Это был длительный период, но теперь как наука стала во главе, 
как считали в 19 веке, философия должна обслуживать ее.

Направления. Классическая стадия (Агюст Конт, Герберт Спенсер, Джонс 
Кьюарт Миль) середина 19 до конца 19. В конце 19 века происходит 
открытие радиоактивности, x-rays, электрон. Позитивисты уже осмысливают 
процесс с своих позиций, это теперь эмпирио-критицизм, критика 
эмпиризма, стремление сделать науку чистой, свободной от метафизических 
сущностей. Это Эрнст Мах и психолог Ричард Авинариус. Дальше следуют 
такие уже менее широкие подходы. Это конвенционализм (Пуанкарэ), 
лингвистическая философия и логический атомизм (Бертран Рассел, Людвиг 
Витгенштейн), прагматизм (Америка, начало 20 века, Чальз Пирс, Джон 
Дьюи, Уилмет Джеймс). Это промежуточные между 2 и 3 стадиями. Третья 
стадия -- логический эмпиризм (Гедель).

\subsection{Агюст Понт}

Основатель этого направления, провозгласил главные тезисы этого 
направления философии. Знать, чтобы предвидеть, предвидеть, чтобы мочь. 
Знать это знать именно с точки зрения науки. Всю науку можно представить 
как совокупность утверждений, каждое из которых можно подвергнуть 
операции истинности. Для науки вполне достаточно двух функций: 
описательные функции (знать это описать... с помощью аппарата, 
уравнений, наблюдений итд.. и из этого прогнозировать, связь между 
явлениями). Но не надо науке лезть в объяснение. При попытках объяснить 
получается матрешка, и метафизика... А философия должна помогать науке, 
прояснять логику описания, обобщать выводы и знания из разных наук. 
Философ должен выполнять функцию медиатора, посредника между 
дисциплинами, сводить выводы естествознания в единую картину мира, но 
ничего изобретать не надо...

Дальше важнейшая его мысль -- объяснение логики самого исторического 
процесса. Как сама история развития науки приводит к такому положению 
вещей. Закон трех стадия (принцип трех стадия) развития, созревания 
интеллектуального, созревания того, что называют сейчас наукой. Закон 
фиксирует, что человечество и каждый человек проходит в своем духовном 
развитии три стадии. Первая -- теологическая, когда человеческое 
сообщество пытается все объяснить с помощью наличия потусторонних 
сверхмощных неконтролируемых потусторонних сил. Внутри этого тоже есть 
свои стадии (фетишизм, анимизм, монистические религии...), но 
.........., но пришел 19 век и прочее, и теперь наука может обходиться 
без метафизических сущностей. Критерий является эмпирическая 
проверяемость. Объяснение как описание, которое удостоверяется 
эмпирически, и можно делать предсказания. Описать и уметь предсказать.
% }}}

\end{document}
