\documentclass[a4paper, 12pt]{article}

% Configuration {{{
\usepackage[utf8]{inputenc}
\usepackage[T2A]{fontenc} % T1 for English
\usepackage[english, russian]{babel}

\usepackage{enumitem}
\setlist{nolistsep}
\usepackage{mathtools}
\usepackage{xcolor}
\definecolor{dimblue}{HTML}{1010aa}
\usepackage[
	colorlinks=true, 
	allcolors=dimblue
]{hyperref}
\usepackage[
	vmargin=1in,
	hmargin=1in
]{geometry}
\linespread{1.3}
\usepackage{indentfirst}
\usepackage{graphicx}
\usepackage[multidot]{grffile}
\usepackage[labelsep=period]{caption}
\usepackage{subcaption}

%\usepackage{times} % for English
% }}}

\begin{document}

% Title Page & Table of Contents {{{
\null
\vfill

\begin{center}
	\begin{Large}
		\textbf{Общие вопросы преподавания физико-математических дисциплин в ВУЗе}
	\end{Large}

	\vspace{\baselineskip}

	Татьяна Андреевна Бушина

	Адрес для домашних работ:
	\href{mailto:metodika2020@rambler.ru}{metodika2020@rambler.ru}
\end{center}

\vfill

Литература
\begin{enumerate}
	\item
\end{enumerate}

Заниматься будем на Зуме, но с перерывом вот таким характерным для 
бесплатного зума...
Теперь про организацию работы. Старостам будет отправлять ссылку на зум, 
она будет рассказывать всякие вещи, иногда будут семинары аж, когда мы 
пишем. Домашку в середине или конце лекции, выполнять лучше в течение 
недели, отправлять на особую почту для домашки, она может задать в ответ 
вопрос, сказать, что неправильно понято задание, чтобы вы исправили. По 
этой работе в семестре выставляется зачет. В теме письма писать номер 
задания.

\clearpage

\tableofcontents

\clearpage
% }}}

\section{Введение}
% {{{

\hfill\textbf{Sep 6}

Предмет более гуманитарный, но в целом вот существуют целые 
педагогические университеты для подготовки преподов в школе, а для 
преподов универа как-то нет. Обычно просто человек заканчивал свой вуз, 
решал остаться там и в конце неизбежно становился преподавателем. Однако 
его особо ведь не учили объяснять и все такое. Для решения этого 
создавались институты повышения квалификации, начали заниматься вот 
в рамках повышения квалификации, потом создали целый факультет 
педагогического образования ФПО. Туда можно пойти учиться, если уже 
закончил универ, и хочешь пойти туда как преподаватель вуза или школы. 
Тем не менее это как-никак дополнительное образование, совершенно 
другое, не каждый готов. В результате молодой преподаватель первые пару 
лет просто экспериментировал над студентами.

В общем наш курс молодой, читает его Бушина третий раз, все еще 
развивается, старается сделать нескучным. Обычно вообще такой тематики 
курсы близки к философии, гуманитарные, разговорные. Создатели курса 
постарались сделать курс адаптированным для людей 
с физико-математическим складом ума.

Теперь про организацию работы. Старостам будет отправлять ссылку на зум, 
она будет рассказывать всякие вещи, иногда будут семинары аж, когда мы 
пишем. Домашку в середине или конце лекции, выполнять лучше в течение 
недели, отправлять на особую почту для домашки, она может задать в ответ 
вопрос, сказать, что неправильно понято задание, чтобы вы исправили. По 
этой работе в семестре выставляется зачет.

К. Д. Ушинский писал ``Знай хорошо свой предмет и излагай его ясно''

Общие задачи в данном курсе: получим знания, умения и навыки для 
семинаров, экзаменов, ....

Частные задачи: познакомимся со структурой общей физики в универе; 
Модели обучения в вузе; формы учебных занятий; формы опроса, контроля 
знаний; научимся грамотно составлять определения; увидим разновидность 
задач; научимся использовать вспомогательные средства; приемы 
запоминания, удержания внимания; мотивация студентов; система 
образования эволюция ее.

Будем вспоминать свой опыт и писать о нем, вспоминать ошибки и находки. 
Подумаем как улучшить что-либо где угодно.

% }}}

\section{Принципы Дидактики и основные вопросы преподавания}
% {{{

Дидактика это что такое. Это основы преподавания. Вольфганг Ратке 
(1571--1635) дидактика -- искусство обучения. Вопросами педагогики как 
науки, как системы, люди занялись не очень скоро. Ратке как раз ввел это 
слово, он был первым по сути. Тогда преподавание начало вычленяться из 
философии вообще. Принципы дидактики -- природосообразность 
(непротиворечивость природе), последовательность, повторение, без 
принуждения, заучивать нужно только понятное, идти стоит от частного 
к общему. Дальше принципы были сформулированы более формально. Ратке 
считал, что разум ребенка это чистая доска, на которой можно написать 
все что угодно. Процесс познания состоит из двух частей: восприятие 
и переработка. Дедукция и индукция, дедукция это выведение, индукция от 
частного к общему. Необходимость педагогических знаний, уделял много 
времени подготовке учителей, учителям тоже нужно неплохо устроиться.

Дальше Ян Амос Коменский (1592--1670) Чешский педагог, написал Великую 
дидактику, книгу, где сформулировал основные принципы дидактики. Дальше 
иоганн Фридрих Гербарт (1762--1841) ввел понятие воспитывающее обучение, 
что воспитание и обучение не могут быть разделены. Министр просвещения 
наш говорил как раз, что школа должна вернуть себе воспитательную 
функцию. Адольф Дистверг (1790--1866) выступал против сословных 
и национальных ограничений, был в этом плане весьма впереди остальных. 
Про Дистверга можно услышать на философии, он в общем-то философ, но дал 
вклад в педагогику. Высшая цель воспитания -- самостоятельность 
в служении истине, красоте и добру. Задачи школы: воспитывать гуманных 
людей, воспитывать сознательных граждан, ... Иоганн Генрих Песталоцци 
(1746--1827) уже больше педагог, чем философ. Он занимался вопросами 
раннего воспитания детей, что сейчас можно было бы назвать детским 
садом. Школы итд, говорил, не могут полностью взять на себя воспитание, 
без родителей. Он сам посвящал себя занятиям с детьми, ввел геометрию 
в начальные классы. Фридрих Фрёбель (1782--1827) по сути родоначальник 
детских садов, придумал их и педагогику детского сада. Детский сад 
должен осуществлять всесторонее развитие детей, ядром считал игру. Для 
ребенка, говорил и доказывал, это инстинкт и все такое. Ну и наш человек 
Ушинский Константин Дмитриевич (1824--1870) считал нравственность 
важной, сохранение культуры и самобытности. В его время было много 
французского языка, иностранных преподавателей, фактически русская 
культура вытеснялась. Он много сил потратил на то, чтобы сохранить, 
в частности, в педагогике.

Дальше интересно знать, что согласно ЮНЕСКО есть 4 педагога, 
определивших способ педагогического мышления в 20 веке, среди которых 
есть Макаренко, хотя в 20-е годы. Джон Дьюи (1859--1952) демократичный 
подход, что ребенок это отдельная личность, цель обучения считал умение 
решать жизненные задачи, овладение творческими навыками, про личность 
было революционно. Мария Монтессори (1870--1952) методика ее заключалась 
позволяла реабилитироваться особенным детям, развивающее обучение, 
ребенку позволяют познавать мир, путем тактильных ощущений, сыпучести, 
температуры, колется ли, взрослый лишь помогает, а не направляет. Георг 
Кершенштейнер (1854--1932) его идеи близки к нашему Макаренко, трудовое 
воспитание, что в школе очень важно, каждый человек должен иметь 
профессиональную эту. Говорил учить воле, разумному образу жизни и проч. 
Народную школу и армию рассматривал как важнейшие гос воспитательные 
учереждения. Дальше Антон Семенович Макаренко (1888--1939) работал сразу 
после революции, перевоспитывал беспризорников, чтобы они не стали 
уголовником, трудовое воспитание у него было. Педагогическая поэма, 
флаги на башнях. В целом, эти имена это очень неполный перечень, но 
ясно, что педагогическая мысль вычленилась из философской, стала 
отдельной областью знаний.

В настоящее время дидактика -- наука, изучающая взаимосвязь процесса 
образования и ??. Под образованием понимают социокультурный феномен, 
способствующий накоплению знаний итд. Принципы дедактики: научность, 
доступность, целенаправленность, систематичность и последовательность, 
нарядность, связь с повседневной жизнью, сознательность и активность, 
прочность знаний, воспитание и развитие. Научность в том, что содержание 
обучения должно соотв реальным фактам и отражать современные научные 
данные, требования в том, чтобы формировались системы теоретических 
знаний, достоверность изучаемых фактов, подверженность действий 
и выводов педагога. Доступность в том, что обучение должно 
соответствовать индивидуальным чертам, не сложно, не просто, от неизв 
к изв, требования, что нужно учитывать образовательный уровень, 
возможности, подготовку, характер, опыт, возрастные особенности, 
потребности и интересы учащихся. Целенаправленность в том, что нужно 
осознанно создавать организационные, методические и содерж основы 
педагогич процесса, требования, что нужен план более-менее четкий, 
всегда обучение начинается с плана, нельзя подавать знания хаотично. 
Принцип систематичности и последовательности в том, что подача должна 
планироваться, разбита на разделы, нужен не только план, но и внутри 
плана нужно структурировать, разбивать на разделы и модули темы, чтобы 
не только теоретические занятия, но и практический выход. Наглядность, 
нужно опираться на зрительные органы, 80\% считается, что через 
зрительного, говорят, что важно то, что человек увидел, поэтому лекции 
сопровождаются демонстрациями, для лучшей иллюстративности, требования 
в том, что нужно с определенными целями, не обязательно лекционные 
демонстрации, можно еще схемы, картинки, видео, прочее. Связь обучения 
с повседневной жизнью в том, что нужно всегда подвергать сомнению 
и проверять с помощью практики, требования, что учебно-воспитательный 
процесс должен иметь явно выраженную профессиональную направленность, 
в ходе нужно отвечать на вопросы, когда где и как можно применять 
полученные знания. Сознательность и активность в том, что есть две 
стороны всегда, обе должны быть активными, учащийся тоже, учитель тоже, 
двусторонность, требования, чтобы педагог использовал активные формы 
обучения, побуждать к самостоятельной работе. Прочность знаний в том, 
что нужно закрепить содержание в сознании, систематически повторять, 
регулярно контролировать, требования, что знания должны закрепляться, 
умения и навыки применяться на практике, должен обеспечиваться 
систематический контроль, сочетаемый с индивидуальным подходом к каждому 
учащемуся. Знания нужно не только хорошо дать, но и хорошо взять. 
Педагог в общем-то должен проконтролировать. Воспитание и развитие 
в том, что воспитание и это неразделимые, нужно воспитать достойного 
члена общества, и даже в ВУЗе продолжается процесс воспитания, 
преподаватель личным примером воспитывает. В целом в процессе обучения 
нужно формировать научное мировоззрение, творческое мышление, 
инициативность и самостоятельность, способность делать выводы, 
сопоставлять, сравнивать, выделять основное, обобщать, анализировать. 
Нужно еще воспитывать дисциплинированность, культурное поведение, 
интеллигентность, гуманность, гражданскую ответственность и патриотизм. 

\textbf{Задание номер 1.} Вы прослушали и сдали множество учебных 
курсов. Нарушались ли преподавателями принципы дидактики при 
преподавании какого-либо курса? Какие именно принципы? Как именно 
нарушались? Ответ присылать на metodika2020@rambler.ru

% }}}

\section{Основные вопросы методики обучения физики}
% {{{

\hfill \textbf{Sep 13}

Адольф Дистверг говорил, что нужно, чтобы студент сам чувствовал сам 
потребность думать, учиться, ровно как не может никто за него есть 
и пить.

Методика обучения предмету -- это педагогическая наука, являющаяся 
приложением принципов дидактики к преподаванию данного учебного 
предмета. Задача методики обучения -- поиск ответов на:
- зачем учить?
- кого учить?
- чему учить?
- когда учить?
- как учить?

Зачем учить? -- о целях обучения, от ответа на этот вопрос зависит 
вообще все остальное. Образовательное учреждение выполняет социальный 
заказ, то есть цели определяются запросами извне, государства или 
общества или чего-либо еще. В результате появляется и пропадает всякое.

Кого учить? -- вопрос о выборе объекта обучения. Не каждый учащийся 
в состоянии освоить произвольную программу, делятся не только по уровню 
подготовки, но и там по склонностям и талантам. Создаются отдельные 
классы с сильным углублением куда-либо. Всех подряд брать нельзя, не от 
капризов, а потому что просто не всякий может учиться в художественном 
училище, на физфаке итд.

Чему учить? -- вопрос о содержании обучения. Если одной из целей 
является формирование научного мировоззрения, нужны включаться вопросы 
мировоззренческого характера. Если ставится цель сформировать 
представление о современном научно-тех прогрессе и направлениях, то 
соответствующий материал должен быть. На содержание также влияют 
псих-педагогические особенности и возраст обучающихся, развитие 
информационной среды, создающего возможности для неформального 
образования.

Когда учить? -- вопрос о выборе времени обучения. Насколько 
целесообразно в данный момент начинать или завершать обучение. Например 
теор мех разумно начинать на 2 курсе, когда студент уже прослушал мат 
анализ и диффуры. Однако у нас на факультете курс физики не как раньше, 
механика, ТД, э-м, оптика, атомка, ядерка, а ядерка переехала на 2 курс 
внезапно.

Как учить? -- методы, средства и формы обучения. Нужно учитывать цели, 
содержание, контингент. Например, если цель формировать 
экспериментальные дела, то нужен практикум, демонстрации. В мед вузах 
безусловно имеется тоже практика. В инженерных вузах людей отправляют на 
предприятия проходить практику тоже.

Цели, содержание, методы, средства и формы обучения -- методическая 
система. Методы, средства и формы --

Метод -- система целенаправленных действий преподавателя, ведущих 
к достижению цели. Средства -- источники получения и формирования 
знаний.

% }}}

\section{Актуальные вопросы технологии преподавания физики в ВУЗе}
% {{{

\begin{enumerate}
	\item Понятийный аппарат в физике. Определения и формулировки.
	\item Законы и теоремы в курсе общей физики.
	\item Типовые вопросы в курсе физики (и не только).
	\item Модели в физике.
	\item Вспомогательные средства: опорные фразы, мнемонические правила, 
		аналогии, подсказки.
\end{enumerate}

Понятийный аппарат в физике, определения и формулировки. Виды 
формулировок: словесная формулировка (масса, тепловая машина, ...), 
формула (коэф трения), сочетание словесной формулировки и формулы (закон 
Кулона), небольшой развернутый рассказ (броуновское движение), сведение 
к частному случаю (не самый хороший способ, часто используется в школе). 
При составлении формулировок лучше нучинать формулировку с определяемого 
понятия, про что спрашивают, про то и отвечать. Дальше, определения 
должны быть логически однозначными, под это определение не должно 
подойти что-либо другое. Дальше, для физических величин лучше включать 
указание на способ экспериментального определения. Дальше, формулировать 
лучше в именительном или творительном падеже (процесс это / процессом 
называют). Дальше, в тексте формулировки полезно использовать ``это''. 
Дальше, не следует вместо полной формулировки использовать сокращенную 
версию.

Законы и теоремы в курсе физики. Законы и теоремы, в математике звучит 
сильнее, аксиомы и теоремы, а законы по сути то же самое. Законы это 
постулаты, выводятся лишь обобщением экспериментальных фактов. Теоремы 
как раз выводятся из всего принятого. Есть еще названия законы, 
причисленные теоремам ввиду особой важности, как сохранения энергии, 
сохранения импульса. Теорема как доказуемое утверждение, как закон 
(закон Паскаля), теорема в виде формулы (уравнение Бернулли), теорема 
в виде аналогии (формула Томсона для частоты колеб контура из периода 
маятника), теорема в виде задачи.

Типовые вопросы в курсе физики (и не только). Объяснять легче, чем 
спрашивать. Дайте определение, сформулируйте, напишите формулу, 
нарисуйте график, приведите пример, изобразите схему опыта, как 
соотносятся, сколько?, почему?, найдите ошибку в утверждении. По мере 
своего рассказа нужно задавать вопросы а как а что а откуда и прочее, 
чтобы держать во внимании и проверять насколько усвоено. Вопросы по типу 
нарисуйте график, расскажите, приведите пример, выше они написаны...

Задание 2. Сформулируйте по одному вопросу каждого типа из какого-либо 
спецкурса по специальности или какого-либо математического курса. 
Напишите, какие еще типы вопросов можно использовать при работе со 
студентами на семинаре, в практикуме, на экзамене.

\hfill \textbf{Sep 20}

Принципы дидактики: научность, доступность, целенаправленность, 
системность и последовательность, наглядность, связь обучения 
с повседневной жизнью, сознательность и активность, прочность знаний, 
воспитание и развитие. Всего 9 получается. Ян Амос Коменский говорил, 
Так как только упражнение делает людей искусными, а мы хотим сделать 
людей сведущими во всем и годными ко всему, поэтому заставляем студентов 
такие вот вещи проходить на практике... Нужно в общем призывать 
тренироваться, чтобы закреплять знания, упражняться на практике, тогда 
мы достигнем прочности знаний. И наша текущая тема, актуальные вопросы 
в вузе, как раз нацелены на то, чтобы правильно направлять студента. 
Сегодня в плане только лишь модели в физике.

\subsection{Модели в физике}

Разновидности моделей в физике: абстрактная, физическая, математическая, 
компьютерная, демонстрационная.

Абстрактная модель -- это абстрактный объект или явление, которое мы 
рассматриваем в данной задаче взамен реального объекта или явления. 
Примеры вот такого: материальная точка, идеальный газ, газ 
ван-дер-Ваальса, абсолютно .... У школьников с этим не очень хорошо, для 
студентов уже хорошо. При работе с первыми нужно конечно доносить, чем 
абстрактная модель отличается от реальности.

Физическая модель -- реальный объект или явление, которое 
рассматривается взамен другого объекта или явления, причем той же 
природы. Аэродинамическая труба, модель крыла самолета.

Математическая модель -- модель реального объекта или явления, 
основанная на системе уравнений, описывающих свойства изучаемого объекта 
или явления. Например колебания, рассеяние и все вот такие вот приколы.

Компьютерная модель -- модель изучаемого объекта или явления, основанная 
главным образом на возможностях используемой техники. Например решение 
диф уравнений.

Демонстрационная модель -- любая модель, удобная для показа. Введение 
в квантовую теорию.

Задача о машине отвода как тест. Те, кто имеет какое-то отношение 
к школьникам, в этом году нужно будет обосновать правомерность 
использования тех или иных моделей. Как раз будет задача например про 
блоки и грузы. Эту задачу хорошо обсуждать с первокурсниками, задача про 
блоки и грузы довольно избитая, но понимают ее не прям до конца, при 
малом усложнении уже проблема возникают. Нить невесомая, нить 
нерастяжима, блок невесом, трение в оси отсутствует, сопротивление 
воздуха отсутствует. И уравнения в задаче: второй закон Ньютона для двух 
блоков, использует только про воздух; уравнение кинематической связи, 
использует только нерастяжимость нити; равенство сил натяжения, 
использует все, кроме растяжения. Уравнения кинематической связи удобно 
записывать через длины нитей и их дифференцирование.

Одна и та же задача может быть очень разной при выборе разных 
абстрактных моделей. Возможно сильно усложнить что угодно. Можно 
привязать пружины к нижним концам грузов. Можно включить вязкое 
сопротивление, тогда изменится циклическая частота. Можно теперь ввести 
массу блоку.

\textbf{Задание 3} Придумать одну задачу с решением из любого курса 
общей физики или спецкурса по вашему выбору, которая допускает различные 
решения в зависимости от выбранных абстрактных моделей.

\hfill \textbf{Oct 4}

\subsection{Вспомогательные средства}

Вспомогательные средства: опорные фразы, мнемонические правила, 
аналогии, подсказки. Преподаватель -- это пророк.

Опорные фразы: На какую тему задача? -- помогает начать решать задачу. 
Если тело движется по окружности, то у него есть... -- помогает записать 
II закон Ньютона в случае рассмотрения движения по окружности). 
Последствия противодействуют причине -- вспомнить правило Ленца. Знания 
надо не только хорошо дать, но и хорошо взять -- вспомнить принцип 
дидактики о сознательности и активности.

Мнемонические правила, от мнемоника -- искусство запоминания. 
Мнемотехника -- совокупность специальных приемов и способов, облегчающих 
запоминание нужной информации и увеличивающих объем памяти образования 
ассоциаций. Каждый охотник желает знать, где сидит фазан. Как поле брани 
усеяно белыми костьми, так и дорога -- мгновенными осьми (мгновенная ось 
вращения). Формулы для прямых и обратных преобразований Лоренца: там где 
штрих -- должна быть дополнительная палочка где плюс, значит, штрих..... 
В сто сорок солнц закат пылал, в июль катилось лето, эр квадрат на синус 
тета, дэ эр дэ фи дэ тета.

Способы запоминания: многократные повторения, использование различных 
видов памяти (моторной -- шпаргалка, зрительной -- записи на стенах, 
слуховой), хоровые упражнения, ритмические фразы (на каждую степень 
свободы...), мнемонические правила, метод аналогий, цифровые и буквенные 
комбинации, регулярное повторение всего сначала...

Примеры аналогий: уравнения движения и моментов, молекула среды 
и броуновской частицы, опыты Кулона и Кавендиша, электростатический 
и гравитационный потенциалы, граничные условия электрического 
и магнитного полей.

Подсказка -- это передача или получение информации, которую можно 
использовать для решения задачи. Отсутствие в условии задачи лишних 
данных тоже подсказка. Обозначения буквенные в задаче тоже подсказывают. 
Рисунок часто подсказывает. Ознакомление с ответом, в крайнем случае, 
тоже помогает решить. Вторая группа подсказок -- от преподавателя -- 
использование опорных фраз.

Домашнего задания не будет сейчас... Нужно выполнять старые три.

% }}}

\section{Состав, цели, задачи учебных дисциплин}
% {{{

Составные части учебной дисциплины
Теоретическая и практическая части и их содержание
....

Составные части учебной дисциплины. Любая учебная дисциплина делится на 
теоретическую и практическую. В состав теоретической части входят 
лекции, семинарские занятия и дополнительные виды конференции, 
. Практическая часть это практикум, общий специальный, специальные лаб 
праки, практики учебная производственная полевая преддипломная, и другие 
виды пед практики экскурсии, наблюдения, работа с макетами, коллекциями. 
При реализации как того, так и другого, реализуется контроль. Опрос сам 
работа, контрольная работа, тестирование, коллоквиум, проверка 
самостоятельно выполненных заданий, зачет, экзамен.

Лекции, семинарские занятия, практикумы. Основные формы учебной 
деятельности. Лекция -- учебное занятие, в течение которого лектор 
излагает материал. Отличительная особенность -- минимальная обратная 
связь. Достоинство лекций как формы учебной деятельности: большой охват, 
лектор очень умный, основной источник информации в целом, во время 
чтения лекций можно проводить демонстрации. Недостатки, однако, лекций 
в том, что препод не может отвечать на вопросы этого большого числа 
обучающихся, отсутствие возможности осуществлять текущий контроль, 
сложно понять осознается материал или нет, недостаточно сильная 
мотивация для посещения лекций. В то же время это очень большая 
ответственность. Польза от лекций зависит от ораторского мастерства, 
теоретическая подготовка, степень содержательной и методической 
проработанности лекций, заинтересованности его. Но одновременно на 
студентах лежит не меньшая ответственность. Семинар обычно подразумевает 
небольшую группу до 30 человек, существенно более тесное взаимодействие 
между преподом и обучающимися, то есть во время семинара реализуется 
значительно большая обратная связь. Особенность семинаров еще в том, что 
даже когда это теоретический элемент курса, там есть практические навыки 
решения задач. В ходе семинара происходит вот такая форма учебной 
деятельности: повторение и обсуждение с наилучшего усвоения основных 
теор положений, которые излагались на лекциях, изучение новых теор 
положений -- тех, которые лектор не успел или не имел возможности 
изложить во время лекции, применение знаний, полученных на лекции 
применительно к семинарам по физике это освоение методики решения задач 
и собственно решение их, ответы на вопросы студентов.

Дальше теперь практикум. У нашего практикума можно пожаловаться конечно, 
что задачи и опыты эти не идут одновременно с лекциями, но с другой 
стороны это заставляет работать самостоятельно. Практикум -- занятие, 
проводимое в спец оборудованной лабе, в ходе... Основной особенностью 
является самостоятельная работа. При проведении практикума препод: 
проверяет готовность обучающегося к самостоятельной работе, выясняет 
уровень владения теор материалом, при необходимости излагает нужный 
материал, дает инструкции и рекомендации по работе, дает обучающимся 
задание, которое нужно выполнить самостоятельно, контролирует сам работу 
и при необходимости руководит ей, проверяет результаты выполнения работы 
и обсуждает итоги с обучающимся.

Теперь есть практики. Учебная, производственная, полевая, преддипломная, 
педагогическая. Учебная практика -- целью является освоение новых 
практических навыков и приобретение практический умений. Примером 
учебной практики может служить работа в лаборатории 
научно-исследовательского института в должности лаборанта -- не с целью 
получения научных результатов, а исключительно с целью приобретения 
навыков обращения с экспериментальным оборудованием. Фактически данный 
вид практики может рассматриваться как разновидность спец прака. 
Производственная практика -- работа в реальном трудовом коллективе 
в условиях реального производства (научного или где угодно еще). Цель 
это приобретение опыта применения полученных теоретических знаний 
в реальных производственных условиях. Полевая практика -- работа 
в природных условиях, экспедициях, метеостанциях, обсерваториях итд. 
Цель -- приобретение практических навыков сбора, хранения, 
систематизации и изучения научного материала для исследовательской 
работы. Преддипломная практика уххх теперь -- совокупность практических 
видов деятельности, направленных на сбор, анализ и систематизацию 
материала, необходимого для подготовки выпускной квалификационной работы 
(диплома). Педагогическая практика -- практическое ведение пед работы 
с целью приобретения и закрепления навыков преподавания различных 
учебных дисциплин. Дополнительные виды пед подготовки -- экскурсии, 
наблюдения, работа с макетами, коллекциями, образцами оборудования 
и подобное. От практик эти виды отличаются эпизодичностью: практика 
обычно продолжается в течение 2--3 недель, а экскурсия это 2--3 часа.

\textbf{Задание подумать, писать не обязательно}, в каких видах 
дополнительной практической подготовки мы участвовали в процессе 
обучения на физфаке? Было ли это полезно и интересно? Что бы вообще еще 
хотелось, чтобы жизнь студенческая была такой вот объемной, интересной 
и прочее.

\hfill\textbf{Oct 18}

Обсудили составные части учебной дисциплины, теоретическую 
и практическую части и их содержание и лекции, семинарские занятия, 
практикумы. Теперь модели построения учебных курсов. Линейная, 
концентрическая, блочная модели.

Линейная это модель, где темы изучаются в логической последовательности, 
но каждая тема затрагивается лишь один раз, не возвращаемся на другом 
взгляде или глубине. Хорошо для коротких курсов полгода--год. Мат анализ 
наш как раз вот так организован. В школе например литература, ОБЖ, 
история.

Концентрическая модель, где все темы курса изучаются два или более раз, 
и на каждом этапе глубина увеличивается. Каждый круг называется 
концентром. Пример в школе -- физика, а у нас -- квантовая механика.

Блочная модель, где темы разбиваются на отдельные изолированные блоки, 
внутри каждого изложение линейное. Особенность -- изолированность боков. 
При помощи этого можно независимо подкручивать разные блоки и даже 
менять преподавателей. В целом курс математики университетской так 
ведется: мат анализ, ан геом, диффуры... Научные дисциплины тоже так 
спецкурсы организованы.

Цели и задачи лекций, теперь, значит. Цели вот такие вот. Закладывание 
общей основы для изучения учебной дисциплины: знакомство и предметом, 
целью и средствами изучения; введение базовых понятий; актуализация мат 
аппарата, необходимого для изучения; знакомство со специфическими 
методами, нужными для дисциплины. Знакомство с новыми фактами: изложение 
научных фактов, рассказ о научных открытиях и истории их, описание 
нерешенных проблем и возможных подходов к ним. Знакомство с новым теор 
материалом: формулировки теорем, свойств, проведение доказательств, 
объяснение научных фактов, явлений, экспериментов. Создание системы 
взаимосвязей с другими дисциплинами: рассказ о наличии связей между 
изучаемой дисциплиной и другими дисциплинами, примеры использования 
знаний в этой и других дисциплинах. Демонстрация связи с практикой: 
эксперименты, видео фото рисунки иллюстрирующие применение знаний, 
информирование об областях применения в науке технике промышленности. 
Для достижения этих целей хорошо бы решать следующие задачи. Определение 
тематики лекции, при этом нужно стараться уместить необходимый материал 
в одну лекцию, если не умещается, приходится разделять логичным образом. 
Составление плана лекции после определения тематики, определить какие 
нужно вещи сказать показать и прочее, что нужно доказать что просто 
сказать. Подготовка конспекта лекции. Подбор демонстрационных 
экспериментов и иллюстраций один или несколько способов... Подбор 
наиболее важных и интересных научных фактов, примеров, описаний 
открытий, чтобы было интереснее и лучше запоминалось. Подбор наиболее 
ярких примеров получаемых знаний на практике. Хронометрирование лекции 
и корректировка ее плана. Подготовка презентации при необходимости, но 
основным материалом презентация быть не должна, презентация не должна 
быть особо большой, особенно где схемы и формулы, включать не следует 
все подряд, не следует помещать туда очень много текста, не слишком 
много декораций, крупный шрифт. Репетиция, апробация при необходимости, 
проверить оборудование. Ответственное исполнение обязанностей лектора, 
не опаздывать, говорить громко и четко, сделать материал доступным, 
четко формулировать мысли, в достаточной степени объяснять, отвечать на 
вопросы.

Структура лекции по дисциплинам общей физике, теор физике и математике. 
Различия обусловлены использованием разных подходов. В физике 
обсуждаются сначала явления и опыты, а потом законы... На теор физике 
по-другому, сначала формулируются общие принципы, а после из них 
выводятся прочие закономерности, то есть от общих к конкретным 
закономерностям, все это основано на аксиоматике, но аксиомы имеют 
реальное обоснование. В математике определяется система понятий и потом 
доказываются разные вещи, потому что изучаются в общем-то вымышленные 
объекты. Далее идут примеры общих планов лекций общей физики, теор 
физике, математики, см. презентацию.

\textbf{Домашнее задание 5} (а где 4?) Напишите, какие методические 
находки и ошибки лекторов вы отметили бы по вашему опыту посещения 
лекций в дистанционном формате?


\hfill\textbf{Oct 25}

Семинарское занятие: цели и задачи, основные этапы. Практикум: цели 
и задачи, натуральный и модельный эксперимент.

\subsection{Семинарские занятия}

Слушает небольшая группа человек, сильная обратная связь. Повторение 
теор материала, знакомство с новым, обсуждение теор положений, контроль 
знаний, практические приемы применения, развитие навыков применения, 
навыков обсуждения и дискуссии, самостоятельная работа.

При подготовке семинарист решает следующие задачи. Определение темы 
семинара, чаще всего уже заранее определен в методичках, но его можно 
чуть корректировать. Планирование содержания семинара и подбор 
материала, определить новый и старый материал, какие вопросы нужно 
обсудить, понадобится ли время для самостоятельной работы. Составление 
плана семинара, планирование всякого, чтобы студенты лучше усвоили все 
это. Составление конспекта семинара, где содержатся теор материал, 
решение задач, заметки о дискуссионных вопросах, домашнее задание 
и прочие материалы для контроля знаний. Хронометрирование семинара 
и корректировка плана, когда конспект уже составлен, прочитать его 
и оценить. Репетиция и апробация при необходимости. Ответственное 
исполнение обязанностей преподавателя, не опаздывать на семинары, 
излагать четко и ясно, записи разборчивые, выписывать отдельно и не 
стирать особые вещи, уделять особое внимание подбору задач (предельные 
случаи, особые всякие, максимальная польза из каждой задачи), полезно 
решать одну задачу несколькими способами.

Основные этапы семинарского занятия. Повторение ранее изученного 
материала, в том числе во время лекций, минут 15. Изложение нового 
материала, что не успел рассказать лектор, до 30 минут наверное. 
Изучение трудных моментов, частные случаи, мелкие задачи, тонкости. 
Закрепление практических навыках, решение самостоятельное, решение 
у доски, вопросы. Самостоятельная работа, в аудитории или дома. Контроль 
качества усвоения материала, опрос, проведение самостоятельных работ, 
контрольных работ.

\subsection{Практикум}

Цели и задачи. Натурный и модельный эксперимент. Повторение теор 
материала и знакомство с новым, знакомство с оборудованием, с физ 
явлениями, эффектами, процессами, знакомство с методиками, развитие 
и тренировка эксп умений и навыков, знакомство с методами обработки 
и интерпретации, приобретение умений по подготовке лаб отчета и его 
защите. В презентациях достаточно текста.

% }}}

\section{Оценивание учебных достижений и мотивация}
% {{{

\hfill\textbf{Nov 1}

\subsection{Система оценивания. Зачет, зачет с оценкой, экзамен}

Перед собой ставит цели преподаватель. Проверка усвоения теор материала, 
практ умений, ... Существует пять уровней усвоения -- узнавание 
воспроизведение воспроизведение перенесение творчество. Знания, умения 
нужно проверять на определенном уровне усвоения.

\hfill\textbf{Nov 8}

Балльно-рейтинговая система.

\textbf{Задание 7} Вспомните и напишите, встречались ли Вы во время 
обучения на физическом факультете (или в другом вузе) с БРС, которая 
по-нашему была устроена несправедливо. Если да, в чем состояла 
несправедливость?

Зачет, экзамен, тестирование.

% }}}

\section{Мотивирование}
% {{{

Мотивация это от латинского движение.

% }}}


\end{document}
