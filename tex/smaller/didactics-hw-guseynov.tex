\documentclass[a4paper, 12pt]{article}

% Configuration {{{
\usepackage[utf8]{inputenc}
\usepackage[T2A]{fontenc} % T1 for English
\usepackage[english, russian]{babel}

\usepackage{enumitem}
\setlist{nolistsep}
\usepackage{mathtools}
\usepackage{xcolor}
\definecolor{dimblue}{HTML}{1010aa}
\usepackage[
	colorlinks=true, 
	allcolors=dimblue
]{hyperref}
\usepackage[
	vmargin=1in,
	hmargin=1in
]{geometry}
\linespread{1.3}
\usepackage{indentfirst}
\usepackage{graphicx}
\usepackage[multidot]{grffile}
\usepackage[labelsep=period]{caption}
\usepackage{subcaption}

%\usepackage{times} % for English

\def\task#1#2{
	\vskip2\baselineskip
	\phantomsection
	\addcontentsline{toc}{section}{#1}
	\begin{center}
		\textbf{#1}
	\end{center}

	\begin{center}\begin{minipage}{.8\linewidth}\small
		#2
	\end{minipage}\end{center}

	\vskip\baselineskip
}
% }}}

\begin{document}

\noindent Керим Гусейнов \hfill группа 213М

\tableofcontents

% hw 1 {{{
\task{Задание 1}{
	Вы прослушали и сдали множество учебных курсов.
	Нарушались ли преподавателями принципы дидактики
	при преподавании какого-либо курса? Какие именно
	принципы? Как именно нарушались?
}

Мой длительный ответ.

Киров вызывал к доске
Володин вызывал к доске
Галлямова....

%%%%%%%%%%%%%%%%%%%%%%%%
%%% FROM THE LECTURE %%%
%%%%%%%%%%%%%%%%%%%%%%%%
% 1) Научность.
% 2) Доступность.
% 3) Целенаправленность.
% 4) Систематичность и последовательность.
% 5) Наглядность.
% 6) Связь обучения с повседневной жизнью.
% 7) Сознательность и активность.
% 8) Прочность знаний.
% 9) Воспитание и развитие. 
% 
% 1 -- Научность
% 
% Сущность принципа: содержание обучения должно
% соответствовать реальным фактам и отражать
% современные научные данные.
% 
% Требования принципа: формирование у учащихся
% системы теоретических знаний, достоверность
% изучаемых фактов, подтвержденность действий и
% выводов педагога наукой. 
% 
% 2 -- Доступность
% 
% Сущность принципа: обучение должно соответствовать
% индивидуальным особенностям учащихся и имеющимся у
% них знаниям. Обучение не должно быть ни очень легким,
% ни чрезмерно сложным. Сам Ян Коменский писал, что
% обучение должно идти от простого к сложному, от
% известного к неизвестному, от близкого к далекому.
% 
% Требования принципа: нужно учитывать образовательный
% уровень, познавательные возможности, профессиональную
% подготовку, характер, опыт, возрастные особенности,
% потребности и интересы учащихся. 
% 
% 3) Целенаправленность.
% 
% Сущность принципа: необходимо осознанно создавать
% организационные, методические и содержательные
% основы педагогического процесса, направляя его к
% достижению поставленных образовательных целей.
% 
% Требования принципа: содержание образовательного
% процесса должно соответствовать содержанию
% воспитания, а обучение должно соответствовать
% учебному плану, рассчитанному на достижение
% определенных результатов.
% 
% 4) Систематичность и последовательность.
% 
% Сущность принципа: необходимы особый порядок и система
% преподавания, основанные на логике и хронологии. Подача
% информации должна планироваться, информацию нужно
% разбивать на разделы, модули, темы, выделять идейные
% центры и главные понятия, расставлять акценты.
% 
% Требования принципа: учебный материал нужно
% преподносить в строгой логической последовательности,
% обеспечивая одновременное применение полученных знаний
% на практике с целью их закрепления.
% 
% 5) Наглядность.
% 
% Сущность принципа: при обучении нужно в первую очередь
% опираться на зрительные органы, и лишь затем – на остальные
% органы чувств. Поэтому крайне необходимо применять
% средства повышения наглядности. Следует помнить, что
% максимальной информативностью обладает именно зрение,
% т.к. оно даёт человеку 80% знаний.
% 
% Требования принципа: демонстрирование нужно проводить в
% определенном порядке с определенными целями, разные
% виды наглядности должны сочетаться друг с другом,
% наблюдаемое должно подвергаться анализу учащимися и
% педагогом, наблюдаемое должно соответствовать культурным
% и психологическим требованиям. 
% 
% 6) Связь обучения с повседневной жизнью.
% 
% Сущность принципа: процесс обучения нужно сопровождать
% постоянным сомнением и проверять теорию с помощью
% практических критериев. Данный принцип называют иначе
% принципом связи теории с практикой.
% 
% Требования принципа: учебно-воспитательный процесс
% должен иметь явно выраженную профессиональную
% направленность, в ходе процесса обучения нужно отвечать
% на вопросы – когда, где и как в жизни можно применять
% полученные знания. 
% 
% 7) Сознательность и активность.
% 
% Сущность принципа: поскольку в педагогическом процессе принимают
% участие две стороны – педагог и учащийся, то обе эти стороны должны
% быть активными, понимать свои цели. Педагог является субъектом
% образования, а учащийся – объектом.
% 
% Активность учащегося состоит в усвоении содержания обучения, в
% самостоятельной организации своей работы и проверке её результатов.
% Активность педагога состоит в мотивации обучения, формировании
% познавательных склонностей учащегося, использовании разных методов
% обучения.
% 
% Требования принципа: учебный процесс должен быть двусторонним,
% педагог должен использовать активные формы обучения, побуждать
% учащихся к самостоятельности и творчеству, развивать у них научное
% мышление и навыки применения полученных знаний для решения
% практических задач. 
% 
% 8) Прочность знаний.
% 
% Сущность принципа: необходимо стремиться закрепить
% содержание обучения в сознании учащихся. Для этого
% нужно стимулировать стремление к познанию,
% систематически повторять материал, регулярно
% контролировать результаты обучения.
% 
% Требования принципа: знания должны повторяться и
% закрепляться, умения и навыки должны применяться на
% практике, должен обеспечиваться систематический
% контроль, сочетаемый с индивидуальным подходом к
% каждому учащемуся.
% 
% 9) Воспитание и развитие. 
% 
% Сущность принципа: педагогический процесс должен быть
% направлен на воспитание и развитие в учащихся не только
% профессиональных качеств и навыков, но и на развитие
% учащегося как адекватной, здоровой, пристойной, состоятельной
% и живой личности. То есть образование и воспитание должны
% «идти рядом». Этот принцип называют также принципом
% воспитывающего и развивающего обучения.
% 
% Требования принципа: нужно помнить, что основные цели
% обучения – развивающая, воспитывающая и познавательная –
% должны достигаться параллельно.
% 
% ------
% В процессе обучения нужно формировать у учащихся научное
% мировоззрение, творческое мышление, инициативность и
% самостоятельность, способность делать выводы, сопоставлять,
% сравнивать, выделять основное, обобщать, анализировать.
% Нужно также воспитывать дисциплинированность, навыки
% культурного поведения, интеллигентность, гуманность,
% гражданскую ответственность и патриотизм.
%  }}}

% hw 2 {{{
\task{Задание 2}{
	Сформулируйте по одному вопросу каждого типа из какого-либо спецкурса по 
	Вашей специальности или из-какого-то одного математического курса (мат. 
	анализ, ТФКП, линейная алгебра и т.д.). Напишите, какие еще, по вашему 
	мнению, типы вопросов можно использовать при работе со студентами на 
	семинаре? в практикуме? на экзамене?
}

Вопросы по общему курсу дифференциальных уравнений.
\begin{enumerate}
	\item \textit{Что это такое? (Дайте определение ...)}

		Что такое характеристическое уравнение линейного дифференциального уравнения с постоянными коэффициентами?

	\item\textit{Сформулируйте ...}

		Сформулируйте теорему Коши существования и единственности решения дифференциального уравнения первого порядка.

	\item\textit{Напишите формулу (уравнение) ...}

		Напишите формулу для общего решения неоднородного линейного дифференциального уравнения.

	\item \textit{Нарисуйте график ...}

		Изобразите фазовый портрет дифференциального оператора вблизи устойчивой точки покоя.

	\item \textit{Приведите пример...}

		Приведите пример автономного дифференциального уравнения второго порядка.

	\item \textit{Изобразите схему опыта...}

		Опишите метод последовательных приближений.

	\item \textit{Как соотносятся...}

		Как соотносятся функция Грина дифференциального оператора и решение неоднородного уравнения с этим оператором?

	\item \textit{Сколько?}

		Сколько элементов содержится в фундаментальной системе решений дифференциального уравнения $n$-го порядка?

	\item \textit{Почему?}

		Почему для краевой задачи, в отличие от задачи Коши, не существует теоремы существования и единственности решения?

	\item \textit{Найдите ошибку в утверждении...}

		Рассмотрим систему дифференциальных уравнений
		$$\frac{d x_i}{d t} = \sum_{j=1}^{n} A_{i,j} x_j,\ \ i=\overline{1,n}$$
		и решения $\lambda_k$, $k=\overline{1,n}$ уравнения
		$$\mathrm{det}(A-\lambda\,1_{n\times n}) = 0.$$

		Найдите и исправьте ошибку в следующем утверждении:

		Решение системы $x_i = 0$, $i = \overline{0, n}$ называется точкой покоя типа фокус, если существует хотя бы два числа $i, j$ от $1$ до $n$ такие, что $\mathrm{Re} \lambda_i \cdot \mathrm{Re} \lambda_j < 0$.
\end{enumerate}
% }}}

% hw 3 {{{
\task{Задание 3}{
	Придумать одну задачу (с решением) из
	любого курса общей физики или из спецкурса по
	Вашему выбору, которая допускает различные
	решения в зависимости от выбранных абстрактных
	моделей. 
}

Мой длительный ответ.
% }}}

% hw 4 {{{
\task{Задание 4}{
	В последнее время все мы вынужденно столкнулись с необходимостью 
	проведения занятий в дистанционном формате. Несмотря на то, что до 
	пандемии специалисты по дистанционным образовательным технологиям 
	утверждали, что дистанционное занятие ничуть не хуже очного, 
	в реальности все оказалось не так просто. Вопрос. Напишите, какие 
	методические находки и ошибки лекторов Вы отметили бы по Вашему опыту 
	посещения лекций в дистанционном формате?
}

Лекция 6
Мой длительный ответ.
% }}}

% hw 5 {{{
\task{Задание 5}{
	Придумайте аналогичный пример проверки элемента знаний из курса общей 
	физики, математики или спецкурса по выбору (составьте 4--5 вопросов по 
	предложенному образцу).
}

Лекция 8, слайд 28
Мой длительный ответ.
% }}}

% hw 6 {{{
\task{Задание 6}{
	Вспомните и напишите, встречались ли Вы во время Вашего обучения на 
	физическом факультете (или в другом вузе) с БРС, которая была, по 
	вашему мнению, устроена несправедливо? Если это так, то в чем состояла 
	несправедливость?
}

Лекция 9, слайд 9
Мой длительный ответ.
% }}}

\end{document}

