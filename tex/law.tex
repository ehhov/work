\documentclass[a4paper, 12pt]{article}

\usepackage[utf8]{inputenc}
\usepackage[T2A]{fontenc}
\usepackage[english,russian]{babel}

\usepackage{cmap}
\usepackage{calc}
\usepackage{enumitem}
\setlist{nolistsep}
\usepackage{mathtext,mathtools,amsmath,amssymb}
\usepackage{xcolor}
\definecolor{allrefs}{HTML}{1010aa}
\usepackage[
	linktoc=page,
	colorlinks=true,
	allcolors=allrefs
]{hyperref}

\frenchspacing
\linespread{1.3}
\usepackage{indentfirst}
%\setlength{\parindent}{10pt}
\usepackage{graphicx}
\usepackage[multidot]{grffile}
\usepackage[labelsep=period]{caption}
\let\origRef\ref
\def\ref{\unskip~\origRef}
\let\origCite\cite
\def\cite{\unskip~\origCite}
\makeatletter
\g@addto@macro\@floatboxreset\centering
\makeatother

\usepackage[
	vmargin=1in,
	hmargin=1in
]{geometry}
\usepackage{multicol}
\usepackage{cuted}
\setlength{\columnsep}{.25in}
\usepackage{flushend}

\def\datee#1{\hfill\textbf{#1} \par}
\def\mailto#1{\href{mailto:#1}{#1}}



\begin{document}

\thispagestyle{empty}
\noindent
Александр Александрович Долганин
\hfill 
\mailto{al.dolganin@gmail.com}
\\ \null \hfill 
\mailto{legalphysics@gmail.com}
\\ 
Пн 16:30--18:00 А-??? на юрфаке
\hfill
+7 985 225 7205

\vfill

\begin{center}
\Large\bf
Правоведение
\end{center}

Зачет это ответы на вопросы, которые дадут заранее, и беседа. Требуется посещать лекции, но списков групп не будем составлять, будет выборочно проверять какие-то группы. Будут даны какие-то задания, аналитика, поиск информации, их приносить распечатанными, отправлять на почту или даже высказываться устно на лекции, если на почту, домашку сдавать до 23:59 понедельника (лекция в среду). Творческую работу он дает на других факультетах, но у нас нет, ибо семестр короткий. Вместо будет одна внезапная проверочная работа, по одной из уже изученных тем. Один общейший вопрос, например, совпадающий с названием предыдущей лекции, и мы за 5 минут пишем что знаем. Но строго пять минут, не больше. 

Литература: 
\begin{enumerate}
\item Базовый учебник, в библиотеке -- Е.А. Абросимова, В.А. Белов, ``Правоведение для студентов не юридических факультетов'' (2016).
\item Первая часть курса -- Б.И. Пугинский, ``Теоретические положения о государстве и о праве'' (2005).
\end{enumerate}

Нормативные акты и судебную практику не гуглить, а консультант плюс и грамота. \href{http://garant.ru}{garant.ru}, \href{http://consultant.ru}{consultant.ru}, \href{http://pravo.gov.ru}{pravo.gov.ru}

\vfill

\begin{center}
2020
\end{center}

\clearpage
\tableofcontents
\clearpage

%%%%%%%%%%%%%%
\datee{Feb 12}
В рамках нашего небольшого курса правоведения правоведение не профильное, но к концу курса, когда будут отрасли разбираться, какие-то элементы будут возникать. Например про патентное право. 

Курс состоит из двух частей: общей и особой. Любая отрасль юридическая делится на такие две части, это древнеримская система. В общей части -- общие положения о государстве и праве. В особой части -- отрасли: конституционного права и гражданского права, постараемся успеть трудовому и семейному. Дальше если останется время либо его пожелания, либо наши, если найдутся достойные просьбы. Конституционное право, неожиданно для всех, особенно для чиновников высшего ранга, актуальность этой части права резко возросла для всех совершенно. 

\section{Общие положения о государстве}
Понятие государства можно использовать любое приличное с нашей точки зрения. Например, организация публичной власти на определенной территории. Другие подходы связаны с политологической точкой зрения, в смысле политического института, элемента политической системы, надстройка в обществе, связанная с политической системой. 

Признаки государства. Территория с четкой границей. Один из трех ключевых признаков, необходимых и достаточных для государства. Территория это не просто пространство, а скорее граница, замкнутая система, определенность границ, контур обязательно должен быть замкнутым. Границы это контроль и внутренний суверенитет. Границей заведуют силовые структуры, ФСБ у нас. 

%%%%%%%%%%%%%%
\datee{Feb 19}
11 марта в 12:35 начинается пара, идет до 14:45. 

Живое право может быть полезным собранием курсов, говорят о составлении документов. 

Признаки государства 
\begin{itemize}
\item территория граница и это еще национальные акты и международные акты. Вниз у государства территория либо до ядра говорят, либо до мантии. До мантии рациональнее, никто даже так глубоко еще не копал. Вверх до стратосферы, до нижнего космоса. В морях и океанах 12 морских миль от берега. Дальше вопросы регулируются конвенцией 88 года, международного права. Дальше потом в зоне шельфа континентального есть привилегии странам рядом. Дальше есть особая экономическая зона, там можно странам рядом добывать ресурсы. Дальше свободное море, эта область уже решается международным делом. 
\item население, без населения естественно невозможно представить это дело. Людей должно быть 2 или более, иначе института и отношений не будет 
\item публичная власть. Вы можете соорудить отношения субординации. Почему власть называется публичной? Потому что это государственная плюс муниципальное. У нас местное самоуправление. Местное самоуправление это, например, мэр. Назначение мэра -- совет органов или выборы, губернатор не имеет прямого доступа. Вмешательство губернаторов мешает. Публичная власть публичная потому что распространяется не на персон, а на всех сразу, неопределенный круг лиц. Публичная власть в рамках территории юрисдикции все равно кто попадает под действие юрисдикции. 
\end{itemize}

Все другие признаки, и суверенитет да, вытекают из этих. Суверенитет есть внутренний и внешний, внешний это независимость в международных отошений. Внутренний суверенитет это верховенство публичной власти на территории, одной единственной публичной власти. В настоящем мире не один сейчас в Ливии. Государство потеряло признаки из-за того, что в Ливии два суверенитета. Майкл Бей 13 часов бенгайза??? \textbf{фильм} неплохо описывает. Герб, гимн, флаг -- не признаки государства, а атрибуты. Налоги не относят к первичным признакам -- следствие публичной власти, производный признак. Право лучше не считать признаком государства, потому что право и государство равноценны, нельзя понять что было сначала, а что потом. Это симбиоз. Право это параллельный институт. 

Формы государства. Традиционно три элемента. Форма правления, форма административно\,\!-\,\!территориального устройства, форма чего???. Форма правления. Виды республик. Президентская, Парламентская, авторитарная. Президентская-Парламентская (Смешанная). Почему часто говорят, что россия республика президентская, президент очень сильная фигура с кучей полномочий. Здесь трудности перевода. Президентская республика не означает сильную президентскую власть. Президентские республики характеризуются борьбой законодательной и исполнительной власти, здесь президент принадлежит к одному из центров силы, исполнительной власти, и в постоянной борьбе. Президент всегда зависит от законодательной власти, у нее есть инструменты. Типичные примеры президентской республики не имеют супермощного президента. Если президент встроен в систему каких-то органов, на него куча ограничений. Если президент обособлен, это совершенно другие способности по полномочиям и расширению. В США президент может быть ограничен, ему могут вставлять палки в колеса и прочее. Наш президент -- система сдержек и противовесов, у него для такого контроля президенту по конституции кучу всего разрешено делать. Во Франции точно такая же история, там вождистская конституция. Просто нет потребности выжимать из конституции все. Это свойство смешанной республики. В смешанных у президента нет ???, замещает собой руководство исполнительной властью, может во всяком случае, премьер-министр есть, для распыления управления властью. Правительство в смешанных республиках имеет дуальную ответственность, и перед парламентом, и перед парламентом. Посмотреть в учебнике. 


Государственный строй, унитарное государство, федерация. Федерация это объединение государственно-подобных субъектов. Федерации: Россия, США, Германия, Швейцария, Индия. Италия государство унитарное, но регионалистское. Там для северных провинций привилегии, но это не значит, что у провинций есть правосубъектность. У федераций есть. В частности могут даже самостоятельно заключать международные договоры. В разных федерациях по-разному, больше полномочий может быть. Великобритания это монархия конечно, поэтому нет такой уж правосубъектности. Нельзя считать полноценной федерацией. Унитарное все же. 

\section{Право}

Под правом терминологически понимается в субъективном смысле и в объективном. Объективное право -- закон -- система норм, действующая в государстве. Правовая система, то же самое. Субъективное право это более значимо для повседневной жизни это возможность определенного поведения, закрепленная объективным правом, в законе. В советском подходе субъективного права это мера дозволенного поведения. Примеры субъективных прав: льготный проезд, на жизнь, здоровье, благоприятную окружающую среду, свободы вероисповедания и так далее все из второй главы конституции. Помимо всего есть еще и права гражданско-правовые, как право на высшее образование (в конституции его нет, есть право на среднее, то есть школьное, и право на получение образования на конкурсной основе), право собственности. Эти права статические, они есть всегда, не нужно для их появления что-либо делать. Есть еще права динамические, когда вы вступаете в какие-то правоотношения и на время их проведения получаете права, как например при купле-продаже, трудовые права, перевозка на общественном транспорте. 

Признаки объективного права: 
\begin{itemize}
\item общая обязательность (уникален для правовых норм в сравнении с другими социальными нормами). 
Общеобязательность правовых норм случилась потому что блага получаем от этого договора с государством, соблюдения норм. Это благо должно быть уникально, потому что если не уникально, то зачем государство, если можно получить благо от кого-то другого. Защита, защищенность, можно рассчитывать на поддержку, в том числе физическую, если что-то случилось. Только государство может такое предоставить. У государства монополия на насилие. 

\item всегда санкционируется и обеспечивается государством (самый простой с точки зрения правовых норм и неправовых, если есть государство, есть право).
\item нормативность (право в целом и норма есть модель поведения, для оценки поведения).
\item определенность права, формальная определенность, формализованность. Значит, что право формирует формы своего существования. Формы права, источники права, где оно изложено, в других местах не может быть закреплено. 
\item содержательная определенность. Это значит, что правовая норма не только закреплена в документе определенного вида, но она еще и однозначна, объективна, идентична для всех субъектов, применяющих их. Никакие дискуссии не могут быть проведены. Интерпретации могут быть. 
\item системность (очень важный, необходимый). Это признак, который кучу слов сказал, но я ничего не понял 
\item нормы действуют длительно и применяются неоднократно. 
\item действуют на неопределенный круг лиц. Индивидуальные акты это санкции индивидуальные, решения суда и подобное. 
\end{itemize}

Определение права -- система и совокупность норм, которая обладает этими признаками. 

Другие нормы: моральные, религиозные, традиции, этикет, политические, международного общения (международный протокол), корпоративные, экологические, технические (использование предметов). 

Система права. Уточнение признака системности. Система права не равно правовая система. Правовая система это правопорядок в государстве. Система права это внутренняя структура права, то, как нормы упорядочены и взаимосвязаны. Первый уровень, базовый уровень, это просто правовая норма. Правовая норма это во-первых по структуре состоит из трех элементов. Диспозиция -- само правило поведения, грубо говоря сказуемые, долженствование, следует и т.д.. Гипотеза -- предпосылки, условия, обстоятельства действия нормы, начало действия нормы. Санкция -- та часть нормы, которая закрепляет последствия неисполнения диспозиции. 

\textbf{ДЗ} \textit{Найти три правовые нормы, любые, и расчленить на три части, гипотезу, диспозицию, санкцию. }


\datee{Feb 26}

\subsection{Институты и отросли}

Кроме нормы, как первичной единицы, есть институты и отросли. Институт -- совокупность норм, регулирующих одно общественное отношение. Вот есть норма владение пользование итд -- институт права собственности. Отросли это еще более высокий уровень -- совокупность норм, регулирующих сферу однородных общественных отношений, регулируют единым методом. Права собственности как институт -- гражданское право как отрасль. Есть суб и над -- субинституты, суботросли, надинституты. Не путать отрасли права и отросли законодательства. Отрасль законодательства это совокупность источников права, которые регулируют одну и ту же форму общественных отношений. Отрасль законодательства есть лесное право -- все акты и другие источники права, регулирующие взаимодействие с лесом. Лесное право уходит в административное право. Государственное регулирование чего угодно, культуры, сельского хозяйства, есть административное право. Публичное и частное право. Это две надотросли, по суди других нет. Все право можно поделить на них. Изобретение критериев по распределению на публичное и частное началось в Римском праве. Критерий интереса -- ради кого создана, чей интерес обеспечивает -- для государства публичное, для частных лиц частное. Но государство всегда маскирует свои интересы под интересы частных лиц. Если гражданин чувствует, что он исполняет свой интерес, естественно делает все лучше. Есть еще критерий метода регулирования, частноправовые отношения горизонтальны, публично-правовые вертикальные, потому что у кого-то есть власть, вступаете в отношения субординации, а не координации. Горизонтальными могут быть отношения даже между государством и чем-то, есть в гражданском праве субъекты -- муниципальные образования и подобное -- с частными лицами частным правом взаимодействуют. Например, это госзаказы. Отдельное регулирование есть, но государство спускается на уровень частника. Можно использовать даже обе нормы сразу. В учебнике есть еще несколько критериев, но мы не будем о них общаться. Разделение права на публичное и частное не только теоретически важно. 

Источники права (иногда Формы права). Определение -- материальный и формальный подход. Материальный смысл источника права это некая совокупность общественных сил, ценностей, интересов, которые предопределили появление права. Потребность, определенная какой-то ценностью. Хороший пример лоббистская деятельность. В России не регулируется, а в США регулируется. Продвижение каких-то интересов. Насколько можно злоупотреблять деятельностью -- разница между нами и США. Формальный смысл источника права -- форма объективного выражения и закрепления правовых норм. Способ придания норме юридической силы. Они не противоречат друг другу, потому что норма должна быть формально определена. 

Виды источников. Нормативно-правовые акты. В нашей системе права (континентальная правовая семья, или же романно-германская правовая семья) на первой месте говорим о нормативно-правовых актов. Нормативный правовой акт это документ, принятый государственным органом в пределах его компетенции, содержащий нормы права. Мы их уже знаем по первому домашнему заданию. Признаки: 
\begin{itemize}
\item письменный характер. Это включает электронный вид, если что. \href{http://pravo.gov.ru}{pravo.gov.ru} сейчас содержит это самое все такое,
\item могут принимать только органы публичной власти,
\item и только в пределах компетенции,
\item обязательные реквизиты. Это подпись уполномоченного лица или подписи лиц, дата принятия, название, номер. Дата нужна особенно для того, что описывается далее, 
\item структурная определенность, определенная структура. Не хаотичный набор норм, а определенная структура. В нормативных актах всегда бывает сфера регулирования и предмет регулирования, определение терминов, используемых в акте, субъекты, на которых распространяется акт. Завершается каждый нормативный акт как правило заключительные положения (или переходные), там указан срок вступления в силу. Переходные положения это что происходит с регулированием предыдущих законов. 
\end{itemize}

Виды нормативных актов, они делятся все на законы и подзаконные акты. Закон отличается несколькими признаками. Самый главый -- субъект. Принимает закон законодательный орган -- представительная власть, представительный орган. Представительный значит законы непосредственно исходят от органа, который представляет населения, наибольшей части населения. Законы обладают высшей юридической силой. В широком смысле конституция -- закон. Предмет регулирования особый тоже -- законы регулируют только наиболее важные общественные отношения. У нас в России с этим проблемы, законы принимаются не только по поводу особо важных проблем. В результате законов очень много. Виды законов. На первое место можно поставить конституцию. На второе законы о поправке в конституции. На третьем федеральные конституционные законы. Это законы, принимающиеся только по предпосылке из конституции по отсылке в конституции. Это по сути детализация положений конституции. Например это утверждение каки-то высших органов. Дальше федеральные законы. Кодексы это тоже федеральные законы. Есть несколько этапов, называемых чтениями, которе проходят федеральные и федерально конституционные законы перед принятием. 

Подзаконные акты это акты исполнительной власти, принимаются во имя исполнения закона какого-то. Подзаконные акты всегда связаны с какими-то законами. Это нужно для того, чтобы закон применялся надлежащим образом. Примеры. Каждый тип взаимодействие государственного органа с гражданином регулируется подзаконными актами. Административный регламент декларирует даже время ожидания в очереди наряду со всеми документами. Все же примеры. ПДД это источник права, закреплены они в постановлении правительства. У президента есть указы, у правительства постановления (распоряжения уже не нормативные акты), министерства дают приказы, это все подзаконные акты. 

\datee{Mar 4}

Источники. Актуален непосредственно для России. У нас нет источника права, который закрепляет список источников. Норма без формы не является правовой нормой. У нас нет системы источников. После нормативно-правовых актов идет правовой обычай. По сути это обычай, который имеет юридическую силу. Обычай это исторически сложившееся, устойчивое правило поведения, регулирующее общественное отношение. Обычай естественен, связан с наиболее эффективным поведением. Соблюдение правовых всяких это тоже обычае. Самый простой путь создания правового обычая просто внесение в куда-нибудь. Дальше закон, допускающий существование каки-либо обычаев. Эти обычаи в казуальных формах. Когда в суде апеллируют к обычаям, суд выясняет какие обычаи существуют где-то там как-то там и принимает решение. Это существование обычаев позволяет государству не лезть. И во всем мире, например, не лезут государства в морское право, морские перевозки. Все, что связано с мореплаванием. И еще добавляется блок в коммерческую перевозку. Морское право очень сильно влияется обычаями. Это потому что в деле абсолютно необходима адекватность. В морском праве: общение судов, выгрузка, загрузка, пребывание в порту. Правово закреплен свод портовых обычаев. Вообще всего правовых обычаев очень мало. Советские инструкции 60х годов, инструкция госарбитража. Место обычаев в системе источников. В Германии и Франции обычай намного сильнее. У нас обычай слабее и закона, и даже договора. Обычай сейчас не просто редко встречаемая, но еще и слаба. Подзаконные акты и обычаи это все же разные. Обычай сильнее него. 

Источники, которые являются наиболее актуальны в России завершаются. Нормативный договор. Нормативный договор это не просто договор, регулирующий отношения двух сторон, а договор с правом, который распространяет действие на неопределенный круг лиц. Например, это международный договор. Между государствами, между правительственными организациями. Важно что нормативный договор все же для неопределенного круга лиц. В международных договорах дела распространяются на всех жителей. Соотношение международного договора и конституции. Выше у нас считается конституция. Во Франции и Германии тоже так. У нас в конституции появляется поправка, разрешающая не исполнять международные договоры, примат???. Например, ЕСПЧ может говорить России заплатить, а она может не платить из-за таких. Такие государственные местечковые интересы создают картину. Примат конституции над тем международным не редко, но такое использование и узаконивание нехорошо. Нормативные договоры есть еще в рамках национального правопорядка. Между федерацией и субъектами или между субъектами. Федеративный договор был у нас, а потом его почти все содержимое вошло в конституцию. Еще есть коллективный договор -- источник трудового права -- между работником и работодателями. В отличие от трудового договора коллективный распространяется на всех, в частности на будущих, он не противоречит трудовому договору, он связан с дополнительными гарантиями по защите прав работников. Это страхование, корпоративные пенсии, разные поблажки семье и вообще привилегии. У всех крупных корпораций есть мощные коллективные договоры. И важно, что если не предоставляют написанное там, можно потребовать, это право. 

Дальше источники разбирать не будем, но есть еще вообще источники. Религиозные нормы, во многих государствах мира являются источниками светского права, и даже не только в исламских странах, например, Израиль, старый еще свободный Тибет, в Европе самая яростная католическая страна это Ирландия, там разводы стали легальны только в 1990х. Религиозные нормы удобно использовать как юридические потому что все формализовано, есть общепризнанные формы. Еще источник это случай когда наша правовая система пытается имплементировать источник. Это прецедент. Это решение гос органа по конкретному делу, которое является обязательной моделью для всех последующих аналогичных дел. Прецеденты бывают административные (решение вынесено административной властью) и судебные (решение вынесено судом). Административные прецеденты могут вносить федеральная налоговая служба, федеральная антимонопольная служба. Последний может по антимонопольным спорам устраивать разбирательства и выносить прецедентные решения. Судебные прецеденты возникли в Риме. Там это было решение претора, который с несколькими еще лицами по сути осуществляли представление государства в местах, было что-то типа современного нотариуса. Претор мог решать спор и вносить дикт. Но эта система не была строгой и четкой. Родиной считается все же Англия. Появился он по сути вместе с Англией как она есть. После Норманского завоевания начинается отсчет и Англии и прецедентов. Это связано с тем, что появился институт путешествующих судей, в XII веке. Это были подвижные центры королевской власти в каких-то местах, у них были не только полномочия говорить за короля, но и ресурсы и так далее. Это похоже на наши институты дружины и князей во времена Киевской Руси. В Англии акцент был на правосудии. Приходил судья, разбирался в делах локальных и решал все, и все должны были исполнять, иначе встреча с дружиной. Судьи решали довольно типовые споры, поскольку и общественная, и экономическая жизни были просты. И вот судьи стали с собой таскать модельные решения, так формировалась база модельных прецедентов. Это такие решения уже сделанные, используемые просто для скорости и удобства. К XIV веку накопилось очень-очень много прецедентов, начали проявляться недостатки метода. Удобно было: скорость, удобство, предсказуемость (если право нарушено, знаем точно, как его восстановят). Недостатки были связаны с некоторой косностью и неуклюжестью, отсутствием адаптивности. Существуя в пределах прошедших прецедентов, нижестоящие суды скованы в имеющейся замкнутой системе источников. Они не могли создавать новые прецеденты так просто. Нельзя гибко истолковать норму и применить к ситуации. Потом стали появляться новые социальные классы, новые сословия, новые социальные отношения. И когда они шли в суды за поиском прецедентов, разумеется ничего не находили, потому что непонятно кто они вообще. Решалась эта проблема в XIV веке точно так же, как у нас решается все, когда непонятно что происходит. Обращаются к президенту. А тогда в Англии обращались к королю. Все эти недовольные граждане или даже коллективы, организации стали писать жалобы королю. Письма стали накапливаться, причем среди писем были не только левые неважные люди, но и финансовая элита типа купечества. Королю надо было как-то разбираться, он поручил это той должности из их кабинетов, типа секретного, внутреннего и т.д., которая стала как раз из-за этого почетной и крутой, ключевой и известной, голосом короля. Это лорд канцлер. С той поры эта должность возвышалась, она и прежде была заметной, но тогда и большее могущество появилось. Лорд канцлер вот что делал. К нему обращались с теми проблемами, что у них там несправедливо разрешили спор. Он выносил решение на основе собственного чувства справедливости, как он считал нужным, так он и решал. Потом канцлер ввел аппарат, который назвали канцлерским судом, и используемое там стало называться правом справедливости. Система внутри себя стала формировать параллельную прецедентную систему. И они потом просто объединились. Право справедливости как часть прецедентного права существует до сих пор. Из чего состоит прецедент. Мотивировочная часть и результивная. Мотивировочная часть это размышление о праве. Результивная часть это решение по итогам анализа обстоятельств и норм. Прецедент и современная российская правовая система не соотносятся, в России нет. У нас есть слова ``судья подчиняется закону'', нет места прецеденту. Но есть основная проблема у нас -- отсутствие устойчивости и единообразия. Нет предсказуемости. И поэтому пропадает доверие к правосудию. Как этот вопрос пытается решить судебная система? Основная выравнивающая вещь -- разъяснения пленумов высших судов о нормах. Пленум это высший орган суда, среди обязанностей которого есть то разъяснение. Это акты судов, но ключевое отличие в том, что это абстрактное разъяснение и толкование, как лучше использовать законы. Подобные инструменты были даже в Советской системе. Еще постановления и решения конституционного суда. Эти решения по сути содержат нормы права. Как источник права. Обязательны для исполнения всеми после принятия. Проверка конституционности нормативных актов один из примеров деятельности. По вновь открывшимся обстоятельствам. Если вдруг суд нижестоящий вынес решение по чему-то, и потом через время верховный суд вынесет по аналогичному делу с другим исходом, это основание для пересмотра старого дела. У нас такое связано с налогообложением. У нас вряд ли будет система прецедентов, потому что много почему. В Англии есть принцип абсолютной состязательности, а у нас она ограничена. Нас судья должен установить объективную истину. И если не получается, начинается волюнтаризм. Судья сталкивается один на один со своим усмотрением. Чувство справедливости развитое есть очень далеко не у всех. Сейчас предлагается другой подход в Америке. Судья имеет вообще другую цель, не пытается достичь объективной истины. Судья решает кто прав кто виноват, кто смог убедить судью. Судья это не исследователь, который ищет истину, а арбитр. Это состязательная система. 


\section{Правовое регулирование и реализация права}
Появляется норма, начинает регулировать какие-то отношения. Что значит регулировать. Нужно вернуться к общеобязательности нормы, принуждение и страх перед наказанием это не единственная и даже не основная причина соблюдения нормы. Право не регулирует отношение в смысле трансформирования и врастания в него, оно не подминает отношению. Детерминация не про регулирование. Наше поведение так просто не детерминируется нормами. Регулирование это не детерминация. Регулирование это оценка, норма это ориентир, модель, критерий оценивания. Когда мы при помощи нормы регулируем, оцениваем, какие варианты оценки есть? Позитивная, нейтральная и негативная. Если правовая норма находится в связи с отношением и оценивает произошедшее позитивно, последствия могут быть: какие-то поощрения, награды, льготы. Например если мы помогаем в задержании преступника и вдруг он ломает руку, учитываются дополнительные права в этих обстоятельствах, вас не накажут за перелом. Если право оценивает отрицательно, есть возможность привлечь к ответственности. Две модальности. Еще есть не регулируемое. 

\section{Конституционное право}
Конституционное право -- отрасль, регулирующая основы конституционного строя и правового статуса человека и гражданина, федеративное устройство, систему органов местного самоуправления, общий порядок их образования, функционирования и взаимодействия. Метод регулировки в этой отрасли -- императивный, то есть это обязательные предписания и субординация, основанная на вертикальности отношений. 

Конституция РФ -- верховный закон. Признаки ее:
\begin{itemize}
\item Верховенство,
\item Высшая юридическая сила,
\item Прямое действие,
\item Является базой действующего законодательства,
\item Учредительный характер,
\item Стабильность.
\end{itemize}
Источники:
\begin{itemize}
\item Акты, изменяющие конституцию (законы о поправках), 
\item Федеральные конституционные законы (если наличие законов прямо указано в конституции), 
\item Федеральные законы, 
\item Международные законы, 
\item Конституции республик и уставы других субъектов, 
\item Акты конституционного суда РФ. 
\end{itemize}

\subsection{Глава 1. Основы конституции}
Основы конституционного строя. Государственный строй -- совокупность характеристик, определяющих существование государства, связанных с его структурой, территориальной организацией, системой государственных органов, правами, свободами и обязанностями человека, политическим режимом и т.д. Конституционный строй -- ключевые характеристики или принципы гос. строя, закрепленные в конституции. Государственный строй более широк. В России ключевые элементы конституционного стоя закреплены в первой главе: 
\begin{itemize}
\item республиканская форма правления, 
\item приоритет прав и свобод человека, 
\item народовластие, 
\item федерализм, 
\item разделение властей на законодательную, исполнительную и судебную, 
\item социальность государства, 
\item идеологическое многообразие и политический плюрализм, 
\item свобода хозяйствования и многообразие форм собственности
\end{itemize}

\subsection{Глава 2. Права граждан}
Конституционно правовой статус личности закреплен во второй главе конституции. Основные элементы: гражданство, правосубъектность, основные права, свободы и обязанности человека и гражданина, принципы правового статуса личности, гарантии статуса личности. Далее в частности. Гражданство -- политико-правовая связь между человеком и конкретным государством. Гражданин обладает максимальным набором прав и свобод, предоставляемых данным государством (в том числе политического характера). Основные способы приобретения гражданства: по рождению (по родителям ``право крови'', по месту рождения ``право почвы''), натурализация (вступление в гражданство по личному желанию, в общем и упрощенном порядках). Правосубъектность -- способность лица выступать участником правоотношений (т.е. урегулированных правовыми нормами). Правосубъектность делится на правоспособность и дееспособность. Правоспособность -- неотчуждимая с момента рождения и до смерти потенциальная возможность обладать правами и нести обязанности. Дееспособность -- способность гражданина реализовать права и обязанности своими действиями. Гражданин РФ может самостоятельно осуществлять в полном объеме свои права и обязанности, то есть обладает дееспособностью, с 18 лет. Основные права, свободы и обязанности. Права
\begin{itemize}
\item личные (гражданские): на жизнь, свободу, личную неприкосновенность и др.
\item политические: избирать и быть избранным и др.
\item экономические: право собственности, право заниматься предпринимательством и др. 
\item социальные: свобода труда, образование, и др.
\item культурные: право на доступ к музейным фондам и др. 
\item экологические: право на благополучную окружающую среду и др. 
\end{itemize}
Обязанности -- соблюдение конституции и законов, сохранение окружающей среды, уплата налогов и так далее. 

Принципы правового статуса личности: признание прав и свобод человека и гражданина, гарантированность, неотъемлемый и неотчуждаемый характер, равенство, непосредственное действие. Гарантии прав и свобод. Общие: социально\!\! -\!\! экономические, политические, организационно-технические и др. Специальные юридические: общее право на судебную защиту, система прокурорского надзора, право на возмещение вреда, причиненного государством и его представителями, право на обжалование незаконных действий государства и его представителей и др. 

\subsection{Глава 3. Федеративное устройство}
Обсудим федеративное устройство РФ. Согласно Конституции, Россия -- симметричная федерация, субъекты ее равноправны. Суть федерации вообще -- в объединении государственноподобных образований (но это не стоит путать с регионалистскими унитарными государствами, где некоторым частям государства даются большие полномочия). Субъектами РФ являются государства (республики), национально \!\!-\!\! территориальные образования (автономные области и автономные округа), территориальные образования (края, области, города федерального значения). Некоторые отличительные характеристики республик не дают им особый правовой статус по сравнению с другими субъектами РФ. Территорией России считается совокупность территорий входящих в Россию субъектов. 

В поправке вводится возможность создания территорий, подчиняющихся напрямую федерации. 

Единство федерации выражается в нескольких вещах. Первая -- единый суверенитет и полнота государственной власти над всей территорией Федерации, реализуемая единой системой федеральных органов (президент, правительство, высшие суды и др.). Вторая -- единая правовая, экономическая (в частности, кредитно-денежная) системы. Третья -- единое гражданство, вооруженные силы, государственный язык и иные государственные атрибуты (символика). 

В поправках вносятся новые единые ценности = идеалы, упоминание о русском народе в качестве ``государствообразующего'', уточнение и расширение компетенции федерации (как регулирование информационной системы, системы воспитания и образования). 

Основные принципы функционирования федерации 
\begin{itemize}
\item государственная целостность (единство территории и суверенитета, отсутствие права субъектов на выход),
\item равноправие и самоопределение народов России (признание и поддержка национального многообразия, возможность образования автономии в виде автономной области или округа и т.д.),
\item равноправие субъектов федерации (у отдельных субъектов нет привилегий и преимуществ, особенно в их отношениях с федерацией),
\item единство системы государственной власти (даже на уровне системы органов власти субъекта -- основы этой системы установлены федеральным законом), 
\item разграничение предметов ведения и полномочий между федерацией и субъектами (федеральная компетенция -- ст. 71, совместная -- ст. 72, субъектов -- ст. 73, компетенция субъектов основана на остаточном принципе, к ней относятся вопросы, не отнесенные к федеральной и совместной). 
\end{itemize}

\subsection{Глава 4. Президент}
Президент РФ -- гос. орган, не включенный ни в одной из трех ветвей власти. Он является гарантом Конституции, координирует функционирование гос. органов. Ключевым полномочием, определяющим значение Президента в жизни государства и общества, является ``определение основных направлений внутренней и внешней политики государства''. 

Что нового в поправках? 
Президент теперь поддерживает гражданский мир и согласие в стране, координирует деятельность органов всей публичной власти (не только государственной, но и местного самоуправления). 
Ужесточены требования к кандидатам (25 лет проживания в РФ вместо 10, отсутствие иностранного гражданства итд). 
Исключено слово подряд из нормы об ограничении количества сроков, введена специальная оговорка о неприменении ограничения количества сроков к лицу, занимавшему должность президента на момент вступления в силу данной поправки (обнуление). 

Значительная часть полномочий, определяющих взаимоотношения президента и других органов гос власти, сосредоточена в ст. 83 конституции (назначение председателя правительства, федеральных министров, судей и др.). 

В поправках. Кандидатуры председателя правительства и несиловых федеральных министров -- предварительно утверждаются Гос Думой. 
Президент осуществляет общее руководство правительством (ранее была норма только о председательствовании на заседаниях). 
Президент имеет право вносить представления в совет федерации о прекращении полномочий судей высших и некоторых других судов в связи с совершением, в частности, порочащих поступков. 
Президент формирует гос совет -- новый гос орган, также уточнены функции совета безопасности. 

Нахождение президента вне трех ветвей власти означает неприменение в его отношении классических сдержек и противовесов, единственным ограничением можно считать процедуру импичмента. В поправках новое: неприкосновенность бывших президентов. 

\subsection{Глава 5. Федеральное собрание}
Федеральное собрание -- двухпалатный представительный и законодательный орган. Постоянно действующий орган, палаты функционируют раздельно, за исключением случаев ст. 100 ч. 3 конституции. Представительность означает выборность нижней палаты (гос думы) населением государства и учет интересов субъектов федерации в верхней палате (совет федерации). 

В предстоящих поправках. 
Совет федерации будет состоять из сенаторов, ранее были просто члены совета федерации; среди сенаторов: по два представителя каждого субъекта федерации (от законодательного и исполнительного органов), бывшие президенты РФ пожизненно, сенаторы РФ (не более 30, назначаемые президентом, их них 7 можно пожизненно). 
Ужесточены требования к кандидатам в депутаты. 
Появился институт парламентского контроля. 
Усилилось влияние президента на законодательную власть -- роспуск думы возможен президентом не только по основаниям, предусмотренным ст. 1111 (несогласие думы с кандидатурой председателя правительства) и ст. 117 (вотум недоверия правительству), но и теперь в случае несогласия думы с более чем 1/3 кандидатур федеральных министров. 

Законодательных процесс:
\begin{enumerate}
\item Законодательная инициатива -- внесение законопроекта в гос думу. Это могут делать только упомянутые в ст. 104. 
\item Рассмотрение законопроекта в гос думе. Как правило, в трех чтениях: в первом принятие концепции, значения и целей проекта; во втором внесение поправок; в третьем окончательное голосование за проект в целом. 
\item Принятие закона гос думой -- либо простым большинством голосов, либо 2/3 голосов (для ФКЗ и поправок к конституции). После принятия в третьем чтении законопроект становится законом. 
\item Одобрение советом федерации -- закон одобрен, если более половины сенаторов за или же если в течение 14 дней закон так и не был рассмотрен СФ. 3/4 необходимы для ФКЗ и поправок к конституции. Некоторые законы подлежат обязательному рассмотрению СФ, как например закон о бюджете. 
\item Подписание и обнародование президентом -- закон в течение 5 дней направляется президенту РФ для подписания и обнародования. В течение 14 дней президент либо подписывает федеральный закон и обнародует его, либо отклоняет его и направляет для повторного рассмотрения палатами федерального собрания (отлагательное право вето -- его можно продлевать повторным голосованием). 
\end{enumerate}
В поправках. В случае использования права вето после повторного голосования гос думы или совета федерации президент может отправить запрос в конституционный суд о проверке конституционности закона -- в случае неподтверждения его конституционности президент возвращает закон в гос думу без подписания. 

\subsection{Глава 6. Правительство}
Правительство -- высший орган исполнительной власти, состоящий из председателя правительства, заместителей его и федеральных министров Основные полномочия связаны с реализацией конституции и актов законодательной власти в рамках текущего управления государством. Правительство подотчетно и президенту, и гос думе. Президент может отправить правительство в отставку, дума -- выразить недоверие (вотум) правительству. 

В поправках.
Исполнительную власть будет осуществлять не правительство, а правительство под общим руководством президента, при этом подотчетность правительства парламенту на президента не распространяется. 
Ряд новых полномочий в статье 114, в том числе сохранение традиционных семейных ценностей, поддержка волонтерского движения и институтов гражданского общества в целом, экологические полномочия и др. 

\section{Гражданское право}
Гражданское право -- совокупность правовых норм, регулирующих имущественные и личные неимущественные отношения. 
Основная часть предмета -- имущественные горизонтальные отношения, то есть между частными лицами по поводу вещей, денег, прав на имущество и прочего. 
Основополагающие принципы построения таких отношений:
\begin{itemize}
\item Юридическое равенство -- одна сторона отношений не может принуждать другую сторону к чему-либо и пользоваться привилегиями, с которыми другая сторона заведомо несогласна.
\item Автономия воли -- каждая сторона свободно и по своему усмотрению определяет, вступать ли в отношения, с кем вступать, на каких условиях. 
\item Имущественная самостоятельность -- каждая сторона отношений, как правило, является обладателем какого-либо имущества, в связи с которым и вступает в отношения, на это имущество не могут посягать другие лица. 
\end{itemize}
Основной метод гражданского права -- дозволительный, также называемый методом координации. 
В отличие от конституционного права с преобладающим императивным (предписывающим) методом. Как правило, императивный метод в гражданско-правовом регулировании применяется в качестве исключения, например, для защиты слабой стороны в отношении. 

Источниками гражданского права являются гражданское законодательство:
\begin{itemize}
\item Гражданский кодекс РФ, 4 части, кодифицированный федеральный закон, 
\item Иные федеральные законы, к которым отсылает гражданский кодекс, например, федеральные законы о банкротстве, об актах гражданского состояния, об опеке и попечительстве, о государственной регистрации прав на недвижимое имущество. 
\end{itemize}
и неписаные источники, которые исходят не от государства, а от самих участников отношений:
\begin{itemize}
\item Обычаи, как упомянуто в статье 5 ГК РФ,
\item Деловые обыкновения, которые применяются по соглашению сторон применительно к их договорным отношениям,
\item Акты мягкого права -- результат научной деятельности по международной унификации источников гражданского права, применяются аналогично деловым обыкновениям; например, принципы европейского договорного права. 
\end{itemize}

Субъектами гражданского права являются граждане. Согласно главе 3 ГК РФ, 
Правосубъектность частных лиц (физических и юридических) -- способность участвовать в отношениях. урегулированных правом. 
Правосубъектность включает два элемента. 
Правоспособность гражданина -- способность иметь гражданские права и нести обязанности. 
Это неотчуждаемый, возникающий в момент рождения и завершающийся смертью потенциал участия в гражданско-правовых отношениях. 
Дееспособность гражданина -- способность своими действиями приобретать и осуществлять гражданские права, создавать для себя гражданские обязанности и исполнять их. 
Это реализация упомянутого потенциала. 
Дееспособность может быть 
\begin{itemize}
\item Полной (при достижении 18 лет, эмансипации или вступления в брак до 18 лет),
\item Частичной несовершеннолетних (14--18 лет),
\item Частичной малолетних (6--14 лет),
\item Ограниченной, 
\item Отсутствующей ввиду возраста (до 6 лет) или признания недееспособным.
\end{itemize}
Частичная дееспособность несовершеннолетних 14--18 и малолетних 6--14 предполагает, что определяются перечни сделок, которые дети могут совершать самостоятельно; 
сделки за пределами перечня совершают либо законные представители (6--14), либо дети с письменного согласия законных представителей (14--18);
имущественную ответственность за вред, причиненный детьми, по общему правилу, несут либо законные представители, либо сами дети (если постарше). 
Ограничение дееспособности и лишение дееспособности возможно лишь в судебном порядке. Основания ограничения: пристрастие к алкоголю, наркотикам или азартным играм, которое ставит семью в тяжелое материальное положение; психическое расстройство, в связи с которым гражданин может осознавать свои действия или руководить ими только при помощи других лиц. 
Гражданам с ограниченной дееспособностью назначается попечитель, последствия в целом близки к частичной дееспособности детей. Дееспособность может быть восстановлена в случае исчезновения оснований. 
Основания признания человека недееспособным -- только психологическое расстройство, в связи с которым гражданин не может осознавать свои действия или руководить ими. 

Юридическое лицо -- организация, которая имеют обособленное имущество и отвечает им по своим обязательствам, может от своего имени приобретать и осуществлять гражданские права и нести гражданские обязанности, быть истцом и ответчиком в суде. 
К признакам юридического лица относятся организационное единство (внутренняя структура), имущественная обособленность (для создания юридического лица необходимо имущество), самостоятельная имущественная ответственность (обычно участники юр. лица не несут ответственность за его действия и наоборот), фирменное наименование. 
Для юридических лиц правосубъектность не разделена на правоспособность и дееспособность, а возникает в момент сразу в момент государственной регистрации. 
Гос. регистрация -- завершающий этап процесса создания юр. лица, который также предполагает единогласное решение участников, в котором указывается, помимо прочего, сведения об имуществе лица, органах управления. 
В отличие от физ. лица, юр. лицо может быть реорганизовано или ликвидировано. 
Ликвидация юр. лица осуществляется по решению учредителей, органов управления юр. лица или по решению суда. 
Разновидностью ликвидации является признание арбитражным судом юр. лица несостоятельным. 
Основные классификации юр. лиц происходят по критерию основной цели и вида деятельности и по критерию членства. 
В первом случае говорят о коммерческих организациях (цель извлечь прибыль) и некоммерческих (общественно-полезные цели).
Во втором случае говорят о корпоративных (имеющих членство, участников) и унитарных (чьи учредители не становятся участниками). 
Унитарными юр. лицами являются государственные и муниципальные унитарные предприятия, фонды, учреждения 

Объекты гражданского права -- материальные и нематериальные явления (блага), по поводу которых складываются имущественные и неимущественные отношения, способные удовлетворять потребности частных лиц и составляющие предмет гражданского права. 
Круг объектов закреплен в ГК РФ:
\begin{itemize}
\item Имущество -- вещи, в т.ч. наличные деньги и документальные ценные бумаги, иное имущество, в том числе имущественные права (бездокументальные ценные бумаги, безналичные деньги, цифровые права),
\item Результаты работ и оказание услуг,
\item Интеллектуальная собственность -- произведения, изобретения и прочее,
\item Нематериальные блага -- деловая репутация, честь и достоинство и прочее. 
\end{itemize}

Вещи -- ключевой объект гражданского права. 
Вещь -- материальная субстанция (в том числе живая), обособленная и с определенными границами в пространстве, являющаяся результатом человеческого труда. 
Вещи бывают движимые и недвижимые (неотделимые от земли без разрушения), родовые и индивидуальные, в обороте свободные и ограниченные, потребляемые и непотребляемые, делимые и неделимые, главные и принадлежности, простые и сложные (составные), капиталы и доходы. 
Не все объекты одинаково участвуют в экономическом обороте, урегулированном правом. Существуют полностью изъятые из оборота (как ядерное оружие) и ограниченные в обороте. 

Представительство существует в двух аспектах: предоставление одним лицом другому лицу полномочия, то есть возможности действовать от имени, в интересах и за счет представляемого; деятельность представителя по реализации полномочия. 
Сделка, совершенная представителем от имени представляемого в рамках полномочия, непосредственно создает, изменяет и прекращает гражданские права и обязанности представляемого. 
Представительство иногда поэтому называют расширением юридической личности представляемого. 
Представительство бывает договорное (добровольное, в силу сделки или решения собрания), должностное (специальный административный неправовой акт), законное (прямое указание закона, как у опекунов). В добровольном случае полномочия представителя удостоверяются доверенностью. 

Доверенность -- письменный документ, представляющий конкретные полномочия представителю, выход за пределы полномочий влечет за собой, по общему правилу, приобретение прав и обязанностей самим представителем. 
Нотариальное удостоверение доверенности не требуется, за исключением предусмотренных законом случаев. 
Отсутствие в доверенности даты ее совершения обесценивает ее. 
При отсутствии конкретного срока длительности -- он 1 год. 
В доверенности может быть предусмотрено право передоверия, то есть переложения представителем полномочий на другого. 

Право собственности -- институт гражданского права, то есть совокупность норм, регулирующих отношения экономического обладания и принадлежности вещей участникам оборота. 
Одновременно право собственности -- субъективное конституционное право, одно из базовых. 
Одновременно еще это одно из так называемых вещных прав, устанавливаемых в отношении вещей, а не проистекающих из отношений между субъектами, для которых право называется обязательным 
Объект права собственности -- всегда индивидуально определенная вещь.
Право собственности -- абсолютное, то есть в отличие от относительных, ему соответствуют обязанности не конкретного лица, а неопределенного круга лиц, всех, кто может взаимодействовать с данной вещью. И собственник вещи может использовать правовые средства защиты против любого лица, посягающего на его вещь. 
Право собственности включает владение, пользование, распоряжение (юридическая судьба). Собственник может осуществлять это, если не запрещает закон. 
Закон может запрещать, если человек обязан, например, ухаживать за вещью. 

Два способа приобретения собственности: первоначальный и производный. Первоначальный включает создание новой вещи, переработку, находку. Производный подразумевает правопреемство. 
В случае отчуждения вещи право собственности возникает у приобретателя в момент передачи вещи. Если отчуждение происходит с гос регистрацией, регистрация и есть момент получения. 
Защищать право собственности можно по специальным вещным искам. Два основных из них: виндикационный и негаторный. 
Виндикационный это  иск невладеющего собственника к владеющему несобственнику. 
Негаторный это все, что не связано с лишением владения. 





\end{document}
