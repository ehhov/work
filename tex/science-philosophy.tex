\documentclass[a4paper, 12pt]{article}

% Configuration {{{
\usepackage[utf8]{inputenc}
\usepackage[T2A]{fontenc} % T1 for English
\usepackage[english, russian]{babel}

\usepackage{enumitem}
\setlist{nolistsep}
\usepackage{mathtools}
\usepackage{xcolor}
\definecolor{dimblue}{HTML}{1010aa}
\usepackage[
  colorlinks=true,
  allcolors=dimblue
]{hyperref}
\usepackage[
  vmargin=1in,
  hmargin=1in
]{geometry}
\linespread{1.3}
\usepackage{indentfirst}
\usepackage{graphicx}
\usepackage{tikz}
\usepackage[multidot]{grffile}
\usepackage[labelsep=period]{caption}
\usepackage{multirow}

%\usepackage{times} % for English

\def\-{\babelhyphen{hard}}
% }}}

\begin{document}

% Title page. Contents {{{
\thispagestyle{empty}

\null\vfill

\begin{center}
  \begin{Large}
    История и философия науки
  \end{Large}

  Владимир Анатольевич Яковлев

  \href{mailto:goroda460@yandex.ru}{goroda460@yandex.ru}?

  И. О. Эрекаев

  \href{mailto:perova.n@physics.msu.ru}{perova.n@physics.msu.ru}
\end{center}

\vfill

Литература, прочая информация
\begin{itemize}
  \item Степин В.С. Философия науки: общие проблемы. М., 2006.
  \item Современные философские проблемы естественных, технических 
    и социально-гуманитарных наук / Под ред. В.В.~Миронова. М., 2006.
  \item Энциклопедия эпистемологии и философии науки / Под ред. 
    И.Т.~Касавина. М., 2009.
  \item Кузнецова Н.И. Проблема возникновения науки // Философия 
    и методология науки / Под ред. В.И.~Купцова. М., 1996. Гл.~2 
    (С.~38--56).
  \item Кузнецова Н.И. Статус и проблемы истории науки // Философия 
    и методология науки / Под ред. В.И.~Купцова. М., 1996. Гл.~15 
    (С.~333--361).
\end{itemize}
\hrulefill
\begin{itemize}
  \item Зотов А.Ф. Современная западная философия.
  \item История философии / Под ред. В.В.Васильева, А.А.Кротова, Д.В.Бугая
  \item \url{http://khmelevskaya.tilda.ws/}
  \item \url{http://filosof.historic.ru/}
\end{itemize}

\clearpage
\tableofcontents
\clearpage
%}}}

\section{Первая лекция. Выяснить тему}
%{{{

\hfill \textbf{Oct 13}

Достать у кого-нибудь

Общая картина того, что понимается под наукой. В самой науке 
прагматические определения, социокультурный контекст, матрица бинарных 
противоположных оппозиций, только в таком контексте можно говорить 
о естественной науке. Общепринятого определения нет, в разных смыслах, 
контекстах прочее.

Культурное творчество, одна из духовных что-то. Философия как рефлексия 
над основаниями науки. Модели генезиса науки. Те же модели, что 
и появления философии в древней Греции, натурфилософии, первая наука как 
раз натурфилософия. Модели ее. Далее более изощренные подходы -- 
изохронный подход, исторический, в истории разные были представления 
и определения. Классификационная сетка философских картин мира, научные 
картины мира, начиная от Аристотеля и до современной квантово 
реляционной. Критический рационализм как основной метод науки, основные 
теории истины.

\hfill \textbf{Oct 20}

Сцеинтизм и антисцеинтизм. Сцеинтизм -- представление о том, что 
прогресс общества основывается на прогрессе науки, идея высказана франц 
рационалистами, царство разума на земле только с помощью науки. Это были 
декларации. Если рассуждать социалогически, научная деятельность вплоть 
до 20 века была не более, чем эпифеноменом (могла быть, а могла и не 
быть), и могла оказывать лишь незначительное воздействие на 
установившиеся образцы других видов социальной деятельности. 
Действительно, только в конце 19 века появляются плоды науки -- Карно 
КПД, радио, электричество -- первые эпохальные изменения благодаря науки 
-- спустя 300 лет. В этот же период франц националистов была белая 
ворона -- Жан Жак Руссо -- первым предостерег от развития прогрессивного 
общества от науки. Знаменитая работа от академии наук единственный 
ответил нет. Все науки происходят от человеческих пороков: физика 
любопытство, математика и геометрия из скупости, астрономия от 
астрологии, риторика из песен, этика тоже из стремления показать какой 
ты светский человек. Исторический пример его соответствует реальности -- 
там, где наука развивалась сильнее всего: Греция, затем Рим, постепенно 
кончились. Грецию Рим завоевал, Рим потом вообще... Так что Руссо первым 
сказал, что развитие науки может привести к негативным последствиям. 
Много кто считает, что развитие науки только умножает проблемы 
человечества, обостряет их, и нужно всегда ограничивать развитие науки 
в этом смысле с помощью законодательства. Экспертизу научных проектов не 
должны проводить только лишь ученые, нужны еще специалисты по праву, по 
этике, культурологии, политики. Сейчас уже много таких ограничений, типа 
клонирование человека, разработка хим оружия. Конечно это не панацея, 
кто-то нарушает, но их клеймят. Это антисцеинтизм, который сейчас имеет 
тоже большое значение.

Дальше синхронный подход к определению науки. Некий срез, какой бы этап 
развития науки мы ни взяли, должны видеть что-то инвариантное. Выделяют 
следующие направления анализа науки: малая наука (по значимости для 
общества, до середины 20 столетия), большая наука (с появления уже 
атомных устройств, как мирных, так и военных, с Манхеттенского проекта. 
Вторая классификация в отношении синхронных подходов -- номопитические 
науки или генерализирующие или диаграфические. Стремление обобщать, 
абстрагировать, прежде всего естествознание, и диаграфические? про 
обществознание. Естественные, гуманитарно\-социальные, когнитивные, на 
западе под science прежде всего имеют в виду блок естествознания, 
физика, химия, биология, а другое говорят там social science и т.д. 
Другая градация тоже к синхронному -- фундаментальные и прикладные. 
Начиная с Бэкона, который ввел светоносные опыты и плодоносные. Сейчас 
прикладные получили четкую аббревиатуру -- NBICCS -- nano bio 
informational cosmic cognitive (в т.ч. искусственный интеллект) social. 
Социальное в смысле технологий, управление обществом, организация 
всякого -- дело рук социальных технологов. Это прикладные науки, а еще 
одна их особенность -- они конвертируемые, конвергентные?, они опираются 
друг на друга, дополняют друг друга. Дальше это не кончается, потому что 
технологии переходят в технические или опытно\-конструкторские, и один 
еще новый блок наук появился, который трудно отнести к фундаментальным 
или прикладным -- мониторинговый, ПДК предельно допустимые концентрации 
веществ, станции всякие на земле, в космосе, отслеживают то, что только 
с помощью науки, предсказание землетрясений итд, хотя и только 
мониторинг (пока что). Заодно подчеркнем специфику соц наук. Быстро 
преходящий характер исследований, не сравнишь с объектами 
в естествознании. Стат методы исследований. Социально\-моральные 
ограничения или морально\-этические (тоже плавающая вещь). В логическом 
плане можно на каждом этапе развития науки выделить три важнейших блока. 
Авангард научных исследований где рождаются идеи, фундаментальные или 
прикладные не важно, где спорят, где истина вообще отсутствует, обмен 
мнениями только, острие науки. Твердое ядро науки 2 блок, то, что 
пишется в учебниках, учебники переписываются, но не так быстро, люди 
учатся по учебникам по тому, что считается апробированным, держащим 
науку, тут корпус знаний, методик, оценка результатов и прочее. 
И наконец история науки, то, что раньше считалось наукой, но сейчас не 
считается, большое спасибо, что были такие люди в науке, но авторитет 
носит исторический характер. Но надо знать историю науки, хотя бы чтобы 
не изобретать велосипеды. Итак, это три значительных блока.

Выделим отличительные черты современного понимания науки:
\begin{enumerate}
  \item Специфический вид деятельности, теоретический, 
    экспериментальный. Конечный продукт науки всегда объективное знание 
    -- то, которое не зависит ни от человека, ни от человечество, 
    и соответствует корреспондентскому пониманию истины. По характеру 
    объективное знание может быть фундаментальное прикладное и т.д.
  \item Тип организации людей, которые называются научным сообществом. 
    Начиная с Пифагорейского союза, который был полурелигиозной 
    организацией, академией Платона, Александрийская школа, древние 
    универы, наши универы, академии, которые появляются с средневековья 
    и ренессанса. Научно\-исследовательские институты, центры -- тип 
    организаций, где развивается наука.
  \item Дисциплинарные и формы организации науки. Первое дисциплинарное, 
    но сейчас организация, сейчас говорят нужно много разных ученых, 
    например в космосе вообще все собраны, куча специальностей, чтобы 
    выйти в космос, а потом работать. Или лекарства, где химики, 
    биологи.
  \item Правовые и морально\-этические регулятивы научной деятельности. 
    Сначала наука была плодом самоорганизации, хотя и там были правила, 
    что наука не вмешивается в полит дела общества. Сейчас конечно 
    научная деятельность регламентирована, многие проблемы табуированы 
    на законодательном уровне, деятельность ученого регламентирована 
    трудовыми кодексами.
  \item Морально\-этические регулятивы, воспитание, за всем не уследишь, 
    каждый ученый руководствуется сознанием и т.д. Есть понятие этос 
    науки. Этнос не путать. Этос это особенности, связанные с моралью, 
    этикой. Имплицитно он всегда присутствовал, и главным в этике 
    ученого считался поиск истины. Роберт Мертон написал кодекс ученого. 
    Понимается как набор морально\-этических норм с целью повысить 
    эффективность труда ученых. Труд ученого конечно благородный, на 
    пользу обществу в целом, но лучше, если ученый будет сознательно 
    руководствоваться некими нормами, которые не пропишешь в кодексе, но 
    морально\-этический должен быть.
    \begin{enumerate}
      \item Универсализм. Плоды должны быть доступны всем. Лейбниц 
        написал писем больше чем своих трудов, был такой способ обмена 
        информацией. В античности не было, поэтому ораторы, школы. 
        А сейчас это очевидная вещь, через интернет можно что угодно 
        обмениваться как угодно, хотя и есть ограничения в виде уровней 
        секретности, допуска к информации, те же статьи не сразу 
        появляются, но норма -- то, к чему надо стремиться.
      \item Коллективизм. Признание того, что наука всегда коллективный 
        труд, и если отметили чем-то, грант дали, все равно нужно 
        понимать, что ты обязан большому количеству людей. 
        И предшественникам, и коллегам
      \item Бескорыстность. Несмотря на все трудности, ведущим мотивом 
        должен быть поиск истины, и у нас есть примеры даже, когда люди, 
        несмотря на пертурбации в обществе, занимались своим делом. Тот 
        же Лаплас что только ни пережил, и монархию, и революцию 
        французскую, и Наполеона, и после Наполеона, и везде оставался 
        Лапласом. И наш Александр Фрезин, который поправил уравнения 
        Эйнштейна, тоже жил в эпоху революции, скончался от болезни, но 
        работал не покладая рук. Таких примеров много.
      \item Организованный скептицизм. Морально\-этическая норма, 
        позволяющая отсеивать лжеученых, лженауку с помощью 
        профессионально отобранных фильтров. Научное общество всегда 
        внутри себя решает. Идеи принимаются или не принимаются, но 
        всегда обсуждаются. Всегда скепсис по отношению к новому, чтобы 
        лженаука не проникала. Хотя конечно всякое бывает.
    \end{enumerate}
  \item Интердисциплинарность. Науку исследует огромное количество 
    других наук. Что такое наука. С разных сторон. В нашей стране есть 
    академический институт истории естествознания и техники. Там 
    работают в том числе ученые, закончившие естественные направления. 
    Науковедение. Конкретные, основанные на документах, эпизоды развития 
    науки. Журнал выходит под их именем. Целая серия была посвящена 
    появлению атомного проекта в России.
  \item Социология науки. Изчение деятельности людей, которые 
    организованы в коллективы. Как становятся лидерами в науке, как 
    оценивается результативность, эффектиность, кпд деятельности 
    ученого, цитирования и так далее. Есть соответствующие журналы, где 
    все это прописывается. Макрология? как государство относится 
    к науке. Мертон или кто-то сказал, что нормы можно свести к двум. 
    Запрет на плагиат и подтасовку экспериментов. Тут ученый на своей 
    совести. Вернемся к социологии. Микросоциологические исследования 
    проводятся в лаборатории, вслепую, ученые в лабе не знают, что среди 
    них импостер социолог. Так и макросоциологические исследования. 
    Наука -- это бренд государства, количество нобелевских челов, 
    стоимость установок и т.д. В развитых странах где-то 4--6\% от ВВП 
    расходуется на науку. Это не считая частных пожертвований. В нашей 
    стране тоже 4 записаны, но в последнее время 1 выбивается (что?). 
    В целом число ученых в мире растет, тут тоже критерий 
    соответствующий, хотя бы первую научную степень получил, уже ученый, 
    раз какой-то вклад в науку внес, поскольку все эти работы должны 
    обладать новизной. В целом в мире распределение неровное. У нас до 
    начала перестройки цифры были сопоставимы с американскими, где-то 
    под 900 тыщ, в Америке сейчас идет рост, у нас меньше в 3 раза 
    стало. Наука у нас если не деградирует, то стагнирует. Продолжается 
    отток, ректор наш жаловался (бедный...). Как из материальных, так 
    и из технических соображений. Не так просто оборудовать современные 
    лаборатории. На этом фоне где-то публикационная активность на 
    мировом уровне низкая для нашей страны стала, называют где-то 2\% от 
    общего массива публикаций. Многие еще другие страны, которые раньше 
    даже близко не считались научными, как Бразилия, Китай, выходят 
    сейчас. Социологический подход тоже довольно объективный, можно 
    делать вывод для того, чтобы регулировать гос политику в отношении 
    этом. И кроме того называют экономикой науки, эргономикой науки.
  \item Психология. Психология научного творчества, коэффициенты 
    интеллектуальности, креативности, исследование жизни корифеев науки, 
    их эскапизм? как Эйнтшейн говорил, что ему бы пришлась жизнь 
    смотрителя маяка. Насколько коммуникативные возможности помогают 
    ученым или наоборот тормозят. Тоже активно занимаются люди наукой 
    и проблемами рождения идей, что способствует, что тормозит.
\end{enumerate}

Дальше раздел лекции: принципы классификации науки. Метод научные, что 
используют. Метанаучный метод -- наиболее абстрактные, научные методы. 
Механицизм был первый, идеал, мир как гигантская машина, часы, машина 
мунди. Органицизм прямо противоположный, представление, что мир -- живой 
организм, начиная от Платона, что космос -- одушевленный объект, 
и современные теории предлагают рассматривать нашу планету как единый 
живой организм. Раз есть живое, то влияет на абсолютно все. 
Элементаризм, что все можно в конце концов свести к элементарному. 
В физике сейчас кварки, в биологии сейчас уже составные части клетки. 
Прямо противоположное холизм -- целостное рассмотрение объекта, 
изначально предполагается, что система целая, нельзя свести 
к совокупности частей, целое всегда больше, особенно касается живых 
организмов. Французы даже придумали витальную силу. Общенаучные методы 
или междисциплинарные. Классические как индукция, дедукция, анализ, 
синтез, мысленный эксп, наблюдение, опыт. Современные как 
системно\-структуный (19 век появился), в 20 синегнетический подход, 
вычислительный эксперимент тоже стал, компьютерное моделирование. 
Проникает во все науки, даже гуманитарные, историю моделируют, битвы.

Принципы современной науки: не меняются, просто существуют. Как на 
основе них что-то развивается или не развивается. Самый первый 
мировоззренческий -- принцип объективности мира, устойчивости, 
изменчивости, существования законов, их познаваемости. Если взять 
принцип, можно заниматься наукой. Общенаучные принципы: сохранения, 
дополнительности, симметрии, причинности, инвариантности, много 
других..., которые для науки во всех ипостасях? считаются очевидными. 
Без них наука как деятельность невозможна. Частные принципы: для физики 
инерции, несепарабельности, линейности нелинейности, хз чего еще. 
С квантовой физикой тоже появилось всякое.

Итоги и осн направления развития науки. Отличия науки от остальных форм. 
\begin{enumerate}
  \item Стремление выйти в новые миры за сферу обыденного опыта.
  \item Концептуальный аппарат, базируется в современной науке на 
    математических дисциплинах, моделях, различных конструкциях.
  \item Методология принципиально отлична от других хз чего. Использует 
    все те принципы выше, причем методология отрефлексированная.
  \item Акцентирование внимания на проблемах как внутри развития науки, 
    так и в значимости для решения проблем социума. Современная наука 
    и то, и другое учитывает.
  \item Принципиально новая материально\-приборная база. Индустрия, уже 
    не циркуль и линейка. И индустрия тоже объект науки...
  \item Появление и бурное развитие техно науки.
  \item Глобализация и стандартизация на процессе подготовки кадров. 
    Наука готовит сама себе кадры. От школ (где учебники разрабатываются 
    академиками) и кончая современными университетами.
\end{enumerate}

Основные направления современной философии науки. Плюрализм, но что-то 
можем зафиксировать.
Кумулютявистская (аддитивная) установка. Оппозиция классической 
науки рационализм (Спиноза...) эмпиризм. Трансцендентализм vs 
что-то.
В современной науке есть ли. Нормативная философия науки 
(осмысливание, как фальсифицируемость). Дистрективность (Кун), идея 
парадигм, идея дискретного развития науки. Эволюционизм -- мб наиболее 
масштабное, прослеживается эволюция как когнетивных структур на уровне 
индивида. Онтогенез. А еще филогенез -- как в эволюционном плане 
сформировались когнитивные способности. Те же связи, правое и левое 
полушария и т.д. (Дэвид Кембол). Герминевтическое направление (Гадамор), 
что такое текст для ученого, что вкладывает автор, чем отличается 
научный текст от художественного текста. Ситуационный или ситуативные 
(case-studies), исследования эпизодов развития науки. Как на макро 
уровне, как появилась классическая механика Ньютона, а можно на микро 
уровне, как рождаются знания на уровне лаборатории. Микро уровень сейчас 
называют когнетивно социологические исследования. Авторы кстати считают, 
что мотивы не поиск истины, а деньги власть слава как и у всех. 
И недавно инновационно\-синегнетическое направление науки. Никита 
Моисеев, Морозов, Пригожин.

%}}}

\section{Философские парадигмы античной рациональности. Основные 
программы, категориальный аппарат что-то}
% {{{

Рациональность -- открываемое, что-то еще разумом. Рациональное 
противопоставляется иррациональному (Кьеркегор, Шопенгауэр, Ницше, 
Бергсон), где совсем не разум играет роль, а страх у Кьеркегора, воля 
к жизни у Шопенгауэра, воля к власти у Ницше, либидо у Фрейда.

В историческом аспекте, эволюция от мифов до современных научных теорий. 
В мифах появляется рациональность. Способ теоретической организации 
практического опыта, там пространство время, высшие силы и прочее. 
Социокультурные предпосылки рациональности: труды, показывающие, почему 
наука не появилась в Китае, потому что там другое представление 
о природе и о законах (только от Императора, в природе законов быть не 
может). Гносеологические подходы к рациональности связаны с пониманием 
элементов прежде всего языка, как устроены языки, в чем отличие от языка 
науки, у Бэкона это даже идол рынка, там разговаривают на обыденном 
языке, а наука должна на своем языке. И структура логики, и прочее. 
Дальше ключевые термины это понятия парадигмы и исследовательской 
программы. И то, и другое сейчас общекультурные понятия. Общекультурное 
можно услышать сейчас где попало, но старт этому дал Томас Кун в работе 
"структура научных революций", где ввел понятие как общенаучное. Хотя 
корни можно найти и у Платона, где для творения мироздания есть демиург 
как творитель, хора как материал и парадигма, образец для подражания. 
Кун это систематизировал, детализировал, а его понятие парадигмы -- 
признанные всеми научные достижения, которые в течение опрпделенного 
времени задают модель постановки проблем и их решения научному 
сообществу. Под влиянием критики дальше Кун сказал, что парадигма -- 
дисциплинарная матрица, включающая не только фундаментальное положение, 
но и философские принципы, на которых основано, понятия, методы, 
соответствующие приборно\-измерительные механизмы и аппараты, а также 
критерии оценки научных результатов. Главным потребителем парадигмы 
является научное сообщество. Наука это то, чем занимается научное 
общество, а научное сообщество занимается наукой, но парадигмально. 
Будем Куна подробнее изучать, сейчас просто введение.

\hfill \textbf{Oct 27}

На последней лекции закончили тему первую сцеинтизм и анти-, стремление 
построить царство разума, синхронный подход к анализу науки, 
конвергентные технологии, био нано информационные космические 
социальные, темп фундаментальной науки замедляется, а эти технологии 
наоборот. Важнейшие черты понимания науки, Вид деятельности, тип 
организации, продукт, техническая база. Роберт Мертон запрет на плагиат 
и подтасовку, запрет на проведение экспериментов, принципы 
классификации, принципы научных методов, принципы мировоззренческие. Ну 
и итоги определения науки, места науки в современном обществе, немного 
статистики по нашей науки, ее доля к сожалению на протяжении последних 
20 лет уменьшается, в общем тренде научных исследований. Уже место нашей 
науки девятое в мире, а недавно было шестое, а в конце 80-х второе 
и конкурировали только с Америкой. Доля наукоемких технологий в мире 
вообще 0.31\%... Не очень позитивная статистика. Вторая наша тема 
философские парадигмы и программы. Ввели понятие рациональности, 
история рациональности, начиная от мифологии, чем круче мифологи, тем 
больше шансов появиться науки. Современное понимание рациональности, 
прежде всего, логико-мат моделирование, комп моделирование. Кант 
говорил, что в науке ровно столько науки, сколько в ней математики. 
Этапы: парадигма (приобрело уже соц культурное значение даже), 
основатель понятия Платон, зазвучало у Куна, в конце концов матрица 
научных исследований, основана на философских предпосылках и кончается 
на оценке результатов, научных методах и прочем.

Представление о науке как о некотором процессе дискретных образований, 
которые меняют друг друга. Переход от одной парадигмы до другой это 
революция. Как происходит переход такой Кун затрудняется сказать. Для 
него вообще иррациональное это, но есть и другие подходы. Дальше 
научно-исследовательская программа, Локатос (Венгрия) и другое 
представление о характере развития науки -- конкуренция между 
программами, которые появляются и конкурируют между собой, а мы можем 
оценивать успешность в истории. В программе есть всегда твердое ядро, 
принимается конвенционально, это философские предпосылки (дискретность, 
причинность и подобное). Это ядро окружено  двумя поясами: негативная 
эвристика и позитивная эвристика. Негативный пояс призван принимать 
и отражать контр аргументировать все нападки на программу, искать контр 
аргументы, защищать ядро. Позитивная эвристика должна уметь делать 
предсказания новых явлений, что-то новое открывать. Об успешности 
исследовательской программы с точки зрения Локатоса судят на основе 
предсказаний чего-то нового, открытия чего-то нового. Классическое 
сравнение Ньютона и Декарта. Декарт мог объяснить все, что и Ньютон, но 
у Ньютона были предсказания, а у Декарта нет, несмотря на мгновенные 
силы, бесконечные расстояния и все такое. И второй пример это 
представление о свете: корпускулярная и волновая. Сначала успешной 
казалась корпускулярная (Ньютон, хотя и тогда были известны волновые 
свойства), а с конца 18 середины 19 века у Френеля волновая пошла. 
Кончилось понятное дело дуализмом в 20 веке с де Бройлем. Локатос 
говорит, что нет вечных истин и абсолютной победы какой-либо программы. 
Говорит, надо иметь интеллектуальную скромность. В этом смысл развития 
науки, не дискретность, а континуальность развития программ, защита или 
принцип методологического упорства.

Теперь подробно рассмотрим как в истории происходило. Главное здесь 
понять, что и исходные парадигмы, и исходные программы все зародились 
в Греции. До всего додумались и до сих пор все парадигмы и программы 
в ходу современной науки, развиваются, конкурируют и так далее. Сначала 
мировоззренческие парадигмы ниже 4. Представление об архэ их всех 
различает. В последней инстанции сейчас кварки.
\begin{itemize}
  \item фиксизм

    Бытие вечно, неизменно, неподвижно, неделимо и так далее. А все, что 
    мы видим, движение например, это видимое, а бытие вот такое. 
    Парменид это заявил впервые. Бытие есть а небытия нет, все, что мы 
    обозначаем. Ничто сущее не возникает и не уничтожается, это нам 
    только кажется. У Платона тоже бытие другого типа, но неизменно -- 
    мир идей. Количество их неизменно, они постоянны, а мир наш это лишь 
    эманация от идей. Платон сформулировал закон сохранения идеального 
    бытия, душ даже можно сказать, а у элеатов это закон сохранения 
    бытия. Первый закон сохранения у древних Греков... Надо сказать, что 
    современная наука имеет парадигму фиксизма, она была и у Ньютона, 
    у кого все ограничивалось солнечной системой. У Эйнштейна в ОТО тоже 
    настаивала на стационарном характере мироздания, и только после 
    Фридмана и неохотно Эйнштейн пересмотрел. В биологии мор растений 
    и животных раз и навсегда создан, Карл Линей тоже имеет 
    представление о неизменности. В физике тоже мечтают о теории всего, 
    и Хокинг, и Вайнберг. Она же не может изменяться, поняли и все, 
    дальше ехать некуда.

  \item циклизм

    Сменил фиксизм. Циклы -- ничего абсолютно постоянного нет, все 
    развивается, но циклами. Циклы сами можно понимать как угодно. 
    У Гегеля спираль и надстройка над предыдущим. а можно просто 
    замкнутые представлять. Первым сказал тоже Грек Гераклит со своим 
    мировым годом 10800 лет, архэ огонь, все возникает из огня. Этот 
    космос один и тот же для всех не создал никто и богов никто из 
    людей, но он всегда был, есть и будет вечно живой огонь, мерно 
    возгорающийся, вечно угасающий. У стоиков пневма это архэ состоит из 
    рождающих все вещи логос, бог, разум одно и то же. У Гераклита 
    все-таки после коллапса и сгорания в огне появляется новое, 
    а у Стоиков все совершенно по кругу. Все исчезает, но появляется 
    потом в том же виде, и снова Платон будет беседовать с Аристотелем. 
    Представление о судьбе как неизбежной вещи, жесточайшая 
    причинно-следственная связь, можно называть богом, природой, 
    судьбой, и вот Синека пишет: ``Не может быть природы без бога и бога 
    без природы. Захочешь назвать бога судьбой и не ошибешься. 
    Провидением тоже верно. Природой тоже верно, ибо от него все 
    рождается. Миром тоже можно...'' Бог природа судьба едины, Спиноза 
    потом тоже пример это.

    В современном это в геологии циклы образования Земли, в химии сейчас 
    открыли ячейки Бенара, реакции Белоусова-Жаботинского, 
    климатология тоже большие циклы, малые. Биология тоже цикличность 
    наблюдается, особенно в популяции микроорганизмов. В астробиологии 
    об этих циклах говорил и Чижевский про солнечную активность, 
    и Вернадский с его идеями ноосферы тоже.

  \item эволюционизм

    сейчас она является основной. Именно к ней тяготеет наука. Идет 
    опять от греков, от Импедокла, кто первый начал говорить, что 
    в живом мире существует эволюция, и не обязательно ссылаться на 
    бога, а вообще могли существовать и отдельные органы, и монстры, 
    а природа в результате конкурентной борьбы нашла и отобрала вот так. 
    Не кто-то отбирает сверху, а сама природа. Это основная идея. 
    В дальнейшем эта теория разрабатывается у французских материалистов, 
    Дидро повторяет идеи Импедокла, тоже появляются сначала 
    разнообразные организмы, у Ламарка тоже уже считается первой научной 
    эволюцией (почему у жирафа шея длинная), у Кювье теория катастроф, 
    периодически стирается все с Земли, а потом могут появиться 
    принципиально новые виды, как динозавры. И Дарвин конечно, сначала 
    в русле конкретной науки была, а потом была экстрапалирована на все 
    науки. Эрих Янч выпускает книгу самоорганизующаяся вселенная. 
    В науке утверждается идея глобальной эволюции как самоорганизации 
    форм как в биологии, так в и так далее. Недавно даже эволюционная 
    эпистемология, и в настоящее время наиболее активно разрабатывается. 
    Там тоже свои направления онтогенеза, филогенеза, участвуют 
    в разработке и философы, и ученые, но общим для всех подразделов 
    является Blind Variation Selective Retention -- слепой подбор 
    и селективное удержание. Универсально, можно к космосу приложить, 
    появление новых миров, теория хаотичной инфляции, можно применить 
    к Земле, а можно и к эпистемологии. Как Поппер скажет, механизм 
    познания инфузории и человека один и тот же. Эйнштейн тоже перебирал 
    много вариантов своих уравнений, пока не остановился на своем. 
    Инфузория тоже перебирает...

  \item креационизм (иногда креативизм)

    Тоже в античности зародилась, Анаксагор ввел идею мирового разума -- 
    нус, который и упорядочивает гомеомерии. Это структурные единицы 
    всего и вся, хотя все и вся можно делить до бесконечности, но 
    гомеомерия это структурная единица, а упорядочивание всего и вся, 
    почему что-то золото, что-то свинец, что-то живое и нет, решает все 
    это нус, оно упорядочивает природу и заставляет развиваться. Подобно 
    тому, как из крупиц составляется золото, так из подобия частных 
    телец образовалась вселенная, движущей причиной является ум, поэтому 
    золотом феноменально является то, в чем много золота, хотя при этом 
    в нем содержится все. Вот так он говорил... Идея все во всем 
    появилась у Анаксагора получается. То же самое у Платона. У него 
    есть демиург, есть парадигма, есть хора как материал, и сам Демиург 
    творит мир, космос, земную жизнь через богов Олимпийцев и самих 
    людей. В средневековье на какое-то время стала преобладающей 
    парадигмой, поскольку Августин говорил, что бог создал мир из 
    ничего, но в современном варианте креативизм. Илья Пригожин придал 
    новый импульс. Его философская парадигма опирается на определенную 
    науку термодинамики, энтропия возрастает, стрела времени, тепловая 
    смерть. И в космологии, начиная с Эдингтона появилась стрела 
    времени, а после большого взрыва эта стрела даже стала полноценно 
    присутствовать. Есть тоже мистические ходы мысли, представление 
    о каком-то там антропном принципе, но есть и просто как у Пригожина 
    идея самоорганизации, которая совсем не похожа на эволюцию. 
    Самоорганизация это от начала и до конца идет в определенном 
    направлении, хоть могут быть флуктуации. Пригожин пишет, я считаю 
    необходимым переосмыслить законы термодинамики запрета на обращение 
    времени как частный случай более общего, его всегда шокировало даже 
    что фундаментальные законы обратимы и детерминированы. Августин еще 
    сказал, что время это не кружение собаки ухватить свой хвост, 
    а стрела, пущенная богом. Есть точка отправления, есть точка 
    приземления -- страшный суд.

    В современной науке об этом начинают говорить не только на уровне 
    примеров, а об информационной матрице мироздания. С самого начала 
    развития вселенной предполагалось наличие наблюдателя. Принцип 
    данных в биологии, увидели некий парадокс времени, что если 
    опираться только на биологию, теорию Дарвина, то нельзя объяснить ту 
    скорость, с которой образовались высшие существа. Когда существа 
    появлялись, они не деградировали, а только развивались, превращались 
    в гомо сапиенса. Слепые мутации так бы не смогли. Значит что-то было 
    другое, это уже стали называть креативистской программой. Сторонники 
    даже существуют, говорят, что мы что-то не понимаем.
\end{itemize}
Дискретность, каждая новая парадигма является более высокой.

Теперь анализ науки и ее истории с точки зрения исследовательских 
программ, сам категориальный аппарат, который нарабатывается 
и в парадигмах, и в исследовательсих программах, а ключевые понятия, 
категории сформировались в античности. Ключевые это космос, хаос, логос, 
материя, потенциальное, актуальное, атомы, идеи, пространство, время, 
виды движения, энергия, причинность и многое другое. Маркс Ратовский 
сказал еще, что все эти понятия движения, материи, силы, поля, 
элементарные частицы, концептуальные структуры механицизма, атомизма, 
неизменности и изменения первоначально имели философскую природу, но 
оказали громадное влияние на построение науки и ее теор понятия. Наш 
академик Степин тоже считает, что именно развитие натурфилософии 
в Греции показало прагмастическую что-то там. Исследовательские 
программы это прогноз на развитие науки. Степин пишет, что сопоставление 
истории философии и истории естествознания показывает, что философия 
обладает прагмастическими чем-то, поскольку вырабатывает необходимые 
категориальные структуры. Все эти структуры и ключевые понятия оказались 
что-то там.

Теперь конкретно об исследовательских программах. Типы -- тематические, 
содержательного типа, далее методологические, далее дисциплинарные. 
Особенность в том, что не менее двух, конкурируют и остаются навсегда. 
Парадигма, особенно в первой парадигмы главным были поиски архэ, главным 
было появление исследовательских программ.
\begin{enumerate}
  \item Первые исследовательские программы это конкуренции склад по 
    линии того, что лежит в основании мира, материальное и идеальное. 
    С одной стороны Демокритовцы, доСократики, с другой стороны 
    пифагорейцы и Платон: что первично, материальное или идеальное. 
    Программы конкурируют и до сегодня в оппозиции. Что первично, физика 
    или математика, волновая функция объект физики или математики. 
    ПОДРОБНЕЕ Эвристический потенциал натурфилософских креативов 
    античности -- вестник МГУ, серия философии, номер 4, 2019 год.
  \item Вторая программа -- конкуренция по принципу понимания 
    структурности мироздания. Мир дискретный (Эпикур, Платон, Демокрит) 
    или континуальный (Аристотель, стоики).
  \item Третья пара -- механизмы взаимодействия и взаимосвязи, могут ли 
    взаимодействовать, не входя в контакт (дальнодействие) или пневма 
    заполняет мир, и следовательно импульс передается только от точки 
    к точке. В современной физике тоже обсуждается. Может ли 
    взаимодействие носить информационный характер (??) Телепортация. На 
    этом построен известный парадокс ЭПР.
  \item Четвертая тип бинарности -- принцип движения, как объяснить, 
    либо атрибут материи, либо все движется чем-то, и в конце концов 
    должен быть исходный импульс, а откуда берется уже есть варианты 
    типа большого взрыва.
  \item пятая программа -- понимание причинности. Или только в нашей 
    голове в конечном счете (гносеологическое понимание случайности), 
    или есть в самом мироздании. Классический пример это Демокрит 
    и Эпикур. У Эпикура клинамен например. Да и сейчас эти споры 
    продолжаются, недавно были споры Бора и Эйнштейна, Лапласовский 
    у нас детерминизм или индетерминизм (детерминизм вероятностного 
    типа)
  \item космоустройство. Или мы на земле, земля в центре вселенной 
    (большинство античных исследователей), или земля лишь одна из планет 
    солнечной системы, космос бесконечен (Демокрит и Эпикур) или 
    замкнут. В современности тоже полно вариантов: или мир единственный, 
    который знаем, или есть параллельные миры, мультиверсы и прочее.
  \item семь -- происхождение жизни. Спор ведется опять же с греков. Или 
    жизнь это атрибут материи, глухая чувствительность, спящие монады 
    как у Лейбница, или жизнь появляется спонтанно на определенном 
    этапе, или господь бог создает, или какие-то условия порождают, или 
    как у французских материалистов, а в античности это пантеизм 
    у натурфилософов большинства, представление о том, что ``и трупы 
    чувствуют''... Жизнь не исчезает и не появляется. Или у Аристотеля 
    уже разделение на царства, неорганика, растения, животные и человек. 
    Может ли жизнь зародиться из неживого или должна быть витальная 
    сила. Все мы состоим в конечном счете из атомов и молекул, но одно 
    живое, а другое нет.
\end{enumerate}
Тематические программы это. Конкуренцию видим, зародились давно, и до 
сих пор конкурируют. Вроде одни кажутся более убедительными, а потом 
другие.

Теперь методы, методологические программы. Самая главная -- формирование 
рефлексивного критического мышления. Метод критического рационализма по 
сравнению с догматическим обучением.%%%
Второе это отличие теоретического мышления от обыденного опыта: есть ли 
отличие или нет. Поставлена у Элейцев, у Зенона стрела летит, но 
попробуйте доказать. Эпистеме и докса, может ли докса противостоять 
мнению или должно опираться.%%%
Различение первичных и вторичных качеств. Наша сенсорика насколько точно 
дает инфу об окружающем мире. Есть ли искажения, и если есть, насколько 
сильные. Есть качества протяженность, плотность, несомненные, или же все 
первичные, или же первичных нет вообще.%%%
Диалектика как понимание метода исследования, как понимать диалпеткику. 
Как метод исследования самой природы как у Гераклита все движется, или 
только метод субъективной диалектики, маевтика Сократа, диалектика 
нужна, чтобы помочь родиться истине. Платон развивает тоже.%%%
Дальше программа теоретического моделирования. Здесь моделирование может 
быть как моделирование самого метода исследования, или это метод анализа 
полученных результатов. Можем ли смоделировать сам процесс творчества, 
чем занимался Платон, или можем анализировать только результаты 
творчества, чем занимался Аристотель. Модели изложения результатов его 
действительны даже сейчас.%%%
Методологическая программа классификации наук, что считать эмпирической, 
что теоретической. У Платона и Аристотеля разные. Плюс еще что 
практические... Своя классификация была у Бэкона. Долго держалась 
классификация Боэрса(??), тривий, куда входили грамматика, диалектика 
и риторика, и квадририи: арифметика, геометрия, астрономия, музыка.%%%
Дальше принципы различения гуманитарных и естественнонаучных знаний. 
Первыми задумались софисты, Протагор сказал, что человек мера всех 
вещей. Вот с того времени пошла дискуссия о том, насколько четко можно 
отличить гуманитарные.%%%
К прошлой программе примыкает скептицизм. Школа скептиков, Пирон, 
насколько можно доверять тому, что называет себя наукой.%%%
Также можно отметить программы соотнесения сущности и существования 
(номинализм и реализм), доказательства бога через существования, 
определение универсалий, известная бритва Оккама.%%%
Программа различения двух типов истин: религиозные и научные. 
Сформировалась на востоке, перешла в Европу, где превратилась в книгу 
сотворения и книгу творения (библию и книгу природы).

Дисциплинарные и образовательные программы. Опять из античности. 
Эллинизм, когда орудовала империя Александра Македонского, образовались 
царства, в Египте правили Птолемеи и образовался научный центр 
с конкретными научными исследованиями. Академия Платона -- образец теор 
мысли, а практическими программами занимались астрономы и куча всяких 
ребят еще, главный Птолемей... Медицина, география, история. Мусейон 
(??) видим четкое различение науки по дисциплинам. Конические сечения 
пригодились даже в 20 веке!...

Итоги подведем. Академик Байденков описывает предпосылки науки нового 
времени на основе развития этих парадигм, программ, конкретных дисциплин 
в античности. Была подготовлен целый ряд предпосылок науки нового 
времени. Такие понятия как пустота, бесконечность пр-ва, движения, 
устранение из объяснений даже живой природы теологического принципа, 
ограничение дальнодействующими причинами способствовало становлению. 
В средние века не так много было сделано, но важные наработки тоже. 
Теория Ипетуса (??), промежуточная между Аристотелем и принципом инерции 
Галилея, счеты и первая идея компа, анализ парадокса пустоты, как 
понимать ничто, может ли бог создать то, где его нет. И различные идеи 
магического ответвления науки. Имеется в виду алхимия, астрология, 
ядохимия. Новые элементы типа сера, ртуть, были открыты, с востока 
конечно пришли, но все равно. Главное отличие лженауки от науки в том, 
что во всех этих направлениях искался практический результат, а не теор 
знания. Период 1277 год средневековья, когда в Европе стало 
распространяться Аристотелевское учение, и парижский епископ выступил 
против него, вывесил 219 тезисов, где было несколько смелых, которые 
предвосхитили классическую науку. Первым сказал про прямолинейное 
равномерное движение и что существовать может много миров.

И последнее это образовательные программы, тоже конкурирующие, 
зарождаются в античности всегда существовали и существуют самостоятельно 
наука подготавливает себе кадры, и самый яркий пример это Пифагорейский 
союз конечно. Дальше это академия Платона с лозунгом не гиометр да не 
войдет. Сад Эпикура. Стоя тоже школа. Также неоплатоники. 
Александрийский Музей тоже. В средневековье академия Карла Великого, где 
он собрал самых мудрых на свой взгляд. Тогда уже ученых содержали, хоть 
практической пользы моментально не было. Появляются светские школы 
и в конце концов первые европейские универы: Болонья, Париж, Оксфорд. 
Тут тоже надо вспомнить из курса. Все эти курсы были основательно 
разработаны, классическая структура университета: разбивка по 
факультетам, в то время 4 факультета было (искусствоведения = философия, 
медицинский, юридический и теологический). Факультет искусств был 
подготовительным, там разрешались дискуссии, обмен мнениями между 
студентами и учителями, в конце концов появились степени бакалавра, 
магистра, деканы, доценты. Экзамены, защиты диссертации. Университеты 
конечно были под церковью, программы утверждались епископами, иногда 
даже самим Папой. Соколов наш сказал, что статус университетов, 
включающий значительную автономию внутриуниверситетской жизни, 
определялся санкцией королей и даже Римских Пап. Определенная защита со 
стороны церкви. Не случайно строили сначала на отшибе кампусы, чтобы 
очередная бесноватая толпа не контролировала. В 13--14 веках, позднее 
средневековье,


\hfill \textbf{Nov 3}

Почти закончили 2 тему... Говорили о важном понятии исследовательских 
программ Локатоса, охарактеризовали структуру, рассмотрели парадигмы 
античности, 4 парадигмы фиксизм, циклизм, эволюционизм, креационизм. 
Подчеркивают определенную дискретность мировоззренческого 
и методологического развития. Подчеркнули доминантность эволюционистской 
сейчас, но после Пригожина креационистская тоже набирает определенную 
силу. Дальше рассмотрели понятие исследовательских программ, сложившихся 
в античности вместе с категориальным аппаратом, Вартовский хорошо 
охарактеризовал его. Программы подчеркивают континуалистский характер 
развития корпуса научных идей. Конкурирующий. Важным является то, что 
появившись еще тогда, они и сейчас продолжают конкурировать, никто не 
отменяет, то одна вырывается вперед, то другая. Критерием по Локатосу 
является предсказательная способность. В историческом аспекте не всегда 
та, которая впереди, будет и дальше впереди. Программ довольно много. 
Основные отметили: 7 программ от архэ до mind-body. Дальше отметили 
программы методологические, тоже много, тоже конкурируют. 
Дисциплинарные, которые начали складываться в Александрийской школе, 
разделение труда произошло, в академии Платона теория была, философия, 
а в Александрийской школе складываются дисциплины конкретные. Дальше 
что-то сказали про средневековье. Прогресса там собственно не было, 
стагнация идей. Но кое-что было, теория импетуса, парадоксы Буритана, 
Генриха Генского, смелый епископ парижский, пошедший против Аристотеля. 
Ну и конкретные научно-образовательные философские институты от 
пифагорейского союза до академии Карла Великого. Образовательные 
учреждения, система образования начала складываться в средневековье, от 
монастырских школ до университетов. Структура универов, особенность 
факультета искусств, который стали потом называть философским. Осталось 
сказать только что в это время в 13-14 веках Европа существенно отстает 
от восточной науки. Основные события в научном плане происходят на 
востоке: открываются новые элементы (сера, ртуть), храм мудрости 
в Багдаде, теория двойственной истины (позволившая размежевать науку 
и религию), уравнения третьей степени решаются, трактат по арифметике 
Мухамеда Альх Эрзми, Омар Хайям математик, начала алгебры, Альберуни 
высказывает мысль о вращении Земли вокруг Солнца, обсерватории 
в Самаркаде, медицина Ибн Сина, и пожалуй главное -- введение 
позиционного исчисления (в Европе только через 2.5 века будет 
внедряться, и то с большими трудностями). Восток дальше не пошел потому 
что появился Ислам и запретил вольность мысли. Потом верх взяли 
фундаменталисты и теорию двойственной истины прикрыли. Ислам стал более 
консервативным, и развитие хорошего не получилось. В Европе еще ряд 
Фибоначчи появляется, тоже особенность развития... Программы не просто 
научного типа, а цивилизационного. Восток конечно перенял много что от 
Греции, но вот все равно. Начиная с эпози возрождения цивилизационные 
программа европейская оказалась вне конкуренции.

Сформулируем.... Программы тематические, методологические, ... Возникло 
в античности и не прерывалось в средневековье, а приобрело религиозную 
форму. Был разработан концептуальный аппарат, от культурных универсалий 
перешли к концептуальному аппарату. Произошло разделение физики 
в отличие от первых натурфилософов на физику неба и земли. Эксперимент 
понимался только как наблюдение, созерцание, природа устроена 
гармонично, если вмешиваемся, то искажаем. Аксиологическая установка 
тоже установка на то, что цель познания -- приобщение к гармонии мира. 
В целом, наука античности и средневековья имеет качественный тип. 
В античности появились новые тоже философские аспекты, характеризующие 
в целом средневековую науку:
\begin{itemize}
  \item универсализм (все -- божественное творение 
    и из ничего),
  \item мир устроен символично (надо разгадывать, все есть знак, 
    с помощью герменевтики, вся геральдика средневековая отсюда, 
    рыцарство итд, а для расшифровки нужна божественное благословение, 
    искра божья, просто так нельзя приступать к этому),
  \item иерархизм (мироздание устроено иерархично, как на небе, так и на 
    земле, на земле высшее -- церковь, в самой церкви тоже иерархия, 
    а потом только миряне)
  \item телеологизм -- все существует не просто так, а направлено на 
    какую-то цель, которая известна конечно только богу, а человеку 
    простому никогда замысел бога до конца не понять.
\end{itemize}
% }}}

\section{Философские идеи ренессанса -- теоретический базис становления 
новоевропейской науки}
%{{{

14--16 века, мизерный исторический период, но произошли тектонические 
сдвиги, и Европейская культура выходит на первый план и с тех пор никому 
не уступает... Быстрый подъем духовно-теоретической активности, можно 
конечно объяснять причинами какими-то, но в целом тоже уникальный 
процесс, как и появление культуры в Греции, сингулярность в мировой 
культуры. Стало моделью, паттерном для подобного в других местах -- 
стали говорить о грузинском, российском итд. Смысл -- вернуться 
к истокам античной культуры, возродить мысль, сделавшую Европу крутой. 
Но быстро поняли, что они уже круче, и вот поэтому возникло уникальное 
-- наукоцентризм. Все началось с того, что пересмотрели отношение 
к человеку, антропоцентризм сперва. Memento vita -- помни о жизни. 
Виндельбанд пишет: возрождение чисто теор духа есть истинный смысл 
научного ренессанса. В этом сродство с Греческим мышлением, оказавшим 
влияние на его развитие. Основные черты:
\begin{itemize}
  \item ориентация на гуманизм (пересмотр понимания человека, роли, 
    места в истории),
  \item активизм (совсем другой подход к жизни, проявившийся 
    в разнообразных ипостаси от бурного развития городов и кончая 
    кругосветными путешествиями) Например часы на ратушах даже били 
    каждый час, а в средневековье совсем другое. Время деньги стали 
    говорить.
  \item светское образование. В средневековье универы конечно появились, 
    но победили церкви, все кадры контролировались, а тут светское 
    образование начиная от школ, но потом и в универах перелом. Зачатки 
    научного познания.
\end{itemize}

Что поспособствовало ренессансу, причины:
\begin{itemize}
  \item соц экономические: разделение труда, мануфактуры, шелковый путь, 
    богатые города Италии, где путь заканчивался. Складывается новый 
    слой людей -- интеллигенция -- зарабатывали не физическим трудом.
  \item соц политические: первая буржуазная революция в Нидерландах, 
    проявление серьезно демократических процессов.
  \item технические -- три новшества пришли в Европу: порох, компас, 
    книгопечатание. В Европе нашли наибольшее применение. Порох даже 
    например требовал науку для расчетов полета. Книги позволили 
    ослабить цензуру, повысить образование.
  \item кросс-культурные: войны с арабами, конкистадоры, 
    колонизаторство, пираты. Столкновение культур было в конечном счете 
    в пользу европейцев.
  \item религиозные: явное ослабление, реформация католичества, начиная 
    с английской (Генрих 8, которому запретили жениться), лютеранство, 
    войны религиозные, ночь длинных ножей. Повоевали и в конце концов 
    север стал реформирован, стал свободным, а юг остался за 
    католичеством. Динамическое равновесие, столкновения были, но 
    победить никто не мог.
  \item сингулярность и синергетика.
\end{itemize}

Дальше пошла теория. Не совсем, сначала искусство... Новое. Смысл 
ренессанса, основные причины фундаментальные. Новые образы, искусство 
наглядно, и книги Данте Оливьере?? Божественная комедия, эпос, где папы 
сидят в самых страшных кругах ада, а философы древней Греции в легких 
кругах. Возвеличение или признание роли женщины (Биатриччи), ведет 
путешественника в рай. Стихи Петрарха, де Камерон, Рафаэль, 
Микеланджело, да Винчи открывает перспективу. Вот это все новое 
искусство. Еще архитектура, готика, устремленность ввысь, не купол со 
звездами говорил об ограничении, а шпили в космос в бесконечность. 
Дальше появляются теоретики гуманизма. Искусство раскрепостило человека, 
а теоретики гуманизма это тоже известные фамилии: Мирандо пишет речи 
о достоинстве человека, а до этого Иннокентий 3 писал о презрении 
к человеку. О наслаждении Лоренцо Вал возрождает эпикуреизм в вульгарной 
форме. О достоинстве и превосходстве человека Жанот Самонетти. Дальше 
Вальтер отмечает Эразма Роттердамского о свободе воли, трактат, 
поразивший Европу. Лютор написал свой трактат о рабстве воли. Похвала 
глупости того же Эразма Роттердамского, сатира, Эразм издевается 
говорит, что правят дураки, сама вера говорит о глупости. Мишель Монтель 
тоже. В Мыслях воскрешает скептицизм античности, трактат опыты тоже, что 
люди часто довольствуются только предрассудками, верой, самодовольны 
в своей глупости, люди ничему не верят так твердо, как тому, о чем они 
меньше всего знают.

Дальше первое научное сообщество, сначала на уровне кружков 
интеллектуалов, Мирен Мильсен??, через которого шла вся переписка 
интеллектуалов, потому что универы были еще под церковью, а это 
светское. Дальше первая академия гуманистов во Флоренции, основал ее 
православный грек и Константинополя, Георгий и кто-то там еще. 
Занимались переводами с древнегреческого на латынь. Потребность потому 
что пал Константинополь, и большое количество интеллектуалов хлынуло 
в европейские города, и надо устанавливать аутентичность текстов, 
переводить, и вот вся современная что-то там основано на этих трудах. 
Много текстов появилось Платона, и сразу Платон стал составлять 
конкуренцию Аристотелю, в Европе стали дискутировать, кто круче, 
Божественный Платон или Энциклопедист Аристотель. Академии оказывал 
поддержку Казима Медичи, тоже богатый человек, правитель Флоренции. 
Вторая академия рядом с Неаполем -- занималась естественнонаучными 
проблемами. Первый трактат назывался о природе вещей по ее что-то там. 
Величие природы, необходимость выводить знания из природы вещей. В 1603 
году появляется Римская академия Деи линчеи (рысьеглазых) на базе 
Ватиканской библиотеки. Теологи там занимаются астрологией, математикой, 
физикой. Галилей даже там участвовал в свое время. Дальше Никола 
Кузанский епископ пишет трактат свой об ученом незнании, тоже поминали 
Сократа, я знаю, что ничего не знаю. Его воспевание и серьезная 
разработка математической программы с уклоном в космологию. Бог 
бесконечный абсолют, природа безграничный абсолют, бог использовал 
трансцендентальные инструменты (существовали наряду с богом). Бог 
пользовался арифметикой, геометрией, музыкой, астрономией. Всеми 
искусствами, которые мы так же применяем при изучении вещей и отношений. 
Целая глава в трактате посвящена математике. В 20 веке Юджин Вигнер 
напишет знаменитую статью о непостижимой эффективности математики 
в естествознании. Недавно наш философ науки Виктор Везгин?? 
О непостижимой эффективности математики в квантовой механике. Математика 
фактически первична получается. И также знаменитые аналогии Кузанца -- 
бог центр мироздания, но само мироздание границы не имеет. Центр нигде 
и везде, окружность, хорда, касательная -- одно и то же. Вселенная 
начинает представляться как гомогенная и изотропная. Раз вселенная 
такая, то миров может быть много. Хоть сразу начали на него наезжать, 
его друг детства, ставший папой, того защищал. Кузанский тогда признал, 
что миров много, но наш лучший, любимый богом. Приоритет за земным миром 
остается.

\subsection{Коперник}
Дальше Коперник, а может быть его ассистент, делается труд для упрощения 
религиозных праздников, Пасхи. На основе этого труда Юлианский календарь 
на Григорианский сменен, у нас нет, поэтому расхождения пасхи, 
рождества. В конце концов церковь разобралась в ереси, произведение 
занесли в индекс запрещенных книг, и просуществовало там аж до 
1800-какого-то года, если я правильно услышал. Надо сказать, что идея 
гелиоцентризма после Коперника с его книгой дала свои плоды. Как 
математическая теория оказалась несовершенной теорией конечно. 
Коперниканцы и Птолимеианцы, но были вот две порочные ошибки 
философского типа, которые не позволили сделать более точные расчеты. 
А именно, круговые орбиты и равномерность движения, что он взял от 
Аристотеля. Поэтому в расчетах были расхождения с экспериментом, 
приходилось вводить круги и остальное. Тем не менее, называют 
революцией, а именно, значение в том, что снят онтологический приоритет 
Земли как центра вселенной. У Николы Кузанского это мат теория, а тут 
физическая теория. Земля как рядовая планета вокруг Солнца. И второе -- 
окончательное расхождение между эпистеме и доксой -- научным знанием 
и обыденным мнением. Каждый видит, что Солнце восходит и заходит, но на 
самом деле все совсем наоборот. Все не так, как нам видится и кажется, 
на самом деле. Коперниканский переворот. Тем не менее, процесс пошел, 
и следующей фигурой становится Тихо де Браге, датский ученый, которому 
король дает в аренду остров для занятий, там он строит обсерваторию, 
король понимает, что там, где занимаются наукой, государство имеет 
приоритеты, Тихо помогают материально, и он создает первый всеобъемлющий 
каталог небесных тел, тысячи небесных тел, уже не шутка. Причем 
с помощью механических приборов. В помощниках у него работает техник. 
Сначала скажем о Джордано Бруно, сбросивший рясы доминиканец?? первым 
стал обсуждать физическую значимость теории Коперника. Сам 
астрономического образования не имел, но грубо говоря поверил, 
пропагандировал. Идеи простые, но абсолютно революционные: все, что 
написал Коперник -- физика, физика одна принципиально (у Коперника две 
физики как у Аристотеля), что на небе то и на земле, мир одушевлен 
и состоит из монад, бог в вещах природных, никакого отдельного бога нет, 
этика героического энтузиазма, надо познавать мир и преобразовывать, 
отказ от какой-либо приоритетности земного мира. Недостойно могуществу 
бога создать только один мир.

Итак, это уже натурфилософия с вплетенной математической программой. 
Непосредственная предпосылка развития дальнейших конкретных программ. 
Конкретные науки позднего возрождения.
\begin{itemize}
  \item Итальянец Лука Пачори: божественная пропорция сумма арифметики. 
    Узаконивает позиционное исчисление. Наконец входит в Европу как мат 
    инструмент, а потом и в физику. Приоритет математики, что применима 
    везде и вся.
  \item Леонардо да Винчи вводит практику золотого сечения, не только 
    построение механизмов, но и искусства. Вводит практику чертежей.
  \item Тарталье и Картана (фамилии или имена) дополняют 
    тригонометрические и астрономические таблицы. Начинают решать 
    уравнения третьей степени. Возникает первый в истории науки спор 
    о приоритете между Тарталье и Картана, кто первый открыл, почему 
    опубликовал, а не известил, целая история. Развивает кроме того 
    теорию импетуса, важная заслуга Тарталье, что первый утверждает, что 
    все движение на Земле имеет в конечном счете криволинейный характер. 
    Важно для стрельбы хоть из пушек хоть откуда. До него были картинки 
    с ломаными линиями полета ядра пушки. Тарталье показал 45 градусов.
  \item Франсуа Ливье?? первооткрыватель алгебры или чего-то.
  \item Стивен Сивин?? вводит десятичные дроби. Потом он же закон 
    свободного падения.
  \item Рафаэль Бомбели?? вводит комплексные числа, его книгу потом 
    используют
  \item труд о магните Гильберт какой-то. Гильберт проводил свои опыты 
    часто как шоу, даже при королеве. Первым выступил с мыслью, что 
    Земля -- большой магнит, поэтому компас показывает так направление. 
    Первое теоретическое утверждение.
  \item В астрономии конкретные научные программы разрабатывает Тихо 
    Браге и немец Регио Монтан. Последний составил таблицы для 
    мореплавателей, практическое применение астрономии того времени. 
    Таблицами пользовались все.
  \item В медицине Парацевс?? знаменитый, лекарства основаны на сущности 
    человека ртуть дух, соль тело, что-то душа. Гомеопатия, подобное 
    лечится подобным.
  \item Гум науки: Макиавеллевская программа, принцип эгоистичности 
    природы человека, правитель может быть все и вся, свирепым как лев, 
    хитрым как лисица, частная собственность во всем, идеология. 
    Противоположная программа коллективизма, Томас Мор, Кампанелла, 
    какой-то еще чел. Тоже важная социальная проблема.
\end{itemize}

Подведем итог ренессансу. Сам феномен Европейского ренессанса -- 
первоимпульс и точка отсчета новой эпохи во всех человеческой 
деятельности. На протяжении нескольких столетий мощнейший резонанс во 
всех частях света, европейцы несли не только грабеж и колонизаторство, 
а идеи возрождения. Возрождение разрушило смирительную рубашку чего-то 
там. За 2 века сформировалась новая система ценностией, на новый уровень 
поднят критический реализм, здесь уже и борьба внутри религии, и зачатки 
науки и борьба с религией, Брруно жертва, древнегреческий язык появился 
наравне. Общий вывод Виндельбанда -- современное естествознание -- дитя 
гуманизма.

%}}}

\section{Научная революция 17--18 веков и становление механической 
картины мира}
%{{{

% Начало {{{

Начинается с соц поражений. Революции по всей Европе, самые значительные 
в Англии, где свергают одну монархию, потом диктатура, потом 
восстановление монархии. Возвышение Голландии, освобождение их от 
испанского угнетения. Первая конкретная фигура Борис Михайлович Гессен?? 
в 30-е годы история и философия науки говорит Ньютон соц корни имеет. 
Механика Ньютона возникла из соц экономических потребностей Европы, 
Ньютон должен был появиться. Это направление -- экстернализм, всегда 
импульс извне. Гессен считал, что такими импульсами для появления 
механики -- потребности в строительстве путей сообщения, инфраструктура, 
дороги, горное дело, полезные ископаемые стали цениться, военное дело, 
где без математики стало ясно не обойтись. Переход 17--18 веков есть 
в то же время переход от антропоцентризма к наукоцентризму. Обосновывал 
все это уже довольно тщательно известный философ Френсис Бэкон. Основные 
его достижения в этом смысле. Переосмысление понятия эксперимента от 
созерцания к преобразованию, испытывать или пытать природу, тогда 
что-нибудь ответит. Второе -- признание окончательное онтологической 
значимости теории Коперника, гелиоцентризм не новость для Европы. Третье 
-- признание важнейшего значения научного метода, не молиться надо, 
а использовать научный метод. Четвертая -- новая аксиология, наука не 
для созерцания и гармонии, а для преобразования в интересах человека. 
Бекон подчеркивал всегда, что знание -- сила.

\hfill \textbf{Nov 10}

Говорили о восточной науке и перешли к ренессансу, кратко вспомнили 
основные причины явления в Европе, до конца не выясненные, но какие-то 
были, экономические, соц культурные, кросс-структурные, ослабление 
католичества, появление протестантизма. Этапы: начинается с искусства, 
появляются новые течения в искусстве, в литературе, дальше появление 
гуманистов, теоретиков гуманизма, Эразм Роттердамский, Мишель Монтель, 
а затем серьезные размышления Николы Кузанского с привлечением 
математики, изотропность вселенной, бог везде и нигде, признание 
математики как инструмента бога при создании мира. Появление первых 
академий, универы, появляется большой слой интеллигенции, затем переводы 
с древнегреческого на латынь после падения Константинополя, затем 
папская академия, какая-то еще до нее, ориентированная на 
естествознание. Галилей даже работал в папской академии. А затем 
Коперниканский переворот, но с физическим подтекстом, и церковь 
запрещает. Недостатки концепции мы отметили круговые орбиты и движение 
планет в виде аксиом. Затем Джордано Бруно, хоть к науке отношения не 
имел, верил в физ значимость концепции, говорит о бесконечности космоса, 
раз бог всемогущ, мог создать и много миров. Первый выдвигает идею 
о единстве физики между небом и землей. Пантеизм Бруно тоже важная вещь, 
жизнь бесконечна и вечна. Дальше Тихо де Браге с экзотической моделью 
вселенной, все планеты вертятся вокруг солнца, а солнце вокруг земли. 
Существенный сбор эмпирического материала, небесных тел, более 1000 
объектов. И на этом мы закончили и начали научную революцию 
и становление механической картины мира. Здесь Гессен на физическом 
факультете выступил на конгрессе произвело фурор, он сказал, что 
достижения классической механики и математики связаны 
с капиталистическим обществом и потребностями общества, расчет для путей 
сообщения, для военных дел. Говорили о конкретных уже естественнонаучных 
достижениях позднего ренессанса: узаконивание позиционного исчисления 
в Европе, Тарталье и Карданн, комплексные числа, десятичные дроби, карты 
звездного неба, идея о земле как о огромном магните, алгебра появляется. 
Прежде чем дальше говорить о классиках классической науки впереди идет 
философия. Вспомним значит философов, которые внесли вклад в разработку 
важнейших общенаучных методов. Если в средневековье нужны были 
благословения, молитвы, то сейчас нужны методы.

%}}}

\subsection{Френсис Бэкон}
%{{{

Первый конечно Френсис Бэкон. Обобщенные результаты это критика 
схоластики -- первый схоласт и враг для него Аристотель, замедлил 
развитие науки. Бэкон уже понимает, что эпоха превзошла натурфилософов 
Греции, идея ренессанса была правильной, но дальше пошло с ускорением. 
Эта разрушительная критика Бэкона выражается конечно в теории идолов. 
Идолы человечеству враг. Идол рода, связанный с ограниченностью 
сенсорики нашей, барьеры, пороги, поэтому единственный способ узнать мир 
-- преодолеть барьеры, а значит использовать научные приборы, которые 
должны расширить наши органы чувств, гири к разуму надо подвесить. 
Второй идол -- пещеры, аллегория Платона, люди видят не реальные 
события, а тени на стене пещеры, у каждого своя пещера, каждый человек 
подвержен своим предрассудкам, здесь для науки важнейшим моментом 
является коммуникация, каждый должен разрушать свою пещеру, дискуссии 
в научном сообществе, различного рода правила. Третий самый тяжкий идол 
-- площади или рынка, здесь поднимается проблема именно о языке науки, 
хоть сам он не считал, что математика должна быть языком науки, но 
признавал, что у науки должен быть особый язык, и это проблема, то, что 
сейчас называется лингвистическими кругами, все равно нужны понятия типа 
атомов и молекул после всех вычислений, с одной стороны язык науки 
специфический, с другой стороны один из семиотических каналов, который 
в конечном итоге должен возвращаться в семантику, в осмысление, а тут 
всегда социокультурное значение. И четвертый идол -- театра, в театре 
самая главная фигура режиссер, он ставит спектакль, и этим идолом театра 
является конечно Аристотель, поставил спектакль длиной 2 тысячелетия, 
и необходимо менять режиссера. Вот эти четыре идола входят в критику. 
А в положительное значит это его эмпирико-индуктивный метод, самый 
первый противопоставляется схоластике. Полная индукция не имеет 
значения, а научная индукция вот как Бэкон описывал в таблицу, признаки 
присутствия, отсутствия, степени, всего остального. И это он проделал 
применительно к теплу: присутствие в Солнце, теплокровные животные, 
можно по отсутствию: светлячки, ... И так он перешел к идее, которая 
живет больше чем 2 века, движение мельчайших частиц. Это его 
непосредственный вклад позитивный уже. Третье его достижение -- 
переосмысление самого понятия опыта, эксперимента. У Аристотеля опыт это 
созерцание, пристальное, внимательное, но созерцание. Потому что мир 
устроен гармонично, и мы должны гармонию понять, иначе искажаем картину. 
Бэкон предлагает другое. Испытатель, экспериментатор, пытатель -- пытать 
природу надо, ставить в стесненные условия, вот тогда она что-нибудь да 
ответит. Знание сила, эксперимент активное вмешательство в природу. Ну 
и последнее это новая аксиология. Опять-таки, вся древнегреческая наука 
стояла на аксиологическом принципе, что это прекрасно, но это приобщение 
к гармонии мира, даже Архимед стыдился практических результатов, которых 
достигал по заказам высокопоставленных лиц, считал, что наука там, где 
чистое знание, в этом смысл познания мира как приобщения к гармонии, 
отсюда развивается тезис, что знание сила, сила преобразования мира 
в интересах человека. И отсюда идея о том, что существуют опыты 
плодоносные, а существуют и светоносные. Знаменитое выражение ``нет 
ничего практичнее, чем надежная теория.'' Без теории сейчас никто не 
будет строить ускорители, запускать что-то в космос и прочее.

Дальше идеи пошли в жизнь, Бэкона оценивали высоко, и основатель 
Лондонского королевского общества, и Гюйгенс, портрет в обществе висит, 
конечно его идея об организации науки тоже имеет значение. Новая 
Атлантида -- короткое произведение, но фундаментальное. Дом Соломона, 
приоритет науке, статуи при жизни ставят, все благое от науки, 
выращивание новых сортов злаков, передача звука на расстояния и куча 
других чудес, которые сейчас уже не чудеса. Государство должно 
заботиться о нас, во главе государства стоят ученые, 
и материально-техническая база, приборы и инструменты, уже в то время 
было понятно, необходимо вкладываться в науку, развивать в базу, 
и сейчас очевидные вещи тоже, в развитых странах от ВВП идет 4\% на 
науку. Ну и другая сторона медали это образование. Бэкон первый 
понимает, что наука должна сама себе готовить кадры, ни по какому 
принципу крови, знатности, ничего не будет. Нужно готовить все самим. 
И причем раз наука носит универсальный характер, то и образование 
должно. Ему принадлежит идея консолидации образовательных учреждений, 
то, что сейчас называется болонской системой. Универсальность 
образования, о которой уже говорил Бэкон, вот она. Ну а философия 
у Бэкона тоже есть. Первым пытается разделить философию, изучающую бога, 
теологию, философию-метафизику, изучающую целевые причины, а вот у науки 
материальные причины и движущие причины должны быть. Бэкон конечно 
далеко не атеист, и говорит, что лишь поверхностное взаимодействие 
с философией ведет к атеизму, но чем глубже, тем яснее, что нужен бог, 
но изучать все равно можно, вынося бога за скобки. Ну вот это кратко 
о Бэконе.

\subsection{Декарт}
Дальше у нас Декарт идет. Прежде чем к нему переходить вспомним общую 
картину философских рассуждений. Бэкон -- основоположник эмпиризма, 
экспериментально-индуктивный метод. То, что можно проверить в опыте, это 
и есть наука. Наблюдаю, значит, существую. Наблюдать можно и с помощью 
органов чувств, и с помощью приборов. К этому направлению Бэконовскому 
относятся Томас Гоббс и Джон Локк. Вот три основные фигуры направления 
эмпиризма. Наиболее разработанная система эмпиризма уже у Локка, теория 
познания, его тезис, что нет ничего в разуме, чего до этого не было 
в органах чувств. Значит это эмпиризм, и у них главный метод 
эмпирико-индуктивный. Второе направление -- рационализм. Это Декарт, 
Спиноза и Лейбниц. Тоже у каждого оригинальная система, но общим 
является то, что в разуме есть все-таки нечто, что от опыта не зависит, 
что называется априорным или трансцендентальным. Лейбниц как раз добавит 
``кроме самого разума'' к тезису Локка. Разум не дан в опыте, а лишь 
в рефлексии над самим собой. С точки зрения рационалистов, можно 
покопаться в разуме и дойти до априорного. В опыте можно тоже дойти, но 
от опыта оно зависеть не будет. Рацио -- нечто, рожденное в человеке. Ну 
и рационалистический метод, основатель конечно Декарт, это 
аналитико-дедуктивный метод. Дойти до каких-то с помощью анализа 
сущностей, которые не разлагаются, и на основе них строить различные 
конструкции. Это второй принципиальный метод. Два других метода это уже 
что сами ученые предлагают, и определенный компромисс по отношению 
к этим двум. Это гипотетико-дедуктивный метод Галлилея, состоит в том, 
что можно выдвинуть любую гипотезу, но дальше нужно оформить ее 
математически, модель, и на основе модели сделать выводы, а потом 
проверять в опыте. Это такой вот компромисс в математике и опыте, 
а сколько опытов делать... загадка. И четвертый метод аксиоматический, 
его развивает Блез Паскаль, развивает его вместе с Ферма, классическая 
теория вероятностей. Не случайно сказано в начале лекции, что эти методы 
и являются решающими, чтобы общаться с природой. Не молитвы, не 
заклинания, не алхимия, не рецептура, а вот методы. Виндельбанд уже 
прямо говорит, что в этот период происходит война методов, каждый 
считает, что его метод универсален. Но вот оказалось, что все методы 
хороши. И используются все методы лишь бы достигнуть цели. Итак, это 
общая картина эмпиризм--рационализм. Теперь основные идеи всех этих.

Декарт -- основатель новоевропейской философии, так сказал Гегель. 
Основатель рационалистического метода. Первая важнейшая идея -- единство 
всего научного знания. У каждого алхимика было свое знание, но у ученых 
уже не так. Своей идеей метода универсального он излагает в произведения 
правила чего-то там плохо сказано. И дальше в рассуждениях об отыскании 
истины в науке он пишет 4 правила основных. Анализ надо вести до тех 
пор, пока положения не станут для вашего ума ясными и понятными. Это 
интеллектуальная интуиция. Принципиальное отличие рациональной интуиции 
от мистической интуиции из средних веков, где все было в храме 
в результате молитвы. Дальше от этих понятий. Дедукция, цепи логических 
выводов, где одни положения вытекают из других. Энумирация, внимательное 
отслеживание шагов этих процедур. Если упустить что-то, все пойдет 
насмарку. Начинать надо с собственного познания эти процедуры, путь 
к истине лежит через анализ собственного сознания. Декарт пишет, что 
если кто-то задается целью познать все истины, доступные разуму, то он 
поймет, что ничего нельзя познать прежде самого разума. Здесь это метод 
рефлексии. Отсюда, когда этот метод погружения в самого себя происходит, 
мы приходим, по Декарту, к обнаружению врожденных истин. Это могут быть 
идеи типа протяженности, единства, длительности, бога. В плане бога 
возвращается к Ансельму Кентерберийскому, раз бог совершенный, то он 
должен существовать. Фома Аквинский критиковал конечно, но вот Декарт 
вернулся. Говорил, что это не кентавры, которые у каждого свои, а общий 
бог, все согласны, что это совершенное существо. Врожденные также могут 
быть аксиомы: тожества, транзитивности, непротиворечивости. Высшей 
аксиомой врожденной является его тезис cogito ergo sum. Никто же не 
скажет человеку, что чел мыслит, лишь сам чел может до этого дойти. Из 
этих значит аксиом возникает первая конструкция. Декарт говорит, что 
2+3=5 истинно, априорно и не зависит ни от какого опыта, ни от палочек, 
ни от пальцев. Априорное оно, но приходить можно и через опыт. Истины 
даны диспозиционно, как некая предрасположенность, нужно заниматься 
алгеброй, чтобы прийти в конечном итоге к ним. Прийти к этим истинам 
Декарт предлагает через метод сомнения: сомневайся во всем. Метод 
Декартовского скептицизма становится научным методом. Школа скептиков из 
античности была разрушительной, ничего нельзя познать, и лучше помолчать 
вообще. У Декарта конструктивный скептицизм: усомниться можно во всем, 
в боге, в математике, но нельзя усомниться в cogito ergo sum. Ваше 
сомнение есть показатель мышления. Но раз есть одна истина, то возможны 
и другие и дальше раскручивание существования бога и остальных истин. 
И дальше его знаменитая постановка проблемы mind body, сознания 
мышления, на уровне человека признание дуализма материи и духа. Материя 
все, что протяженно и можем измерять, а дух -- не можем измерить никаким 
образом. Проблема состоит в том, как же можно управлять телом, где это 
пересечение. Декарт ее ставит, но и сейчас тоже проблема актуальна. Все 
углубилось, нейронные сети, хим анализ, э-м анализ, прощупывание 
активности разными приборами, проблема последнего нейрона, где 
появляется не-материя, остается проблемой. Последнее это его физика. 
Первым создает физическую картину мира из этих. Вводит понятие 
корпускулы, мельчайшего элемента, и запускает с помощью бога. Импульс 
изначальный вносит господь бог, а дальше по законам механики, теория 
вихрей, корпускулы создали вот такое движение на основе импульса от 
бога. Наша галактика тоже спиралевидная, много других таких. Главное 
здесь подчеркнуть, что это первый в истории серьезный мысленный 
эксперимент и объяснение формы нашей галактики. Еще мысль, что 
постулируется континуалистская программа физики: мир устроен 
континуалистски, корпускулы соприкасаются друг с другом, никакой пустоты 
нет, импульс передается от точки к точке, принципиальное отличие от 
программы дискретности того же дальше Ньютона. Забегая вперед, скажем, 
что это фактически программа была очень убедительной, особенно на 
континенте, существовала вплоть до 18 столетия, и только уже Вальтер 
фактически раскритиковал существование этих двух физик, описал в своем 
саркастическом духе ``дыня, тыква, корпускулы, притяжение, непрерывное, 
атомы и пустота'' Главное конечно, что у Декарта можно было объяснить 
все, но предсказать нельзя было. Вывести Кеплера нельзя было, 
предсказать планету тоже. Объяснить можно все, а вот предсказать 
с помощью теории уже совсем другое дело, и физика Декарта потерпела 
фиаско. Тем не менее, многие идеи потом обнаружили значимость в э-м 
картине мира, континуализм, передача импульса от точки к точке.

%}}}

\subsection{Пьер Гассенди}
%{{{

Пьер Гассенди. Защита от Бэкона, авторитет имеет исторический характер, 
а не нормативный, и при этом физика Аристотеля имеет место, дала первое 
объяснение, а потом уже сделали лоскутное одеяло. Введение молекулы, 
потом уже Бойль. Ну и возрождение идей Демокрита и Эпикура именно 
в физическом плане. Гассенди критикует и Декарта за континуалистскую 
картину мира, считает, что атомы и пустота между ними вполне возможна. 
Тем не менее, как епископ, хоть и занимался физикой и математикой, сами 
атомы с его точки зрения сделаны и порождены богом, и складывание атомов 
в молекулы по принципу складывания букв алфавита в слова. Количество 
букв в алфавите ограничено, а сочетание букв по сути безгранично. Отсюда 
и многообразие идей. Бог создал атомы, на основе них молекулы, а на 
молекулах все многообразие. Ну и понятие пространства-времени.

Не знаю написано ли уже, но Гассенди еще и ввел молекулы из атомов 
в стиле как слова составляются из букв. Атомы созданы самим богом 
у него. Легли в основание физики Ньютона и химии Бойля.

%}}}

\subsection{Спиноза}
%{{{

Спиноза. Трагическая судьба, отлучение от сообщества евреев 
в Амстердаме, наложение херениума, проклятья фактически, запрет на 
общение. Всю жизнь шлифовал линзы. Рационализм у него практически как 
у Декарта, но не дуализм, а монизм и пантеизм. Причина самой себя -- 
мысли об однородности природы, атрибутами, а уже не субстанциями, 
являются дух и вещество, а модусы -- конкретные вещи. В плане онтологии 
это. Кроме того, жесткий детерминизм: случай, лишенный смысла слова??. 
Порядок и связь идей те же, что порядок и связь вещей. В природе нет 
творения, а только порождение, отвержение идей креационизма. Отторжение 
космической религиозности. Если бог, природа и материя одно и то же, то 
вот религия вот такая космическая. Ее признавал даже Эйнштейн потом. 
Если бог, природа и материя есть одно и то же, то рациональность лежит 
в самом мире. Атрибутов может быть бесконечно много, но нам известно вот 
только два: материальный и идеальный. Модусы уже в конкретных вещах 
проявление. Порядок и связи идей такие же, как порядок и связи идей. 
Можно начинать с самой природы, можно с идей, будет одно и то же. 
Жесткий детерменизм, случайности только у нас в голове. Потом назвали 
космическая религиозность: в самой природе есть логика, именно поэтому 
можем познавать.

%}}}

\subsection{Лейбниц}
%{{{

\hfill \textbf{Nov 17}

Готфрид Лейбниц -- немецкий ученый тоже. В отличие от Спинозы, Лейбниц 
отличается бурной организационной деятельностью, член королевского 
общества, Парижской академии наук, организует Берлинскую академию наук. 
Поспособствовал образованию Петербургской академии наук. Петр I с ним 
встречается. 1725 год образование Петербургской академии наук. Внимание 
на то, что в Европе академии наук вырастали естественным образом, мы 
даже перечисляли этапы, а в России академия была так сказать спущена 
сверху, по указу Петра. Указ можно было легко издать, но ученых особо не 
было. Первыми учеными были иностранцы. Преподавали на немецком, 
например, учили в гимназиях, Эйлер был даже.

Сам Лейбниц тоже кроме своей организаторской деятельности и полемист 
хороший, вступает в полемику с Кларком, защищал приоритеты интегрального 
исчисления. Вторая дискуссия Кардан и Тарталья, обсуждали решение 
уравнений третьей степени. До сих пор консенсуса нет, рабочей теорией 
считается, что Ньютон был первый, но не опубликовал, и вот Лейбниц так 
опередил. Дошло до того, что Ньютон в то время уже президент Лондонского 
королевского общества, создал комиссию специальную, и та комиссия 
конечно понятно что вынесла. Тем не менее, вся мат символика пошла 
в науку именно в исполнении Лейбница.

Конкретный вклад в проблематику философии науки. Лейбниц конечно 
критиковал философию Ньютона за абс пространство и время, а сам выдвинул 
принцип пространства и времени как субъектно-чувственной организации 
точки зрения всех монад. У Ньютона если все убрать из пространства, 
пространство останется. У Лейбница пространство и время связаны с самими 
монадами. Его теория монад принципиально тоже отличается от дискретной 
программы Ньютона. У Лейбница континуализм, импульс от точки к точке, 
монады соприкасаются, сами монады тоже даже не пантеизм, а панпсихизм. 
Все монады созданы богом и более того одушевлены. Некоторые монады 
спящие, некоторые круче других, самая крутая монада -- бог. Действуют по 
законам диалектики: с одной стороны созданы богом, с другой стороны 
активны; с одной стороны нет окон и дверей, с другой действуют 
в согласии, в резонансе. Должны обустроить мир. Все к лучшему в этом 
лучшем из миров -- итог размышления Лейбница о мире, состоящем из монад. 
Соответствующее произведение взяло за основу это положение: несмотря на 
зло в мире, оно носит лишь субстанциальный характер, это лишь отсутствие 
добра. Мы можем всего не знать, замысел бога никогда известен не будет, 
то, что мы видим злом, в космических масштабах может быть очень даже 
добром, ну и другие аргументы.

Гносеологически развивает идею чего-то. Логико-мат истины считает 
априорными, не зависят от опыта, но приходят от опыта. Пример с глыбой 
мрамора с прожилками, люди проходят мимо, а скульптор взял и увидел. 
Чтобы увидеть, нужно отсечь все лишнее. Логико-мат истины находятся 
у нас диспозиционно, но не всем дано отсечь лишнее и увидеть. Сами 
истины могут быть истинами факта, с этого познание начинается, посмотрел 
в окно, увидел, идет дождь. Истинные вечные -- логико-мат истины, но 
путь познания от этих фактов к логико-мат истинам. Факт идет дождь нужно 
объяснить через там модель погоды по крайней мере на какой-то период. 
Видим факты и, отталкиваясь от них, строим модели. Идеал науки 
и философии: хватит спорить, нужно сесть и посчитать.

Кроме того, достижения в исчислении, сохранение движения тела при 
соударениях, обнаружение бессознательного в сознании человека, задолго 
до Фрейда высказывает, что есть не только рацио, но и неосознанные вещи. 
Но выводил их порогов сенсорики. Сознание это некое лучезарное ядро, 
а бахрома за этим ядром уже неосознанный процесс. Шум одной волны не 
слышим, а прибой слышим. У Фрейда конечно все наоборот.

%}}}

\subsection{Эмпирики}
%{{{

Гоббс и Локк идут по стопам Бэкона. Локк сказал, что нет ничего 
в разуме, чего не было в чувстве. У Гоббса идея конвенционального 
происхождения языка. До этого язык просто дал бог. У Гоббса идея тоже, 
что материю нужно отождествлять с природой. Никакой материи-прима быть 
не может. У Локка более интересно конечно это познание, теория познания, 
теория опыта, но может быть внутренний и внешний, внутренний на основе 
рефлексии, внешний на сенсорике, получаем идеи, простые сложные 
первичные вторичные качества. Проблема познания -- проблема комбинации 
идей, что внутренних, то и внешних. Аргументация простая: если бы нас 
сенсорика обманывала, мы бы просто не выжили как биологические существа. 
В животном мире такие животные долго не живут. Сенсорика дает хорошую 
информацию, а ошибается ум. Знаменитый Декартовский пример, что чувства 
нас обманывают, что палка в воде кажется такой вот поломанной, а на деле 
закон преломления. Видим то, что должны видеть, а что палка сломана -- 
наше суждение.

Это значит два философско-научных направления. Они первые, кто стоит 
у истоков формирования новой научной картины. Теперь пара слов конкретно 
про ученых, которые конкретно закончили рассуждения и выкладки философов 
и как ученые внесли фундаментальные идеи в эту механистическую картину 
мира.

%}}}

\subsection{Кеплер}
%{{{

Учился у Тихо Браге, на основе карты Браге, на основе его таблиц пытался 
создать всеобъемлющую картину нашей солнечной системы. Его произведение 
-- гармония мира, где он попытался обосновать, что наша система 
гармонична (конечно потому что создал господь бог). Но с помощью 
геометрии, математики, сама геометрия отражает мир потому что бог 
пользовался ей при создании мира. Гармония мира через геометрию вложил 
господь бог. Нестыковки конечно были. Модель его, он пытался провести 
идеи Коперника, а вообще Аристотеля, о круговых орбитах и равномерных 
движениях. Но было больше эксп материала, и он пробовал разные 
комбинации, и в конце концов пришел к выводу об эллипсах, и в конце 
концов на основе этой предпосылки открыл свои три закона чисто 
эмпирических. Солнце находится в фокусе эллипса, движение неравномерное, 
а вот такое-то, удаляясь замедляются, а сам цикл движения планет вокруг 
Солнца зависит от расстояния между планетой и Солнцем. Он задумался 
все-таки почему планеты движутся по орбитам, почему в конце концов не 
сваливаются. Идея притяжения уже была, но была в стиле магнита. Признал 
и центробежное движение, ввел понятие инерции, но конечно у него это 
чисто спекулятивные вещи. Инерцию он определял как лень планеты, лень 
планетам падать на Солнце, поэтому вращаются. Но это была уже мат 
модель, гораздо меньше нестыковок по сравнению с Коперником, но церковь 
конечно запретила эти труды и не согласилась.

%}}}

\subsection{Галилей}
%{{{

Лекции по топографии ада читал даже, очень образованный человек... 
Непосредственно автор гипотетико-дедуктивного метода. Один из основных 
методов науки: выдвигание гипотезы и проверка предсказаний. Философские 
идеи совершенно новые тоже, корни идей есть, одна из основных идей, что 
мироздание написано на двух книгах. Философия написана в величественной 
книге, я имею в виду Вселенную, и эта книга всегда открыта нашему взору, 
но читать может только тот, кто знает язык ее, а это язык математики. 
Идея, что есть книга откровения -- библия, там свое изложение творения 
мира, знаменитые шесть дней. Здесь книга сотворения -- книга природы. 
Одна книга написана на латыни, а потом на национальных языках, а другая 
-- на универсальном языке, на языке математики. Хочешь читать Библию -- 
учи латынь. Хочешь природу -- учи математику. Читаем по страничке, 
и каждая страничка открывает нам все новое и новое на языке математики. 
Всякая книга имеет границы, тут даже просматривается идея того, что 
сейчас называется теорией всего. Если прочтем всю книгу, каждую 
страницу, то поймем всю природу. Посыл тот, что математика фактически 
становится первичной. Платон начал первым об этом говорить, и у Галилея 
снова возрождается эта идея платонистская, что великие уравнения уже 
были написаны, а мы лишь приобщаемся. То, что пифагорейцы ставили выше 
всего числа.

Дальше, он окончательно размежовывает науку и теологию, хотя сам 
является глубоко верующим. Послание герцогине: профессора богословы не 
должны регулировать профессии, бла бла бла...

Еще одна важная гносеологическая идея Галилея -- о двух видах знания. 
Интенсивные и экстенсивные. Книга знаний написана богом, сам автор знает 
все, полные знания -- экстенсивные. Дойдет ли сам человек до последней 
страницы -- проблематично, но где-то вот мы уже начали ее читать. И это 
у нас интенсивное знание. Те истины, которые мы открываем, те же, что 
знает бог. Истины одни и те же, просто бог знает все, а мы какие-то 
фрагменты. Но важная мысль, что по мере познания мы приближаемся к богу. 
В библии совершенно наоборот: умножая познания ты умножаешь боль. 
У Галилея человек так приближается к богу.

Теперь его научные достижения. Это известные открытия закон свободного 
падения тел, мнение историков расходится, что он там бросал шары 
с Пизанской башни считается нет, и считается, что он делал это как 
мысленный эксперимент. Если мы привяжем к тяжелому шару легкий шар, 
сделаем связку и бросим, получим парадокс -- с точки зрения Аристотеля 
тяжелые шары падают быстрее, но вот от легкого шара должно замедляться. 
Закон инерции тоже плод мысленного эксперимента, но Галилей формулирует 
закон инерции, и хоть входит в 1 закон Ньютона, Ньютон признавал. 
В астрономии первый зеркальный телескоп у Галилея, он сам 
усовершенствовал, и его трубы давали 3--6 раз, использовались на море, 
у Галилея просто поворот идеологический, повернуть и посмотреть в небо. 
Просто потому что. Сам совершенствует телескоп до 30 раз увеличения. 
Получает вид спутников Юпитера, а это уже модель, вот спутники вращаются 
вокруг тела. Неровности на луне, изрезанный рельеф. Милки вей, млечный 
путь наш, не просто белая полоса, а огромное скопление звезд на небе. 
Все это конечно косвенные доказательства, но собранные вместе заставляют 
Галилея доказать, что Коперник и Бруно правы. Ну и пятна на Солнце тоже 
невероятная вещь. Все открытия Галилея папские ребята потом 
перепроверили, но конечно не сделали онтологических выводов. Галилей же 
сделал этот шаг и стал настаивать на новой физике неба. Он пишет в своем 
произведении диалог о системе мира, где три собеседника, два яйцеголовых 
и один простак. Они обсуждают эти системы, приводят аргументы, и в конце 
концов те оказались правы. В 1632 году произведение вышло, а 1633 
состоялся суд над Галилеем. Много было судей Галилея заставили отречься 
на коленях, он покаялся, сказал, что не прав, тем не менее, его осудили, 
он попал под домашний арест до конца жизни, но написал еще книгу, издал 
ее в Голландии. Там уже коперниканский переворот на основе этих двух 
книг можно сказать завершился. Сам Галилей пишет, чтобы не ссориться 
с церковью, что занимается доведением до ясности аргументов Коперника, 
добавляя новые соображения, основанные на наблюдениях неба, чувственном 
опыте. Чтобы принести результаты на суд церкви, чтобы она дала им то 
применение, которое покажется нужным верховной мудрости. В Европе 
большинство отвергло эти идеи конечно, только интеллектуальная элита, 
а многие называли Галилея и сумасшедшим, и дураком. Для завершения 
истории скажем, что в нашем контексте 350 годовщину публикации диалогов 
сделал обращение, что церковь состоит из индивидов, ограниченных в своих 
возможностях, в том числе интеллектуальной. Непогрешим папа, а окружение 
может ошибаться, и церковь пришла к выводу, была создана 1984 году 
комиссия, что церковные ребята ошиблись, заставив Галилея отречься.

%}}}

\subsection{Блез Паскаль}
%{{{

Разработчик классической теории вероятности, теории исчисления 
интегрального, метод аксиоматико-дедуктивный, но вот разочаровался 
в познании. Глубоко верующий, но если Галилей гносеологический оптимист, 
то Паскаль пессимист, считает, что, поскольку верит в теорию Коперника 
и Галилея, космос бесконечен, но и вглубь тоже бесконечность, приходит 
к выводу, что познания не имеет смысла. От бесконечности ничего не 
убудет. Жизнь человеческая коротка, ученым нужно думать прежде всего 
о своей душе, а величие человека осознается признанием ничтожности перед 
всем этим, мыслящий тростник. Две бездны, на одной бесконечности, на 
другой бог, и сердце подсказывает, что надо идти к богу. Если ставите на 
бога, то выигрываете все, поскольку бог обещает вечную жизнь. Если бога 
нет, то проигрываете небольшую часть ничтожной жизни. Большой вклад 
в науку, научная премия большая, кратер на Луне в его честь назван и так 
далее.

%}}}

\subsection{Исаак Ньютон}
%{{{

Биография показывает, что довольно случайно попал в науку, мать хотела 
сделать из него фермера, забрала бы из школы, если б не дядя, который 
настаивал, тринити колледж тоже случайно попал. Открытия свои сделал 
в молодом возрасте, в 40 лет выходит его основной труд: мат начала нат 
философии. У Декарта уже был труд начала философии, и Ньютоновский 
отличается сразу, что мат именно. Известные тезисы его физика что-то 
метафизики, гипотез не измышляю, но понятно у Ньютона и метафизики 
полно, и гипотез достаточно. Перрен это показал. При объяснении природы 
тяготения Ньютон пишет, что силу тяготения не смог природу найти, 
а гипотез не измышляет. Дальше свет открыли, конечную его скорость, 
Ремер попытался его измерить. А у Ньютона раз тяготение мгновенно, то 
превосходит скорость. И Гук, и Гюйгенс критиковали эту идею, но вот из 
этой идеи получалась физика. Кроме того, абс пространство, которое 
Ньютон отождествляет с чувствилище (типа чувства...) бога, абс время, 
абс движение. Историк физики Тредер?? так пишет по этому поводу. ... не 
буду зачитывать слишком долго... Еще важно, что Ньютон описывал 
солнечную систему как стационарную, но не мог объяснить генезис, 
и вынужден был ввести господа бога. Бог на основе закона тяготения 
создает солнечную систему, но чтобы придать ей устойчивость, чтобы не 
падали на Солнце, Бог придумывает центробежное движение, причем так, что 
располагает все планеты по орбитам, и дает щелчок одновременно всем, 
чтобы крутились в одном направлении. Пишут еще, что Ньютон считал, что 
неразрешим в научной позиции, но с теологической был даже рад, что видел 
необходимость повторного вмешательства творца в мироздание. Бог не 
только сотворил систему, но и вынужден подправлять. Импульс должен был 
понемногу исчерпываться, и Ньютон считал, что не может бог создавать 
заново все, а просто каждому атому придает импульс и координату, как 
часто непонятно, но вот бог вынужден подправлять. По этому поводу возник 
спор, дискуссия, кто больше прав, кто больше атеист. У Декарта деизм, 
бог просто придал импульс корпускулам, а дальше они сами. Ньютонианцы 
упрекали типа что это за бог, если не вмешивается вообще. А те упрекали, 
что бог какой-то несовершенный получается, раз сделал какую-то солнечную 
систему и вынужден ее подправлять. Еще добавим, что у Ньютона картина 
мира дискретная, атомы и пустота, в этом контексте происходит развитие 
всей системы, и конечно неточности его небесной механики были 
существенные, он рассматривал только планету с солнцем, а на деле все 
планеты взаимодействуют, конечно теория была недостаточно точна. Но 
достаточно, чтобы можно было много что сделать и предсказывать, например 
полет кометы Галлея. Ньютон сам и алхимиком был, искал всю жизнь 
возможность превращать элементы в золото, и астрологией грешил, и от 
веры был строгим арианцем, тем не менее дослужив директором монетного 
двора, при жизни стал великим ученым, главной фигурой Лондонского 
королевского общества, похоронен в Вестминстерском аббатстве. В 1713 
году письмо отправил Петру первому, 6 экземпляров второго издания 
математических начал.

После смерти Ньютона продолжали конечно критиковать и ученые, 
и философы. Беркли прежде всего выступил против Ньютона, против именно 
сил. Беркли считал, что это возвращение к средневековью, что на деле все 
силы у бога, а наука должна лишь искать связь между явлениями. Это 
сейчас называют феноменологией. На таких же позициях и Юм стоял, 
говорил, что причинно-следственных связей в мире нет, есть принцип 
ассоциаций, ассоциации идей, пространство и время, другие похожие. 
Беркли и Юм эксплицировали 4 важных принципа для классической науки, 
которые сами по себе не являются научными принципами.
\begin{enumerate}
\item Мир существует объективно. Как сказал Беркли, доказать нельзя. 
  Даны лишь наши ощущения, а наши мысли, что это объективный мир -- наши 
  догадки.
\item Мир имеет регулярности. Тоже надо поверить.
\item Регулярности носят устойчивый характер. Не меняются. Мир един 
  в своих закономерностях.
\item Человеческий ум может познавать эти регулярности. Тоже ниоткуда не 
  следует, постулируем и начинаем познавать.
\end{enumerate}

%}}}

\subsection{Другие ученые}
%{{{

Христиан Гюйгенс, открыл спутник Сатурна, Титан. Вместе с Гуком выдвинул 
волновую теорию света. Торричелли ученик Галилея доказал наличие 
давления. Бойль и Мариотт открыли закон тоже. Братья Бернулли. Эйлер. 
И новое направление открылось в конце этого века в парижской академии 
наук -- прикладная математика и судостроительство, теория 
непотопляемости. Это уже целое отделение фр академии наук было, 
разрабатывало это.

%}}}

\subsection{Французское возрождение}
%{{{

\hfill \textbf{Nov 24}

Философы просветители. Деисты: Вальтер, Руссо, Кондильяк, Монтескье. 
Материалисты: Дидро, Гельвеций, Гольбах, Ламетри. Открыто провозгласили 
себя материалистами. До этого все опирались в той или иной мере на бога. 
Идея деистов в том, что бог один раз только вмешался в мироздание, дал 
импульс мельчайшим частицам, все они признают, что атомы корпускулы или 
самые мелкие частицы, а дальше все по закону механики произошло, вот мир 
и работает как часы, как говорил Лаплас. Переводится книга Ньютона 
благодаря подруге Вальтера, мадам Дюшакле??, мат начала нат философии. 
Вальтер сам пишет философские письма, где сравнивает две картины мира 
и вынуждает ученых и мыслителей на континенте сравнивать эти картины 
и через 50 лет фактически картина Ньютона побеждает. Кроме того, Вальтер 
известен как автор идеи всемирной истории: впервые мысль, что несмотря 
на разные скорости развития стран, продвижение идет в одном направлении. 
Прогресс. Вводится понятие прогресса. Мир, несмотря на все бедствия, 
злоключения, развивается в направлении прогресса. С помощью науки. 
Французы сказали, что с помощью науки можно построить царство разума на 
земле. Была конечно белая ворона одна -- Руссо. Он усомнился, что наука 
связана с развитием, говорит, что все научные страны завоевываются. 
Приводил пример Греции, Рима.

Конкретные идеи научные. Кратно, потому что повторяем только. Монтескье 
-- основатель школы географического детерминизма. Указал, что тип гос 
устройства во многом определяется естественными факторами: ландшафт, 
почва, климат. Там, где все это в оптимальном состоянии, появляются 
более развитые культуры, чем в странах, где выходит за рамки. Скудная 
почва плохо, слишком плодородная тоже. Средняя заставляет искать другие 
виды деятельности, диверсифицировать деятельность. Так же с климатом, 
так же с ландшафтом. Где есть естественные образования для границ, там 
растут культуры, а где степи культуры расползаются, более вероятно 
возникает диктаторство. Ну а материалисты и их идеи. Материя понимается, 
как совокупность природных тел и явлений, самодостаточных в своем 
развитии. Движение воспринимается как движение вообще, никаких начальных 
импульсов не нужно. Где материя, там движение. Движения тоже разные, от 
механических до социальных. Развитие есть в природе, не только движение. 
Развитие от неорганики к органике. Космогенез. Постулируется первая 
идея, что космос тоже не стационарный. Земля тоже, говорят, не сразу 
образовалась, были значительные периоды, даже спор был между 
нептунистами и вулканистами. Одни говорили, что главный фактор 
в становлении Земли были вулканы, и сейчас тоже не отвергают, а другие 
считали, что колебания уровня океана существенно меняли рельеф земли, 
в Альпах находили ракушки... с какой стати в Альпах ракушки, что-то 
другое значит там было. Ну и развитие от неорганики к органике. Дидро, 
например, решил, что жизнь изначально есть в материи, он говорил даже 
``и камень чувствует''. В муке мошки образуются, из грязи черви 
вылезают. Только в 19 веке Пастер доказал, что все живое на земле 
происходит только от живого. Хотя гипотеза нашего академика Апарина?? 
могло произойти нечто, что из неограники могла произойти органика. Но 
это все еще гипотеза лишь. Развитие животного мира тоже у Дидро 
присутствует. Дидро говорит о связи электрических и магнитных явлений, 
первым из философов высказал, что они взаимосвязаны. А теория эволюции 
почти как у Импедокла, что могли происходить всякие уроды, гермафродиты, 
а природа сама выбраковывала, выживали сильнейшие, естественный отбор 
присутствует. Дидро первый указал на приматов и на людей, но у него идея 
была, что приматы это выродившиеся люди. Пространство и время понимаются 
как объективные формы существования материи, а движение если оно везде 
и всюду, то что-то там. Причинность тоже считают, что в мире существуют 
объективные причинно-следственные связи, и это есть показатель 
творческой мощи природы. Все силы в самой природе, именно поэтому не 
просто в движении, а развивается. Идея причинно-следственной связи 
подчеркивал Гольбах в знаменитом произведении система природы. Долгое 
время звали библия материалистов. Церковь сразу включила в индекс 
запрещенных книг, Гольбах стал врагом церкви. Гольбах стал предвестником 
детерминизма Лапласовского типа. В каждом облаке пыли, в каждой волне 
океана все частицы совершают определенные движения, и если бы могли их 
вычислить, знали бы все.

Еще идея единства всего живого. Знаменитая работа Ламетри человек -- 
машина и человек -- растение. У Декарта например все животные и растения 
это автоматы, а психическими особенностями наделен только человек. 
Ламетри как врач пишет, что все высшие познавательные функции должны 
произрастать из низших функций. И здесь должно быть развитие. Если 
животные машины, то и человек машина. Если у человека психические, 
когнитивные функции, то и у животных не запрещены в результате эволюции.

В заключение скажем, что просветителями они называются потому что делали 
ставку на науку, но и на образование. Гельвеций прямо пишет, что 
показатель развития общества является образование народа. Нельзя 
образовывать по принципу крови. Ум, мозг, человеческие способности 
познания так сложны, что мы не можем по внешним признакам говорить, кто 
способен. Нужно предоставлять равные вещи, а потом увидим, кто более 
способен, кто менее. Тоже очень важная прогрессивная идея, как можно 
шире дать всем возможность образовываться. Их педагогическая 
деятельность образовательная это написание энциклопедии, первой научной 
энциклопедии, научной можно сказать, раньше только церковные. Даламбер, 
Дидро. 20 томов с 1751 по 1772 год, в том числе в России была. Знакомили 
широкого читателя с наиболее важными достижениями науки. Сыграла большую 
роль в деле секуляризации. Книгу природы надо научиться читать, для 
этого тоже энциклопедия. Впечатляющие цифры в развитых странах того 
времени насчитывалось 250 тыщ учащихся в одних лишь воскресных школах 
в Англии. Во франции 80 тыщ вроде. Так вот это натурфилософские идеи. Не 
зря говорят, что они подготовили великую Французскую революцию. А они 
перевернули можно сказать Европу, систему права, конституцию.

Коротко о естественнонаучных исследованиях 19 века. Первая промышленная 
революция была изобретателей, которые опирались на ежедневный опыт, 
знание изобретательного типа, не теоретическое знание, а утилитарное. Но 
тем не менее изобретения были впечатляющие. Принципиально новые 
устройства, которые собственно делались по законам механики. Не было 
конечно чертежей, но тем не менее. Конструируют паровую машину! 
Происходит промышленный бум, огромное количество сфер. Хотя объяснять 
начали только совсем позже. Вот был толчок, но дальше пошла уже 
дифференциация научных направлений, а затем и дифференциация наук. Уже 
не идея Декарта, что ему удобно осваивать все науки, а происходит 
дифференциация. Физика и математика. Вводится принцип наименьшего 
действия. Даламбер рассматривает время как четвертую штуку. Лагранж тоже 
переход в 19 век, его знаменитый труд аналитической механики. Ну 
и Бенджамин Франклин, первый ученый на Американском континенте, 
самоучка, известен своими работами по электричеству, его изобретения 
молниеотвода, первое доказательство однородности атмосферного 
электричества и эл. от трения. Введение символики плюса и минуса 
в теории электричества. У него предложение, которое вошло в географию 
назвать течение Гольфстрим. В области политики он конечно один из 
авторов американской конституции и создатель первого философского 
общества в Америке. Гальвани итальянец тоже открыл электричество 
в опытах над лягушками. Кулон тоже с его знаменитыми крутильными весами, 
точечные заряды, закон его, очень идентичный закону всемирного 
тяготения. Александр Вольта тоже еще один источник электричества, 
химический. Ну и Лаплас. Самый большой научный чел того времени. 
Разработка небесной механики. Устанавливает почему все-таки можно 
обойтись без божественной силы для устойчивости солнечной системы. Ну 
и его развитие идей Гольбаха и введение в физику того демона, которого 
стали называть демона Лапласа. Он пишет, ``ум, которому было бы известно 
все...'' Его путь ученого был необычный, потому что он жил в непростое 
время, выходец из крестьянской семьи, даже потом оказался от своих 
родителей, тем не менее успешно занимался наукой в условиях феодального, 
Наполеона, реставрации власти Бурбонов. Политическая индифферентность 
порой препятствовала занятиям наукой, но он сумел ужиться при всех этих 
режимах. Хотя он разными решениями был выведен из метрической комиссии 
со словами слишком слабая ненависть к тиранам. При Наполеоне занимал 
должность министра внутренних дел, но потом уволил со словами, что он 
рассматривал мир с точки зрения бесконечно малых...

Первая теория происхождения вселенной, вернее солнечной системы, от 
атомов и всего такого. Биологические и химические исследования, физика 
математика, космология, биологические, химические. Карл Линей, шведский 
человек, классификация живых организмов, работа системы природы, где 
изложил основные принципы систематизации, которые и сейчас в ходу. Там 
как раз роды, виды, отряды, царства. Ввел понятие видов, которые 
и сейчас основную роль играют. Более 10 тыщ видов растений, 4400 видов 
животных описал. Латинские названия дал многим конкретным видам, 
а главное поместил человека в один отряд с приматами, введя название 
homo sapiens. Главный недостаток конечно в том, что он предполагал, что 
все виды изготовлены богом, а поэтому стабильны и не могут смешиваться. 
В 36 томах кто-то еще там издал. Автор теории катастроф, согласно 
которой жизнь резко меняла свои формы в результате глобальных 
катаклизмов планетарного масштаба. Лавуазье дальше опирался на открытие 
кислорода, Пристли. Его считают уже автором научной теории горения 
кислорода или чего-то там. В гуманитарном не будем говорить, там теория 
Руссо общественного договора. Как раз в начале в обществе все было 
хорошо, но кто-то оградил участок и сказал это мое. Дальше пошло 
поехало, частная собственность начала раздуваться, а железо и зерно 
испортило человека, вся конкуренция, ну и после этого появляется 
государство. Но встает на сторону богатых, а люди нищают, потом 
революция, а потом народ получает снова суверенные права и уже должен 
следить, чтобы никто суверенные права не забрал, и народ имеет право на 
свержение правительства, если оно перестает отвечать нуждам большинства. 
Хотя его понятие общей воли тоже метафизическое представление, что 
оппозиция будет лишь тормозить. Не произойдет от общей воли снова 
диктатура. Из теории Руссо могла вырасти и собственно выросла 
Гитлеровская диктатура. Никакого переворота не было, чтобы Гитлер пришел 
к власти. Отметим также из общественно-полит теорий Томаса Мальтуса. 
Английский экономист, лондонское королевское общество, франц академия 
наук, основатель лондонского статистического общества. Опыт о законах 
народонаселения. Рост народонаселения ограничен соц и природными 
катаклизмами. Прогресс не компенсирует истощение почвы. При 
благоприятных условиях рост населения в геометрической прогрессии, 
а развитие лишь линейно. Существование нищих и богатых что-то там тоже. 
Все тезисы обсуждаются и сейчас. Сейчас даже говорят 
о неомальтезианстве, потому что тот говорил об отдельных странах, а есть 
смысл о всем мире. Борьба бедных и богатых неизбежна. Если в богатых 
странах накал можно снизить, то между странами, между югом и севером, 
никак. Переселение народов в Европу ни к чему хорошему не приведут.

%}}}

\subsection{Генезис науки в России}
%{{{

Коротко о генезисе науки в России... Императорская академия наук создана 
в каком-то году на основе казны государства. Ломоносов прошел 
образование в Германии. Почетный член шведской королевской академии 
и академии наук Боллонского института. Был ближе к Декарту, а не 
к Ньютону. Также считал, что язык науки -- математика, придерживался 
Галилеевской теории двух книг. Закон сохранения вещества. Явление 
рефракции света. Атмосферное электричество. В опытах погиб его помощник 
Рихмант. Работа о приумножении народа российского.

%}}}

\subsection{Выводы}
%{{{

Становление нового типа мировоззрения: наукоцентризм. Задали французские 
просветители. Наука определяет вектор человеческого прогресса. 
Становление механистической картины мира. Определены базовые эталоны, 
идеалы из механики, сама наука понимается как самоорганизующаяся 
система. Готовит сама себе кадры из первокружков до академий. Появляются 
разные организации в Англии, Франции.

В философском можно выделить следующее.
\begin{enumerate}
  \item Утверждается онтологическая значимость теории Коперника. 
    Устраняется качественное различие между физикой земли и физикой 
    неба. Переход от геоцентризма к гелиоцентризму. Пространство и время 
    как объективные однородные сущности.
  \item Эпистемическое учение -- обыденное знание на уровне чувств -- 
    может существенно отличаться от научного познания. В мире может быть 
    много чего непознанного, но не принципиально непознаваемого. Никто 
    не может поставить пределы человеческому разуму. Становление 
    классической механики. Претендует на редукционизм. Желательно 
    вывести все и вся из простейших законов механики. Метафора мир -- 
    машина, а еще более точно часы. Где все в конечном итоге сводится 
    к законам механики.
  \item Познание в целом понимается как кумулятивный или аддитивный 
    процесс. Наука строится на основе прочного раз и навсегда 
    заложенного фундамента. О Ньютоне стали говорить, что он счастливый 
    человек, потому что фундамент можно только один раз создать. Наука 
    рассматривается как богоугодное дело, все ученые в основном, кроме 
    франц материалистов, верующие люди. Происходит постепенное 
    размежевание науки с философией. Не просто натурфилософия, а наука 
    и философия. Философия признается как основа и истоинчк научного 
    познания, операционализация. Знания это то, что можно изложить 
    с математики. За философией квалификационная характеристика, а за 
    наукой квантификационная.
  \item Переосмысление научного опыта, начиная с Френсиса Бэкона. Не 
    наблюдение, а постановка лабораторных испытаний и опытов. Развитие 
    базы науки, термометры, хронометры, другая чепуха. Бэконовский 
    лозунг начинает действовать. Наука развивается по мере развития 
    базы. Кроме того, в опыт включается и мыслительный эксперимент, 
    мысленный. Мы это отмечали уже и Декарт пробовал построить 
    с мысленными экспериментами, а Галилей явное прямолинейное движение, 
    закон падения. Субъект и объект точно и четко различаются, 
    считается, что субъект не влияет на объект, хоть и ставит в разные 
    условия. Но познает сам объект, как он сам по себе существует. 
    Изменится в 20 веке с квантовой физикой.
  \item Аксиология -- верх большинства ученых бога как служба богу через 
    науку. Наука рассматривается позитивно, как сила власть над природой 
    с целью прогресса человеческого общества, наука должна принести 
    плоды через различные прикладные исследования. Профессия ученого 
    становится престижной, оценка государства, престиж государства 
    соизмеряется с достижениями в научной сфере и в сфере образования.
\end{enumerate}

Подытоживая это, наш академик Степин говорит. Дисциплинарно 
организованная наука с 4 основными блоками математика естествознания 
техническими и соц гум науками завершила долгий этап формирования науки. 
Сложились внутридисциплинарные и междисциплинарные механизмы, обеспечили 
прорывы в ноые предметные миры. Прорывы открыли новые возможности для 
инноваций в различных сферах человеческой деятельности.

%}}}

%}}}

\section{Философиские и естественнонаучные программы 19--20 веков. 
Переход от механицизма к электродинамической картине мира}
%{{{

Сначала наиболее важные идеи перечислим. В первую очередь немецкая 
классическая школа вплоть до Маркса и Энгельса. Потом Марксизм в России 
в лице Плеханова, Ленина, Богданова. Материализм эмпириокритицизм 
Ленина. Потом перейдем к научным достижениям времени, на основе которых 
происходит Фарадей и Максвелл, введение в науку принципиально нового 
понятия поля, которое не совмещается с понятием атомов и пустоты.

Начинается с Канта, его начала научных работ, история и теория неба. 
Затем в целом ряде работ, начинающиеся со слова критика, мы тоже что-то 
скажем. Он опередил Лапласа в идее происхождения солнечной системы из 
туманности. Работа по Лиссабонскому чему-то, физика математика 
астрономия все есть. Молодой самонадеянный Кант говорил дайте мне 
материю, и я покажу, как из нее должен возникнуть мир. Но он понимал, 
что жизнь принципиально другое, и говорил, что нужно что-то сверх 
механики. Юм пробудил его от догматической спячки и заставил задуматься 
как это оно вообще такое вот возможно. А что было дальше потом скажем.

\hfill \textbf{Dec 1}

На предыдущей лекции вспомнили франц материалистов, две группы: деисты 
и материалисты атеисты, основная группа. Первая группа Вальтер много 
сделал для пропаганды Ньютона на континенте. Основные идеи конечно 
у материалистов. Материи, бесконечность ее, движение, развитие, 
совокупность всех тел природы. Движение атрибут материи, пространство 
форма?? движения. Идея развития на уровне космогенеза, жизни на земле, 
Дидро, что жизнь фактически всегда присутствовала на земле (и камень 
дышит). Их тоже понимание причинности у Гольдбаха в духе Демокрита -- 
гносеологический статус. Пропагандистская деятельность выродилась 
в написании многотонной первой научной энциклопедии. Затем коротко 
остановились на конкретных программах в науке. Это, начиная от физ мат 
блока, Даламбер, Ранк??, Лаплас, ключевая фигура периода, обошелся без 
бога, чтобы обосновать устойчивость солнечной системы. У Ньютона был 
необходим, Лаплас первый показал, что можно обойтись без него. Дальше 
социальные тоже науки, Руссо теория общественного договора, Мальбуса, 
первое серьезное разъяснение о демографии земной, Запасы земли 
ограничены, рост населения и ограничение природы, борьба за 
существование. Имея в виду предыдущие наши лекции по теме, сделали 
выводы по формированию первой научной картины мира. До этого был только 
Аристотелизм как картина мира, а тут первая механистическая картина. 
Основана на классиках науки и франц материалистов. Напомним идеи.
\begin{itemize}
  \item Онтологическая значимость теории Коперника. Никто больше не 
    сомневался.
  \item Эпистемологическая значимость, здравый смысл, обыденные знания, 
    существенно расходятся с теорией.
  \item Понимание познания как кумулятивного процесса.
  \item Переосмысление понятия опыта. Уже не созерцание, не приобщение 
    к гармонии мира.
  \item Аксиология. Познание рассматривается как необходимый элемент 
    развития общества. Французы эти заявляли, что с помощью науки можно 
    построить царство разума на Земле. Те, кто верил в бога, кроме 
    Паскаля, считали, что это богоугодное дело, и чем больше познаем, 
    тем ближе к нему.
\end{itemize}
Дальше перешли к новой теме, это текущая тема как раз. Переход от 
механицизма к электродинамической картине мира. Начинается с Канта, 
логика тоже будет, идеи философские, связано с процессом познания, 
а потом посмотрим на конкретные, более разветвленные, научные прежде 
всего программы.

\subsection{Кант}
Начинается с Канта. Канта докритического. Ключевое произведение -- 
всеобщая история и теория неба. Опережает Лапласа с небулярной теорией 
солнечной системы (из туманности), занимается сейсмологией, 
метеорологией и прочим. Исходит из объективности мира, метод индукции 
правомерный, наука на основе этих методов познает мир. Но затем как он 
сам признает идеи Юма, что наука преувеличивает, когда говорит 
о причинно-следственных связях, они ассоциативные, по сходству, 
контрасту итд. Кант задумывается и в конце дает три критики. Чистого, 
практического разума и способности суждения. Пытается доказать 
возможность теоретического познания. Физ мат блока.

Критика чистого разума. Три вопроса: как возможна математика как наука 
(положительный), как возможна физика как наука (положительный), 
трансцендентальная диалектика (метафизика -- отрицательный).

Первые две части, эстетика и что-то еще, считает, что обязательно должны 
в нашем сознании, мышлении, существовать трансцендентальные структуры. 
Априорные и синтетические. Априорные не зависят от опыта, хотя можно 
с помощью опыта прийти. Синтетические поскольку дают приращение знания. 
$7+5=12$ в арифметике, можно на пальцах, на камушках, прочее. Но поняв 
это видим, что от нас не зависит. Синтетическое потому что потому. 
Трансцендентальная структура для математики это время. Любой счет 
происходит во времени. Самый мощный компьютер все равно во времени. Но 
время само никакого отношения не имеет. Счет происходит во времени. 
Время есть главная трансцендентальная априорная структура, которая 
делает возможной любой счет. Так же геометрия, можно измерять углы 
треугольника, проверять на опыте, но убедимся, что от опыта не зависит. 
Тоже априорное синтетическое. Главной структурой априорной тут уже 
является пространство. Любые построения происходят в пространстве. Таким 
образом, трансцендентальная эстетика показывает как возможна математика.

Трансцендентальная аналитика дальше. Качества, количества, модуса, 
отношения. Каждая делится на 3. Эти категории также априорные, в нашей 
голове, но тоже можем приходить через опыт. Приходит мысль, что все это 
возможно, что существует все благодаря количеству, при счете. И так 
с остальным. Кант сравнивает категории с сеткой, которую мы набрасываем 
на мир. Или с картой, глобусом, где нанесены параллели и меридианы. 
В природе их нет, но для ориентирования наносим. Категории есть те же 
параллели, меридианы, итд, в нашем сознании. Никакая физика невозможна 
без оперирования этими категориями. Это трансцендентальная аналитика.

Третья это трансцендентальное единство апперцепции. Единство 
самосознания. Единство рефлексивного мышления. Человек не только познает 
мир, но и осознает, что познает мир. Это осознание и есть т.е.апп. Это 
некий плавильный котел. Одновременно эстетика, аналитика, осознание 
процесса познания. Субъект играет ключевую роль в процессе познания. 
Кант пишет, что мы не столько отражаем законы природы, сколько 
навязываем ей их. С помощью априорных структур.

Трансцендентальная диалектика, метафизика, невозможна. Замахиваемся на 
познание целостности вообще. Вещей самих по себе. Ставим вопросы почему. 
Наука начинается с вопроса как. Вещь-в-себе -- ноумен, вещь для нас -- 
феномен. Наука должна ограничиваться познанием феноменов, как связаны 
эти явления. Если вопрос почему, в чем сущность вещи, то выходит за свои 
пределы. Любой объект науки дан в наших ощущениях, а если выходим за 
это, то выходим в метафизику. Кант демонстрирует это на антиномиях. Мир 
не имеет начала во времени -- что было до начала? пустое начало? как от 
него перейти к нашему времени?. Если не имеет начала -- как наступило 
настоящее? до нас успела пройти бесконечность. И еще три похожих 
антиномии. Кант все равно говорит, что попытка выйти за феномены -- 
необходимое свойство человека. Метафизика нужна как воздух человеку, 
метафизика становится разведчиком для науки. Ставит перед наукой 
вопросы, а наука потом идет и пытается перевести метафизику в науку. 
Знаменитый пример с атомами. Долгое время были конструкцией, а в конце 
19 века стали рабочим понятием науки.

Таким образом, Кант строит свою теорию познания и показывает место 
реальных наук, это для него физика (одна -- Ньютона), логика (одна -- 
Аристотеля), мораль (одна -- его собственное изобретение, категорический 
императив). Про императив не будем повторять.

\subsection{Фихте}
Фихте пытается улучшить систему Канта и создает эпистемологию, которую 
называют эпистемологический конструктивизм. У Канта есть априорные 
структуры, которые даны, и вещь-в-себе. А Фихте считает, что мы все 
конструируем. Показывает, что абсолютно тождественного не существует, 
даже закон логики А = А тоже. Фихте говорит, что человек осознает Я, 
а потом, отталкиваясь от себя, осознает остальной мир -- не-Я. 
Соединение Я и не-Я дает супер-эго, осознание всего мира. Единственной 
априорной функцией является функция конструирования. Конструировать себя 
и мир и остальное... Фихте делает важный вывод -- ни одна наука не может 
обосновать саму себя. Потом теорема Гёделя подтвердит это. В этом 
состоит эпистемологический конструктивизм.

Отметим важную идею о роли ученого в обществе, встает в полемику 
с Руссо. Знаменитая лекция о роли ученого. Прямо показывает, что ученые 
с одной стороны в ссоре земли, высший интеллектуал, потенциал нации, 
государство должно выделять максимум внимания слоя, а с другой стороны 
на ученом лежит ответственность за те знания, которые он получает. 
В конце концов всем понятно, что знания получаются для использования на 
практике. Наука делает лишь первые шаги, а Фихте уже задумывается, что 
продукт познания всегда имеет двойственный характер: как во вред, так 
и в добро. Ученые должны сами отвечать со стороны. Могут быть навязаны 
решения, но сам ученый несет ответственность. И еще отмечает 
просвещенческую деятельность ученых. Ученые должны нести знания в массы. 
Бороться с предрассудками, популяризировать научные знания, чтобы 
ускорить развитие общества.

\subsection{Шеллинг}
Фридрих Шеллинг. Основное произведение -- идеи философии природы 
и система трансцендентального идеализма. Середина 19 столетия, когда уже 
многие открытия были в области электричества, Гальвани, Гольбах, 
Гансье??, в химии тоже успехи. Приходит к мысли, что априорность 
и трансцендентальность лежит в самой природе. В природе лежит творческая 
сила. Ученые лишь открывают. Главными трансцендентальными структурами 
природы Шеллинг называет закон полярности материи (притяжение 
отталкивание, северный южный полюса, плюс и минус, адаптации 
и изменчивости в биологии). Полярности обусловлены законом полярности, 
говорит. На основе него можем открывать и классифицировать различные 
формы самого окружающего мира. На основе этого у Дидро это была просто 
гениальная догадка, а Шеллинг приходит к мысли, что эл. и магн. явления 
должны быть неразрывно связаны. Кроме того говорит, что природа на 
основе принципа полярности развивается. Вводит эволюцию -- 
разворачивание диспозиций в природе из точки неразличия. Из точки 
неразличия следует разворот, эволюция начал. Может бы и обратный виток 
-- инволюция. Схлопывание полярностей в точку безразличия. Сейчас 
называем теорией пульсирующей вселенной... Это у нас Шеллинг. Пантеист.

\subsection{Гегель}
Шеллинг это конечно пантеист. Последний великий пантеист, в природе 
видит креативное начало, а система Гегеля конечно последняя великая 
система объективного идеализма. До него Платон и Лейбниц были в такой 
сфере, а Гегель венчает направление развития объективного идеализма. 
Постольку поскольку это идеализм, вводится исходное идеальное начало. Их 
может обнаружить только философия. Философию называет царицей наук, ибо 
только она может обнаружить такие вещи. Произведений у него много, 27 
томов. Главная из них -- феноменология духа, прослеживает эволюцию 
индивидуального сознания от примитивных до общественно значимых форм. 
Энциклопедия?? где рассматривает и философские вопросы, и науку.

Общая схема. Идеальное исходное начало -- абсолютный разум, абсолютный 
дух. Первая стадия развития это развитие идеи в самой себе, как раз то, 
что называется наука логика, начиная от понятия чистого бытия, которое 
равнозначно ничто, а их синтез дает категорию становления. Понятие 
сущность и понятие... Категория количества, качества, синтез дает 
категорию меры, форма, содержание, причина, следствие, больше ста 
категорий, которые идут последовательно. По ниточке через все его 
произведения. Заканчивается диалектической логикой, сформулированной 
в трех законах.
\begin{enumerate}
  \item Источник развития. Закон единства и чего-то противоположностей. 
    Везде есть зачатки противоположных начал, которые развиваются, 
    вступают в противоречия, взаимодействия, появляется в результате 
    нечто новое. Его упрекали, что противоречие не может таким образом 
    развиваться, он говорил, что весь мир соткан из противоречий 
    и смешно думать, что мы не можем их мыслить. Исходное противоречие 
    -- духа.
  \item Переход количественных изменений в качественные. В чем состоит 
    механизм развития. Точка бифуркации, происходит скачок, нарушение 
    меры. Объект или процесс переходит в новое качество. Иллюстрирует 
    это на примере фазовых состояний воды, где простейшая вещь точка 
    кипения, замерзания, а потом вода переходит вот так вот.
  \item Есть ли, а если есть, каково направление развития? Закон 
  отрицания отрицания. Есть голое отрицание. Зерно перемололи, 
    образовалась мука, зерна больше не будет. Диалектическое отрицание 
    -- зерно сажается в землю, из него вырастает стебель, из стебля 
    колос, а там снова зерна. Не старое зерно, но тем не менее. 
    Произошел виток в развитии. В таком частном примере вывод, что любое 
    развитие имеет форму спирали. Каждый последующий виток как бы 
    повторяет исходные положения, но на новом уровне. Совсем не как 
    раньше через линейную схему или еще как-то.
\end{enumerate}
Эти три закона диалектики, диалектической логики, выводятся из духа, 
иллюстрируются на примерах, взятых из обыденной жизни так сказать.

Чистая логика при сотворении мира. Гегель подчеркивает принципиальное 
отличие диалектической логики от формальной. Он не отрицал формальную 
логику, только говорил, что формальная -- рассудочная. Рассудка. Без нее 
не обойтись в обыденной жизни. Чтобы понимать друг друга, люди должны 
пользоваться ей. Диалектическая -- открытия. Разума. Ее достижения -- 
высшего уровня теор достижения. Без разума рассудок нечто. Но без 
рассудка разум ничто. Необходимы два этажа в развитии теор мышления. 
Первый этаж развивает формальную логику, обязано для каждого. Логика 
рассудка уже высшая, с помощью нее человек может заниматься 
строительством теорий. Сравнивает с обычной математикой и высшей 
математикой. Высшая математика как раз необходима для построения высшей 
математики. Диалектическая есть диалектическое отрицание формальной??

Дальше второй виток -- отчуждение высшего разума для природы. Инобытие 
в природе. Философия природы в энциклопедии его так названа. Дух 
привносит природу. Дух порезвился в природе. Гегель выделяет структуры 
такие. Механика, философия природы, наука занимается проблемами 
пространства, времени, движения. Первым критикует Ньютона за понятие 
абсолютного движения, пространства, эталона времени. У Ньютона на фона 
абсолютов происходит развитие и движение природы. Гегель отрицает 
развитие природы. Движение от духа только. Гегель пишет, что атомы 
и пустота это варварские идеи. Основа всего -- идея. Как в ней могут 
быть атомарные структуры? Это механика. Теперь физика. Отрицание 
атомистической природы приводит к экзотическим объяснениям физический 
явлений. Например свет. Есть призма Ньютона, свет делится на цвета, 
преломляется, свет оказывается сложная субстанция. Гегель неприемлет 
это. Для него свет -- простейшее физ явление. Народная субстанция. Самая 
простая мысль под формой природы. Звук это механическая душевность, 
а магнетизм -- проявление умозаключений в природе. Все силы носят 
духовный характер у Гегеля... Призывает вернуться не к атомизму, 
а к трем элементам природы из Греции. Тем не менее в области химии 
Гегеля считают предвестником периодической таблицы. Последняя структура 
природы -- органика. Разделение на растительную и животную природу. Но 
для объяснения самой сути вынужден предположить, что дух, который 
существует в живых организмах, существует и в неорганической природе. 
Называет точечные живые организмы.

Третья стадия -- возвращение духа к самому себе. Снова триада, как 
в тезис-антитезис-синтез. Война-мир-война. Мир лишь пауза между войнами. 
Война показатель развития. Так вот третья стадия духа. Возвращение 
к самому себе через формы культуры. Такие как право, мораль, философия, 
религия, наука. Все это развитие духа на третьей стадии. Анализ роли 
личности в истории, но это уже не трогаем.

\subsection{Фейербах}
Последний крупный философ. Сначала был Гегельянец, но потом 
разочаровался, но стал материалистом. Единственный материалист 
классической немецкой философии. Критику религии не рассмотрим. А вот 
ценная идея в нашем курсе это введение термина антропология и, в отличие 
от других философов, акцент внимания и на чувственно телесной природе 
человека. Не просто cogito ergo sum, но и соматические процессы, 
мыслительный и телесный комплекс всегда нужны. Человек может становиться 
объектом науки. Знаменитое Сократовское познать самого себя это благое 
пожелание. Раз объект науки, нужно изучать полноценно всеми методами. 
Сейчас понятное дело много что вот так вот так сказать ну вы поняли.

\subsection{Марксизм}
Сначала Маркс построил исторический материализм, а потом Энгельс 
построил систему диалектического материализма. Но диалектический конечно 
первичен, поскольку претендует на большие вопросы. Есть идеалистическая 
диалектика, как у Гегеля, а есть материалистическая, марксистская, где 
говорят, что диалектические законы присущи самой природе. Маркс говорит, 
что у Гегеля диалектика стоит на голове, надо поставить на ноги. 
Диалектика -- наиболее общая наука о природе, обществе, мышлении. Дальше 
общие законы науки, естествознания. Закон сохранения энергии. Частные 
научные законы, закон Бойля Мариотта. Конечно и Маркс и Энгельс, и Ленин 
работают середина 19 века и позже?? Клетка открыта уже. Доказательство 
схожести всего живого на земле. Закон сохранения энергии, которые 
открывали всевозможные люди. Теория Дарвина тоже Энгельс строит свои 
рассуждения во многом на теории Дарвина. И в теории Дарвина можно найти 
диалектику. Периодическая система Менделеева тоже (конец 19 века). И так 
далее...

В высшей стадии здесь считается материализм и эмпириокритицизм, там 
и радиоактивность, и электрон, и все самое такое вот. Классическое 
определение материи: философская категория для объяснения внешнего мира, 
существует объективно и воспринимается органами чувств. На 
``исчезновение материи'' при радиоактивности говорил, что материю 
неправомерно связали с веществом. Говорил, ``электрон так же 
неисчерпаем, как и атом, материя бесконечна.'' Кроме того, в этой книге 
Ленин показывает развитие самого аппарата чувствительности и когнитивных 
структур это тоже процесс, и сознание это тоже не данное богом, а высшая 
функция такой материальной структуры как мозг. В то время были уже 
вульгарные материалисты, которые поспешили отождествить мышление 
с разными выделениями разных органов. Мозг выделяет мысль так же, как 
печень выделяет желчь. Этих натурфилософов Ленин называет вульгаризмом 
и говорит, что перед наукой еще очень большой путь, чтобы показать, как 
из неживого переход осуществляется к живому.

В самом общем плане напомним, что дается в диалектическом материализме 
классификация форм движения. Основные идеи из франц материализма. 
Движение неразрывно с материей (не нужны первоначальные толчки). 
Пространство и время это форма существования материи. Но развитие 
материи можно структурировать, и Ленин повторяет сетку Энгельса: 
механическая форма, физическая, химическая, биологическая, социальная. 
Существует иерархия. Каждая последующая надстраивается, и социальная 
вбирает в себя все предыдущие формы. А второй принцип нередуцируемости. 
Любая высшая форма движения не сводится к низшей. И именно это говорит 
об особенном и новом в этом движении. Ставит крест практически на теории 
всего и тем более на Ньютоновской мечте.

Причинность понимается в двух формах. Лапласа (динамическая) 
и статистическая. В термодинамике одно время полагалось, что все равно 
можем свести к законам механики. Фундаментальными признавались тем не 
менее динамические, как в механике. Но в 20 веке кванты и переворот. 
Фундаментальные статистические, а динамические это определенные 
округления. Теория отражения -- фундаментальная теория, на которой 
строится познание диалектического материализма. Отражение есть на всех 
формах существования материи. Свет босой ноги на листке. Или зеркальное 
отражение. Но переход от физ формы отражения к простейшим формам неживой 
природы это тоже барьер, который до сих пор преодолеть не удалось. 
Когнитивные это раздражимость (самая простейшая), чувствительность (чуть 
выше, более широкий спектр организма задействован) и, наконец, 
чувственные формы отражения (дифференциация сенсорной системы, разные 
колебания, у одних одно, у других другое, у человека свои барьеры 
восприятия). Итак, это чувственная форма познания -- ощущения отдельных 
воздействий из отдельных объектов, вкус итд, восприятия (комплексная 
сенсорика, а не отдельные ощущения) и представление (третья форма, когда 
можем воспроизвести в памяти объект без непосредственного контакта). Это 
эмпирическая ступень познания. Рациональная теперь. Понятия -- 
абстракции, общие, важные элементы, в реальности нет, но на основе них 
можем охватывать много. Суждения -- понятия не сами по себе, а всегда 
связаны, что-то утверждаем, всегда есть субъект и предикат (стол стоит 
в аудитории, субъект стол, предикат где находится). Подчеркивается, что 
чувственное и рациональное всегда связаны. Только в теории познания 
разделяем. Но ощущение красного еще не есть знание. Знанием становится 
тогда, когда говорим, что видим красное. С одной стороны переплетение 
чувственного. С другой стороны формулирование суждений, никаких суждений 
в мире нет, это наша конструкция. Обыденная ступень познания.

Высшая ступень познания -- наука. Здесь тоже свои определенные элементы. 
Наука начинается с фактов. Фиксирование на языке науки объективное 
положение дел. Это красное уже не удовлетворяет науку. Будут определять 
красное в рамках того, что принято считать красным. Дальше ступень -- 
гипотеза. На основе фактов не просто коллекции, а обобщения и выдвижение 
гипотез. На одном и том же объеме фактов могут быть построены разные 
гипотезы, как случилось в электродинамике. Третья ступень уже теория. 
Любая наука в конечном счете стремится к построению теории. На 
специфическом языке науки. Можно объединять и получать мировоззрения, 
картины мира и остальное, о чем говорил Планк.

Критерии истинности, а наука всегда ищет истину, диалектический 
материализм ищет в практике. Это опыт и эксперимент, но высшая практика 
это социально-общественная. Классический пример это развитие атомной 
теории. Догадки и теории, затем опыты Резерфорда, но а потом атомные 
бомбы, электростанции, ледоколы и прочее. Последние убедительные 
доказательства теории.

\hfill \textbf{Dec 8}

Спросить у кого-нибудь конспект.

Гегель.

Иррационализм (Кьеркегор, Ницше, Бергсон, Фрейд, Шпенглер).
Рационализм только европейская культура, но на закате говорят...
У Кьеркегора страх перед смертью, ничего не меняется от теории Коперника 
или Птолемея, ничего не меняется для души.
У Шопергауэра наука способствует разрушению человеческого общества, 
создает предпосылки закон опережения потребностей, вывод все к худшему 
в этом худшем из миров.
У Ницше наука лишь проявление науки, белокурая бестия по мнению Ницше не 
ученый, это другой тип существа.
У Бергсона разведение понятий длительности и одновременности, 
геометризирует время, время становится 4-й координатой, 
кинематографический метод, движение иллюзорно, истинное время наука 
с точки зрения Бергсона не понимает. Подчеркнул, что идеи Бергсона нашли 
отражение у Пригожина в философии науки 20 века. Он подчеркнул, что 
в фундаментальных уравнениях идет как оператор, с плюсом или с минусом 
разницы нет, а стрела только в термодинамике и космогонии.
У Шпенглера также еще предупреждение человечества о машинизации культуры 
и возможность попадания человека в зависимость от мега-машины.

\hfill \textbf{Dec 15}


\subsection{Гуссерль}
Эдмунд Гуссерль. Создатель современной феноменологии. Феноменология еще 
с Бэкона, Беркли и Юма??, которые считали, что наука должна заниматься 
явлениями, объяснять последовательность явлений.

Гуссерль идет дальше. Идея идет от Бэкона с теорией идолов. Идолы 
остаются у ученых до сих пор, желают причинно-следственные связи. Нужно 
избавиться с помощью феноменологического анализа. Чтобы усмотреть 
незыблимые сущности в самом сознании субъекта. Ноэмы. Аналогии 
с Декартом, у которого человек рефлексирует и обнаруживает. Процесс 
выявления этих сущностей ноэзис. Очищение сознания от идолов с целью 
оставить только вечные сущности. Логико-мат сущности, как и у Декарта. 
Остальное вынести за скобки. Называют это эпохэ. Интеллектуальная 
интуиция. Через интеллектуальную интуицию получаем представление 
о треугольнике вообще, даже если его не ощущаем.

У позднего Гуссерля критика: Кризис европейских наук. Считает, что 
европейские науки пошли по неправильному пути, заданному Галилеем. 
Галилей задал акцент на физ мат науках, книга природы написана на языке 
математики по Галилею. Гусерль говорит, что такой подход науки не дал 
ничего обычному человеку. Человеку важен жизненный мир, а не вот эта физ 
мат чепуха. Наука так далеко ушла от самого человека, что обычный 
человек ничего от науки не имеет, а абстракции науки никому не 
интересны. Поправить можно с помощью трансцендентальной феноменологии. 
Подход близок к экзистенциализму далее.

\subsection{Неокантианская философия науки}
19 век. Германия. Два направления: маркбургская школа, баденская школа 
(Виндельбанд, Рикерт).

Маркбурцы рассматривали некие априорные структуры. Кант нашел их в самом 
субъекте, а эти ребята считают, что они находятся в самой природе, 
а наука в конце концов их лишь отражает. Наиболее сильный аргумент 
в пользу таких рессуждений они видят в математике. Мат уравнения не 
всегда сразу конструируются, но когда появляются, становятся вечными. 
А математика лежит в основании физики, а значит априорность в самой 
природе. Наука в конце концов ищет эти структуры в природе и выражает 
с помощью математики.

Баденская школа рассматривала науку как состоящую из 2 частей. 
Естественнонаучные и гуманитарные. ЕН существуют как науки отражающие 
законы природы и минимально зависят от самого субъекта познания. Субъект 
познает законы и выражает. Гуманитарные науки аксиоологически очень 
сильно нагружены. Описания фактов всегда связаны с оценкой фактов. 
Аксиологическая значимость знаний. Цезарь перешел рубекон (речку). Смысл 
высказывания не виден, если не знать все детали времени. Отсюда 
различение двух методов. Номотетический (генерализирующий) 
в естествознании и идеографический (индивидуализирующий). Номотетический 
метод создает обширные и всеобъемлющие теории, восходящий процесс, мы 
это действительно видим, в физике в той же теории. А в гум науках 
индивидуализирующий метод. Это две школы и эти идеи и сейчас конечно 
в ходу.

\subsection{Экзистенциализм}
20 век. Отталкиваются от феноменологии Гуссерля. И от нашего Николая 
Бердяева, его произведение филоиофия свободы и философия творчества тоже 
лежат в фундаменте. У Бердяева суть в том, что он тоже считает 
естествознание имеет дело только с оставшим, невозможно творчество как 
таковое. Ученые открывают то, что уже есть. Творчество начинается 
с изобретения того, чего в природе нет. В этом плане отличие. 
Естествознание это всегда послушание, послушание законам, а настоящее 
творчество открывается в гум науках и в философии. Раз творчество 
возможно, то должно на чем-то основываться. У Бердяева основой 
творчества является свобода. Свободе придается метафизическое начало, 
существует до бытия, до бога. Бог ведь тоже творил свободно. Бог знал, 
что Адам съест яблоко, но позволил это сделать ему, потому что если бы 
не позволил, то не было бы человека. Наличие свободы у человека 
определило, даже с точки зрения религии, ход человеческой истории. На 
основе этой свободы человек может приблизиться к богу: 6 дней бог творил 
мир, небеса, свет, животных, растения, на 7 день отдыхал. На 8 день 
человек может вступить в сотворчество с богом. Идти к богу или идти во 
грех. Апокалипсис как прогностическое произведение христианства, 
грешники в ад, праведники в рай, Бердяев объявляет предостережением 
человечеству. Этого не случится, если человек вступит в сотворчество 
с богом.

Хайдеггер, бытие и время, подчеркивает тоже эту фундаментальную вещь, 
свободу. Прямо говорит, что есть некие экзистенциальные истины, которые 
в принципе не доступны для познания. Свобода и ничто. Наука всегда имеет 
дело с нечто, никакая наука не может заниматься ничто. А ничто для 
человека важно. В любом языке есть приставки не и положение нет. А что 
такое нет? Это отрицание какого-то существования всегда. С точки зрения 
лингвистики, это оппозиция какому-то утверждению, а с точки зрения 
метафизики -- путь в ничто. Хайдеггер есть конструкция здесь и сейчас. 
Он сам себя конструирует, но ведет себя к смерти неихбежно всегда. 
Отсюда все человеческие проблемы. Человек открытый проект, открытый 
в том смысле, что наука ничего не может поделать с ничто, но выделен 
к небытию, к смерти.

Вторая проблема: я в связи с человеческой свободой. Фейербах свобода 
в действии. Познать самого себя... А кто будет познавать самого себя? 
Надстраивается еще одно я над первым я. А кто будет познавать второе я? 
Третье я? Дурная бесконечность. Я нельзя познать, только чувствовать. 
Разрыв каузальной физической замкнутости. Дырка в бублике. К этой 
свободе наука ничего не имеет.

Карл Ясперс последователь Хайдеггера. Вносит то овое, что человеческое 
я имеет различные сферы своего действия. Различия Ясперс называет 
ролями. Человек всегда играет свои роли. Эти роли тоже сам выбирает, но 
когда выбрал, вынужден играть по этим правилам. Выбрали поехать 
в общественном транспорте, но дальше-то нужно следовать правилам. 
Выбрали создать семью, но дальше опять по правилам общества. Поступили 
учиться, но дальше опять играете эту роль в данном учебном заведении. 
Ролевой принцип функционирования Я. Основной рифмотип человеческой 
жизни. Шекспир говорил, что вся жизнь игра. Ясперс соглашается, но 
говорит еще о пограничных ситуациях. Здесь обнажается его истинное 
экзистенциальное я. Это всегда ситуации между жизнью и смертью. Без 
четкого обозначения, какое решение будет правильным. Не принять решение 
нельзя. Кроме того осевое время человечества. Древний восток и древняя 
Греция. И там, и там появилась философия. Оси появились именно в это 
время. Одна ось не сложилась. Европейская ось дала то направление, из 
которого появилась наука. Именно в философии появился главный метод 
науки -- критический рационализм. Без этого никакая наука не появилась.

Два французских экзистенциалиста. Сартр и Камю. Сартр занимается 
проблемами бытия и ничто. Экзистенциализм это гуманизм -- произведение. 
Сартр тоже имеет претензии к науке, что она носит слишком аналитический 
характер. При этом теряется целостность объекта. Любое живое не сводится 
к сумме составляющих. Наука не может понять живой организм. 
Прогрессивно-регрессивный метод. Задаются противоположности, в рамках 
которых исследуется объект на этапе его развития. Движение мысли должно 
следовать от одной оппозиции к другой, по синусоиде затухающей. В итоге 
даст представление объекта в его сложности. Приводит пример на 
марксистской философии. Сартр хорошо относился к марксизму. Тем не менее 
считает, что марксизм имеет недостаток из-за работы со слишком крупными 
блоками. Теряется сам человек. Свобода воли человека очень ограничена. 
Человек может немного ускорить, немого замедлить, но изменить особо не 
может. Сартр считает, что человек может изменить все. Поэтому считает, 
что регрессивной тенденцией в этом отношении будет психоаналитика. Фрейд 
имеет дело с отдельным человеком. Метод заключается анализ от больших 
блоков до анализов человеческого сознания (или бессознательного). Это 
выражение сущности прогрессивно-регрессивного метода. Сартр углубляет 
мысль Ясперса. Человек не просто выбирает, а человек в этих ситуациях 
обречен на свободу. Его существование здесь и сейчас предшествует его 
сущности. Пример. Французы никогда не были так свободны как во время 
Гитлеровской оккупации Франции. На уровне индивида появляется 
пограничная ситуация. Не выбирать нельзя, причем сам выбор не 
гарантирует правильность. Не выбирать нельзя, сам выбор ничего не 
гарантирует, но в этом и проявляется твоя экзистенциальная свобода. 
Абсолютная свобода. Носит метафизический характер.

Альбер Камю. Бунтующий человек, миф о Сизифе. Прочее.. Считает, что сам 
мир во всех проявлениях абсурден. Если Эйнштейн и Спиноза говорили, что 
человек может познавать мир, человек часть рациональности, Галилей со 
своей книгой математики тоже. Рациональность это приближение к богу. 
Камю же говорил нет. Какая рациональность? Мир абсурден для человека, 
в любой момент может произойти много всего, землетрясение, комета, чума, 
подобные вещи. Какая рациональность? Мир устроен абсурдно. И тем более 
абсурдно в науке. Наука способствует приумножению средств уничтожения 
самого человека. Тезис как у Паскаля. Наука в этом отношении не просто 
бесполезна. Бесконечность нельзя понять. Наука не только бесполезна, не 
важно земля или солнце крутится, наука абсурдна по своей сути, поскольку 
умножает возможность уничтожения человечества. Ни о какой рациональности 
говорить нельзя. Основным вопросом становится не абстрактный вопрос, 
а стоит ли жить в этом абсурдном мире. Камю отвечает все-таки стоит. 
Главное в человеке все-таки не рацио, а стремление бороться. Абсурдом 
абсурд попрах??? (что это значит что за слово такое...). Абсурд надо 
попирать тем же абсурдом. Свободный человек может это делать. Борьба 
бесполезна, но Камю заявляет, что все равно надо бороться.

На этом закончим иррациональность. Но не абсурд...

\subsection{Позитивизм}

Продолжение французских материалистов. Если проблемы и есть, надо решать 
их с помощью науки все равно. Наука способна дать решение при нужном 
отношении.

Позитивизм сформировался в 1830-е годы. Несмотря на различные теории, 
течения, всех авторов объединяет установка одна. Философия теперь только 
лишь инструмент анализа положительного знания. Само знание получается 
специальными дисциплинами. Наука, философия должна только обобщать, 
систематизировать, анализировать знания. Выполнять функцию, какую 
выполняла по отношению к религии в средневековье. Была служанкой 
религии, теперь служанка науки.

Различают три направления основных. Классический позитивизм (1830-е -- 
конец 19 века). Эмпириокритицизм, прагматизм, конвенционализм (до конца 
20 века). Неопозитивизм, логический атомизм, лингвистическую философию. 
Наряду с третьим этапом появляется постпозитивизм. Не исторически 
следует за позитивизмом, а вот и сейчас сосуществует с неопозитивизмом. 
до сих пор активная борьба. На всех конгрессах, конференциях, 
постпозитивисты спорят с неопозитивистами.

Коротко расскажем...

\subsubsection{Классический позитивизм}

Название было дано Огюстом Контом. Курс позитивной философии -- 
произведение. Основополагающие мысли. Философия внесла свой вклад 
в подготовку к появлению науки, но теперь в 19 веке наука сама имеет 
позвожность для анализм и своих методов, и результатов деятельности. 
Наука должна сама себе стать философией. Философы профессиональные 
должны ограничиваться тем, что выше. Наука должна описывать связь между 
явлениями, предвидеть развитие тех явлений, которые описывает, а все 
вместе должно быть, чтобы в конце концов на основе науки что-то 
происходило в человеческом обществе. Для обоснования разрабатыавет 
теорию развития человеческого духа.
\begin{enumerate}
  \item От древнего мира до 1300 года. Объяснение всех причин 
    и сущностей посредством обращения к высшим силам. Мифологические 
    установки. Религиозные.
  \item С 1300 до 1800 года. Объяснение всего сущего с помощью 
    метафизических понятий. Монада, абсолютный дух, творческий порыв. 
    Сущности эти принципиально нельзя проверить, но с их помощью человек 
    что-то объяснял и успокаивался.
  \item Развитие наук в 19 веке. Принцип: наблюдаю, следовательно, 
    существую. Это и есть позитивная стадия развития. Наука 
    основывается, и вообще признаком науки является 
    опытно-экспериментальная методика и мат аппарат. Если можно доказать 
    экспериментом, значит, наука.
\end{enumerate}

Вторая аргументация. Закон сдерживания воображения опытом. Еще Бэкон 
говорил, что к воображению ученых надо бы гири подвесить.

Третий принцип. Классификация наук по определенным установкам. Например, 
от абстрактного к конкретному. Математика, астрономия, физика, химия, 
биология, социология. Везде предлагает использовать физ мат метод. 
Физическая социология даже говорил.

Конт создает социологию. Принципиальное отличие в проверке на 
эксперименте. Свои методы: интервью, опрос, левесцит???. Три основные 
структуры, которые поддаются социологическим методам: семья, государство 
и церковь. Можем как в синхронном плане исследовать (прям сейчас, что 
такое семья). Можно создавать систему и прогнозировать. А можно 
в диахронном аспекте, смотреть, что являлось семьей в истории. С чего 
начиналось. Так же по отношению к церкви. Тоже используются мат методы. 
И конечно государство тоже.

\subsubsection{Милль}

Джон Стюарт Милль. Системы логики. Принципы систематической экономики. 
Продолжает дело Бэкона. Главным фунд методом науки является индукция. 
Все остальные так или иначе опираются на нее. Что прямая линия 
кратчайшая между точками выяснили на эксперименте, а потом возвели 
в аксиому.

Четыре принципа
\begin{enumerate}
  \item Логика следует из повторов и единообразия самого мира. Наблюдая 
    повторы и единообразия, человек откладывает.
  \item Логика не открывает. Она судит. Никакой логики открытий быть не 
    может. Диалектической логики у Милля нет.
  \item Логика основана не на уверенности мышления, мысли, а на 
    доказательствах. Есть определенные правила логические 
    доказательства. На основе них судим о правильности суждений. Нет 
    мистической уверенности мысли, интуиции итд быть не может.
  \item Из индукции частных наук следует идея индукции как метода науки 
    вообще. Раз все науки основаны на индукции, это научный метод 
    вообще. Милль проанализировал методы частных наук, увидел, что 
    основаны на индукции в конечном счете, а значит это и есть научный 
    метод.
\end{enumerate}

\subsubsection{Герберт Спенсер}

Теория глобального эволюционизма. Развита потом в 20 веке кем-то. 
Глобально он начинает с космической эволюции прогресс. Считает 
спекуляции типа существовала ли вселенная всегда, из чего появилась, 
куда денется. Ничего общего с наукой. Настоящая наука должна изучать что 
там здесь и сейчас. Сейчас только о двух процессах: интеграция материи 
и дифференциация. Звезды появляются и взрываются, галактики летают, а мы 
можем изучать. Каждый новый виток -- подъем на новый порядок. Но нет 
никогда уверенности, что не последует обратный виток. Эволюция носит 
цикличный характер. Вводит задолго до Дарвина положение, что в эволюции 
выживает всегда сильнейший или наиболее приспособленный. К космическим 
системам или системам земли не важно.

Классификация наук, уточнение. Делит все науки на абстрактные (логика, 
математика), абстрактно-конкретные (механика, физика, математика?), 
конкретные (геология, социология, че-то еще).

%}}}

Тут что-то пропущено, находится в других файлах...

\hfill \textbf{Jan 12}
% {{{

Вторая волна позитивизма, Авенариус, принцип экономии мышления: приколы 
после Ньютона.

Авенариус: нет объекта без субъекта, координация, с разных точек зрения 
нужно посмотреть на ощущения, которые с помощью чувства принимают.

Далее логический атомизм Рассела. Философия оказалась на нейтральной 
земле, и проблемы для логиков (а все можно свести к логике, по его 
мнению), должен быть логический анализ научных суждений. Если поддаются 
сведению к атомарным предложениям, то перед нами научное суждение. Если 
нет, то суждения близки к религиозным. Все философские системы теряют 
смысл после логического анализа такого. Но он оптимист, считает, что 
наука дает нам адекватную картину мироздания, и эту картину можно 
упорядочить.

Витгенштейн расширяет поле этого анализа: лингвистический поворот 
в философии. Философия должна проводить работу по прояснению мысли 
ученого, поскольку как любой человек ученый уязвим, надо показать мухе 
выход из мухоловки. Логика тоже на первом месте. Язык основывается на 
логике, в этом смысле: все, о чем может быть сказано, должно быть 
сказано ясно. О чем нельзя быть сказано следует молчать. Логика и язык 
это вещи чисто человеческие, поэтому ученые не должны требовать слишком 
много, а за логику мы выйти не можем.

Это ранний Витгенштейн. В наследии его нашли то, что он не опубликовал. 
Там идеи, что логика это только часть языка, что в науке логика не все 
может, что язык основывается на игровых ситуациях, через логику 
обучиться нельзя. В самом большом философском масштабе понимается игра. 
Но можно зафиксировать какие-то черты их. Наука получается одна из форм 
языковой игры. В науке логики много, но тоже не все укладывается.

Это основные мысли прошлого.

\subsection{Логический эмпиризм. Венский кружок}

В 1922 году в Вене стал работать. Существовал до аннексии Австрии 
Германией. Семинар возглавлял Муррис Шли. Все естественные науки 
считались индуктивными науками. Кружок расширялся, установил связи 
с единомышленниками по всей Европе, особенно в Германии, Чехословакии, 
Великобритании. В 1929 году выпустили манифест научное миропонимание. 
Большая роль... Многие члены кружка были известными учеными, Гёдель. 
Докторские степени по физике шли Моррис Шли (ученик Планка), Карнап. Не 
было чисто философов. Были ученые, которые начали размышлять о науке. 
Формы каркаса были идеи Витгенштейна. Но тот брал весь язык, а кружок 
ограничился языком науки.

Идеи:
\begin{enumerate}
  \item Четко разделить язык науки на эмпирический и теоретический. 
    Считали, что это важнейшие ориентиры для анализа языка науки. Нужно 
    систематизировать экспериментальные факты на протокольные 
    предложение. Задача эмпириков -- сбор и обобщение фактов. 
    Фиксировались в лабораторных журналах. На этом уровне факты 
    обобщаются. Обобщения можно строить все выше, глубже, абстрактнее. 
    Задача философа -- показать, что самые высокие абстракции в науке 
    можно свести к протокольным предложениям. Методом логической 
    редукции. Каждая наука похожа на некую пирамиду, в основании лежат 
    эмпирические факты, а вверху самые высокие абстракции. В каждой 
    науке можно увидеть эмпирический и теор уровни, и тогда (наука не 
    завершена) можно все пирамиды свести к одной большой пирамиды. Это 
    и будет модель науки. Не философия природы, а философия 
    естествознания. Протокольные предложения основываются на наблюдениях 
    и т.д. Теор законы помогают объяснить эмпирические законы. Научность 
    осмысливается таким образом, что мы можем верифицировать в конечном 
    счете всё и вся.

    Наряду с верификацией как таковой введен принцип верифицируемости. 
    Например, Луна имеет обратную сторону. Наблюдать просто так не 
    можем, но в принципе можем облететь Луну и посмотреть... А вот 
    существует круглый квадрат нельзя верифицировать вообще.

    Карнап пишет работу преодоление метафизики с помощью логического 
    анализа. Там показывает, что научные предложения строятся на основе 
    так называемых языковых каркасов. Это научный язык, в рамках 
    которого ученые и работают. Извечный спор между между материалистами 
    и идеалистами бессмыслен, поскольку философы работают в разных 
    каркасах. Например бессмысленным является вопрос существует ли 
    электрон на самом деле. Вопрос некорректный, поскольку ``на самом 
    деле'' метафизическая штука. Также не имеет смысла ставить вопрос 
    существуют ли числа на самом деле. Итак, для неопозитивистов 
    существовать значит существовать в рамках какого-либо каркаса.

  \item Еще одна важная мысль. Вопрос о языке науки. Наука, несмотря на 
    дисциплинарную организацию, будет являться наукой только тогда, 
    когда будет иметь единый общий язык. Задача философии науки в том, 
    чтобы постараться построить унифицированный язык науки. Галилей 
    считал, например, что это математика. Неопозитивисты не согласны... 
    Они рассматривают мат. структуры только как аппарат науки, но не 
    язык. Единым языком науки считается язык физики как наиболее 
    корректный и адекватный, где есть разделение между эксп и теорией. 
    Делают вывод, что все естественные дисциплины должны получить 
    обоснование через физику, как это уже произошло с химией. Такой 
    подход был назван физикализмом. Физикалистский язык станет 
    универсальным языком науки. В конце концов произойдет обоснование 
    всех наук вот так... Карнап реагирует на Хайдеггера, что такое 
    физика. Исследованию должно подлежать только сущее и еще ничто. 
    Сущее и ничто. Сущее одно и далее ничто. Сущее единственно и сверх 
    этого ничто. Как обстоит дело с ничто? Имеется ничто только потому 
    что имеется связка нет. Или наоборот, имеется отрицание, и поэтому 
    связка нет? Ничто первоначальнее, чем нет и отрицание. Мы не ищем 
    ничто, а просто находим, мы знаем ничто. Ничто было тут. Ничто само 
    себя ничтит... Карнап пишет, что это абсолютно бессмысленный набор 
    предложений. Псевдопредложений. Они противоречат логике. Вопросы 
    и ответы относительно ничто противоразумны. По Хайдеггеру если эти 
    предложения не вписываются в логику, то это проблема логики. Мнимая 
    рассудительность автора не должна приниматься в расчет настоящей 
    науки. Экзистенциализм с точки зрения третьего позитивизма абсурден 
    просто.

    Ошибка языковая по Карнапу в игнорировании неоднозначности глагола 
    ``есть''. Когда используем, получается двусмыслица, на которую 
    Гассенди указывал Декарту. Мыслю, следовательно, существую. Надо 
    Мыслю, следовательно, мысли существуют. Но от мыслей до физического 
    существования огромная дистанция. В методологии необходимо прояснять 
    глагол, где для связки используется, а где понимается физическое 
    существование.

    Что остается от философии по Карнапу? Не универсалии, абстрактные 
    понятия, а выражение чувства жизни. Состояния, в котором живет 
    человек. Метафизика выражает эмоционально-волевое отношение к миру. 
    Наиболее близко метафизика отражается не в понятиях, а в искусстве, 
    музыке. Лучше всех это понял Ницше, по его мнению.
\end{enumerate}

% }}}

\section{Становление и развитие науки. Философские и методологические 
концепции науки 20 и 21 веков}
%{{{

% Общий обзор? {{{

Немного искусственные определения картин мира. Ньютон пишет, что хорошо 
бы все свести к механике. Дальше электродинамическая картина, тоже 
с трудом, со спорами. Даже у Максвелла была мысль, что можно свести 
к механике. В 20 веке, начиная с конца 19 века, формируется новая 
картина мира, которая и сейчас остается главной: 
квантово-релятивистская. Дуалистическая, очевидно. Переход к этой 
картине связан тоже с научными открытиями, которые не могли вписаться 
в э-м картину: радиоактивность, Рентгеновские лучи, атомное ядро. Также 
начались большие вопросы к эфиру... Опыты показали, что нельзя 
зарегистрировать движение эфира относительно Земли. Лоренц продолжал 
настаивать на эфире и говорил, что просто эксперимент недостаточно 
точен. Но вот Эйнштейн в 1905 году показал, что можно обойтись без 
эфира. В статье была изложена спец теория относительности.

Минковский тоже заявил о своем воззрении на пространство и время. Дальше 
дискуссия Бергсона и Эйнштейна. Появилась спец теория относительности, 
потом общая. В 1917 году выходит первая космологическая работа Эйнштейна 
с так называемым Лямбда-членом. Стационарное значит вечное, в покое, 
замкнутое в себе, что еще Эллеаты говорили... Но из ОТО вдруг оказался 
и другой выход. Увидел его Фридман Александр Александрович, до открытия 
Хабблом расширения Вселенной Фридман показал, что есть другое решение, 
и Вселенная может сжиматься в точку и снова расширяться. Эйнштейн сперва 
сказал, что это вздор у Фридмана, но потом согласился, что результаты 
правильные и проливают свет. Появилась идея движения и развития самой 
вселенной. Независимо от Фридмана бельгийский Жан Леметр построил свою 
модель расширения Вселенной. Далее Гамов построил уже модель горячей 
Вселенной. Согласно этой теории, Вселенная образовалась большим взрывом. 
Реликтовое излучение тоже было открыто. В 1970--90 годы была открыта 
темная материя и темная энергия, которые составляют более 95\% от 
вещества во Вселенной. Далее в 2011 году открыли разбегание галактик 
с эффектом антитяготения, ускоренное расширение, непонятно куда 
расширяется, как долго будет. Черные дыры, все недавние открытия. 
Настало время, когда человечество пытается распространить принцип 
Коперника не только на солнце и галактику, но и на всю Вселенную.

Сама идея квантов была высказана Планком для излучения абс. черного 
тела. Использовал для облегчения расчетов. Потом Столетов показал, что 
свет излучается, распространяется и поглощается квантами. До этого была 
теория Френеля. Де Бройль получает нобелевскую премию. Появились две 
интерпретации квантового мира. Шредингер. В 1918 году Бор в работе 
о линейчатых спектрах выдвинул принцип соответствия. В 1927 году 
сформулировал другой метафизический принцип... принцип дополнительности.

Квантовая механика и теория относительности сформировали новую 
реальность. Все изменилось... абсолютность пространства и времени 
и прочее. На протяжении 20 века и до сих пор идут попытки объединить ОТО 
и кванты.

Кратко подчеркнем биологию. Революционное открытие тут -- открытие 
структуры молекулы ДНК. Дальше открытие способов кодирования 
наследственной информации. Геном человека.

Химия тоже получила обоснование с помощью физики. Когда атомы начали 
расщеплять.

Подведем итоги.
\begin{enumerate}
  \item Линейность и обратимость природных процессов подвергаются 
    критике. Нелинейность, компактивация, несепарабельность, бифуркации.
  \item Проблематичным оказывается объективистский подход к природе, где 
    считалось, что субъект не влияет на объект. Прибор рассматривался 
    как продолжение диапазона сенсорики. Есть кто считает, что даже 
    сознание влияет, поскольку проекция квантовой реальности...
  \item Коренным образом меняется представление о типах законов. 
    В классической физике фундаментальными считались детерминированные. 
    Стат законы только как упрощение. Сейчас же принципиальное 
    отсутствие детерминированности. Не знает даже природа...
  \item Принцип дополнительности Бора в отношении квантовых систем 
    постепенно стал принимать мировоззренческое и междисциплинарное 
    значение.
  \item Важный момент в квантово-релятивистской картине мира -- утрата 
    наглядности. Эмпирическая невесомость естественнонаучных теорий 
    и самих объектов теории. Запрещен геометрический образ частиц. 
    Подобное физическое иконоборчество распространяется, по мнению 
    Ефремова, и на массу, заряд, спин. Измерено значение величины, но 
    непонятно, что оно означает. Механическое действие до сих пор 
    остается абстракцией, хотя и уравнение Гамильтона-Якоби абсолютно 
    идентично имеющемуся.
  \item Принципиальная ограниченность точности знания. Роль прибора 
    в научном исследовании. Макроприборы.
    \\ Кроме того, невозможно полное что-то. Как показал Гедель.
    \\ Всегда есть скрытое, изначально неустранимое от человека.
  \item Вплоть до 20 века развитие науки считалось кумулятивным 
    процессом. Принцип преемственности. Фундаментальная несоизмеримость 
    теории. Кроме того, никто не мог представить, что найдут общую 
    теорию относительности, ее нельзя было вывести. В настоящее время 
    тоже есть те, кто считает, что квантовая физика -- последняя 
    принципиально новая теория, и можно будет построить теорию всего.
  \item В науку вводятся телеологические представления. Казалось бы, они 
    были давно отброшены. Классическая наука в первую же очередь 
    отбросила энтелегиальную причину. В современных науках снова принцип 
    целесообразности. Антропный принцип космологии, причем в двух 
    вариантах.
    \\ Первый вопрос: почему физические константы именно такие, какие 
    они есть. Ответ, что по-другому не было бы кому задать вопрос. 
    Альтернативный ответ, что изначально в большом взрыве была программа 
    такая, чтобы Вселенная такова, что в ней неизбежно появляется 
    наблюдатель.
    \\ В биологии также появляется телеология, номогенез. Принцип, 
    согласно которому есть парадокс, если следуем принципу Дарвина, где 
    слепые мутации. В развитии когнитивного аппарата никогда не было 
    отката назад. Как так? Эволюция не просто слепая.
  \item Постепенно происходит замена натурного эксперимента 
    вычислительным. Задача ставится на уровне параметров, строятся мат 
    модели, проигрываются на компьютере. Так был сыгран вариант ядерной 
    зимы.
\end{enumerate}

\hfill \textbf{Jan 26}

Новая квантово-механическая картина мира, связана с новыми открытиями. 
Является дуалистической картиной мира. Гравитация до сих пор не создана 
такая. Два пути к этой картине мира. Один путь связан с новым пониманием 
пространства и времени, начиная от Минковского, Лоренца и кончая 
Эйнштейном СТО, а затем и ОТО. Переосмысления космологических концепций. 
Сложный переход от стац понимания Вселенной. Гамов, Фридман тоже помогли 
взойти этой новой теории. Второе направление связано с введением Планком 
понятий кванта, сперва чисто какое-то, потом наш Столетов показал, что 
реально кванты. Потом благодаря Де Бройлю понимание, что все это вот 
такое. Копенгагенская ортодоксальная школа, которая и сейчас является 
доминирующей, но есть и другие. Сейчас усиливается Эвер-какая-то. 
О физическом смысле спорят очень активно.

Крупнейшие открытия в биологии, ДНК, в химии. Естествознание поднялось 
в общем. Ну и эта картина мира до сих пор превалирует, хотя и много есть 
маргинальных альтернатив.

На этом фоне мы еще разобрали последнюю стадию позитивистскую стадию, 
логический эмпиризм, венский кружок, много деятелей и науки, 
и философии. Идея та же, что у предыдущих: очистить от метафизических 
концепций. На основе идей Рассела, Витгенштейна. Создают теорию 
принципиального различения эмпирического и теоретического уровней. 
Особенно на примере физики говорят ясно все. На основе науки нужно 
решать базисные, протокольные, предложения, а все, что над ними 
надстраивается, нуждается в проверке. Задача философии в том, чтобы 
проследить как возможно логически непротиворечивым образом прийти от 
абстракций к базисным предложениям. Если не сводимы, то метафизика, 
и надо избавиться. Принцип верификации. Принцип принципиальной 
верифицируемости. Если проверить нельзя никак, то утверждение ненаучное. 
Языком науки, считают, надо признать физику, а не математику, поскольку 
именно тут видны эмпирические и теоретические уровни. В каждой науки 
теперь столько научности, сколько в ней физики (раньше про математику 
так говорили). Вот кратко предыдущая лекция...
% }}}

\subsection{Постпозитивизм}
% {{{

% Общие слова {{{
Это не что-то пришедшее на смену позитивизму. Постпозитивизм существует 
параллельно позитивизму. Второе важное замечание, что большинство 
авторов этого направления должны быть рассмотрены на уровне докладов, 
а наша задача здесь -- кратко осветить общие, ключевые положения 
наиболее авторитетных челов. Остальное на семинарах (было)... Важно 
подчеркнуть принципиальные отличия от третьего позитивизма.

Основатель Карл Поппер. Сам он тоже на первых порах был членом венского 
кружка, и по жизни вообще был другом со многими членами кружка, прежде 
всего с Карнапом, хотя в теоретическом плане считал его оппонентом.

Постпозитивизм начинается в конце 1930-х годов, полное развитие 
в 1950-60-е. Особенно заметен в Американской философии, где существуют 
центры, особенно важный -- Бостонский, где работают видные ученые 
и выпускается много литературы. В Питсбурге тоже центр мощный. Итак, 
несмотря на полемику между членами направления, есть общие черты.
\begin{enumerate}
  \item Отказываются выделять в любой науке, в том числе физике, 
    эмпирический базис и теоретический уровень науки. Девиз, что любой 
    факт теоретически нагружен. Любой факт существует только в рамках 
    теории. Нет смысла проводить демаркацию науки вот так. Если все 
    является эмпирическим или теоретическим, как можно свести так.
  \item Отказ от науки как кумулятивного процесса наращивания знаний. 
    В позитивизме эволюционное представление развития науки. 
    Постпозитивизм считает, что в науке происходят революции, причем они 
    наиболее важны. Во время революций рассматриваются самые глубоко 
    лежащие. Можно при определенных математических манипуляциях 
    представить теории Эйнштейна и Ньютона как взаимно дополнительные, 
    но это будет искусственно, поскольку основные понятия у них 
    различные. Направленность на анализ динамики научного знания.
  \item Понимание роли философии. У позитивистов стремление очистить 
    физику от метафизики и в общем-то философии. Постпозитивизм 
    учитывает, как минимум, историческое влияние. А в целом любой 
    естествоиспытатель должен быть философски подкован. Всё должно быть 
    логически продумано, смоделировано и прочее. Некоторые философские 
    догадки (про эволюцию, тепло в виде частиц, связь э-м явлений 
    и подобное) стали частью науки.
  \item Признание важным фактором в развитии науки социокультурных 
    факторов. Позитивизм пытается рассматривать науку в самой себе. 
    Постпозитивизм не отрицает так сказать саму в себе науку, но 
    учитывает, что наука существует в социуме, а не в вакууме.
\end{enumerate}
Это то общее, что объединяет всех авторов.

Теперь введем классификационную сетку постпозитивизма.
\begin{enumerate}
  \item Нормативное направление: Поппер, Локатос. Есть некие императивы 
    в развитии науки, ее рационализации как эталонной формы 
    социокультурное деятельности должна опираться на эти императивы. 
    Философ науки может вычленить в науке некие нормы, которым должны 
    следовать все ученные, чтобы ускорить прогресс науки и повысить 
    уровень рациональности.
  \item Дескриптивное направление: Томас Кун. Теория парадигм. Философы 
    науки должны просто опираясь на историю науки описывать ее ход 
    в историческом плане, выявлять нюансы, связи с социумом, но не 
    предписывать ничего науке.
  \item Историческая эпистемология. Опирается на редактора многих 
    изданий Маркса Бартовского. Идея -- показать, что все ключевые 
    понятия в науке (пространство, энергия, время) возникали сначала на 
    уровне обыденного знания, потом становились социокультурным знанием, 
    потом уже стали наукой. Все имеет историческое происхождение. 
    Эпистемология должна быть исторической. Кай Рен(?) тоже 
    привлекается. Кай Рен показывает как в классической науке сменялся 
    акцент с фундаментальных идей Аристотеля на Платона. Какую роль 
    играл принцип точности, почему стали вдруг говорить об этом вообще.
  \item Методологический анархизм. Пол Фейерабенд. Он считает, что 
    в науке вообще нет однозначного метода, ученые должны использовать 
    весь арсенал своих знаний и эрудиции, принцип контр индукции, 
    принцип все позволено. Наука как образец рациональности -- миф.
  \item Эволюционная эпистемология. Сейчас тоже наиболее развита. На 
    разных этапах познавательной деятельности. У Поппера эволюция -- 
    борьба за существование теорий. Альтернативно Степан Тулмен? 
    рассматривает эволюцию концептуального аппарата, как сложился он для 
    дифференциального и интегрального исчисления. Альтернативно Конрад 
    Лоренц биолог -- стремление рассмотреть эволюцию когнитивного 
    аппарата человека как определенного биологического вида. Как 
    складывался когнитивный аппарат от простейших моментов до такого 
    развитого, как человеческий мозг. Альтернативно Жан Пиаже 
    рассматривает эволюцию когнитивного аппарата на уровне индивида. Как 
    ребенок, рождающийся с нулевым почти что аппаратом, достигает в 14 
    лет развития, не уступающего ученым. Еще есть у него работы по 
    развитию параллельных. Корреляции между развитием ребенка 
    и развитием науки. Как появляются числа в науке и как складывается 
    понятие числа.
  \item Какая-то еще эпистемология, Пригожин, много кто еще. Наиболее 
    перспективным представляется не через эволюционные концепты, а через 
    коммуникативную практику. Главным становится не понятие 
    энерджентности мутаций, а понятие резонанса, который вызывает 
    синергетический эффект. Пригожин говорит о точке бифуркации, после 
    которой развитие идет только к одному аттрактору.
\end{enumerate}
Итак, это важная сетка, ориентир, для рассмотрения конкретных авторов.

Теперь к конкретным авторам, наиболее важные идеи, которые надо 
осмыслить логически, а детали на семинарах.
% }}}

\subsubsection{Карл Поппер}
% {{{

Основатель этого направления. Именно Поппер вводит в философию науки 
понятие критический рационализм -- метод, выделивший науку среди всех. 
Произошло на уровне Милевской школы, где Фалес Анаксимандр Анаесимен, 
где учитель инициирует ученика на критику, а ученик уже сам может 
развивать. Это главное, что философия дала науке. На основе общего 
метода критического рационализма. В античности разработан метод научной 
дискуссии. Чем отличается принципиально научная дискуссия от того, что 
Бэкон называл базаром. Поппер показывает: уважительное отношение 
к оппоненту; возможность правоты оппонента; сближение точек зрения; 
Поппер отрицает, что истина рождается в споре, это редчайшие моменты, 
чаще точки сближаются, потом осмысливаются слова, потом продолжается 
дискуссия. Для Поппера примером была дискуссия Бора и Эйнштейна. Поппер 
прослеживает подробно, что оппоненты твердо придерживались своих мнений, 
но учитывали аргументы чужие.

Вторая мысль Поппера -- анализ функций человеческого языка. Он 
разрабатывает теорию языка, выделяет наиболее важные функции, 
в частности позволяющие вести такие дискуссии и науку.
\begin{itemize}
  \item Эмоциональная, соматическая. Организм вольно или невольно 
    откликается на внешнюю среду, крик от боли или радости итд.
  \item Сигнальная функция, передается информация о состоянии среды, 
    например опасность. У стайных животных всегда есть.
  \item Дескриптивная. Дает возможность описать что-нибудь с помощью 
    языка и всего такого. Описывают маршрут по карте и все понятно. Есть 
    подобное у муравьев и много кого еще.
  \item Уникальное для человека: аргументативная. Именно эта функция 
    позволяет вести научную дискуссию. Критическое отношение к другим 
    мнениям. Ну или одобрительное... Именно благодаря этому возможна 
    наука. Именно благодаря этому возможно появление третьего мира. Мир 
    объективного знания. Знание без познающего субъекта.
\end{itemize}
Идея бессубъектного знания идет к Платону, где души знают все эти идеи, 
но и есть принципиальное различие. У Платона изначально существует, но 
тут создается людьми. Но будучи созданным уже дальше сам. Также 
в третьем мире произведения искусства, концепции этические и вообще все, 
что вырабатывает человек. Но после продукт отделяется от человека, 
а дальше живет сам по себе. Поскольку Поппер эволюционист происходит 
борьба за существование. Принципиальное отличие, что тот, кто проиграл, 
выходит полностью. Выбраковка происходит окончательно. В третьем мире 
по-другому. Одни картины признаются шедеврами, другие нет, но со 
временем может поменяться. В этом смысле третий мир бессмертен. 
Напоминает современный интернет.

На основе этих рассуждений Поппер проводит мысленный эксперимент. 
Предположим, что в результате катастрофы исчезнут все материальные 
продукты, существующие на Земле. Но останется третий мир и наша 
способность обучаться. Тогда культура очень быстро восстановится. А если 
исчезает и третий мир, то человеческая цивилизация должна будет начаться 
с нуля заново.

На основе этих идей Поппер проводит демаркационную линию. У Позитивистов 
это признак верификации. Поппер критикует это, поскольку основано на 
индукции, которая всегда неполна. Отсюда Поппер делает вывод, что не 
принцип верификации является определяющим, а принцип фальсификации. 
Ученые должны искать прежде всего не подтверждения, а те эксперименты, 
которые могут фальсифицировать гипотезу. Если находим такое, то наука. 
Если нет -- метафизика. В этом нормативность. Еще и принцип фалибилизма? 
Любое научное знание погрешимо и может быть в конце концов опровергнуто. 
Позитивисты говорят, что нужен фундамент науки, а Поппер заявляет, что 
нет такого фундамента вообще и быть не может. Ученый всегда должен быть 
готов, что все разрушится. Соответственно критикует теории сенсуализма, 
которые тоже заявляют, что есть сущности, которые можно открыть и не 
беспокоиться. И даже инструментализма, который идет от прагматизма 
и говорит, что все теории это лишь каталоги в библиотеках, и в этом 
смысле теория -- лишь удобная интерпретация. Поппер не согласен ни 
с фундаментом, ни с инструментализмом. Поппер пишет, что прогресс 
в науке бесспорен. Можно фиксировать его, поскольку новая теория 
является более мощной (обобщает более широкий эмпирический материал, 
дает более точные предсказания, выдерживает более рискованные 
экспериментальные проверки). В этом смысле Эйнштейн ближе к истине, чем 
Ньютон. Но ближе, а не абсолютно. Не факт, что завтра рискованный 
эксперимент тоже выдержат теории.

В конце Поппер представляет такую схему теории.
Сначала ставится проблема. На основе нее выдвигаются пробные гипотезы 
(tentative theory). Дальше пытается устранить ошибки (error 
elimination), после этого появляется проблема уже вторая. Дальше от 
второй проблемы то же самое. Это один из циклов. Второй алгоритм 
развития эволюционный Blind variation selective retention -- BVSR. Для 
Поппера тоже актуально.

Дальше замечание -- понимание истины у Поппера. Да, каждая новая теория 
лучше, но достичь истины нельзя. Истина всегда впереди. В этом тоже 
нормативность теории. Ученый тоже работает до тех пор, пока ученый 
считает, что вот-вот приблизится к истине. Но никто никогда не должен 
считать, что достиг истины. В этом смысле надежды на теорию всего 
эфемерны.
% }}}

\subsubsection{Имрэ Локатос}
% {{{

Ученик Поппера, но критикует и идет дальше. Наука развивается не 
линейно, а на уровне конкуренции как минимум двух исследовательских 
программ. Кодекс научной честности, что научное переплетено с чем-то. 
Ученые должны подходить объективно к оценке своих достижений раз, 
и придерживаясь своих установок не стремиться их фальсифицировать, но не 
думать, что соперники никогда не окажутся впереди. Динамика развития 
науки всегда конкуренция как минимум двух. Сам принцип фальсификации 
считает метафизическим. Утонченный фальсификационизм. Ученые никогда не 
знают возможностей развития своих теорий. Сама исследовательская 
программа как методологическая фигура речи рассматривается всегда как 
твердое ядро, основанное на метафизических принципах, негативная 
эвристика. И эвристический пояс, где создаются новые гипотезы, которые 
и являются приращением новых знаний. Прогрессивный сдвиг гипотез, 
критерий успешности исследовательских программ. Показывает как на макро 
уровне, физика Ньютона и Декарта. Но самое главное -- умение 
предсказывать. Теория Декарта могла объяснить +- то же самое, но не 
могла ничего предсказать... А Ньютон смог и Кеплера и все остальное. 
Однако в дальнейшем, как говорит Локатос, при э-м идеи Декарта оказались 
снова востребованы. Динамику и борьбу программ он прослеживает еще 
и в представлении света. Во времена Ньютона казалось свет частица, 
а потом Френель показал успехи волновой теории, а дальше борьба 
продолжалась вплоть до Де Бройля. Локатос, таким образом, считает, что 
важные принципы, вытекающие из честности, обуславливают и динамику 
движения и исследовательских программ и возможности приращения знаний. 
В то же время Локатос подчеркивает, что, пытаясь рационализировать 
историю науки, мы не достигнем полного успеха. Наука развивается через 
аномалии различных социокультурных факторов. Все социально-экономические 
факторы, не говоря уже о характере ученого, учесть невозможно.
% }}}

\subsubsection{Томас Кун}
% {{{

Ключевое понятие -- парадигма. Большой акцент делается на 
социокультурных факторах. В целом нужно подчеркнуть, что парадигма как 
дисциплинарная матрица развития науки состоит из следующих обязательных 
компонентов.
\begin{itemize}
  \item Общепринятое символическое обобщение (второго закона Ньютона, 
    например)
  \item Метафизическая часть парадигмы. Метафизика лежит в основаниях 
    всего.
  \item Ценности данной парадигмы, возможности количественных 
    предсказаний, как проводить эксперименты, как оценивать результаты. 
    Все может быть своё.
  \item Возможности, предусматриваемые для проведения практикума. Чтобы 
    человек не со стороны видел парадигму, а мог зайти и попробовать 
    самому. Нужно, чтобы человек владел парадигмой.
\end{itemize}
В рамках парадигмы и развивается то, что Кун называет нормальной наукой. 
Когда человек освоил это, становится ученым. С этой секунды человек 
начинает решать головоломки. Определенного рода задачи, которые 
вписываются в эту парадигму. Появляются алгоритмы решения. Постепенно 
ученые сталкиваются с парадоксами, аномалиями. Два выхода. Либо отложить 
вопрос, сказав, что несовершенный математический или экспериментальный 
аппарат. Но если аномалии накапливаются, то ученые вдруг начинают 
сомневаться в самой парадигме. Причем сомнение приводит к внезапным 
озарениям. Инсайд (или инсайт???) и гештальт. А если накапливаются 
аномалии, то ученый может переключиться с одного гештальта на другой. 
Его инсайд погружается в другой гештальт... Прогресс науки можно считать 
решение головоломок внутри одной парадигмы. Посмотреть прогресс науки 
при смене парадигмы нельзя так просто. Кун не отрицает, но 
проблематизирует прогресс науки в целом. В рамках парадигмы запросто. Но 
переход от одной парадигмы к другой не является прогрессом. Ну или во 
всяком случае не так просто сказать является ли.

Все это Кун анализирует в работе структура научных революций. Сам 
переход от одной парадигмы к другой проблематизирует.
% }}}

\subsubsection{Пол Фейерабенд}
% {{{

Он доводит до логического конца подход Куна к развитию науки. Если 
у того в науке выражался прогресс, то у Фейерабенда нет согласия ни на 
каком отрезке. То, что называют законами, частные случаи. Математические 
выражения удачно угаданы, но никто не может объяснить, почему они такие. 
Под вопрос ставят наши константы. Никаких нормативов в науке не 
существует вообще. Максимально вводит социокультурные факторы. Все 
остальное считает притянутым за уши. Три рекомендательных принципа, 
которые предлагают ученым
\begin{enumerate}
  \item Принцип контр индукции. Ученый должен сосредотачивать свое 
    внимание на том, что отрицает науку современную.
  \item Принцип пролиферации идей. Термин взят из биологии, 
    разветвление. Максимальное умножение базисных идей и всего такого, 
    в том числе за счет безумных идей.
  \item Для развития науки все подходит. Нет ничего строго научного, 
    строго обязательного anything goes, everything is allowed. 
    Предлагает обращать внимание на паранаучные практики типа 
    иглоукалывания.
\end{enumerate}
Кроме того, наука на современном этапе слишком сблизилась 
с государством, стала служанкой государства. Особенно не проявила себя 
в фундаментальных успехах человечества. Не приложила свою руку наука 
к этому. Колесо, приручение и все такое. А современная наука стала 
служанкой власти. Глубокий пессимистический вывод, что наука обречена на 
застой, если не будет отделена от государства.
% }}}

% }}}

\subsection{Историческая эпистемология}
% {{{

\subsubsection{Александр Койре}
% {{{

Александр Койре. Экстерналистское признание, что внешние факторы влияют 
сильно на науку. Case-studies: изучение различных крупных эпизодов науки 
на основе экстернализма. Философская история науки. Даже религиозное 
учение. Принцип прецезеонности: связано с социокультурными факторами, 
в культуре не было идеи точности, не смогла появиться и в науке до тех 
пор. Отход от преднаучного типа мышления к научному в 17 веке. Прямо 
связывать переворот этот в перевороте философских взглядов. Мутация 
философских взглядов. Перешли от Аристотелевской точки зрения к Платону. 
Реванш Платона. У Платона главное -- математика, числа, вот и идея 
точности. Сравнение науки Декарта и Галилея тоже принципиальное 
различие.

Койре также связывает философию Галилея с тем, что он, как и последующие 
ученые, хоть и религиозно ориентирован, но начинает вдруг 
интерпретировать библейские положения, что бог сотворил мир числом, 
мерой и весом.

Койре пишет следующее про мат начала нат философии. Гипотез не измышляю 
применяют гибко. Мысль человеческая полемична по своей сути и питается 
отрицанием. Отрицанием мысли оппонента. Мысль Ньютона формировалась 
в противостояние Декарту. Чтобы появился Ньютон должен был появиться 
Декарт.

% }}}

\subsubsection{Макс Вортовский}
% {{{

Макс Вортовский. Историко-философские корни науки. Понятие материи, 
движения, силы, поля, элементарной частицы, концептуальные структуры 
атомизма, причинности, прочее, все понятия первоначально имели 
метафизическую природу. И оказали существенное влияние. Основной 
концептуальный аппарат науки изначально имел метафизическую природу.

Вторая важная идея в том, что все научные теории в латентном виде 
опираются на теор деятельность ученых, которые заложили фундаментальные 
метафизические концепции. Они носили фундаментальный характер. Они друг 
другу альтернативны. Но это не минус, а плюс, поскольку в наиболее общей 
форме выражены существенные черты теор знания.

Структура метафизических систем появляется из понятий здравого смысла. 
Двойная функция метафизических структур. Корни в здравом смысле, 
а вершины дают опору для науки. Построение в конечном счете 
концептуальных моделей. Физика в этом плане самокритична. Метафизика 
есть эвристика для науки, поскольку задает основные модели научного 
понимания. Само понятие модели у Вортовского приобретает метафизический 
характер. Модель это преднамеренно создаваемые эвристические артефакты.

% }}}

\subsubsection{Джерод Холтон}
% {{{

Наука как популяция определенных тем, а каждый научный результат как 
событие на пересечении траекторий. Вся наука основывается на творческом 
воображении ученых, которое может направляться личной приверженностью 
к некоторым темам. Понятие ``тема'' включает как естественнонаучное, так 
и гуманитарное. Все специфические научные исследования опираются на 
темы. Тема простоты, гармонии и все такое стали ключевыми для обращения 
небесных тел. Бинарная оппозиция при выборе темы: субъект-объект, 
редукция-холизм, классика-вероятность. Вывод, что в науке существует 100 
важнейших тем, в физике, прежде всего. Последняя -- дополнительности 
и хиральности. Механицизм, телеологический порядок, гармония, 
относительность, эфир, дискретность, континуальность. Эпистемологическая 
концепция Эйнштейна, Эйнштейн считает, что нет прямой линии от 
чувственных данных к рациональным обобщениям. Всегда существует 
иррациональный, интуитивный прыжок для создания аксиом. Эйнштейн по 
такому прыжку и работал. Холтон подчеркивает, что всегда есть логический 
разрыв. Но тематически всё непрерывно. Прыжки могут быть разными, но 
нельзя прыгать в ширину... Темы всегда есть, но какие-то становятся 
запредельными и отфильтровываются учеными.

Кроме того, анализ проводится также как сформировалась тема 
дополнительности у Бора. Играли внешние факторы. Кьеркегор говорил 
о скачках вместо непрерывных переходов. Это было так дико, что такое же 
дикое у Бора? Дальше пишет, что Бор, оседлав тему дополнительности, 
начал развивать ее на все другие темы. Дополнительна любовь 
и справедливость. Наука и религия. Структура -- функция организма.

В стиле кейс-стадис проводится анализ первичного института науки. 
Структура лаборатории тоже играет роль. Проводит сравнение лаборатории 
Ферми и Резерфорда. У Резерфорда по всем направлениям, а у Ферми были 
узко подобраны сотрудники и определенные цели. Организаторские 
способности Ферми тоже играли большую роль при получении результатов. 
Установил хорошие отношения с итальянской властью, а также получил много 
грантов.

% }}}

\subsubsection{Майкл Полани}
%{{{

Полани считает, что не бывает деперсонифицированного научного знания, 
нет прогресса вообще, знания вообще. Мы знаем больше, чем можем выразить 
словами. Критикует нормативно-рационалистическую философию третьего мира 
Поппера, поскольку знание зависит от ученых. Могут быть утеряны 
способности истолкования. Так случилось и с наукой в средневековье.

Концепция личностного знания, где субъект не устраним. Через 
коммуникацию передается знание. Учитель необходим. Личностные знания это 
интеллектуальная самоотдача. К научному познанию применимы понятия как 
мастерство, искусство, интеллектуальная страстность, самоотдача, 
интеллектуальная красота, личная инициатива. В этих понятиях большой 
рациональности и объективности нет. Это связано с личностями. Полани 
считает, что в каждом акте познания есть страстный акт научной личности. 
И это необходимый элемент.

Страстность выполняет еще и эвристическую функцию. Истина есть нечто, 
о чем можно мыслить только что-то там. Без интеллектуальных эмоций мысль 
потонет в тривиальности.

Также связь с фокусом и периферией сознания. Фокус и периферия 
взаимоисключающие. Если пианист тратит фокус на пальцы, то сбивается. 
Также шофер, оратор, скрипач. Есть действия в науке, которые уходят 
в периферию сознания, и это хорошо, поскольку в фокусе остается 
достижение чего-то нового. Наука подобна ремесленническому. Отсюда 
Полани считает, что рациональность не носит какой-то характер, 
а коренится в доверии ученых друг к другу. Доверие как исходный базис 
ученых.

Полани использовал богатый биографический материал для изучения вот 
того-то чего-то. Эйнштейн предугадал результаты экспериментов. Есть 
и казусы в научном мышлении, когда эмоции и излишнее суеверие становятся 
препятствием для науки. Наука борется с суеверием, но есть косность. 
Например французская что-то там отрицала факт падения метеоритов, 
поскольку Аристотель сказал, что с неба ничего не может падать. 
Предрассудок долго сдерживал развитие французской науки. Истина всегда 
содержит элемент веры. Критическая рефлексия с точки зрения Полани 
создает науке угрозу. Сегодня мы снова должны признать, что вера 
является источником знания. Взаимное притяжение братьев по разуму.

Любая рациональность коренится в доверии. Принципиально ненормативный 
характер. Приписывать чему-либо реальность значит утверждать, что это 
проявится еще потом.

%}}}

% }}}

\subsection{Социальная эпистемология}
%{{{

Расширяют влияние социокультурного контекста. Рационализация культуры. 
Появление перспективы в живописи. Правило золотого сечения. Ноты 
в музыке. Рационализация культуры. Битва Греции и Персии тоже оказало

\subsubsection{Роберт Мертон}
%{{{

В 1960-70-е появилась концепция наукометрии. Полевые, этнографические 
исследования. Ученые объявляются племенем. Племя ученых. Надо внедриться 
в племя, чтобы приняли за своего, и изнутри изучать, как ученые сделали 
то или иное, что называют наукой. Есть чисто наукометрический вариант: 
измерение научных открытий, инноваций и так далее. До середины 20 века 
развитие науки происходило по экспоненте, а потом сатурация. В 20 веке 
было сделано около 90\% всех великих научных открытий.

%}}}

Блур, Барнс, Майкл Малкейн, кейс-стадис. Пытались показать, что наука не 
является особым видом деятельности, а просто один из видов 
социокультурной деятельности наряду с другим. Отрицают научный этос 
Мертона. Основным являются деньги, слава и власть. Наука -- один из 
видов верования, ее можно исследовать в социологическом плане. Движение 
-- все, а цель -- ничто. Результаты научной деятельности фиксируются на 
двух уровнях: континджентный (неформальный) и институциональный 
(формальный). Институциональный уровень лишь фиксирует уже сделанное. 
Как и в политике, сначала между собой переговоры, а потом уже 
официальные заявления.

\subsubsection{Майкл Малкейн}
%{{{

Майкл Малкейн пытается показать, что наука это тоже локальность, 
сингулярность. Появляется в Европе. Новые научные утверждения 
оцениваются не по истинности, а по удовлетворению требованиям 
социокультурному контексту.

Переворот корпускулярной теории света произошел не по правилам Куна, 
из-за нарастания проблем, а благодаря хорошей социокультурной работе 
сторонников и создателей волновой теории.

Так же и с Дарвином. Путешествие Дарвина по свету на корабле произошло 
только потому что были деньги. Второй фактор -- общение 
с селекционерами. В социокультурной теории было мальтузианство. Дарвин 
перевел его в натуральную науку. Потом на это опиралась фашистская 
Германия. При чем тут это -- загадка. Но в общем вещи рождаются не 
внутри науки, а вообще.

%}}}

%}}}

\subsection{Эволюционная эпистемология}
%{{{

Просто в общих словах.

Появление этого направления связано с появлением системы новых наук, где 
прослеживается эволюция знания и когнитивного аппарата в эволюции 
и прочих делах. Комплекс наук определил новое направление -- 
эволюционная эпистемология.

Три под-направления.
\begin{itemize}
  \item Эволюция гипотез и теорий.  Карл Поппер, третий мир, эволюция 
    продуктов деятельности общества по Дарвиновской схеме. Новации 
    в науке и новации в природе. Новации как предположения, гипотезы, 
    появление нового в природе. Естественных отбор как опровержение 
    гипотез.

  \item Эволюция концептуального аппарата науки. Стефан Тулмин. Эволюция 
    в науке, человеческое понимание. Идея, что прежде чем дойдет борьба 
    до теорий и гипотез, сначала нарабатывается концептуальный аппарат. 
    Он тоже не сразу нарабатывается. Критикует Куна за революционные 
    скачки от одной парадигмы к другой. Говорит, что это взгляд на науку 
    с высоты птичьего полета. Если спустимся, увидим, что постепенно 
    развиваются символы, понятия. Сама по себе фраза революция -- лишь 
    таксономическая что-то там. Ограничиться фразой ``а затем была 
    революция'' нельзя. Борьба Ньютона и Лейбница 
    в дифференциально-интегральном исчислении. Анализ концептуального 
    аппарата науки с точки зрения эволюции.

    Тулмин. Понятие относительности приходит в науку от понятия морали. 
    Принцип относительности появился сначала в морали. Резонанс между 
    узкой дисциплиной и общественным сознанием. Вследствие резонанса 
    и появляются все новации в науке.

  \item Генетическая эпистемология. Жан Пиаже. Пытается исследовать это 
    на уровне рассмотрения того, как складывается когнитивный аппарат 
    ребенка, который рождается с 0 знаний, а к 14 годам достигает уровня 
    развитого мыслителя, который может заниматься любой профессиональной 
    деятельностью. Эпистемологический конструктивизм. Еще попытка найти 
    параллелизм между развитием аппарата ребенка и развитием науки. 
    Числа, законы сохранения, другое.

  \item Филогенетическое направление. Лоренц, Колмен??. Историческое 
    развитие аппарата человеческого. Идея, что познание как функция 
    жизни, а аппараты служат для гомеостаза организма с внешней средой 
    и развиваются. Все структуры, которые Кант называет априорными на 
    самом деле в видовом развитии апостериорные.
\end{itemize}

%}}}

\subsection{Инновационная эпистемология}
%{{{

Много имен. Последнее рядом с Пригожиным.

Исходный тезис, что креативность заложена в космосе, космос порождает 
новое, если бы не было потенциала, не было бы музыки и стихов. 
Конкретная что-то у Платона, космогенез. Творчество как всякий переход 
от небытия к бытию. Классификация творчества. Исследование механизма 
творчества на примере Пира. Пир моделирует как собравшиеся приходят 
к новации, понимание Бога и что-то там Рота??.

Пример в этом же ключе на уровне понимания того, как появилась квантовая 
физика. В целом само творчество науки на трех уровнях. Проблемные 
ситуации в культуре древней Греции, возможность появления науки. Затем 
угасание науки в средневековье. Появление в эпоху возрождения. Потом 
утверждение во всем обществе.

На уровне отдельного научного направления. Становление неклассической 
физики. Формирование парадигмы, традиции, 17 век. Проблемные ситуации: 
э-м волны, лучи, радиоактивность. Проблемные события: Планк выводит 
излучение как прерывный процесс. Креативная ситуация: Де Бройль. Бор 
и принцип дополнительности. Инновация: идут за Бором. Формируется 
традиция.

Аналогично на уровне ученого. Видна структура инновационных процессов. 
Прогресс в науке фиксируется через структуру и типологию переноса. 
Внутридисциплинарный перенос: Максвелл использует сталкивание шаров для 
модели. Междисциплинарный перенос: цепные реакции переходят в ядерную 
физику, Фрейд тоже закон сохранения энергии. Междисциплинарный перенос: 
системные исследования, структурные исследования, математические 
эксперименты, компьютерные эксперименты.

Экстраординарные открытия дает импульс новым исследованиям науки.

%}}}

%}}}

\section{Эрекаев}
%{{{

%{{{

\hfill\textbf{Feb 8}

Физика изучает модели или же реальный мир?

Является ли философия наукой? Часто говорят о научной философии. Многие 
философы требовали, чтобы философия стала точной наукой. Но возможно ли 
это? Может ли философия стать точной наукой?
%
И еще один важный заход: можно ли философию рассматривать как науку? 
Точную науку? Буквально несколько хорошо известных моментов. В науках, 
тем более в точных, нужен формализм. Математика язык физики, Галилей 
говорил даже природы. Язык нужен, чтобы описать, предсказать, посчитать 
и так далее. Философии формализм необходим? В 17 веке было, интегралы 
и все такое... Непостижимо, что абстрактные объекты хорошо описывают 
реальную природу.
%
Дальше, нужен ли измерительный прибор философии? Нет...
%
Еще один аргумент: нужна ли точность философии? В логике наверно надо 
отслеживать все, энумерация. А в самой физике необходимо ли все доводить 
до числа?

Теперь о физике пару слов. Можно ли дать точное определение физике? Про 
что физика? По этимологии это наука о природе. Но есть много наук 
о природе. Все естественные науки -- о природе. Ботаника, например, 
физика? Радикальные редукционисты говорят, что все должно сводиться 
к физике, и таким образом все -- физика. Естественных наук много (хорошо 
бы перечислить самому 8).
%
Второй момент. Если физика -- наука о природе в целом, то должна 
претендовать на всеобщность. А всеобщность -- удел философии. Значит, 
физика это философия. Вообще есть мнение, что все науки возникли из 
философии, но его тоже можно аргументировано покритиковать. А касательно 
физики даже другое. Первым физиком и философом был Фалес. Люди жили 
в природе, наблюдали ее, изучали, формулировали что-то, друг другу 
передавали знания. Здесь тоже определенная тонкость или трудность, 
является ли физика философией или нет. Но и это еще не все.
%
По поводу определения физики. Вот космология про что? Про вселенную, 
космос. Физика обладает некими основаниями. Основание физики -- основные 
идеи, понятия, принципы, возможно законы, теории, результаты 
экспериментов, и так далее. Ключевое слово -- основные. Идей вообще 
море, но основных нет. А как выбирать основные? Нет критериев. Тем не 
менее физика является своеобразной метанаукой. Десятки и сотни 
естественных наук. Из физики постоянно локализовывается новое поколение 
различных теорий. Идет процесс дифференциации, в то же время интеграции. 
Например появилась биофизика. Все ищут эти стыки наук. А вот открыть 
принципиально новое -- тяжелая задача.
%
Далее, на основах физики было построено три физических картины мира. 
Механистическая, базируется на классической механике. Электромагнитная 
картина мира, даже соударение бильярдных шаров можно посчитать в э-м 
картине. И современная квантово-полевая картина мира. Здесь две 
программы: квантовая и релятивистская. Они бок о бок идут в современных 
фундаментальных исследованиях.

Попробуем объединить это и сформулировать, что такое философия физики. 
Найдем определение. Можно ли ее назвать наукой? Теорией? Лучше говорить 
об учении. Философия науки -- это учение о наиболее общих и глубоких 
законах и формах физического бытия и физического познания.
%
Ориентируясь на это будем пытаться обсуждать предмет.

Дальше заметим, что философия физики -- часть философии науки. 
Большинство концепций философии науки строились самими философами на 
базе физического материала. Это часть философии науки. Но задачи 
и предмет философии науки существенно отличаются от предмета и задач 
философии физики. У философии науки задача -- выяснить, как прирастает 
научное знание. Философия физики занимается основаниями физики. Ключевую 
роль играют философские проблемы физики. Большинство из них не решены. 
Не факт, что вообще будут решены. Вселенная конечна или бесконечна? 
А что вообще такое вселенная? Одним словом, такую проблематику, 
связанную с основанием.
%
Существует определенная классификация философских проблем физики. 
Первое: по различным категориям проблем в физике. Существует ли на самом 
деле трехмерное пространство? Реально ли пространство? А если реально, 
то реальна ли каждая точка? Ну и новые. Квантовая нелокальность. Это 
вообще проблема философии или чего? А квантовые корреляции.
%
Буквально в заключение этого момента. Основания физики изменялись. 
Эволюционировали. В 17--18 веках интересовали вопросы каковы свойства 
атомов, какова структура звезд, что такое тепло, что такое электричество 
и~магнетизм. Это были хорошие философские вопросы. Дальше в 19 веке были 
вопросы в~чем сущность эфира, в~чем сущность э-м поля.

Предмет и особенности философии физики. Разделы философии. Много 
разделов философии.
%
Онтология -- учение о бытие. А физики изучают что-то реально 
существующее или не обязательно? Квантовая онтология, релятивистская 
онтология и прочее часто встречаются в литературе. Второй аспект в том, 
что онтологию сегодня определяют как учение о формах существования 
объектности. Сначала было вещество материя и все такое, потом появилось 
поле, теперь возник квантово-полевой вакуум.
%
Гносеология -- учение о познании. Эпистемология -- часть гносеологии. На 
западе настаивают, что любое познание является научным.
%
Эпистемология -- учение о научном познании.
%
Методология -- учение о методах познания. Часто отождествляют 
с эпистемологией. А методы это как и с помощью чего.
%
Логика -- мир логично выстроен?
%
Герменевтика...
%
Дальше этика, важны ли этические вопросы в физике, касаются ли они ее 
вообще.

%}}}

\subsection{Мысленный эксперимент}
%{{{

Статус мысленного эксперимента.
Мысленные эксперименты помогали создавать фундаментальные физические 
теории (механика электродинамика ОТО СТО итд). Помогали прояснить суть 
теорий.
%
Что же такое: мысленный эксперимент? В философии тоже существует такое 
понятие. Один из ярких мысленных экспериментов придумал Джон Локк. Обмен 
бытия между сапожником и принцем. Человек будет себя вести так, как 
подсказывает ему сознание. Смерть поражает только мыслящее сознание. 
Представим червяк ползет по ветке. Раз и помер. Бывает. А человек? Одна 
проблема, другая, третья, как оформить, что надеть.

Представим себе, что произошла авария, у одного водителя цела осталась 
голова, у другого только тело. Пришили. А что получится? Сейчас более 
холистический подход популярен. Тело тоже много помнит. Спинной мозг 
и все остальное. А что семьям сказать? Как полученный человек будет себя 
вести в отношении жизней его частей? Много проблем.

Теперь к физике. В фундаментальной физике. Но сперва определение. Что же 
такое мысленный эксперимент.
%
Мысленный эксперимент -- теоретическая процедура по созданию 
искусственных объектов и манипулированию ими в искусственно созданных 
условиях. ((!) проанализировать. (!) критически. (!) что можно убрать, 
добавить). Искусственно связанных объектов и процессов? Или реальных 
объектов.
%
Конечно же, современный физик вынужден заниматься философией науки 
в гораздо большей степени, чем приходилось предыдущим поколениям.
%
Знаменитые мысленные эксперименты. Выбрать три и запомнить.
Галилей -- мастер мысленного эксперимента. Нужно исключить все 
лишнее, постараться рассмотреть процесс в чистом виде.
\begin{enumerate}
  \item Падение тел с башни.
  \item Мысленный эксперимент в трюме корабля.
  \item Вращающееся ядро Ньютона.
  \item Мысленный эксперимент Эйнштейна: бег за световой волной.
  \item Поезд Эйнштейна.
  \item Лифт Эйнштейна.
  \item ЭПР парадокс.
  \item Парадокс кота Шредингера.
  \item Парадокс друга Вигнера.
  \item Мысленные эксперименты Карно.
  \item Демон Максвелла.
\end{enumerate}
Проанализировать, рассказать, прокомментировать.

Прежде чем обсуждать мысленные эксперименты нужно выработать 
методологию. Бывают разные мысленные эксперименты, подумаем 
о классификации.
%
Первая классиф: классические, неклассические, метафизические.
%
Нужно приготовить мысленный эксперимент. Нужно ли? Провести тоже важно. 
Получить какие-то выводы. Проинтерпретировать.
%
Проанализируем на описанных примерах как работают эти эксперименты.
\begin{enumerate}
  \item Падение тел с башни.
    Галилей хотел опровергнуть физику Ньютона. Опыт показывал, что сила 
    должна быть предложена для постоянного движения тоже. Дальше падение 
    ядра и пробки.

    Сбрасывать с одной стороны конечно хорошо. Но можно ли мысленно?
    Возьмем два тела. Ядро пушечное и маленькую мушкетную пулю. 
    Попробуем их соединить. Веревкой неудобно. Стержень лучше, но не 
    самый хороший вариант. Прикрепим просто один к другому. Что будет 
    происходить? Нужно выписать два неравенства.
    $v_\text{я} > v_\text{п}$
    $v_\text{я+п} > v_\text{я}$ -- общее тело тяжелее.
    $v_\text{я+п} < v_\text{я}$ -- пуля тормозит ядро.
    Решение: $v_\text{я} = v_\text{п}$.

  \item Мысленный эксперимент в трюме корабля.
    Из этого эксперимента вышел принцип относительности.

    В трюме корабля разные вещи принесли. Бочку подвесили, сделали 
    дырку, тщательно фиксируем результаты. Походили по кораблю, 
    попрыгали, записываем. Бабочек и мух тоже выпустили, фиксируем.
    Далее корабль поплыл. Проведем те же самые эксперименты, 
    зафиксируем. Все процессы будут одинаковыми, все результаты 
    одинаковыми.

  \item Вращающееся ядро Ньютона.
    Ведро Ньютона посмотрим в Википедии, если захотим.
    Но с ним связан принцип Маха.

    Ведро наполнили, привязали, раскрутили. Поверхность воды становится 
    параболоидом. Удаленные звезды задают инерциальные свойства массы 
    воды...

  \item Мысленный эксперимент Эйнштейна: бег за световой волной.
    Спец теории относительности не было на время создания эксперимента.

    Если бегун будет двигаться за световой волной в пустоте с той же 
    скоростью, то он должен увидеть переменное, но постоянное световое 
    поле.
    %
    1 вариант: бегун связан с волновым фронтом.
    2 вариант: линейная оптика, бегун летит в этом чем-то.
    3 вариант: кванты света, бегун привязан к кванту.
    %
    Переменное в пространстве, но постоянное во времени. Это 
    противоречит опыту и законам электродинамики. Вся электродинамика 
    построена для досветовых систем отсчета.

  \item Поезд Эйнштейна.
    Тоже очень простой.

    Поезд запоздалый. Сидит пассажир. Попадают две молнии в края вагона. 
    Рядом наблюдатель. Одновременно ли попадают молнии?

  \item Лифт Эйнштейна.

    Есть лифт, обычный лифт. Покоится на 9 этаже. Внутри наблюдатель. 
    Какие ощущения испытывает пассажир лифта? Пол давит на ступни его 
    ног. (наркоман какой-то). Подвес оборвался. Что испытывает пассажир? 
    Пол уже не давит... Невесомость? Альтернативная ситуация. В космосе 
    коробка, в ней наблюдатель. Как лифт... Тянут снаружи или же просто 
    висит в космосе.

    Уже давно говорили о том, что масса инертная эквивалентна массе 
    тяготеющей. Считалось, это просто совпадение. Но вот пришел Эйнштейн 
    и сказал, что это фундаментальный закон.

    Слегка модернизируем эксперимент. Луч света светит из одной стены на 
    другую. В космосе и на 9 этаже. На земле наблюдатель.

    \hfill \textbf{Feb 14}

  \item ЭПР парадокс.
    Очень нетривиальный.

    Эйнштейн, коллега Розен и аспирант Подольский написали в 1935 году. 
    Причинность, они считали, обязана существовать в физических 
    процессах. Идею предложил конечно Эйнштейн.

    Две частицы провзаимодействовали, потом разлетелись в разные стороны 
    с одинаковыми импульсами. Разлетелись на сколь угодно большое 
    расстояние друг от друга. Одной частицы измерили координату $x$. 
    Импульс значит у нее не определен по Гейзенбергу. Но мы одновременно 
    быстро измеряем импульс другой частицы.

    Бом предложил модификацию: измерять спины вместо координаты 
    и импульса.

    Может быть проблема в том, что мы обсуждаем квантовую механику? 
    Вместо квантовой теории поля. Наше рассуждение принципиально 
    нерелятивистское.

  \item Парадокс кота Шредингера.
    Конечно знаем.

  \item Парадокс друга Вигнера.
  \item Мысленные эксперименты Карно.
  \item Демон Максвелла.
\end{enumerate}

Нужно поразмышлять над возможностью построения и проведения следующего 
мысленного эксперимента. Что увидит наблюдатель, который будет 
редуцирован до размера квантовой частицы?

%}}}

\subsection{Принцип относительности как основание фундаментальной 
            физики}
%{{{

На экзамене 5--6 принципов надо перечислить. Принцип наименьшего 
действия. Принцип соответствия. Принцип Ферма. Принцип относительности 
тоже в этом списке.

Понятие относительности интересное и имеет долгую историю. Самый темный 
философ, который об этом говорил, Гераклит. Вообще непонятно, о чем он 
говорил, поэтому темный. Появился даже такой принцип, что все в мире 
относительно. Красиво, хорошо, похоже на правду, за исключением вот 
чего. Все в мире относительно за исключением самого принципа. Он 
абсолютен.

Кратко о принципе относительности. Говорил Коперник. Говорил Кузанский. 
Был придуман эксперимент в реальной ситуации. Плывет корабль мимо 
берега. На берегу люди в здравии стоят и видят, что мимо проплывает 
корабль. Моряки на корабле стоят и видят: проплывает берег и люди на 
берегу. Заявление, что сила и ускорение тоже относительны.

Галилей формулирует то, что мы сейчас называем принципом 
относительности. Все физические процессы происходят одинаково вне 
зависимости от того, движемся мы или нет. Пуанкаре дальше сформулировал 
это в виде принципа. А дальше попробуем нарисовать блок схему, 
показывающую, что этот принцип эволюционирует. Или эволюционировал. 
Принцип наименьшего действия, например, не эволюционировал.

Разные принципы относительности:
\begin{enumerate}
  \item Все физические законы сохраняют свой вид в любых инерциальных 
    системах отсчета. Не существует никаких экспериментов, которые бы 
    показали, движется тело или покоится. Имеются в виду механические 
    эксперименты.

    Классическая механика Галилея-Ньютона.

    Относительные вещи: движение, покой скорость.

    Абсолютные вещи: ускорение, сила, время.

  \item Специальный принцип относительности.

    Все физические законы сохраняют свой вид в любых инерциальных 
    системах отсчета. То же самое... Но вторая формулировка. Не 
    существует никаких механических или оптических (электромагнитных) 
    экспериментов, которые показывают, движется тело или покоится. 
    Специальный принцип относительности базируется на Галилее, выходит 
    из него.

    На этом принципе относительности базируется вот эта физическая 
    теория. Специальная теория относительности. Также она является 
    расширением классической механики в каком-то смысле, выходит из нее 
    тоже.

    Относительные вещи: все предыдущие + время, размер тел, масса, 
    однвременность.

    Абсолютные вещи: сила, ускорение, интервал.

  \item Общий принцип относительности.

    Все физические законы сохраняют свой вид в любых системах отсчета. 
    Теперь в любых. Выходит из спец принципа относительности аналогично.

    Новая фундаментальная физическая теория, общая теория 
    относительности. Выходит из спец теории относительности аналогично.

    Относительные вещи: все прошлое + гравитация, сила.

    Абсолютные вещи: скорость света, постоянная Планка, интервал ($ds^2$), 
    инвариантная масса.

  \item Ожидается большая теория относительности. Там будет относительно 
    всё.
\end{enumerate}

%}}}

\subsection{Полиинтерпретационная квантовая парадигма}
%{{{

Полиинтерпретационная. Множество интерпретаций. Квантовых. Квантовой 
теории.

Интерпретация это попытка выразить теорию на каком-то языке. 
Приписывание физического смысла чему-либо. Символам, членам уравнений, 
теориям.

Формализм: совокупность уравнений, их решений и правил для получения 
предсказаний.


\begin{enumerate}
  \item Копенгагенская интерпретация.
  \item Эверетовская (многомировая, но не совсем одно и то же).
  \item Бомовская.
  \item Реляционная.
  \item Статистическая.
  \item Брюссельская (термодинамической называют часто).
  \item Интерпретация совместимых историй.
  \item Фейнмановская интерпретация.
  \item Интерпретация со многими разумами (many mind).
  \item
  \item
  \item
  \item
  \item
  \item
  \item
\end{enumerate}

%}}}

%}}}

\end{document}
