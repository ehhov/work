\documentclass[a4paper, 12pt]{article}

% Configuration {{{
\usepackage[utf8]{inputenc}
\usepackage[T2A]{fontenc} % T1 for English
\usepackage[russian, english]{babel}

\usepackage{enumitem}
\setlist{nolistsep}
%\usepackage{mathtools}
\usepackage{xcolor}
\definecolor{dimblue}{HTML}{1010aa}
\usepackage[
	colorlinks=true, 
	allcolors=dimblue
]{hyperref}
\usepackage[
	vmargin=1in,
	hmargin=1in
]{geometry}
\linespread{1.3}
\usepackage{indentfirst}
\usepackage{graphicx}
\usepackage[multidot]{grffile}
\usepackage[labelsep=period]{caption}

%\usepackage{times} % for English

\def\ex#1{\begin{center}\textit{#1}\end{center}\vskip-.3\baselineskip}
% }}}

\begin{document}

\noindent
Kerim Guseynov, 113M
\hfill
Feb 17

\ex{Ex. 2, p. 107}
% {{{
\begin{enumerate}[label=\alph*)]
	\item
		   Сформировать современные информационные технологии -- to shape modern information technology.
		\\ Цифровой датчик изображения -- a digital image sensor.
		\\ Мгновенно получать информацию -- to instantly receive information.
		\\ Принимать что-то как должное -- to take something for granted.
		\\ Необходимое условие необычайно быстрого развития в области коммуникаций -- a prerequisite for the extremely rapid development in the field of communications.
		\\ Изменить условия в области фотографирования -- to alter the conditions for the field of photography.
		\\ Зафиксировать изображение в электронной форме при помощи датчика изображения -- to capture an image electronically with an image sensor.
		\\ Цифровая передача изображений -- digital transfer of images.
		\\ Способный передавать большие объемы информации -- capable of transferring large quantities of data.
		\\ Требовать многочисленных изобретений -- to require numerous inventions.

	\item
		   Бросить вызов воображению многих мужчин и женщин -- to challenge the imagination of many men and women.
		\\ Предоставить ответ -- to provide the answer.
		\\ Коэффициент преломления -- a refractive index.
		\\ Наметить путь промышленного производства медицинских инструментов -- to pave the way for industrial manufacturing of medical instruments.
		\\ Системы дальней связи -- long distance communication.
		\\ Удовлетворять растущие потребности в коммуникации -- to cover the growing communication needs.
		\\ Не принимать во внимание потенциал оптического света -- to disregard the potential of optical light waves.

	\item
		   Решительный шаг вперед к волоконной оптике -- a decisive stop forward for fiber optics.
		\\ Стабильный источник света -- a stable source of light.
		\\ Испускать интенсивный и строго направленный луч света -- to emit an intensive and highly focused beam of light.
		\\ Облегчить оптическую коммуникацию -- to facilitate optical communication.
		\\ Скрупулезно изучать что-либо -- to meticulously study something.
		\\ Представить свои заключения -- to present one's conclusions.
		\\ Упростить процесс -- to simplify the process.
		\\ Осуществимый, но крайне трудный -- feasible but very difficult.

	\item
		   Занять достаточно длительное время -- to take a fair share of time.
		\\ Передаваться на большие расстояния -- to be transmitted over longer distances.
		\\ Покончить с лишними тратами -- to bring an end to unnecessary losses.
		\\ Подвергнуть что-либо рассмотрению с технической точки зрения -- to subject something to technical considerations.
		\\ Сложная взаимосвязь между размером, свойствами материала и длиной световых волн -- a sophisticated interplay between size, material properties, and wavelengths of light.
		\\ Совершенно другой вопрос -- an altogether different question.

	\item
		   Оказаться абсолютно неожиданным -- to appear totally unanticipated.
		\\ Снизить скорость развития -- to take a slower course (not exactly).
		\\ Придумать незаменимую часть современной технологии формирования изображений -- to come up with an indispensable part of modern imaging technology.
		
	\item
		   Превращать оптические изображения в электрические сигналы -- to transform optical images into electronic signals.
		\\ Придумать идею во время коллективного обсуждения -- to get the idea during a brainstorming session.
		\\ Полностью посвятить себя чему-либо -- to dedicate oneself fully to something.
		\\ Собрать прототип -- to assemble a prototype.

	\item
		   Появиться на рынке -- to appear on the market.
		\\ Достаточно высокое разрешение -- a sufficiently high resolution.
		\\ Оказаться коммерчески успешным -- to turn out to be a commercial success.
		\\ Принимать во внимание более высокий уровень шума и потерю качества изображения -- to take into account higher noise levels and the loss of image quality.
		\\ Превзойти уровень чего-либо -- to surpass the limit of something.
		\\ Постоянно развиваемый -- constantly being developed.
		\\ Покрывать весь световой спектр -- to span the entire light spectrum.
		\\ Стать незаменимым в области астрономии -- to become indispensable to the field of astronomy.
		\\ Камера широкого диапазона -- a wide-angle camera.
\end{enumerate}
% }}}

\ex{Ex. 1, p. 107}
% {{{
   Charge-coupled device (CCD)~-- прибор с зарядовой связью (ПЗС).
\\ Optical amplifier~-- оптический усилитель.
\\ Optical waveguide technology~-- оптическая волноводная технология.
\\ Pixel~-- пиксель.
\\ Infrared light~-- инфракрасный свет.
\\ Bubble memory~-- магнитоэлектронное запоминающее устройство.
\\ Complementary Metal Oxide Semiconductor (CMOS)~-- комплементарная структура металл-оксид-полупроводник (КМОП).
\\ Continuous red light laser~-- непрерывный лазер красного света.
% }}}

\ex{Ex. 3, p. 109}
% {{{
\begin{enumerate}[label=\alph*)]
	\item It [glass] becomes strong, light, and flexible, which is 
		a \textit{prerequisite} if the fiber is to be buried, drawn under 
		water, or bent around corners.

	\item When \textit{voltage} is applied to the CCD array, the content 
		of the wells can be progressively read out.

	\item /.../ CCD \textit{breached} the limit of 100 megapixels, and 
		although the image quality is not dependent on the number of pixels, 
		\textit{surpassing} this limit is seen to have brought digital 
		photography a further step into the future.

	\item The electronic eye, the CCD, became the first truly successful 
		technology for the \textit{digital transfer of images}.

	\item During the production of glass, different additives such as 
		\textit{soda} and \textit{lime} are used to simplify the process.

	\item \textit{Notwithstanding} its bulky and primitive 
		characteristics, when compared to contemporary cameras, it initiated 
		a more commercially oriented digitization in the field of 
		photography.
\end{enumerate}
% }}}

\ex{Ex. 4, p. 109}
% {{{
\begin{enumerate}[label=\alph*)]
	\item True.
	\item False.
	\item False according to the text but true when considering 
		photoelectric effect in general (there could be any material as soon 
		as the light beam has high enough frequency and thus energy).
	\item False.
	\item False.
	\item False.
	\item True.
\end{enumerate}
% }}}
\newpage
\ex{3-minute talk}
% {{{
One of the characteristics that make people who they are is the ability 
to communicate. As humanity developed and spread across the planet, 
there appeared a need in extremely long-range communication solutions. 
To connect people in Europe with people in North America, 
a transatlantic telegraph cable was developed and laid in the 19th 
century. The line speed back then was pretty poor. Modern 
telecommunication cables connecting the two continents were developed in 
the late 20th century and make use of optical fiber, which has quite an 
interesting story that even led to Charles Kuen Kao's Nobel Prize in 
Physics.

The basics of optical fiber technology are in the refraction of light 
when it propagates from one medium to another. Because of different 
refractive indexes of materials and the fact that the optical path and 
the geometrical path are not the same, rays of light are bent on the 
surface. The basic equation describing how much light is bent equates 
the products of the refractive index and the falling angle sine for the 
two media, so there may be situations when for light to propagate to the 
other medium the value of sine has to be greater than unity. It means 
that the light does not propagate to the other medium and is entirely 
reflected back into the first one. This way the light can stay within 
a medium with a larger refractive index even when it is bent. It was 
first observed in water when rays of light followed its flow.

However, two problems emerge when we try to utilize this effect to 
develop long-range communication cables. The first is to find a material 
that can be bent as much as needed. And the second is to reduce 
intensity losses within the medium itself so that light can travel long 
distances.

The second problem is somewhat easier than it sounds since we can 
install intermediate amplifiers that would restore the signal strength 
and pass it further. But this increases the cost of cables and may also 
lead to data losses due to imperfections of amplification.

The first problem is also not impossible to solve since the first thing 
to think of when it comes to light and optics is glass. Indeed, glass 
can be manufactured thin enough to be flexible, and its refractive index 
is substantially larger than that of air.

Years of research and development conducted by Charles Kuen Kao have led 
to better understanding of technical problems and ultimately to the 
creation of optical fiber technology as we know it now.
% }}}

\end{document}

facilitate -- облегчить
meticulously -- скрупулезно
