\documentclass[a4paper, 12pt]{article}

% Configuration {{{
\usepackage[utf8]{inputenc}
\usepackage[T2A]{fontenc} % T1 for English
\usepackage[russian, english]{babel}

\usepackage{enumitem}
\setlist{nolistsep}
\usepackage{mathtools}
\usepackage{xcolor}
\definecolor{dimblue}{HTML}{1010aa}
\usepackage[
	colorlinks=true, 
	allcolors=dimblue
]{hyperref}
\usepackage[
	vmargin=1in,
	hmargin=1in
]{geometry}
\linespread{1.3}
\usepackage{indentfirst}
\usepackage{graphicx}
\usepackage[multidot]{grffile}
\usepackage[labelsep=period]{caption}
\usepackage{subcaption}

%\usepackage{times} % for English
% }}}

\begin{document}

\noindent
Kerim Guseynov
\hfill
Mar 12

\begin{center}
	\textit{3-minute talk about Andre Geim}
\end{center}

Andre Geim quotes Samuel Goldwyn: ``I don’t think anyone should write 
their autobiography until after they’re dead.'' Nonetheless, it's 
a tradition to tell about oneself when awarded the Nobel Prize, so Andre 
Geim did.

We are most interested in his career, but it would not have turned out 
the way it did if not his earlier years. So a few words about them. His 
parents wanted him to study well and did a lot for it. He had many 
tutors in maths, physics, and Russian literature. The largest impact had 
the physics tutor, who taught him to solve problems from both ends and 
encouraged his interest in physics and specifically cosmology. He 
applied to Moscow Institute of Physics and Technology, where he had some 
great years and was taught to logically analyze and grasp everything. 
A unique feature of MIPT education was that students were able to easily 
combine different subjects to solve very complex problems that were much 
closer to real life.

Despite Geim's interest in cosmology, at the university and later 
he explored solid state physics. For his Master's degree he studied 
electronic properties of medals by exciting electromagnetic waves in 
spherical samples of ultrapure indium. Geim worked on his PhD in the 
same laboratory and investigated mechanisms of transport relaxation in 
metals by an electromagnetic waves resonance method. This topic was not 
very promising, and Geim says he learned from it that he should not 
force students to work on zombie projects.

After his PhD, he tried different fields of solid state physics and 
finally found his own niche, different from the almost-boring PhD topic. 
Geim constructed a system with highly inhomogeneous magnetic field 
provided by a superconductor and studied electron transitions in a metal 
film subjected to the field. The scale of the magnetic field 
inhomogeneity was the key feature of his work. Geim had investigated 
this type of systems for a few years and then caught an opportunity to 
visit a foreign university. It happened to be Nottingham University, 
where he worked with older samples due to limited time, but still wrote 
two articles as a result. After that Geim didn't want to return to the 
Soviet Union and decided to apply to different universities. Over the 
next few years he had worked at universities at Nottingham, Copenhagen, 
and Bath. It broadened his field of study dramatically and ultimately 
let him conduct his research on graphene, which led to some outstanding 
results and obviously his Nobel Prize.

\end{document}
