\documentclass[a4paper, 12pt]{article}

% Configuration {{{
\usepackage[utf8]{inputenc}
\usepackage[T2A]{fontenc} % T1 for English
\usepackage[russian, english]{babel}

\usepackage{enumitem}
\setlist{nolistsep}
\usepackage{mathtools}
\usepackage{xcolor}
\definecolor{dimblue}{HTML}{1010aa}
\usepackage[
	colorlinks=true, 
	allcolors=dimblue
]{hyperref}
\usepackage[
	vmargin=1in,
	hmargin=1in
]{geometry}
\linespread{1.3}
\usepackage{indentfirst}
\usepackage{graphicx}
\usepackage[multidot]{grffile}
\usepackage[labelsep=period]{caption}
\usepackage{subcaption}

%\usepackage{times} % for English
% }}}

\begin{document}

Kerim Guseynov, group 113M
\hfill
\today

\begin{center}
	\begin{large}
		\textbf{Demonstration of Quantum Brachistochrones between \linebreak Distant States of an Atom}
	\end{large}

	Sections I--VIII.

	The text is about 27k symbols.
\end{center}

\begin{center}
	\textbf{Vocabulary}
	\vskip.3\baselineskip
	\begin{tabular}{rl|c}
		\hline\hline
		paradigmatic & характерный & page 1, abstract \\
		advent & пришествие & page 1, par. 2 (left) \\
		\hline
		conveyor belt & ленточный конвейер & page 2, Fig. 1 caption \\
		conundrum & загадка, парадокс & page 2, par. 2 (right) \\
		\hline
		trough & впадина, минимум & page 3, par. 1 (left) \\
		crest & гребень, вершина & page 3, par. 1 (left) \\
		recoil & отдача & page 3, par. 1 (left) \\
		swift & быстрый, резкий & page 3, par. 2 (right) \\
		hatched & заштрихованный & page 3, 4th line from the bottom (right) \\
		corroborate & подтверждать & page 3, 4th line from the bottom (right) \\
		\hline
		ramp & наклонная плоскость & page 4, Fig. 2 caption \\
		render & превратить & page 4, last line (right) \\
		\hline
		interrogation & извлечение данных & page 5, par. 2 (left) \\
		confine & ограничивать & page 5, par. 2 (right) \\
		\hline
		conjecture & предположение & page 6, par. 4 (left) \\
		\hline\hline
	\end{tabular}
\end{center}

\begin{center}
	\textbf{Questions}
\end{center}
\vskip-.5\baselineskip
\begin{enumerate}
	\item What does the term brachistochrone refer to?
	\item How can the Bernoulli's brachistochrone be generalized?
	\item For what kind of quantum systems the transition speed limit is 
		long known?
	\item What does this limit resemble?
	\item Why can't the same logic be applied to other kinds of quantum 
		systems?
	\item Why do we need a deep trap potential?
	\item Why is the speed limit lower than the adiabatic time?
	\item What is the obtained transportation speed limit?
	\item How does it depend on the trap potential depth?
	\item How can the coherence of the transportation be verified?
	\item What is the relation between classical and quantum minimal times 
		of transporting an atomic wave packet?
	\item Where does the speed limit come from?
\end{enumerate}

\begin{center}
	\textbf{Overview}
\end{center}

Since the birth of quantum physics in the early 20th century, there have 
been discovered many interesting, unusual, and bizarre phenomena. Some 
of them were then found to be quite useful for applications. The 
objective of many modern physics experiments is to manage the control of 
quantum systems to fully exploit their properties. One of such 
objectives is to transport a quantum system in space. Of course, it's 
very important to do that coherently, or otherwise the state is 
distorted and the task of transporting it is failed. So there has to be 
some natural limit for the duration of coherent transportation. This 
topic is long understood for simple two-level systems. But those are 
rare, and we mostly deal with more complex ones. Some would expect 
complex systems to have some kind of modification of the simpler 
results, but experiments have shown a difference. So it is interesting 
both to understand where it comes from and to find a meaningful lower 
bound for the duration. As a complementary task, it would also be 
interesting to compare the results with classical physics.

The task of optimizing the path of an object for minimal time was first 
posed by Bernoulli for a massive object falling under the influence of 
a uniform gravitational field. The solution is well known and called 
a brachistochrone curve. This problem can be generalized to finding the 
shortest time at which a physical system can be changed from its initial 
state to a desired final state. The shortest time would depend on the 
amount of energy available and the type of control.

In quantum physics such a speed limit was derived by Mandelstam and Tamm 
and is reminiscent of Heisenberg's uncertainty principle. It shows that 
the duration cannot vanish unless one has access to unlimited energy. 
This limit was experimentally confirmed for two-level systems, however, 
for complex systems the real limit is much greater, making the existing 
one meaningless. The problem here is that two-level systems can be 
converted through a Rabi oscillation, which is fundamentally 
inapplicable to continuous states like the spatial position.

In this work, a conveyor belt based on optical lattice is used to 
transport an atomic wave packet on a distance 15 times its size. The 
trap potential is adjusted to be deep enough to suppress quantum 
tunneling and also so that the trajectory of the transported system is 
optimal. The optimal trajectory is obtained from numerical simulations.

The fidelity of the transportation, which is the overlap between the 
actual and targeted final states, with 1 being the perfect result, is 
obtained as a function of duration. It is found that the fidelity 
saturates for durations larger than quasi-harmonious oscillation period 
of the trap potential (which is also responsible for energy uncertainty 
of the system to transport). To ensure that the transportation is 
coherent, the authors prepared a specific superposition as an initial 
state, then transported part of it to a target state and back, and then 
extracted the interference constant. The constant was found to be 
exactly the same as the fidelity measured earlier, proving that the 
method keeps coherence.

The same measurements and simulations were made for several trap 
potential depths, and the minimal time is shown to be inversely 
proportional to the depth. Interestingly, the classical and quantum 
cases differ only by 25\%, which shows that the limit comes mostly from 
the Hilbertian metric of quantum states.

\end{document}
