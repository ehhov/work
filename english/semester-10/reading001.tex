\documentclass[a4paper, 12pt]{article}

% Configuration {{{
\usepackage[utf8]{inputenc}
\usepackage[T2A]{fontenc} % T1 for English
\usepackage[russian, english]{babel}

\usepackage{enumitem}
\setlist{nolistsep}
\usepackage{mathtools}
\usepackage{xcolor}
\definecolor{dimblue}{HTML}{1010aa}
\usepackage[
	colorlinks=true, 
	allcolors=dimblue
]{hyperref}
\usepackage[
	vmargin=1in,
	hmargin=1in
]{geometry}
\linespread{1.3}
\usepackage{indentfirst}
\usepackage{graphicx}
\usepackage[multidot]{grffile}
\usepackage[labelsep=period]{caption}

%\usepackage{times} % for English
% }}}

\begin{document}

Kerim Guseynov, group 113M
\hfill
\today

\begin{center}
	\begin{large}
		\textbf{Maximum Refractive Index of an Atomic Medium}
	\end{large}

	Sections I, II, III, IV, and VI (since Sec.~V is very technical and, as 
	the paper says, ``casual readers can consider skipping'' it.)

	The text is about 39k symbols.
\end{center}

\begin{center}
	\textbf{Vocabulary}
	\vskip.3\baselineskip
	\begin{tabular}{rl|c}
		\hline\hline
		reconcile & согласовать & page 1, abstract \\
		dilute gas & разреженный газ & page 1, abstract \\
		implication & последствие & page 1, par. 1 (left) \\
		readily & охотно & page 1, par. 1 (right) \\
		\hline
		granularity & гранулярность, дискретность & page 2, last par. (right) \\
		exquisitely & изящно & page 2, last par. (right) \\
		retain & сохранять & page 2, Fig. 1 caption \\
		\hline
		dyadic & состоящий из двух частей & page 3, par. 3 (left) \\
		foster & стимулировать & page 3, par. 2 (right) \\
		encode & представлять & page 3, par. 3 (left) \\
		\hline
		attenuation & затухание & page 4, par. 3 (left) \\
		decompose & раскладывать & page 4, par. 3 (left) \\
		speckles & пятна & page 4, last par. (left) \\
		\hline
		prescribe & описывать, определять & page 5, par. 2 (left) \\
		inset & вставка & page 5, par. 2 (right) \\
		necessitate & требовать & page 5, par. 3 (right) \\
		\hline
		pictorial(ly) & наглядный & page 6, last par. (right) \\
		\hline
		bulk & большой & page 8, par. 3 (left) \\
		\hline\hline
	\end{tabular}
\end{center}

\begin{center}
	\textbf{Questions}
\end{center}
\vskip-.5\baselineskip
\begin{enumerate}
	\item What is the relation between the optical response and scattering 
		cross-section of a single atom and its size?
	\item How does the refractive index of a smooth medium behave when its 
		density increases?
	\item How can we take into account the local electric field in the 
		smooth medium approximation?
	\item Does this modification change the behavior of the refractive 
		index at high densities?
	\item What can we do to avoid the indefinite growth of the refractive 
		index at the resonant frequency?
	\item Is the optical response of two interacting identical atoms equal 
		to the sum their individual responses?
	\item What causes the change in the optical response?
	\item How can we use this information to simplify calculations for an 
		atomic medium?
	\item How can we verify does the simplification work how exact is it?
	\item What interaction remains between the atoms after the iterative 
		simplification procedure?
	\item How does the refractive index of an ensemble of identical atoms 
		behave with density increase?
	\item What is the ultimate reason for the absence of large refractive 
		indexes of transparent media?
\end{enumerate}

\begin{center}
	\textbf{Overview}
\end{center}

Transparent materials let the visible light come through differently. 
Some of their properties include opacity, reflectively, dispersion law, 
and refractive index. Most of them are related to the optical response 
of the material, which is its low-level characteristic resulting from 
the atomic structure. While the optical response of a single atom is 
easy to calculate, any attempts to derive the refractive index of an 
atomic medium have led to unrealistic results. This paper presents 
a completely new approach to calculating the optical response and the 
refractive index of an atomic medium and throws light on the mystery of 
why refractive indexes are of the order of unity.

To understand the problem, we should first follow the conventional 
approach and find its weak point. To avoid very complex and 
time-consuming calculations, atoms are approximated as semi-classical 
two-level systems. This is fairly precise and incorporates the most 
important characteristics of the atom. The reaction of a single atom is 
well known, so our work is to apply it somehow to a system of many 
atoms. The conventional way is to forget about the atomic, discrete, 
granular structure of the matter, introduce its collective 
characteristics such as density, polarizability, etc., and operate in 
terms of a continuous medium. At this point you may have already gotten 
the largest weakness of the approach, but we should still reach its 
result. One distinction of a continuous medium is that the electric 
field traversing it is always and constantly influenced by it, thus 
making it impossible to experience multiple scattering and interference. 
The result of that is that the refractive index in the leading order is 
just the index of a single atom scaled by a factor proportional to the 
square root of the number of atoms divided by the volume. So at the 
resonance, the refractive index just grows indefinitely with density, 
what is obviously not observed in real life.

The approach suggested in this paper, however, does not forget about the 
atomic nature of any medium. It is implemented for an ensemble of 
perfectly identical atoms, each of which has the same resonant 
frequency, but remains a distinct and point-like particle. The main 
reason why theoretical studies have always considered the smooth medium 
approximation is that it dramatically simplifies all formulas and even 
lets one achieve the result analytically. To keep the atomic structure, 
one has to calculate large sums over all atoms including the values of 
the electric field at each atom's position. To make matters worse, in 
the correct, quantum approach, one has to take into account the mixing 
of atomic states due to the Pauli principle, which adds an 
N x N determinant to every equation, making it exponentially more 
time-consuming for larger systems. And for N even about 10\^3, it takes 
an enormous amount of time and work. The fact that in the system 
considered atoms have only two energy levels makes things easier, but 
not enough for a complete large-scale study. However, an elegant way out 
is also suggested by the authors. I will explain it in a minute, but 
first, you should understand how two atoms behave under a incident plain 
electromagnetic wave.

If there are two perfectly identical atoms with the resonant frequency 
omega zero that interact with each other, their optical response is not 
a Lorentzian centered at omega zero with the height twice as large as 
for a single atom. Instead, the response is two substantially distant 
Lorentzian peaks of normal height centered at shifted frequencies of 
omega zero plus-minus delta omega, where delta omega depends on the 
distance between the atoms and their interaction strength. Based on the 
optical response, which defines the reaction of the system to visible 
light, two identical atoms behave almost like two atoms whose resonant 
frequencies are slightly shifted from the original atoms' frequency.

This way, one can expect that a set of identical atoms can be 
approximated by a set of different atoms with effective resonant 
frequencies depending on spatial parameters of the original system. And 
indeed, as the authors compare the simplification with a straightforward 
calculation of the aforementioned large formulas for smaller densities, 
this test shows good agreement between them. And it also shows that the 
refractive index n of an atomic medium approaches a limit as the density 
increases. For larger densities, only the simplified technique is 
applicable, but it confirms that the index saturates and reaches a value 
of about 1.7, which is of the order of unity.

The result of the study explains the universal rule that the refractive 
index is always of the order of unity. It is because of the granularity 
and atomic structure of everything in the Universe.

So, thank you for your attention. I hope you enjoyed my talk and will 
probably enjoy looking through the article. Goodbye.

\end{document}
