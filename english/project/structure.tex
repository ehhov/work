\documentclass[12pt, a4paper]{article}
\usepackage{style}
\geometry{top=1.2in}
\geometry{bottom=1.2in}
\geometry{left=1in}
\geometry{right=1in}
\usepackage[nottoc]{tocbibind}
\usepackage{multicol}

\begin{document}
\thispagestyle{empty}
\vspace*{-50pt}
\begin{center}
\sc Московский Государственный Университет имени М.\,В. Ломоносова
\end{center}
\vspace*{-1.2\baselineskip}
\hrulefill
\vspace*{-.65\baselineskip}
\begin{center}\sc Физический Факультет\end{center}
\vspace{180 pt}
\begin{center}
\fontsize{22}{22}\selectfont 
Исследование структуры нуклона в экспериментах на Большом адронном коллайдере
\end{center}
\vfill
\begin{flushright}
Выполнил студент 313 группы\\ Гусейнов Керим Демирович %\hspace*{0pt}

%\vspace{5pt}
Преподаватель:\hspace*{3.5pt}%\hspace*{0pt}
\\Поправко Елена Сергеевна
%\hspace*{0pt}
\end{flushright}

\vspace{7cm}

\begin{center}
Москва --- 2019
\end{center}

\newpage
\tableofcontents

\newpage

%список ускорителей в википедии.

%электронные ускорители с 30-х годов; потом циклотрон - протоны - 1932 год

%в 50-х активно стали строить ускориели. 60е - коллайдеры на е-е+, а в 70 протонные коллайдеры. 

%синхрофазотрон в 1957 года в россии

%1959 год протон циклотрон - церн

%в книжке страница 60 --> ускоритель в Протвино.

%Антипротон в 56 году. Большинство частиц было открыто в космических лучах. 

%Основная хар-ка сильного вз - множественное рождение частиц.

%на HERA было получено довольно много объектов. Структура изучается в электронных ускорителях. Точечная структура протонов была обнаружена именно там Фейнманом. Глубоконеупругое рассеяние. 1969 год - предложены партоны. Точечноподобные составляющие 
%лептоны на адронах. Потом были отождествлены с кварками и глюонами. Сами партоны были обнаружены в экспериментах, а кварки и глюоны -- необходимые из-за симметрии элементы. 
%исследовать, откуда писать то, что непонятно

%фейнман партон. партонная модель первое знакомство на нуклфизе. 

%при содударении обнаружили рождение частиц -- сильное взаимодействие -- по рассеянию электронов на протонах надо 

%исследование структуры -- рассеяние электронов на гера -- исследование столкновений протонов с протонами 
%систематическое исследование структуры протона было выполнено на ускорителе HERA в DESY -- научный центр. 
%на н1 и зевс были измерены структурные функции протона -- послужили базой для моделирования экспериментов на баке. Взаимодействие частиц в СМодели эм, сильное, слабое -- все исследуется на баке. 


%современные представления о структуре протона. 
%что такое кхд -- теория сильных вз-й, описывает вз-е между партонами именно. при столкновениях протонов на больших энергиях происходит как бы столкновения именно партонов 

%для описания вз-я нужно знать импульс партонов
%импульсы партонов неизвестны = распределение импульса протона в партонах разной природы называется структурной функцией партона. 

%Проиллюстрировать эти функции. Отдельно распределение валентных и морских. Глюоны тоже очень важная часть - несут половину импульса протона. 
%привести табличку масс кварков. виртуально присутствуют Б и Ц кварки тоже есть -- виртуальные. 
%понятие структурной функйии

%структура адрона меняется действуем на протоны заданной $Q^2$ вз-е происходит с большей передачей импульса, более тонкую структуру видим. Q2 прхоже на длину волны света. 

%Привожу картинки для разных Q2. Проанализировать изменения. С большим ку2 проходим в большие доли импульса $10^{-4}$ происходят реакции для таких долей тоже. 

%теперь при наблюдении этих процессов -- рассеяние партонов -- улетают.  


%кварки из понятия симметрии адронов -- из структуры 

%загуглить точечная структура протона фейнман. 
%фейнман партонная модель.

%скейлинг - следствие партонной модели. 

%Ускорители хорошо работают -- понимание структур и динамики приводит к строению машин для медицины. 

\section{Введение}

Люди всегда интересовались тем, что в повседневной жизни скрыто от их восприятия. Например, к этому относятся предметы, не различимые невооруженным глазом. Открытие увеличительных оптических систем, позволявших человеку своими глазами видеть довольно маленькие объекты, было невероятным прорывом в их исследовании. Однако, ввиду чувствительности глаза только в диапазоне сравнительно больших длин волн и дифракционного предела, даже идеальные оптические системы позволяют рассматривать предметы с линейными размерами лишь порядка $10^{-7}$ метров, а интерес существует и к меньшим. 
В то же время, после тщательного изучения электромагнитного взаимодействия и преодоления некоторых технологических проблем стало ясно, что исследовать форму (а именно, распределение заряда) удобнее с помощью рассеяния зарядов на предмете. Например, в таких экспериментах \cite{Ni-60_charge} было получено распределение электрического заряда в атомных ядрах, например ядре $^{60}$Ni.

Изначально использовались пучки электронов и $\alpha$-частиц природных радиоактивных веществ, поскольку было сложно получить более высокую энергию. Но такие источники сильно ограничивали ученых, поэтому разрабатывались специальные установки, разгоняющие частицы -- ускорители. И в 1932 году Стенли Ливингстон и Эрнест Лоуренс создали первый циклический протонный ускоритель -- циклотрон на энергию 1.2 МэВ. В дальнейшем энергия ускорителей очень быстро росла -- более чем в 30 раз каждые 10 лет. Например, энергия 4.8 МэВ была достигнута в том же 1932 году. Далее ускорители применялись не только как инструмент исследования структуры материи, но и для изучения взаимодействий. В 60-х годах прошлого века удалось сконструировать коллайдер -- ускоритель на встречных пучках. Принципиальный выигрыш по сравнению с обычным ускорителем в том, что энергия не тратится на разгон центра масс системы частиц. А в 1970-х стали строить коллайдеры на протонных пучках. 

Сейчас на ускорителях проводят самые различные эксперименты, нацеленные на многие области физики частиц: от создания ядер новых химических элементов до уточнения параметров адронов и поиска новой физики.

\section{Теоретическое представление}
\subsection{Фотоны и глюоны}

Согласно Стандартной модели \cite{std_model}, мире существует четыре типа взаимодействия: сильное, электромагнитное, слабое и гравитационное. Их основные характеристики можно описать так: сильное взаимодействие имеет константу 1, радиус действия $10^{-15}$ м, переносится восемью глюонами, а участвуют в нем глюоны и кварки; электромагнитное имеет константу $1/137$, действует на неограниченные расстояния, переносится фотонами и действует на заряженные частицы; слабое имеет константу $10^{-5}$, переносится массивными $W^\pm$ и $Z$ бозонами на $10^{-18}$ м и действует на лептоны и кварки; гравитационное взаимодействие имеет наименьшую константу -- $10^{-38}$, но действует на неограниченные расстояния, предположительно, переносится гравитоном и действует на все массивные частицы (последнее остается правдой вплоть до релятивистских энергий, при которых гравитация начинает действовать не на массу, а на энергию, что учитывает общая теория относительности). Свойства каждой силы определяются ее переносчиками. Гравитационное взаимодействие на квантовом уровне не изучено вообще, а переносчики слабого взаимодействия имеют массу. В свою очередь глюоны, как и фотоны, не имеют массу, поэтому есть смысл сравнить их свойства.

%\begin{table}[!b]
%\caption{Основные свойства фундаментальных взаимодействий.}
%\renewcommand{\arraystretch}{2}
%\begin{tabular}{| c | c | c | c | c |}
%\hline
%Взаимодействие & Константа & Радиус, м & Переносчики & Участники \\
%\hline 
%Сильное & 1 & $10^{-15}$ & $g_1\ldots g_8$ -- глюоны & кварки, глюоны \\Электромагнитное & $\dfrac{1}{137}$ & $\infty $ & $\gamma$ -- фотон & \parbox{98pt}{\centering заряженные\\частицы, фотоны} \\
%Слабое & $10^{-5}$ & $10^{-18}$ & $W^\pm,Z$ & лептоны, кварки \\
%Гравитационное & $10^{-38}$ & $\infty$ & $G$ -- гравитон & \parbox{98pt}{\centering все массивные\\ частицы} \\
%\hline
%\end{tabular}
%\renewcommand{\arraystretch}{1.0}
%\label{tab:interactions}
%\end{table}

Изначально кварковая модель была введена, чтобы описать одновременно все характеристики открытых адронов -- спин, изоспин, странность и заряд, тем самым уменьшив число фундаментальных частиц. В 1951 году были проведены эксперименты по рассеянию $\pi$-мезонов на протонах \cite{delta++}. В результате было получено отношение сечений взаимодействия для $\pi^-$ и $\pi^+$ мезонов, и оно не совпадало с рассчитанным теоретически. Однако, если вести расчет, исходя из образования частицы, похожей на протон, но с зарядом +2 и спином 3/2, то результат будет соответствовать эксперименту. Этой частицей был $\Delta^{++}$-резонанс с энергией возбуждения 277 МэВ. Сама кварковая структура $\Delta^{++}$-резонанса, а именно, $\left(u_\uparrow u_\uparrow u_\uparrow\right)$, указывает на наличие ранее не известного квантового числа, поскольку все кварки -- фермионы, причем оно должно принимать три различных значения для каждого аромата. Это квантовое число назвали цветом, а его значения -- красным, зеленым и синим. Глюоны в том виде, в котором мы воспринимаем их сейчас, появились именно после введения цвета. Для цветового заряда, как и для любого существующего на самом деле квантового числа, выполняется закон сохранения. Это значит, что кварк при испускании глюона должен менять цвет, а сам глюон должен переносить цвет и антицвет. Поскольку цветов три: к, з, с, то антицветов тоже три: \textbar{к}, \textbar{з}, \textbar{с}, а их комбинаций -- девять. Из них шесть явно окрашенных, а три имеют неявный цвет, поэтому из них необходимо составить более явно окрашенные комбинации. Таких получается всего две, поскольку комбинация к\textbar{к}$+$з\textbar{з}$+$c\textbar{c} не имеет цвета вообще. 

Как видно из предыдущих рассуждений, как кварки, так и глюоны переносят заряд. То есть глюоны -- переносчики взаимодействия -- также испускают другие глюоны, а значит, взаимодействуют. В свою очередь, для электромагнитных сил характерны только два типа зарядов -- положительные и отрицательные. Поэтому фотонам не нужно переносить с собой электрический заряд -- фотоны не взаимодействуют. Именно в этом отличие электромагнитных сил от ядерных.

Самая первая кварковая модель была сравнительно простой, однако, как оказалось, не вполне верной. Изучение структуры протона выявило довольно большое количество неожиданных его характеристик. Экспериментальные исследования \cite{DIS} показали, что протон содержит в себе гораздо больше типов кварков, чем предполагает структура $(uud)$. Оказалось, что вещество (например, электрический заряд) внутри протона сосредоточено в трех областях. В центе каждой такой области располагается один кварк, называемый валентным, а вокруг него находится огромное количество кварков и глюонов. Такие кварки называют морскими, а область их обитания -- кварковым морем.



%Э-м взаимодействие используется и сейчас. Просто после обнаружения сильного взаимодействия стало интересно изучать именно его. 

% В результате наступает момент, когда рассеяние не успевает произойти. В то же время, как только сильное взаимодействие было обнаружено, начались попытки использовать его в качестве инструмента исследования \colour{(Правда ли это?)}. Проблема оказалась гораздо более сложной, поскольку сильное взаимодействие имеет большое количество особенностей. Несмотря на это, очередной шаг в был предложен в \colour{??? (Когда было предложено сталкивать протоны?)} году. Идея заключалась в том, что для исследования сильного взаимодействия можно использовать само сильное взаимодействие. 

%Изучали структуру вещества все равно. Просто параллельно стало интересно исследовать еще и сильное взаимодействие. 

%То есть можно сталкивать протоны и изучать результат. Таким образом выяснились некоторые особенности сильного взаимодействия и были предложены их объяснения. Например, кварки не могут находиться в свободном состоянии, а при разделении двух кварков образуются так называемые струи, константа взаимодействия уменьшается с увеличением энергии, {\color{red} придумать еще что-нибудь}. Кварки обладают цветом - есть цветовое поле. . Далее будут рассмотрены некоторые из них.

\subsection{Кварковый конфайнмент и константа сильного взаимодействия}

%Можно добавить текст

При столкновении адронов на ускорителях при больших энергиях одним из наиболее часто встречающихся явлений оказывается образование адронных струй -- многочисленного сгустка адронов, вылетающего в очень узкий конус. 

Поскольку изучение струй -- один из наиболее удобных способов исследования нуклонов, целесообразно рассмотреть пример их образования. Для этого проведем более глубокое сравнение с электромагнитным взаимодействием, основанное на уже описанном. Представим себе силовые линии при взаимодействии двух кварков. Как уже упоминалось, глюоны тоже переносят цвет, а значит, взаимодействуют между собой и с кварками. В результате этого при разъединении кварков, то есть увеличении расстояния между ними, глюоны все большее время проводят вдали от них. А значит, роль межглюонного взаимодействия увеличивается -- они притягиваются друг к другу и концентрируются в очень узком цилиндре, соединяющем кварки. Причем число силовых линий не меняется, а значит, их плотность в этом цилиндре становится весьма велика. То же самое касается энергии межкваркового взаимодействия. В результате при попытках разъединить кварки она только увеличивается и становится настолько большой, что в вакууме между ними рождаются вторичные кварки или другие частицы, участвующие в сильном взаимодействии, а при дальнейшем движении первичных кварков этот процесс многократно повторяется, а оставшиеся кварки адронизуются, подобрав другие кварки из вакуума. В итоге образуется струя адронов.

Заметим, что неупругое рассеяние протона на протоне, реакция с образованием большого количества частиц, может быть упругим рассеянием цветовых зарядов одного протона на аналогичных частицах другого.

%\begin{figure}
%\begin{subfigure}{.4\textwidth}
%\includegraphics[width=\linewidth]{colour_field}
%\caption{}
%\label{fig:string}
%\end{subfigure}
%\hspace*{1cm}
%\begin{subfigure}{.4\textwidth}
%\includegraphics[width=\linewidth]{string_quark_creation}
%\caption{}
%\label{fig:string_quark_creation}
%\end{subfigure}
%\caption{(a) Переход от подобного Кулоновскому потенциала между цветовыми зарядами на малых дистанциях к струнному на больших. (b) Разрыв струны посредством создания кварк-антикварковой пары в цветовом поле.}
%\end{figure}

%\begin{figure}[!b]
%\includegraphics[width=.6\textwidth]{antishielding}
%\caption{Полный путь преобразований глюона, испущенного свободным кварком.}
%\label{fig:strange}
%\end{figure}

Соотношение неопределенности Гейзенберга показывает, что, с точки зрения измерительных приборов, закон сохранения энергии может быть нарушен на величину $\Delta E$ на время $\Delta t \approx \hbar/\Delta E$ (разумеется, закон сохранения энергии не нарушается вообще никогда, однако мы не можем с абсолютной точностью утверждать, что система не имеет достаточную для какого-либо процесса энергию, если его временные и энергетические шкалы удовлетворяют соотношению неопределенности). Поэтому в пространстве всегда находится большое количество частиц, возникающих и аннигилирующих в соответствии с соотношением неопределенности. Если на эти частицы не действуют никакие внешние силы, то есть для них все положения в малой области пространства одинаково выгодны, то усреднение по физически значимому промежутку времени даст нулевой результат. Однако если внешние силы есть, то наличие спонтанно возникающих частиц приведет к действительно имеющим место в физике явлениям. 

%\begin{figure}
%\image[width=0.6\textwidth]{alpha_s_running}
%\caption{Зависимость константы сильного взаимодействия от энергии в эксперименте.}
%\label{fig:alpha_s_on_E}
%\end{figure}

Рассмотрим свободный кварк. Он непрерывно испускает глюоны, которые, в свою очередь, рождают кварки или вторичные глюоны. Оказывается, вероятность появления вторичных глюонов выше вероятности появления вторичных кварков. 
%С учетом закона сохранения цветового заряда, один из возможных процессов испускания глюона будет выглядеть, как представлено на рисунке \ref{fig:strange}. 
Причем глюоны не будут располагаться хаотично. Взаимодействие между цветом и антицветом наиболее сильно, поэтому глюон, имеющий в себе антицвет кварка будет ближе к нему. Получается, что цветовой заряд исходной частицы распределяется по объему, превышающему ее объем. Когда в поле кварка попадает другой кварк, он ощущает притяжение со стороны первого с силой, соответствующей цветовому заряду внутри сферы с центром в первой частице, на поверхности которой находится вторая. Этот заряд меньше полного заряда первого кварка. Это и есть причина кваркового конфайнмента: при попытке увеличить расстояние между кварками потенциал их взаимодействия увеличивается, поскольку увеличивается количество заряда в описанной выше сфере. Величина сильного взаимодействия (а значит, и константа) зависит от расстояния между цветовыми зарядами. В свою очередь при увеличении энергии движения зарядов расстояние между ними уменьшается. Поэтому константа сильного взаимодействия зависит от энергии в эксперименте и уменьшается при ее увеличении. Это утверждение было проверено и подтверждено \cite{strong_coupling}.

\subsection{Партонная модель адронов}
В экспериментах по глубоконеупругому рассеянию электронов на протонах на SLAC \cite{SLAC} при энергии 20 ГэВ было обнаружено, что на большие углы рассеивается больше частиц, чем если бы протон был однородным. То есть, аналогично экспериментам Резерфорда с $\alpha$-частицами и золотой фольгой, была обнаружена точечная структура протона. Только она содержала не один центр, а три. Тогда Фейнман предложил, что при больших импульсах протон выглядит, как облако слабо взаимодействующих между собой точечных частиц, а эти частицы назвал партонами. Позже они были отождествлены с кварками, антикварками и глюонами.

%\begin{figure}
%\includegraphics[width=0.5\textwidth]{deepis}
%\caption{Кинематические характеристики при глубоконеупругом рассеянии.}
%\label{fig:dis}
%\end{figure}

%\question{Нужно ли вообще? начало}
%В зависимости от энергии частиц, их взаимодействие может быть упругим или неупругим. При упругом частицы передают друг другу энергию и импульс, но их состояние не меняется, а при неупругом происходит изменение структуры или состояния одной или обеих частиц. Пусть на нуклон массой $M$ с четырехимпульсом $\P$ налетает лептон с массой $m_\ell$ и четырехимпульсом $k$. В результате образуются другой лептон с четырехимпульсом $k'$ и массой $m_{\ell'}$ и какой-то набор частиц с полной массой $W$. Такой процесс, описываемый уравнением $\ell + N \rightarrow \ell' + X$, изображен на рисунке \ref{fig:dis}. При больших энергиях нуклоны представляют собой облака слабо взаимодействующих между собой партонов, поэтому его можно рассматривать, как взаимодействие кварка в нуклоне с фотоном или $Z$-бозоном, выпущенным лептоном. Обозначим $q=k-k'$, а начальный импульс кварка -- $\vec{p}$ и введем параметр кварка $x$ такой, чтобы $\vec{p}=x\,\vec{p}_{\text{адр}}$, где $\vec{p}_{\text{адр}}$ -- импульс адрона, в котором содержится кварк, в нашем случае это нуклон. Число $x$ находится в диапазоне от $0$ до $1$ и называется продольной долей кварка. Пусть в системе отсчета покоя адрона начальная и конечная энергии лептона равны $E$ и $E'$ соответственно. Тогда $E-E'=\dfrac{q\cdot\P}{M}$.
%Запишем теперь параметры этого процесса. 
%
%
%
%Вектор $q$ пространственноподобен, поэтому принято вводить параметр $Q^2=-q^2$ -- квадрат переданного электроном импульса. Тогда $Q^2=2(EE'-k\cdot k') -m_\ell-m_{\ell'}$, причем, если $EE'\sin ^2\br{\theta/2}\gg m_\ell^2,\ m_{\ell'}^2$, то $Q^2\approx 4 EE' \sin^2\br{\theta/2}$, где $\theta$ -- угол рассеяния лептона относительно направления первоначального движения. Также, если лептон взаимодействует с кварком, несущим часть $x$ импульса нуклона, то $Q^2=2xM(E-E')$. $y=\dfrac{q\cdot \P}{k\cdot \P}=\dfrac{E-E'}{E}$ -- доля потерянной лептоном энергии в системе отсчета покоя нуклона.
%Чем больше $Q^2$, тем больше вероятность, что кварк покинет нуклон, поскольку он один берет на себя весь переданный импульс. В результате при больших $Q^2$ кварк вылетает, и образуется адронная струя. 
%Этот процесс называется глубоко $\lt(Q^2\gg M^2\rt)$ неупругим $\lt(W^2\gg M^2\rt)$ рассеянием. 
%\question{Нужно ли вообще? конец}

В наше время для описания структуры адронов вводят так называемые функции партонных распределений. Построим ее для одного кварка, чтобы понять общий смысл.

Пусть импульс какого-либо партона в адроне равен $\vec{p}$, а импульс всего адрона -- $\vec{p}_{\text{адр}}$. Введем параметр $x$ такой, что $\vec{p}=x\,\vec{p}_{\text{адр}}$. Это число называется продольной долей партона в адроне.

Запишем вероятность $dp$ найти партон $i$ в адроне с продольной долей в интервале от $x$ до $x+dx$ в виде $dp=f_i(x)dx$. Определенная так функция $f_i(x)$ называется партонной функцией распределения партона $i$.
%Квантовая хромодинамика не позволяет вычислять вероятности нахождения кварков в нуклонах и их импульсы, значит, эти величины надо рассматривать, как неизвестные, которые необходимо найти из эксперимента. 
Теория возмущений в квантовой хромодинамике позволяет теоретически находить сечения элементарных процессов с участием кварков и глюонов. А значит, если имеются их функции распределения, то, проинтегрировав по всем им и по всем их импульсам, можно найти сечение более сложного процесса \cite{new_phys}. Например, для рассеяния лептона на нуклоне оно запишется следующим образом:
$$\sigma(\ell N(\vec{p}_{\text{адр}})\rightarrow\ell'X)=\int \limits_0^1\br{\sum\limits_i f_i(x)\ \hat{\sigma}(\ell q_i(x\,\vec{p}_{\text{адр}})\rightarrow\ell'X)}dx .$$

%\question{Хотел бы написать хотя бы формулу для определения $f_i(x)$ через сечения}
%С учетом наличия различных путей взаимодействия, дважды дифференциальное сечение записывается в виде 
%$$\frac{d^2\sigma}{dx\,dy}=\br{\sum\limits_i xf_i(x)Q^2_i}\frac{2\pi\alpha^2s}{Q^4}\br{1+(1-y)^2}. $$
%\question{Эту формулу я взял из курсовой Волкова Игоря, которую вы мне присылали}


\subsection{Партонные функции распределения}

Рассмотрим подробнее партонные распределения и то, каким требованиям они должны удовлетворять. Обозначим
$ P_q = \int\limits_{0}^{1}\big(f_q(x)-f_{\bar{q}}(x)\big)dx, $
где $q$ и $\bar{q}$ есть кварк и антикварк того же аромата. Величина $P_q$ является характеристикой того, сколько кварков аромата $q$ содержит адрон. Структура протона, например, должна характеризоваться составом $(uud)$. Значит, для него $P_u=2$, $P_d=1$, $P_s=P_c=P_b=0$. К тому же, по определению функций $f_i(x)$, для любого адрона они должны удовлетворять следующему уравнению, являющемуся законом сохранения импульса: 
$\int\limits_{0}^{1}\sum\limits_i x\,f_i(x)dx=1.$

%\begin{figure}[!p]
%\image[width=.9\textwidth]{pdgPDF}
%\caption{Функции распределения кварков, антикварков и глюонов в протоне при масштабах $\mu^2=10$ ГэВ$^2$ и $\mu^2=10^4$ ГэВ$^2$.}
%\label{fig:pdgPDF}
%\end{figure}

На практике партоны в адроне не существуют все время жизни адрона, а постоянно рождаются и умирают, эти процессы называют партонными флуктуациями. 
Более того, ввиду квантовых флуктуаций и антиэкранировки, партонные распределения зависят от квадрата переданного адрону в процессе реакции импульса.
Это проявляется в том, что, чем выше $Q^2$, тем больше глюонов испускают кварки в процессе реакции, а глюоны, в свою очередь, рождают кварк-антикварковые пары. Этот процесс приводит к смягчению распределений импульсов валентных кварков и росту плотности глюонов и кварк-антикваркового моря в области малых $x$.

На рисунке \ref{fig:pdgPDF} продемонстрированы партонные функции распределения для протона при двух квадратах переданного импульса \cite{pdf_pdg}. В них отчетливо видны валентные кварки, имеющие максимум доли импульса, и морские кварки, доля импульса которых возрастает с уменьшением продольной доли $x$, о чем уже говорилось выше, и практически равна нулю при $x=0.3$. Аналогично дело обстоит с глюонами, но, поскольку их масса равна нулю, а вероятность рождения больше, чем у кварков, они участвуют в большем числе партонных флуктуаций, их плотность значительно превосходит плотность морских кварков. Кроме того, видны глобальные изменения партонных распределений, например, при большем $Q^2$ стал виден $b$ кварк. Он имеет довольно большую массу, поэтому в партонных флуктуациях появляется на очень короткое время. При $Q^2=10$ ГэВ$^2$ он успевал родиться и умереть во флуктуациях, не прореагировав ни с одной другой частицей.

\section{Эксперимент ATLAS}
%\subsection{Устройство детектора}
Для проведения экспериментов сейчас используется большое количество ускорителей и детекторов. Один из них -- детектор ATLAS на Большом адронном коллайдере \cite{LHC}.

%\begin{figure}[!b]
%\begin{subfigure}{0.45\textwidth}
%%\image[width=\linewidth]{ATLASoutside}
%\caption{}
%\label{fig:ATLASoutside}
%\end{subfigure}
%\hspace*{0.3cm}
%\begin{subfigure}{0.45\textwidth}
%%\image[width=\linewidth]{LHClayers}
%\caption{}
%\label{fig:layers}
%\end{subfigure}
%\caption{(a) Общий вид детектора ATLAS. (b) Слои детектора ATLAS и детектирование различных видов частиц.}
%\end{figure}

Это самый большой детектор в мире, он имеет цилиндрическую форму, а его размеры составляют 44 метра в длину и 25 в диаметре. Сам детектор состоит из нескольких слоев, каждый из которых реагирует на определенный тип частиц. В трековой камере на заряженные частицы действует магнитное поле, там остаются треки заряженных частиц. За ней находится электромагнитный калориметр -- там легкие, участвующие в электромагнитном взаимодействии частицы теряют всю энергию и оставляют соответствующий след. Следующая часть -- адронный калориметр. В нем теряют энергию все сильно взаимодействующие частицы -- адроны. Последняя камера предназначена для мюонов: вероятность их торможения на предыдущих детекторах ничтожно мала, поэтому необходим отдельный. 
%На рисунке \ref{fig:layers} изображена реакция детектора на частицы различной природы.

\subsection{Экспериментальная проверка партонных распределений}

%\begin{table}
%\caption{Процессы на большом адронном коллайдере, чувствительные к партонным распределениям, и их характерные особенности.}
%%\begin{tabular}{c}
%%\includegraphics[width=.9\textwidth]{PDFdetProcessesTable}
%%\end{tabular}
%\label{tab:PDFprocesses}
%\end{table}

Для исследования партонных распределений сейчас используют различные процессы \cite{pdf_det}, которые чувствительны в разных диапазонах продольных долей партонов и переданных импульсов, в результате чего покрывают большое количество ситуаций. 
%Основные особенности таких экспериментов на БАКе приведены в таблице \ref{tab:PDFprocesses}.
Среди них образование векторных бозонов, адронных струй, {кварк-"=антикварковых} пар и другие процессы. Рассмотрим подробнее рождение $W^\pm$ и $Z$ бозонов. Время их жизни мало, поэтому распознавать их появление можно только по продуктам распада. Кроме того, при протонных соударениях образуется слишком много адронов, а определить их истинные источники весьма проблематично. В результате самыми точными сигналами появления $W$ и $Z$ бозонов являются конечные состояния следующих распадов: 
$W\rightarrow\ell\nu,\ Z\rightarrow\ell\ell$, где $\ell$ -- электрон или мюон (эффективно регистрировать таоны довольно сложно), а $\nu$ -- соответствующее нейтрино. 

Проследим последовательно за проведением эксперимента и обработкой данных для проверки и уточнения партонных функций распределения через образование $Z$-бозона \cite{render}. Исследование продвигается сразу с двух сторон.

На генераторе событий, устроенному по принципу Монте-Карло, строится модель экспериментальной установки. Она должна учитывать все ее особенности, недостатки и преимущества. Например, образование вторичных частиц в определенных частях детектора является такой характеристикой.
%Например, в рассматриваемом случае необходимо учесть наличие фоновых . 
%\question{Какие именно они, что нужно учесть?}
Довольно большое количество теоретических исследовательских групп по всему миру рассчитывает партонные функции распределения для различных адронов, исходя из диаграмм Фейнмана. 
Из этого набора выбираются подходящие партонные функции распределений (а чаще несколько, от разных групп) и ставится эксперимент по столкновению протонов, которые для генератора представляют собой пучки партонов. Из получившихся результатов при многократном повторении рассчитываются сечения различных процессов, в нашем случае это образование $Z$ бозона. Оно может быть получено, как функция, например, его поперечного импульса.

На ускорителе ставится эксперимент, рассчитываются параметры всех зарегистрированных частиц. Затем определяются события-кандидаты, которые могут соответствовать рассматриваемому процессу. Для изучения рождения $Z$ бозона нужно сначала из всех событий выделить относящиеся к рождению какой-либо частицы -- они должны содержать как минимум один пик на гистограмме инвариантных масс продуктов распада, он должен быть восстановлен по меньшей мере из трех треков. Затем из всех таких событий отбирают относящиеся к рождению $Z$ бозона -- в них должны быть два противоположно заряженных лептона с инвариантной массой $m_{\ell\ell}$ в диапазоне от 66 до 116 ГэВ, который заведомо включает в себя весь пик, соответствующий нашей частице. Получившиеся события -- события-кандидаты. 
%Пример одного и таких показан на рисунке \ref{fig:singleZmumu} -- инвариантная масса мюонов составила 90.2 ГэВ \cite{Z_to_mumu}. 
%
%\begin{figure}
%%\image[width=0.7\textwidth]{singleZmumu}
%\caption{Зарегистрированный на БАКе распад $Z\rightarrow\mu\mu$.}
%\label{fig:singleZmumu}
%\end{figure}

Затем для каждого полученного в результате этих процедур события можно вычислить параметры изначально образовавшейся частицы и тем самым получить сечение образования, как функцию от этого параметра. В нашем случае изучается рождение $Z$ бозонов с различными поперечными импульсами. Для получения зависимости в каждом оставшемся после отбора событии определяется суммарный поперечный импульс лептонов, который и должен соответствовать $Z$ бозону, распавшемуся на них. 
Весь интервал полученных импульсов делится на более маленькие промежутки -- корзины с определенными границами -- и строят гистограмму зарегистрированных событий. Однако не все оставшиеся события соответствуют рождению $Z$ бозона. Для отбрасывания лишних на генераторе рассчитываются вероятности появления сторонних частиц (например, из-за многочисленных столкновений в калориметрах) и их параметры. Затем, умножая число событий в каждой корзине на соответствующую вероятность, получают истинное распределение. В результате одного из таких процессов \cite{pT_distributions} была получена изображенная на рисунке \ref{fig:ZpTHistogram} зависимость. Видно, что наибольшее количество $Z$-бозонов рождается с поперечным импульсом около 5 ГэВ.

%\begin{figure}[!p]
%\image[width=0.7 \textwidth]{crosssection_pT_data}
%\caption{Экспериментальная зависимость $1/\sigma\, d\sigma/dp_T^{\ell\ell}$ от суммарного поперечного импульса лептонов $p_T^{\ell\ell}$.}
%\label{fig:ZpTHistogram}
%\end{figure}
%
%\begin{figure}
%%\image[width=.75\textwidth]{crosssection_pT_comparison_masses}
%\caption{Сравнение полученных в эксперименте и вычисленных на различных генераторах зависимостей $1/\sigma\, d\sigma/dp_T^{\ell\ell}$ от суммарного поперечного импульса лептонов в нескольких областях инвариантных масс последних.}
%\label{fig:comparison}
%\end{figure}

%\begin{figure}
%\begin{subfigure}{.45\textwidth}
%%\image[width=\linewidth]{Zmass_ee}
%\caption{}
%\end{subfigure}
%\hspace*{0.1cm}
%\begin{subfigure}{.45\textwidth}
%%\image[width=\linewidth]{Zmass_mumu}
%\caption{}
%\end{subfigure}
%\caption{Гистограммы инвариантных масс электронов (а) и мюонов (b) и процессы, дающие вклад в распределение.}
%\label{fig:Zmass}
%\end{figure}

Затем производят сравнение рассчитанных на генераторе и полученных при обработке эксперимента результатов и делают вывод о точности использованных в расчетах партонных распределений и уточняют их. 

Аналогично обработав распределение инвариантных масс лептонов \cite{pT_distributions}, можно уточнить массу $Z$ бозона. Также можно восстановить место распада бозона по трекам лептонов, сравнить его с положением вершины, в которой этот бозон был рожден, и затем, зная импульсы лептонов, можно вычислить полный импульс бозона и его массу, а значит и время жизни. 
%Полученные зависимости для электронов и для мюонов отдельно показаны на рисунке \ref{fig:Zmass}.

Также исходя из экспериментальных данных можно определить, например, сечения рождения $W^\pm$ и $Z$ бозонов \cite{ellipses} и, сравнивая их с рассчитанными с помощью структурных функций, уточнить последние.
%Такое сравнение приведено на рисунке \ref{fig:W_over_Z}. 
Имеющиеся сейчас теоретические результаты разумно соотносятся с экспериментальными, то есть партонные распределения действительно описывают протон.

%\begin{figure}
%%\image[width=0.6\textwidth]{sigmaW_over_sigmaZ}
%\caption{Рассчитанные с помощью различных пакетов структурных функций и экспериментальные сечения рождения $W^\pm$ и $Z$ бозонов.}
%\label{fig:W_over_Z}
%\end{figure}

%При обработке экспериментальных данных находятся события, содержащие как минимум один основной пик, восстановленный по меньшей мере из трех треков. События, в которых участвовал $Z$-бозон отбираются по наличию двух противоположно заряженных лептонов с инвариантной массой $m_{\ell\ell}$ в диапазоне от 66 до 116 ГэВ. Затем для всех событий из импульсов лептонов определяется поперечный импульс $Z$-бозона. По этому параметру все события распределяются в корзины с определенными максимальным и минимальным импульсами. Для каждой корзины определяется сечение и дифференциальное сечение процесса, а также статистические и систематические погрешности. 

\subsection{Программа для визуализации событий на ATLAS}
Для более подробного ознакомления с процессом обработки экспериментальных данных можно воспользоваться программным пакетом для визуализации событий на ATLAS -- \textsc{Hypatia}. 
%Внешний вид меню управления показан на рисунке \ref{fig:hyp_menu}

%\begin{figure}
%\begin{subfigure}{0.35\textwidth}
%%\image[width=\linewidth]{Hypatia_screen}
%\caption{}
%\label{fig:hyp_menu}
%\end{subfigure}
%\hspace*{0.2cm}
%\begin{subfigure}{0.5\textwidth}
%%\image[width=\linewidth]{leptonsHistogram}
%\caption{}
%\label{fig:hyp_leptons}
%\end{subfigure}
%\caption{(a) Общий вид меню управления программы \textsc{Hypatia}. (b) Гистограмма инвариантных масс лептонных пар, полученная при обработке событий в \textsc{Hypatia}.}
%\end{figure}

Сама программа представляет собой графическое меню с большим количеством опций, например, возможностью загрузки и выбора набора событий. Одна из панелей программы показывает треки частиц, проходящих через детектор. Положение слоев детектора соответствует реальному, а участвующие в определенных взаимодействиях частицы оставляют характерные для себя треки, то есть возможно распознавание частиц по трекам. Также возможно выделение треков и помещение в отдельный список для последующей обработки. Можно, например, отобрать события с двумя лептонами разных зарядов и получить распределение их инвариантных масс.

Такие события должны соответствовать рождению $Z$ бозона. В графическом интерфейсе отбор событий занимает существенное время, поэтому составить реально значимую выборку невозможно. Однако полученные значения инвариантных масс хоть и немногочисленны, но разбросаны около его табличной массы -- 91.2 ГэВ.

\section{Заключение}
В работе были обоснованы необходимость введения цветового заряда, существование кваркового конфайнмента и асимптотической свободы, а также описаны основные свойства и особенности квантовой хромодинамики -- теории сильных взаимодействий в Стандартной модели. Причиной большинства из них является наличие межглюонных взаимодействий.

Было изучено понятие партонных распределений и рассмотрены их особенности, в том числе, проявляющиеся из-за партонных флуктуаций.

Рассмотрена структура детектора ATLAS Большого адронного коллайдера и описан ход эксперимента по изучению рождения векторных бозонов. Кроме того, приведены сравнения экспериментальных данных с рассчитанными при помощи различных пакетов партонных распределений.

Для более подробного ознакомления с процессом обработки изучена программа \textsc{Hypatia}. В ней получена гистограмма инвариантных масс лептонных пар для событий с рождением $Z$ бозона.
 
%То же самое происходит и при взаимодействиях между адронами. 

%Рассмотрим подробнее сильное взаимодействие в адрон-адронном столкновении. Пусть каждый партон несет долю $x$ от полного импульса адрона, то есть $p_i=x_ip$, а вероятность $\d P$ найти в адроне партон $i$ с долей импульса в интервале от $x$ до $x+\d x$ равна $\d P=f_i(x)\d x$. Определенная так функция $f_i$ называется функцией распределения. Поскольку константа взаимодействия убывает обратно пропорционально энергии или, что равносильно, импульсу, то только партоны с наименьшеими долями импульса $x$ одного пучка будут взаимодействовать с аналогичными партонами другого. То есть сечение взаимодействия адронов определяется как сечением взаимодействия самых медленных партонов друг с другом, так и фукнцией распределения. 

%При моделировании эксперимента удобным оказывается рассмотрение диаграмм Фейнмана.
%На них частицы обозначаются прямыми, взаимодействие частиц -- узлами, а частицы, заключенные между двумя узлами фактически только переносят взаимодействие и называются виртуальными, поскольку не могут быть зарегистрированы. Вероятность того, что между частицами произойдет взаимодействие, пропорциональна квадрату константы взаимодействия. 


%\newpage
%\centr{Кусочки тем} обнаружение структуры нейтрона; резонансы; поэтому нуклоны не фундаментальны; электронные микроскопы; дифракционный предел и необходимая для различения частей нуклона энергия; недостаточность энергии электронов; более простое решение на протонных ускорителях; результаты первых экспериментов и примеры простейших устройств ускорителей; более точная структура протона; понятие партона; образование струи; функции распределения; массовые доли и механистическое представление; моделирование происшествий на компьютерах; антиэкранировка и почему параметр $\alpha_s$ мал; результаты экспериментов; уточнение параметров для будущих расчетов; 

%примеры принципиальных ошибок (Есть ли явления, которые предполагались в теории, но были опровергнуты в эксперименте?); нахождение новых физических явлений и их учет в будущем; 

%\begin{addmargin}[2cm]{0.5cm}
%В 1970-х в теоретической области физики сильных взаимодействий господствовала теория полюсов \question{пару слов о ней}, предсказывающая падение полных сечений адронных взаимодействий при достижении энергии выше 20 ГэВ, однако в 1973 г. в экспериментах на российском ускорителе в Протвино обнаружили их неожиданный рост. Это открытие заставило практически с нуля искать объяснения всем явлениям физики высоких энергий. 
%\end{addmargin}

\clearpage
%\begin{tabular}{rlrl}
%reveal & выявить & focused & сосредоточено
%\end{tabular}

\begin{center}
\bf Глоссарий
\end{center}

\begin{multicols}{2}\centering

\begin{tabular}{rl}
	advantage & преимущество \\
	attraction & притяжение \\
	averaged & усредненный \\
	bias & отклонение \\
	bin & корзина, ячейка \\
	chamber & камера \\
	conservation law & закон сохранения \\
	consistent & последовательный \\
	cross-section & сечение \\
	decay products & продукты распада\\
	deflect & отклонять (траекторию) \\
	density & плотность \\
	distinctively & отчетливо\\
	distribution & распределение \\
	emit & испускать \\
	field lines & силовые линии\\
	flaw & недостаток \\
	focused & сосредоточено \\
	foil & фольга\\
	force carrier & переносчик силы \\
	handle & обрабатывать \\
	in its turn & в свою очередь \\
	inelastic & неупрогое (рассеяние)\\
	interval & промежуток\\
	jet & струя\\
	layer & слой \\
	lifetime & время жизни \\
	longitudinal & поперечный \\
	magnifying & увеличительный \\

\end{tabular}

\centering
\begin{tabular}{rl}
	many times over & многократно\\
measuring device & измерительный прибор \\
momentum & импульс \\
multitudinous & многочисленный \\
neglectable & пренебрежимо \\
order & порядок \\
peak & пик, максимум \\
perceive & воспринимать \\
perception & восприятие \\
point-like & точечный \\
presence & присутствие \\
principled & принципиальный\\
production & образование \\
range & радиус действия \\
realization of an experiment \hspace*{-3cm}& \\
& \hspace*{-1cm}проведение эксперимента \\
refininement & уточнение \\
remaining & оставшийся \\
reveal & выявить \\
scale & масштаб \\
scattered & разбросаны \\
sensitivity & чувствительность \\
smooth & однородный \\
total & суммарный \\
training in & ознакомление с \\
transferred & переданный (импульс)\\
transverse & перпендикулярный \\
verification & проверка \\
wavelength & длина волны \\

\end{tabular}

\end{multicols}


\clearpage
\begin{thebibliography}{10}
\bibitem{Ni-60_charge} 
	H.\,D. Wohlfahrt, \textit{et al.}, ``Systematics of nuclear charge distributions in the mass 60 region from elastic electron scattering and muonic x-ray measurements,''{} Phys. Rev. \textbf{C22} (1980) 264--283.
\bibitem{std_model}
	G.~Altarelli, ``The Standard model of particle physics,''  hep-ph/0510281.
\bibitem{delta++} 
	K.\,A. Brueckner, “Meson-Nucleon Scattering and Nucleon Isobars,” Phys. Rev. \textbf{86} (1952) 106–109.
\bibitem{DIS}
	E. Bloom et al. ``High-energy inelastic $e$-$p$ scattering at 6$^\circ$ and 10$^\circ$'' Phys. Rev. Lett. \textbf{23} (1969) 930-938.
\bibitem{strong_coupling}
	G. Dissertori, ``The Determination of the Strong Coupling Constant,'' 2015 arXiv:1506.05407.
\bibitem{SLAC}
	J.\,D. Bjorken ``Theoretical ideas on high-energy inelastic electron-proton scattering,'' SLAC-PUB-571 (1969). 
\bibitem{new_phys}
	Н.\,В. Красников, В.\,А. Матвеев ``Новая физика на Большом адронном коллайдере.'' М.: Красанд, 2011.
\bibitem{pdf_pdg} %р 11
	Particle Data Group Collaboration, B. Foster et al., ``Structure functions,'' 2017.
\bibitem{LHC} %р 12
	The Large Hadron Collider, CERN-AC-95-05.
\bibitem{pdf_det}%p 12
	J. Rojo et al. ``The PDF4LHC report on PDFs and LHC data: Results from Run I and preparation for Run II,'' 2015 arXiv:1507.00556.
\bibitem{render}
	Measurement of the $W\rightarrow\ell\nu$ and $Z/\gamma^*\rightarrow\ell\ell$ production cross sections in proton-proton collisions at $\sqrt{s}=7$ TeV with the ATLAS detector, CERN-PH-EP–2010-037.
%\bibitem{Z_to_mumu}
%	\url{https://twiki.cern.ch/twiki/bin/view/AtlasPublic/EventDisplayRun2Collisions}
\bibitem{pT_distributions}
	Measurement of the transverse momentum and $\phi^*_\eta$ distributions of Drell–Yan lepton pairs in proton–proton collisions at $\sqrt{s} = 8$ TeV with the ATLAS detector, CERN-PH-EP-2015-275
\bibitem{ellipses}
	Measurement of the inclusive $W^\pm$ and $Z/\gamma^*$ cross sections in the $e$ and $\mu$ decay channels in $pp$ collisions at $\sqrt{s} = 7$ TeV with the ATLAS detector, CERN-PH-EP-2011-143


\end{thebibliography}

\begin{figure}[!p]
	\image[width=.9\textwidth]{pdgPDF}
	\caption{Функции распределения кварков, антикварков и глюонов в протоне при масштабах $\mu^2=10$ ГэВ$^2$ и $\mu^2=10^4$ ГэВ$^2$.}
	\label{fig:pdgPDF}
\end{figure}

\begin{figure}[!p]
	\image[width=0.7 \textwidth]{crosssection_pT_data}
	\caption{Экспериментальная зависимость $1/\sigma\, d\sigma/dp_T^{\ell\ell}$ от суммарного поперечного импульса лептонов $p_T^{\ell\ell}$.}
	\label{fig:ZpTHistogram}
\end{figure}

\end{document}