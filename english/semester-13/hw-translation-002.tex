\documentclass[a4paper, 10pt, twocolumn]{article}

% Configuration {{{
\usepackage[utf8]{inputenc}
\usepackage[T2A]{fontenc} % T1 for English
\usepackage[english, russian]{babel}

\usepackage{enumitem}
\setlist{nolistsep}
\usepackage{mathtools}
\usepackage{xcolor}
\definecolor{dimblue}{HTML}{1010aa}
\usepackage[
  colorlinks=true,
  allcolors=dimblue
]{hyperref}
\usepackage[
  vmargin=1in,
  hmargin=.8in
]{geometry}
\setlength{\columnsep}{.25in}
%\usepackage[nospread]{flushend}
\usepackage{cuted}
\linespread{1.3}
\usepackage{indentfirst}
\usepackage{graphicx}
\usepackage{tikz}
\usepackage[multidot]{grffile}
\usepackage[labelsep=period]{caption}
\usepackage{multirow}

\usepackage{titlesec}
\titleformat{\section}[hang]{\bf\centering}{\thesection.}{.5em}{}[]
\def\thesection{\Roman{section}}
\def\thefootnote{\fnsymbol{footnote}}

%\usepackage{times} % for English

\def\q#1{{\color{red} #1}}
% }}}

\begin{document}

% Title {{{
\begin{strip}
\begin{center}
\vskip -2.4\baselineskip
{\large\textbf{Black Holes and Gravitational Theory}}
\vskip .3\baselineskip
{Roger Penrose}
\vskip .3\baselineskip
\today
\end{center}
\end{strip}
% }}}

% Introduction {{{

Недавно было несколько высказываний, заявляющих об открытии черных дыр. 
В данной статье профессор Пенроуз описывает теоретические основания 
черных дыр и некоторые недавние рассуждения на эту тему.

Спустя примерно 10 в 10ой лет, согласно принятой теории, наше Солнце 
сначала расширится и станет красным гигантом (поглощая при этом Землю 
\q{Earth or the Earth?}), а затем сожмется в белого карлика -- звезду 
лишь планетарных размеров. Дальнейшему сжатию будет препятствовать 
давление вырождения электронов (следствие принципа Паули в применении 
к электронам). Однако, как Чандрасекар показал в 1931, без вращения 
звезда с массой больше 1.3 М (где М -- масса Солнца) не может в конечном 
итоге поддерживать себя в этом положении в противостояние дальнейшему 
гравитационному коллапсу. Известно о существовании звезд вплоть до 50 М, 
так что вопрос их конечной судьбы вполне реален. Хотя на этот вопрос 
в большой степени ответили Оппенгеймер и Шнайдер в 1939, большинство 
астрономов уделили этому мало внимания, по-видимому полагая, что 
в финальной стадии коллапса достаточно материи будет всегда выброшено, 
приводя массу итогового тела до массы ниже предела Чандрасекара -- или 
ниже предела Оппенгеймера-Волкова, который соответствует наибольшей 
массе (по-разному выражаемой от 0.7 до 3 М) нейтронной звезды (более 
плотной даже чем белый карлик, с плотностью сопоставимой с атомным 
ядром), в которой уже давление вырождения нейтронов поддерживает звезду 
против гравитационного коллапса. (Нейтронные звезды наблюдаются 
в пульсарах, а при массе около 1 М радиус нейтронной звезды должен быть 
порядка 10 км.) Картина Оппенгеймера-Шнайдера до недавнего времени не 
была глобально принята из-за своей большой странности. Именно эта 
картина стала называться картиной черной дыры, о которой и пойдет речь 
здесь.

%}}}

% No escape {{{

{\large Нет выхода}

Характерное свойство черной дыры в том, что она представляет область 
пространства, в которую тело может упасть, но из которой никакая форма 
материи, энергии или информации не может выбраться. (Черная дыра массой 
1 М имеет радиус около 3 км, совсем слегка меньше радиуса нейтронной 
звезды). Тело само по себе упадет глубоко внутрь черной дыры и возможно 
будет вытеснено из существования. Тем не менее, черная дыра  сама по 
себе все равно оказывает гравитационное притяжение на другие тела, так 
что она не совсем полностью не наблюдаема. Однако обнаружение черной 
дыры сложно. Лишь в последний год или около того стали всерьез 
появляться утверждения, что невидимые компаньоны некоторых звезд, 
принадлежащих к бинарным системам (или бОльшим системам, таким как 
глобулярные кластеры) могут в самом деле быть черными дырами. 
Обнаружение черной дыры можно произвести по ее гравитационному эффекту 
на материю вблизи одной или обеих ``звезд'' или путем возможного 
испускания рентгеновских лучей (или даже гравитационных волн), при том 
как близлежащая материя медленно, но верно \q{inexorably} направляется 
внутрь черной дыры. Альтернативные ("конвенциональные") объяснения, 
кажется, все еще существуют для всех этих кандидатов на черные дыры, 
которые выдвинули на данный момент, но эта ситуация, возможно, не 
продлится долго.

Стоит обратить внимание, однако, что свидетельства существования черных 
дыр, которые могут выглядеть наиболее убедительными \q{conclusive}, 
а именно, что в тех местах находятся полностью черные объекты массы 
слишком большой для белого карлика или нейтронной звезды, имеет довольно 
негативный характер. Теория предсказывает, что такие черные "объекты" 
(то есть черные дыры) должны существовать, но наблюдения такого вида 
едва ли можно считать серьезными свидетельствами в поддержку теории. 
Можно представить, что теория на самом деле совершенно неверна 
настолько, что она запрещает существование объектов природы белых 
карликов или нейтронных звезд, которые имеют слишком большую массу. 
Если, однако, активный поиск на протяжении нескольких лет не подтвердит 
существование черных "объектов" массы более 3 М, то это бы чрезвычайно 
ограничивало существующую теорию.

Я рассмотрю концепцию черной дыры и исследую теоретические причины 
верить, что черные дыры должны существовать. Хотя большая часть 
обсуждения черных дыр проводится на формализме общей теории 
относительности Эйнштейна, существования феномена такого типа не столь 
критично зависит на использовании конкретной теории гравитации. В самом 
деле, даже в формализме Ньютоновской теории было предсказано, уже в 1798 
году Лапласом, что (как цитирует Эддингтон) "наиболее большие светящиеся 
тела во вселенной могут, по такой причине, быть невидимыми для нас." 
Аргумент Лапласа заключался в том, что для достаточно массивных 
и плотных тел скорость вылета на поверхности превышала бы скорость 
света. Таким образом, любой испускаемый свет притягивался бы обратно, 
превращая объект в невидимый. В частности, Лаплас обращает внимание, что 
"светящаяся звезда, такой же плотности как Земля, чей диаметр должен 
быть в 250 раз больше солнечного, вследствие притяжения, не позволяла бы 
каким-либо своим лучам достигать нас. Стоит отметить, что формула, 
выражающая условие этого, 2mGc-2 > r, в теории Ньютона, где м и ар это 
масса и радиус тела, Г гравитационная константа, а ц скорость света. Это 
та же самая формула, которая дается теорией Эенштейна, выражающей 
условие, при которых тело должно быть окружено черной дырой.

Не все теории гравитации, однако, предсказывают существование такого 
явления. В частности, скалярная гравитационная теория Нордсторма не 
приводит к картине подобного вида. Это происходит потому что в теории 
Нордстрома свет не отклоняется гравитационным полем. В общем, явление 
черной дыры является следстием факта, что свет притягивается 
гравитирующими телами -- наблюдаемый эффект (который, по случайности, 
можно считать наблюдательным опровержением теории Нордстрома). Общий 
принцип при этом, что в достаточно мощном (сферически симметричном) 
гравитационном поле свет не может вырваться и в конечном итоге падает на 
центр притяжения, по большей части не зависит от деталей использованной 
теории гравитации, если свет достаточно сильно притягивается гравитацией 
подобным материи способом. Грубо говоря, имеется в виду, что как только 
вторая космическая скорость превышает скорость света, то (предполагая, 
что специальная теория относительности соблюдается локально, так что 
скорость света является ограничивающей) ничего не может выбраться. 
В самом деле, в теории гравитации, которая на данный момент может 
считаться наиболее серьезным соперником общей теории относительности, 
а именно, теория (Джордана--)Бранса--Дике, черные дыры (по сути 
идентичные черным дырам, предсказанным ОТО) развиваются.

На данный момент, я ограничу внимание к ОТО и опишу первую стандатную 
картину (Оппергеймера--Шнайдера) сферически симметричного 
гравитационного коллапса в черную дыру. Затем я рассмотрю эффект, что 
отклонения от сферической симметрии можно ожидать в свете некоторых 
теоретических результатов в физике черных дыр ОТО. В конце концов, 
я обсужу некоторые более предположительные \q{conjectural} аспекты, 
в частности, относительно выдающихся наблюдений Вебера, по-видимому, 
гравитационных волн, исходящих \q{emanating} из центра нашей галактики.

Для получения картины сферически симметричного гравитационного коллапса, 
представим себе \q{envisage} материю, падающую радиально по направлению 
некоторой центральной точки, и предположим, что давления недостаточно 
велики для противостояния гравитационному притяжению. (В самом деле, 
в ОТО давление само по себе является источником дополнительного 
гравитационного притяжения. При достаточной концентрации материи 
дополнительное гравитационное притяжение, обусловленное давлением, более 
чем превосходит прямые эффекты давления как такового. Большее давление 
затем увеличивает склонность к коллапсу.)

После того как материя упала ниже радиуса r = rm = 2mGc-2 (радиус 
Шварцшильда), происходят занятные вещи, поскольку испущенный 
с поверхности вещества свет падает внутрь. (При r = rm испущенный свет 
может отметить время, вечно летая на одном расстоянии от центра.)

Из рисунка 1 очевидно, что наблюдатель вне дыры не может наблюдать 
материю после того как та упала внутрь. В принципе, однако, он всегда 
все еще может наблюдать материю непосредственно до того как она упала 
сквозь дыру. Это происходит потому что при при приближении поверхности 
к радиусу r = rm испускаемый ей свет все дольше и дольше улетает, время 
улета теоретически бесконечно из радиуса r = rm. Но на практике 
наблюдатель бы по сути не получил света от материи по прошествии 
короткого времени. Это происходит потому что лишь ограниченное 
количество света испускается до того как тело достигает r = rm в то 
время как это ограниченное количество должно быть получено наблюдателем 
в течение бесконечного времени. Таким образом, наблюдатель видит свет 
все более и более смещенный в красную сторону, крайне быстро 
сокращающейся интенсивности, на самом деле, экспоненциально, с масштабом 
времени порядка миллисекунд для случая тела массы близкой к солнечной. 
Черная дыра затем остается всем, что видно наблюдателю.

Но что происходит с самой материей после того как она падает в дыру? 
Согласно сферически симметричной картине ОТО, материя неизбежно 
сжимается до все большей и большей плотности, без какого-либо предела. 
В рамках конечного времени (вновь миллисекунды для солнечной массы), 
плотность становится бесконечной, искривленность пространства-времени 
становится тоже бесконечной, так что результатом становится 
сингулярность в пространстве-времени в центре.

Каким образом подобная сингулярность пространства-времени проявляет себя 
в физике? Это можно ответить представляя наблюдателя, падающего по 
направлению внутрь черной дыры, следуя за материей к центру. Эффект 
кривизны пространства-времени на наблюдателя заключается в том, что он 
испытывает приливные силы. Это чувствует как растягивается в одном 
направлении (а именно, вдоль линии, соединяющей его и центральную точку) 
и сжимается в перпендикулярном направлении. По мере падения к центру 
этот приливной эффект усиливается и, в самом деле, для солнечной массы 
этот приливной эффект с легкостью бы убил человека даже до достижения 
черной дыры. Но для бОльших масс (например, в диапазоне от 10 в 6 до 10 
в 8 М, который по предположениям некоторых астрофизиков подходит для 
черной дыры, расположенной в центре галактики), эти приливные силы были 
бы достаточно малы до тех пор пока наблюдатель не внутри дыры. Но при 
приближении его к центру эффект усиливается неумолимо \q{mounts 
inexorably}. Наблюдатель не только вскоре разрушается возрастающей 
кривизной пространства-времени, но и сами атомы его тела, даже их 
составные элементарные частицы сами по себе будут в конечном итоге сжаты 
до небытия.

Но эта картина предполагает точную сферическую симметрию. Из-за того, 
что все падает внутрь к единой центральной точке, бесконечные плотности 
и кривизны, образующиеся в этой точке, можно поспорить, атипичны для 
общей ситуации коллапса. Любые асимметрии, присутствующие в начальном 
состоянии тела, как ожидается, будут крайне усилены при сжатии тела. 
Возможно разные участки коллапирующего тела пропустят друг друга, даже 
вновь расширятся наружу черной дыры. На какое-то время кажется, что 
надежды подобного типа рассматривались разными теоретиками. Более 
недавние работы показали, что в рамках ОТО (или даже теории 
Бранс--Дике), такие надежды невозможно реализовать. Строгие теоремы 
подразумевают, что как только материя падает в черную дыру, 
сингулярность пространства-времени какого-либо типа неизбежна, хотя ее 
точная природа может очень отличаться от природы, возникающей 
в сферически симметричном случае. (В физическом мире, 
в противоположность математическим моделям, мы можем считать, что фраза 
"пространственно-временная сингулярность" означает область, в которой 
пространство и время стали локально искажены настолько, что имеющиеся на 
данный момент законы физики более неприменимы.)

Может ли какая-либо материя предположительно когда-либо вновь появиться 
изнутри области черной дыры в определенной мере менее однозначно. Но она 
может такое осуществить только если другое, даже более странное явление, 
"голая сингулярность", также существует. Голая пространственно-временная 
сингулярность -- это такая, которая видима внешнему наблюдателю, 
в отличие от той, которая рассматривалась ранее (из которой никакой 
сигнал не может покинуть черную дыру). Существование голых 
сингулярностей представило бы гораздо более серьезные теоретические 
проблемы, чем сингулярности, скрытые черными дырами. Это обусловлено 
тем, что физика вблизи пр-вр сингулярности неизвестна. Необычайные 
локальные эффекты (например, рождение материи из гравитации) можно 
ожидать. С точки зрения наблюдений эти эффекты можно игнорировать, тогда 
(и только тогда), когда информация от сингулярностей не может выбраться.

Естественным оказывается вопрос, в свете настолько странным последствий, 
является ли разумным предполагать, что материя в принципе может быть 
достаточно концентрированной для образования черных дыр. Не может ли 
быть так, что по какой-то причине коллапсирующая звезда или суперзвезда 
всегда должна притормозить или, возможно, обратить собственный коллапс 
до того как достигает критического размера r = rm? Я думаю, довольно 
очевидно, что, хотя вполне возможно, что в некоторых случаях коллапс 
может быть приостановлен (если присутствует достаточное вращение, 
например), всегда найдется множество ситуаций, в которых точка 
невозврата пройдена. В общем, чем больше вовлеченный объект или набор 
объектов, тем меньше неоднозначности, что эта точка может достигаться. 
Например, для набора тел массы порядка галактической, это происходит при 
средней плотности меньше плотности воздуха. Более того, это условие, что 
точка невозврата пройдена (существование "захваченной поверхности"), 
является устойчивым при возмущениях начальных условий. Таким образом, 
при не слишком больших отклонениях от сферической симметрии 
и достаточном развитии коллапса, черная дыра (или голая сингулярность) 
с необходимостью образуется.

%}}}

% Rotating black holes {{{

Вращающиеся черные дыры

Если мы предполагаем, что голые сингулярности не существуют, то можно 
разумно вывести довольно точную картину конечного состояния черной дыры. 
Эта картина основана на решении Керра Эйнштейновских уравнений вакуума, 
описывающих вращающуюся черную дыру. Решение зависит от всего лишь двух 
параметров, м и а, где м это масса и маGc-2 -- угловой момент черной 
дыры. (Если электромагнитное поле тоже присутствует, возникает еще один 
параметр, е, описывающий заряд.) Возникающая картина имеет 
асимметричное, возможно вращающееся тело, коллапсирующее за точкой не 
возврата и производящее черную дыру, которая сперва обладает 
асимметриями, происходящими от асимметрий тела. Но в течение короткого 
времени эти асиметрии уходят с гравитационным излучением, и только 
параметры а и м остаются для описания природы дыры. Высшие моменты, 
такие как квадрупольные и октупольные, все принимают значения, явно 
зависящие от а и м. (Эта картина поддерживается анализом возмущений. 
Более убедительны \q{persuasive}, однако, результаты в рамках точной 
теории Эйнштейна, и они почти что устанавливают, что черная дыра, 
которая достигла стационарного состояния, одлжна быть решением Керра. 
Израэль показал это в случае без вращения (статичном), но потребовал 
некоторых дополнительных условий, которые сейчас эффективно исключены 
неопубликованной работой Мюллера цум Хагена, Боринсона и Шейферта. 
Картер эффективно получил результат для вращающейся аксиально 
симметричной черной дыры, хотя другое семейство решений, не связанное 
с решением Керра все еще не исключено. Хокинг показал, что любая 
стационарная черная дыра должна быть либо не вращающейся, либо аксиально 
симметричной, таким образом ограничивая общий случай до уже 
рассмотренных. Его оригинальный аргумент был основан на слишком сильном 
предположении относительно природы стационарной черной дыры, но 
улучшенная работа, по-видимому, решила эту сложность.)

Гравитационное излучение -- это гравитационный аналог электромагнитного 
излучения (света). Оно путешествует с той же скорость ц, что и свет, 
а его существование -- теоретическое следствие не только теории 
Эйнштейна, но и почти что любой другой релятивистской теории гравитации, 
с наибольшей разницей в том, что некоторые теории, например, теория 
Бранса-Дике, позволяют как скалярные волны, так и Эйнштейновские 
тензорные волны.

Тензорные гравитационные волны, по-видимому, были зарегистрированы 
Вебером, но скалярные волны -- нет. Это можно использовать как некоторое 
свидетельство против теории Бранса--Дикке, но все это зависит от 
механизма, которым природа производит волны Вебера. Хокинг обратил 
внимание, что в столкновении между двумя черными дырами в теории 
Бранса-Дикке испускаются лишь тензорные гравитационные волны, какие 
и испускаются в стандартной ОТО. Это происходит потому что скалярная 
часть гравитационного поля должна быть целиком испущена или поглощена 
при образовании черной дыры в теории Бранса-Дикке, результатом чего 
является черная дыра, идентичная черной дыре ОТО (то есть, по-видимому, 
решению Керра).

В свете астрономических наблюдений, что сильные испускания энергии 
происходят в разных частях вселенной (например, квазарах, взрывах 
галактических ядер, радиационных галактик, источнике гравитационных волн 
Вебера), разумно задаться вопросом, не может ли масса-энергия, 
содержащаяся в черной дыре, при каких-нибудь условиях быть испущена во 
внешний мир. Это не предполагает, что непосредственно материя, 
поглощенная черной дырой, должна каким-либо образом быть выпущена, 
а всего лишь что существенные порции энергетического содержимого ее 
массы возможно восстановить. На самом деле, существует несколько 
процессов, в разной степени достижимых, к которым можно обратиться для 
извлечения энергии. Наиболее простой среди них -- позволить черным дырам 
слиться \q{coalesce} и в процессе испускать энергию в виде 
гравитационных волн. Мизнер указал, как процесс застывания/затвердевания 
\q{congealing} большого числа черных дыр в единое облако может, 
в принципе, позволить (пропорционально) выпустить в виде волн почти 
полную изначальную энергию массы. Также существуют методы, в которых 
вращательную энергию черной дыры можно в принципе извлечь. Один из них 
-- позволить частице упасть с бесконечности в область, называемую 
эргосферой, которая лежит непосредственно вне поверхности вращающейся 
черной дыры. При попадании в эргосферу частица должна разделиться на две 
таким образом, что одна порция улетает на бесконечность, обладая 
энергией больше, чем у изначальной частицы. Другая порция поглощается 
черной дырой. При многократном повторении этот процесс привел бы 
к замедлению вращения черной дыры и существенному уменьшению ее массы. 
Энергию этой потери массы можно восстановить (очень эффективным 
способом, в принципе) в кинематическую энергию конечных частиц. (Я не 
предполагаю, что этот процесс с большой вероятностью важен 
в реалистичных астрофизических ситуациях, всего лишь что, если 
извлечение энергии возможно в принципе, то можно представить, что 
природа нашла какие-то средства, с помощью которых такие же конечные 
результаты могут в самом деле быть достигнуты.)

Один характерный аспект гравитационной физики -- это то, что в то время 
как обычно чрезвычайно малая величина гравитационной постоянной приводит 
к практически полной пренебрежимости гравитационных эффектов при 
сравнении с другими силами в природе, существуют достаточно 
экстремальные ситуации, в которых это верно с точностью до наоборот, 
и превращение массы в энергию достигается с помощью гравитации 
с эффективность близкой к 100\%. Таким образом, стоит изучать любой 
процесс, не важно насколько надуманный \q{contrived}, если он приводит 
к надежде на объяснение почти что волшебной видимой эффективности 
производства энергии в некоторых астрономических явлениях.

Полезным теоретическим принципом для обсуждения извлечения энергии из 
черных дыр является увеличение площади поверхности. Этот принцип 
использовался в различных ситуациях и утверждает, что полная площадь 
поверхности границы черной дыры не может сокращаться (и, по-видимому, 
может оставаться постоянной лишь в стационарном случае). Граница черной 
дыры (горизонт событий) определяется как такая поверхность, из-под 
которой сигнал не может проникнуть наружу, но может вылететь из точек, 
находящихся вне его. (В сферически симметричном случае это поверхность 
r = rm). Поскольку площадь поверхности черной дыры Керра может быть 
с легкостью вычислена (и равна 8 пи м (м + (м2 - а2) 0.5 ) G2 c-2), это 
быстро приводит к нескольким интересным неравенствам. Например, 
существует концепция неприводимой массы Христодулу, определяемая как 16 
pi m\_ir = 8pi m (m + 9m2 - a2) 1/2) для черной дыры Керра, являющаяся 
минимальной массой, к которой можно привести черную дуру с помощью 
процессов извлечения энергии. Хокинг тоже очень успешно использовал 
принцип площади при изучении столкновений черных дыр. Если, например, 
две не вращающиеся черные дыры с одинаковой массой м затвердевают 
и создают единую черную дыру массой М, то масса М больше м делить на 
корень из двух. Если они приближаются друг к другу, не имея 
предварительно значимой относительной скорости, то можно вычислить 
верхний предел на энергию испускаемого излучения, равный (2 - корень 2) 
мс2. Следует отметить, что черные дыры никогда не разделяются, они 
только сливаются.

Другой способ использования черных дыр для образования энергии -- 
извлечение энергии материальных тел, например, звезд. Возможность 
большой черной дыры (допустим от 10 4 до 10 6 М), располагающейся 
в центре галактики, часто выдвигалась. Можно предположить, что подобная 
черная дыра непрерывно поглощает звезды. С каждой поглощенной звездой 
определенная пропорция ее массы-энергии, можно ожидать, будет испущена 
в виде гравитационных волн. К тому же, приливные силы на звезде могут 
оказаться достаточно большими для образования существенных побочных 
эффектов (как электромагнитное излучение и расщепление звезды). Что 
произошло бы сильно зависит от особенностей ситуации. Разные авторы 
пытались использовать эти идеи для объяснения необычайных наблюдений, 
сделанных Вебером, которые, по-видимому, подразумевают огромный поток 
излучения гравитационных волн из центра нашей галактики. Этот поток 
настолько велик, на самом деле, что естественное предположение, что 
излучение испускается изотропно из галактического центра, кажется, 
приводит к абсурду, а именно, что галактика теряет массу со скоростью 
2000 М в год или даже 10 5 10 6 М в год (Кафка, личная беседа). (При 
скорости 2000 М в год, при ее постоянстве, целая галактика бы исчезла за 
менее чем 10 8 лет, что составляет лишь 1\% от ее возраста). Филд, Рис 
и Сиама указали, что наблюдение звездных орбит исключили возможность, 
что такая потеря массы нашей галактикой могла продолжаться хоть близко 
к 10 8 лет. Они утверждают, что потеря 70 М в год гораздо лучше сходится 
с наблюдениями. В свете этой разности (и опуская возможность, что 
галактический центр стал активен лишь недавно или что наблюдения Вебера 
не измеряют то, что кажется), мы приходим к рассмотрению возможности, 
что существуют гравитационные волны, идущие от галактического центра, 
существенно направленные в плоскости галактики. Можно показать, что 
Земля располагается очень близко к галактической плоскости, около 0.1\% 
ее расстояния до центра галактики, и таким образом, потеря массы, 
скажем, настолько малая как 10 М в год может оказаться в согласии 
с наблюдениями Вебера, и в таком же согласии также с вычислениями Филда, 
Риса и Сиама.

Но какой возможный механизм мог бы быть ответствен за такую 
направленность? Интересное предложение Миснера (личная беседа) включает 
экстремальную черную  дыру Керра, такую, для которой a = m. Бардин 
предложил, что такие экстремальные черные дыры могут образовываться 
через аккрецию материи в центрах галактик, при этом предполагается, что 
звезды вращаются по спирали постепенно внутрь по медленно ускоряющимся 
коллинеарным \q{co-planar} окружностям. Согласно механизму Мизнера, эти 
вращающиеся по спирали звезды испускают пучки гравитационного излучения 
вперед в узком угле, в определенном смысле аналогично электромагнитному 
синхротронному излучению, которое образуется релятивистскими заряженными 
частицами в электромагнитных полях. Утверждается, что это испускание 
пучков было бы достаточным не только лдя ограничения радиации эффективно 
до плоскости галактики, но и для образования высокочастотных (1660 Гц) 
коротких импульсов, которые были зарегистрированы Вебером по его 
утверждениям.

Существуют серьезные сложности, появляющиеся, на мой взгляд, при 
попытке заставить подобную модель быть работоспособной, но она 
действительно представляет достойную попытку объяснить наблюдение Вебера 
на основе принятой физической теории. Поэтому ее следствия следует 
исследовать во всех деталях для утверждения полной правдоподобности 
идеи. Возможно, другие предположения, включающие черные дыры, появятся 
в качестве альтернативных объяснений волн Вебера. С другой стороны, 
кажется весьма возможным, что все подобные объяснения в конечном итоге 
провалятся \q{founder}. Если это так, останется возможность 
какого-нибудь ощутимо другого типа гравитационной катастрофы, которая 
может быть ответственна за волны Вебера. Мне уже давно казалось, что 
предположение, что пространственно волновые сингулярности, которые 
рождаются в гравитационных коллапсах, должны неизбежно оказаться внутри 
черных дыр, является в большой степени результатом принятия желаемого за 
действительное \q{wishful thinking}. Возможность периодического 
возникновения голых сингулярностей необходимо всерьез рассмотреть. 
Существует множество решений уравнений Эйнштейна с голыми 
сингулярностями; примером является решение Керра при а больше м. (Это 
решение представляет не черную дыру, а существование за экстремальным 
случаем а равно м.) Проблема, однако, в том, являются ли такие голые 
сингулярности с высокой вероятностью самодостаточными или происходящими 
в коллапсах из совершенно разумных не сингулярных начальных состояний. 
Если, как можно ожидать, образование материи или гравитационных волн 
может происходить вблизи такой сингулярности, астрофизические следствия 
могли бы быть ощутимыми. В этой связи следует отметить, что большое 
решение Керра с а больше м, расположенное в центре галактики, имело бы 
свойство, что сигналы, исходящие вблизи его (решения) (голой) 
сингулярности необходимым образом направлены в одной плоскости, и это 
могла бы быть галактическая плоскость, если сингулярность каким-то 
образом была также ответственна за само существование галактической 
плоскости. Конечно, предположения такого рода было бы сложно оправдать 
в отсутствие проработанной теории сингулярностей. Но если все объяснения 
черных дыр провалятся, возможно, нам придется рассматривать возможности 
такого рода.


%}}}

\end{document}
