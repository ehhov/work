\documentclass[a4paper, 12pt]{article}

% Configuration {{{
\usepackage[utf8]{inputenc}
\usepackage[T2A]{fontenc} % T1 for English
\usepackage[english, russian]{babel}

\usepackage{enumitem}
\setlist{nolistsep}
\usepackage{mathtools}
\usepackage{xcolor}
\definecolor{dimblue}{HTML}{1010aa}
\usepackage[
  colorlinks=true,
  allcolors=dimblue
]{hyperref}
\usepackage[
  %showframe,
  vmargin=20mm,
  left=30mm,
  right=15mm
]{geometry}
\linespread{1.5}
\usepackage{indentfirst}
\usepackage{graphicx}
\usepackage{tikz}
\usepackage[multidot]{grffile}
\usepackage[labelsep=period]{caption}
\usepackage{multirow}

%\usepackage{times} % for English

\usepackage[russian]{csquotes}
\usepackage{longtable}
\usepackage{tipa}
% }}}

% Definitions {{{
\def\-{\nobreakdash-\hspace{0pt}}
\def\Lc{\Lambda_c^+}
\def\ScIz{\Sigma_c(2455)^0}
\def\ScIpp{\Sigma_c(2455)^{++}}
\def\ScI{\Sigma_c(2455)}
\def\ScII{\Sigma_c(2520)}
\def\LcII{\Lambda_c(2625)^+}
\def\LcIII{\Lambda_c(2880)^+}
\def\LcIV{\Lambda_c(2940)^+}
\def\Km{K^-}
\def\pip{\pi^+}
\def\pim{\pi^-}
\def\piz{\pi^0}
\def\rhop{\rho^+}
\def\rhom{\rho^-}
%}}}

\begin{document}

% Title page, Table of contents {{{

\def\thinnewline{\vskip -.3\baselineskip \relax}
%\def\signatureline{\raisebox{-1ex}{\rule{7cm}{.5pt}}\\[2ex]}
\def\signatureline{\vskip 1.5cm \relax}

\thispagestyle{empty}

\begin{center}
  \textsc{Московский государственный университет}
  \textsc{им.~М.\,В.~Ломоносова}
  \\[-.1\baselineskip] \textsc{Физический факультет}
  \\[-.1\baselineskip] \textsc{Кафедра английского языка}
\end{center}

\vspace{.2\textheight}

\begin{center}
  \large
  \textsc{Реферат} \thinnewline
  по дисциплине \thinnewline
  \enquote{Английский язык} \thinnewline
  на тему \thinnewline
  \enquote{Экспериментальная физика высоких энергий}
\end{center}

\vspace{.02\textheight}

\begin{flushright}
  Выполнил аспирант \thinnewline
  Гусейнов Абдул-Керим Демирович \\
  \signatureline

  Научный руководитель: \thinnewline
  д.ф.-м.н.~Бережной Александр Викторович \\
  \signatureline

  Преподаватель: \thinnewline
  к. филол. н. Плотникова Анна Вячеславовна \\
  \signatureline
\end{flushright}

\vfill

\begin{center}
  Москва --- 2023
\end{center}

\clearpage

\tableofcontents

\clearpage

%}}}

\clearpage
\phantomsection
\addcontentsline{toc}{section}{Введение}
\section*{Введение}
%{{{

Современное представление о вселенной основывается на Стандартной модели 
физики частиц. Она содержит три поколения фермионов, но из шести кварков 
и шести лептонов лишь два кварка и два лептона составляют подавляющее 
большинство материи во вселенной. Вклад остальных поколений, возможно, 
заключается в создании асимметрии между материей и антиматерией, без 
которой вещество превратилось бы обратно в излучение еще в ранней 
вселенной. Для ответа на вопрос о предназначении второго и третьего 
поколений фермионов нужны измерения $CP$\-симметрии, но не только они.

Квантовая хромодинамика входит в Стандартную модель и описывает 
взаимодействия кварков и глюонов. Теоретические вычисления в рамках КХД 
существенно осложнены ввиду большой величины константы сильного 
взаимодействия $\alpha_s$. Вычисления с помощью теории возмущений 
оказываются попросту невозможны в области малых квадратов переданных 
импульсов. Именно такой сценарий реализуется при объединении кварков 
в адроны. Сложности теории возмущений заставляют физиков искать другие 
подходы к теории, включая феноменологические модели, вычисления на 
решетке и прочее. Проверка таких теорий позволяет получать и улучшать 
знания о взаимодействии кварков, а следовательно, и о вселенной.

Адроны, содержащие хотя бы один тяжелый $c$ или $b$ кварк называются 
тяжелыми адронами. Они предоставляют уникальные возможности для изучения 
и проверки квантовой хромодинамики.
%
Благодаря большой константе связи $\alpha_s$ при малых передачах 
импульса, теория возмущений в КХД непригодна для теоретических 
исследований каких-либо адронов, включая тяжелые.
%
В то же время, присутствие тяжелого кварка дает хоть какую-то начальную 
точку и позволяет проводить нечеткие аналогии с хорошо известными 
и хорошо изученными электромагнитными системами. А аналогии можно затем 
модифицировать и усилить.
%
Именно поэтому множество различных подходов было разработано для 
описания тяжелых адронов, включая, например, эффективную теорию тяжелых 
кварков, релятивистские потенциальные модели, КХД на решетке и правила 
сумм КХД.

Современная физика высоких энергий проводится на ускорителях частиц, 
включая как электронные, так и адронные. Достигаемые на них энергии 
центра масс поднимаются вплоть до 14~ТэВ на Большом адронном коллайдере. 
Однако отдельный интерес представляют $B$-фабрики, где энергия подобрана 
специально для попадания в резонанс $\Upsilon(4S)$, распадающийся в пару 
$B\bar{B}$. Детекторы в современной физике высоких энергий крайне 
сложны, а для анализа данных применяются многоступенчатые методы, 
требующие тщательного и внимательного подхода на каждом шаге. Данный 
реферат посвящен некоторым актуальным анализам коллабораций CDF, CLEO, 
Belle и BaBar и проведенным ими измерениям в физике тяжелых адронов 
и физике высоких энергий в целом.

%}}}

\clearpage
\section{Измерения очарованных барионов на CDF}
%{{{

% Измерения характеристик барионов $\Lambda_c(2595)$, $\Lambda_c(2625)$, 
% $\Sigma_c(2455)$, $\Sigma_c(2520)$.

Базовую структуру тяжелого адрона можно, в определенном смысле, 
представить как атом водорода в КХД. Мезоны составлены из тяжелого 
кварка и легкого кварка и слегка больше похожи на атомное ядро 
с вращающимся вокруг него электроном по сравнению с барионами, которые 
состоят из тяжелого кварка и легкого дикварка. Однако это сходство 
настолько мало само по себе, что разница между мезонами и барионами 
в конечном итоге пренебрежима.
%
В результате было сделано множество теоретических предсказаний насчет 
многочисленных тяжелых барионов и их резонансных состояний. Изученный же 
анализ имеет дело с шестью резонансными состояниями очарованных барионов 
$\Lambda_c$ и $\Sigma_c$.

% === Можно прикрепить картинку со Стандартной моделью ===

На время проведения анализа~\cite{1-cdf} всемирные средние значения масс 
и естественных ширин изучаемых резонансов уже были измерены несколькими 
коллаборациями. Однако наборы экспериментальных данных еще не были 
настолько большими и точными как те, которые использовались для анализа. 
Это в особенности касается $\Lambda_c$ резонансов.

% === Можно прикрепить картинку с таблицей средних значений ===

Экспериментальные данные были собраны детектором CDF II на Тэватроне, 
который располагался в Фермилабе в США, в период с 2002 по 2009 год. 
Объем данных соответствует $5.2$ фб$^{-1}$ интегральной светимости.
%
Из ключевых элементов детектора CDF можно назвать его трековую систему, 
позволяющую измерять импульсы заряженных частиц, калориметры, измеряющие 
кинетические энергии, а также сенсор времени пролета. При обработке 
информации с детектора совместное использование этих систем позволяет 
идентифицировать частицы.

Отбор событий для анализа проводился в несколько этапов с использованием 
нейронных сетей на каждом из них. Сначала производится отбор кандидатов 
$\Lambda_c$ (основного состояния) в спектре $pK^-\pi^+$. Затем -- отбор 
кандидатов $\Sigma_c(2455)$ в спектре инвариантных масс 
$\Lambda_c^+\pi^\pm$. А затем -- отбор кандидатов $\Lambda_c(2625)^+$ 
в спектре $\Lambda_c^+\pi^+\pi^-$.

% === Можно прикрепить картинку CDF детектора или Фермилаба ===

Отбор кандидатов $\Lambda_c$ сам по себе производится в две стадии. 
Сначала применяются некоторые свободные критерии отбора предназначенные 
для подавления самых очевидных компонент фона. Затем, нейронная сеть 
обучается выявлять характеристики сигнальных событий и отличать их от 
комбинаторного фона. Вся тренировка проводится исключительно на 
экспериментальных данных. Поэтому, для предотвращения перетренировки 
и искажений, полный набор данных случайным образом разделяется на две 
половины, а две независимые нейронные сети обучаются на одной половине, 
а затем применяются к другой.
%
%Результат такого отбора представлен на рисунке красным.
% === Можно прикрепить картинку 001/after-Lc ===

Подобная двухстадийная процедура с двумя под-наборами данных затем 
повторяется для спектра $\Lambda_c\pi$ для улучшения качества сигнала. 
Только наиболее выдающийся пик (соответствующий $\Sigma_c(2455)$) 
используется для отбора, но результат все равно применяется для 
измерения обоих резонансов.

% === Можно прикрепить картинку 001/selection-Scstar ===

И аналогичная же процедура применяется к спектру $\Lambda_c\pi\pi$. Как 
и раньше, только наиболее выдающийся пик (в этом случае соответствующий 
$\Lambda_c(2625)$) используется для тренировки нейросети.

% === Можно прикрепить картинку 001/selection-Lcstar ===

Для аппроксимации спектров $\Lambda_c\pi$ используется модель с тремя 
фоновыми и двумя сигнальными компонентами. Фоны описывают полностью 
случайные комбинации треков, комбинации реальных $\Lambda_c$ со 
случайными пионами, а также распады резонанса $\Lambda_c(2625)$. Фоны 
моделируются полиномами, а сигналы -- функциями Брейта-Вигнера. 
Пограничная область исключена из фита для упрощения модели.

% === Можно прикрепить 001/fit-Scstar ===

Для спектра $\Lambda_c\pi\pi$ используется аналогичная модель, 
содержащая 3 фона и 2 сигнала. Третья фоновая компонента здесь 
обусловлена комбинациями реальных $\Sigma_c$ частиц со случайными 
пионами. Таким образом, присутствует перекрестный вклад между спектрами 
$\Lambda_c\pi\pi$ и $\Lambda_c\pi$. Большой особенностью является то, 
что резонанс $\Lambda_c(2595)$ невозможно описать простой симметричной 
функцией Брейта-Вигнера. Благодаря чрезвычайной близости его массы 
к порогу образования пары $\Sigma_c\pi$ необходима довольно сложная 
модель, в которой естественная ширина зависима от массы.

% === Можно прикрепить 001/fit-Lcstar, хуже fit-Lcstar-simple ===

Таким образом, параметры резонансов измеряются описанными тремя фитами. 
Следующим шагом было измерение систематических погрешностей. CDF 
рассматривает следующие их источники: модель разрешения, масштабирование 
момента, индивидуально определенное для детектора, а также модель 
аппроксимации. В конечном итоге CDF предоставляет следующие измерения 
масс и ширин изученных резонансов:
\[
  \begin{array}{rll}
    \Sigma_c(2455)^{++} : & \Delta M = 167.44 \pm 0.04 \pm 0.12 \text{ МэВ}, & \Gamma = 2.34 \pm 0.13 \pm 0.45 \text{ МэВ}, \\
    \Sigma_c(2455)^{0}  : & \Delta M = 167.28 \pm 0.03 \pm 0.12 \text{ МэВ}, & \Gamma = 1.65 \pm 0.11 \pm 0.49 \text{ МэВ}, \\
    \Sigma_c(2520)^{++} : & \Delta M = 230.73 \pm 0.56 \pm 0.16 \text{ МэВ}, & \Gamma = 15.0 \pm 2.1  \pm 1.4  \text{ МэВ}, \\
    \Sigma_c(2520)^{0}  : & \Delta M = 232.88 \pm 0.43 \pm 0.16 \text{ МэВ}, & \Gamma = 12.5 \pm 1.8  \pm 1.4  \text{ МэВ}, \\
    \Lambda_c(2595)^+   : & \Delta M = 305.79 \pm 0.14 \pm 0.20 \text{ МэВ}, & \Gamma = 2.59 \pm 0.30 \pm 0.47 \text{ МэВ}, \\
    \Lambda_c(2625)^+   : & \Delta M = 314.65 \pm 0.04 \pm 0.12. & \\
  \end{array}
\]

%}}}

\clearpage
\section{Адронная спектроскопия на $B$-фабриках}
%{{{

% Измерения $\mathcal{B}\left(\bar{B}^0\to\Lambda_c^+\bar{p}\right)$ 
% и $\mathcal{B}\left(B^-\to\Lambda_c^+\bar{p}\pi^-\right)$ и изучение 
% резонансов $\Lambda_c^+\pi^-$.

Изучение распадов очарованных и прелестных мезонов представляют большой 
интерес и большую ценность для физики элементарных частиц. Во-первых, 
они хорошо дополняют измерения тяжелых барионов, которые были затронуты 
в предыдущем разделе. Во-вторых, наличие антибарионов в конечном 
состоянии позволяет проверить гипотезу, что более вероятны распады, при 
которых барион и антибарион находятся ближе друг к другу в фазовом 
пространстве. В-третьих, тяжелые мезоны могут быть скалярными, что 
существенно упрощает измерения спинов конечных частиц.
%
В работе коллаборации BaBar~\cite{2-babar-sc} измеряются вероятности 
распадов $\mathcal{B}\left(\bar{B}^0\to\Lambda_c^+\bar{p}\right)$ 
и $\mathcal{B}\left(B^-\to\Lambda_c^+\bar{p}\pi^-\right)$, а также 
резонансы в спектре $\Lambda_c^+\pi^-$.

Экспериментальные данные, использованные для анализа, были собраны 
детектором BaBar, расположенном на $e^+e^-$ коллайдере с асимметричной 
энергией PEP-II в SLAC, США. В точке соударения электроны с энергией 
9~ГэВ сталкиваются с позитронами с энергией 3.1~ГэВ вблизи резонанса 
$\Upsilon(4S)$ при энергии центра масс 10.58~ГэВ. Собранные данные 
содержат $383\times10^6$ реакций $ee\to \Upsilon(4S) \to B\bar{B}$.
%
Среди основных частей детектора BaBar можно назвать вершинный детектор, 
дрифтовую камеру, а также детекторы Черенковского излучения.

Восстановление $\Lambda_c^+$ производится одновременно в 5 каналах:
$\Lambda_c^+ \to p K^- \pi^+$, $pK^0_S$, $pK^0_S\pi^+\pi^-$,
$\Lambda\pi^+$, $\Lambda\pi^+\pi^-\pi^+$ (последний используется только 
при изучении распада $B^-$). Фоновые события в анализе обусловлены как 
сторонними распадами прелестных мезонов, так и событиями континуума 
$ee\to qq$. Для внесения поправок, учитывающих способ подсчета событий, 
используется Монте-Карло моделирование. Эффективности регистрации 
распадов с разными каналами регистрации $\Lambda_c$ рассчитываются 
независимо, а для канала $B^-\to\Lambda_c^+\bar{p}\pi^-$ фазовое 
пространство конечных частиц делится на ячейки согласно кинематическим 
характеристикам. Максимальные вариации между ячейками достигают 50\%, но 
крайне редки.

% === Possibly insert the table from figures/002/table-efficiencies ===

Величины сигналов извлекаются при помощи одновременной аппроксимации 
спектров переменных
$m_m = \sqrt{\left(q_{e^+e^-} - q_{\Lambda_c^+\bar{p}(\pi^-)}\right)^2}$
и $m_r = \sqrt{\left(q_{\Lambda_c^+\bar{p}(\pi^-)}\right)^2} - m_B$,
где $q_X$ -- четырехимпульс комбинаций частиц. Переменная $m_m$ содержит 
разницу между энергиями-импульсами начальной системы (электрон-позитрон) 
и~одного из~продуктов аннигиляции (одного из {$B$-мезонов}) и~связана 
с~энергией отдачи другого $B$-мезона, рожденного в~электро-позитронном 
столкновении. Переменная $m_r$ выражает отклонение массы $B$-мезона, 
восстановленной в~событии, с~известным табличным значением его массы. 
Спектры $m_m$ и~$m_r$ событий с~разными каналами реконструкции бариона 
$\Lambda_c$ аппроксимируются одновременно. При этом распады $\bar{B}^0$ 
и~$B^-$ не связаны между собой, поэтому для каждого из них отдельно 
производится описанная одновременная аппроксимация.
%
Модели спектров строятся в виде суммы сигнала и фона. Фон моделируется 
пороговой функцией, умноженной на полином, а сигнал -- различными 
модификациями функции Гаусса.
%
Числа сигнальных событий в изучаемых спектрах, особенно для распада 
$\bar{B}^0$, малы, поэтому необходимо убедиться в их достоверности. 
Дополнительная проверка с использованием статистических экспериментов 
подтвердила отсутствие искажений.

Для распада $\bar{B}^0$ полученные числа событий в четырех спектрах 
затем корректируются для учета эффективности регистрации 
и восстановления каждого канала и используются для вычисления удельной 
вероятности распада.
%
Для $B^-$ корректировка усложняется за счет необходимости учитывать 
распределение по фазовому пространству. Результат вышеупомянутой 
аппроксимации используется для определения чисел событий в каждой ячейке 
фазового пространства, к которым затем корректировка применяется 
индивидуально.
%
Полученные величины вероятностей распадов, вычисленные с использованием 
разных каналов регистрации $\Lambda_c$, объединяются в конечный 
результат при помощи линейной комбинации, коэффициенты в которой 
подобраны так, чтобы погрешность результата была минимальна.
%
Полученные величины равны
$\mathcal{B} \big(\bar{B}{}^- \to \Lambda_c^+ \bar{p} \big)
= (1.89 \pm 0.21 \pm 0.06 \pm 0.49) \times 10^{-5} $,
$\mathcal{B} \big( {B}{}^- \to \Lambda_c^+ \bar{p} \pi^- \big)
= (3.38 \pm 0.12 \pm 0.12 \pm 0.88) \times 10^{-4} $,
где первая погрешность статистическая, вторая -- систематическая, 
а третья обусловлена погрешностью 
$\mathcal{B}(\Lambda_c^+\to{p}K^-\pi^+)$.
%
Отношение вероятностей распадов дает
$ \dfrac{\mathcal{B} \big( {B}{}^- \to \Lambda_c^+ \bar{p} \pi^- \big)}
{\mathcal{B} \big( \bar{B}{}^0 \to \Lambda_c^+ \bar{p} \big)} =
15.4 \pm 1.8 \pm 0.3 $.
То есть добавление лишь одного пиона в распад увеличивает его 
вероятность в 15 раз. Данный результат поддерживает гипотезу, что 
распады, в которых барион и антибарион находятся ближе друг к другу 
в фазовом пространстве, значительно более вероятны.

Приведенные выше систематические погрешности измеряются отдельно. Среди 
источников систематической погрешности можно выделить неточности при 
подсчете зарегистрированных $B\bar{B}$ событий, неточности 
в относительных вероятностях каналов распада $\Lambda_c^+$, ограниченный 
размер выборки моделирования, погрешности алгоритма вычисления треков 
частиц, вершин столкновения и идентификации частиц. Также присутствует 
вклад от систематики модели аппроксимации спектра инвариантных масс. 
Наибольшими оказываются вклады вероятностей каналов $\Lambda_c^+$, 
трековой системы и модели аппроксимации. Систематические погрешности 
считаются независимыми для разных каналов восстановления $\Lambda_c^+$ 
и складываются квадратично.

Для распада с большим числом событий, $B^-\to\Lambda_c^+\bar{p}\pi^-$, 
есть смысл исследовать резонансную структуру. Для этого с помощью 
методики sPlot строится спектр инвариантных масс $m(\Lambda_c^+\pi^-)$ 
частиц распада $B^-$. В спектре отчетливо видны пики, соответствующие 
резонансам $\Sigma_c(2455)^0$ и $\Sigma_c(2800)^0$, но не резонансу 
$\Sigma_c(2520)^0$. Их спектров удается извлечь величины естественных 
ширин и масс резонансов:
$\Gamma(\Sigma_c(2455)^0) = 2.6 \pm 0.5 \pm 0.3$~МэВ,
$m(\Sigma_c(2800)^0) = 2846 \pm 8 \pm 10$~МэВ,
$\Gamma(\Sigma_c(2455)^0) = 86 \pm^{+33}_{-22} \pm 12$~МэВ.
%
Очарованный изотриплет $\Sigma_c(2800)$ был впервые обнаружен 
коллаборацией Belle в анализе инклюзивных спектров 
$\Lambda_c^+\pi$~\cite{5-belle-sc}.
%
Поскольку пики сравнительно небольшие, требуется дополнительная 
проверка. Было проведено несколько проверок. Во-первых, наблюдаемые 
резонансы не обусловлены отражением $\Delta$ бариона, поскольку пики 
сохраняются при ограничении массы $m(p\pi^-)$. Во-вторых, пики 
действительно соответствуют реальной частице, распадающейся 
в $\Lambda_c^+\pi^-$, поскольку они наблюдаются во всех каналах 
регистрации $\Lambda_c^+$, то есть вне зависимости от конкретного 
распада $\Lambda_c^+$.
%
Наибольшие вклады в систематическую погрешность чисел событий 
и параметров очарованных барионов вносили способ разбиения спектра масс 
на ячейки, а также модели резонансной и нерезонансной компонент.

Найденные числа событий, соответствующие распадам с образованием 
возбужденных очарованных барионов, далее используются для вычисления 
относительных вероятностей протекания распада 
$B^-\to\Lambda_c^+\bar{p}\pi^-$ по данным каналам. Они оказались равны
\[\dfrac{\mathcal{B}\big(B^- \to \Sigma_c(2455)^0 \bar{p} \big)}
  {\mathcal{B}\big(B^- \to \Lambda_c^+ \bar{p} \pi^- \big)}
  = \left(12.3 \pm 1.2 \pm 0.8 \right) \times 10^{-2},\]
%
\[\dfrac{\mathcal{B}\big(B^- \to \Sigma_c(2800)^0 \bar{p} \big)}
  {\mathcal{B}\big(B^- \to \Lambda_c^+ \bar{p} \pi^- \big)}
  = \left(11.7 \pm 2.3 \pm 2.4 \right) \times 10^{-2}.\]
%
То есть четверть всех распадов происходит через резонансы. Для 
$\Sigma_c(2520)^0$, пик которого не наблюдается, оценена верхняя граница
%
\[\dfrac{\mathcal{B}\big(B^- \to \Sigma_c(2520)^0 \bar{p} \big)}
  {\mathcal{B}\big(B^- \to \Lambda_c^+ \bar{p} \pi^- \big)}
  < 0.9 \times 10^{-2} \text{ (90\% C.L.)}.\]

Пик $\Sigma_c(2455)^0$ в спектре достаточно велик, и есть смысл изучить 
угловое распределение конечных частиц, что позволит определить квантовые 
числа $J^P$ резонанса. Это измерение осложняется тем фактом, что 
эффективности регистрации и восстановления событий существенно зависят 
от геометрии распада, что искажает угловые распределения. Проверка 
гипотез с разными спинами осуществляется следующим образом. Идеальные 
теоретические распределения используются для генерации 500 выборок, 
эквивалентных экспериментальной по числу событий. Для каждой выборки 
вычисляется разность логарифмов функций правдоподобия, соответствующих 
двум конкурирующим гипотезам: $J=\frac{1}{2}$ и $J=\frac{3}{2}$. 
Распределение разностей логарифмов затем сравнивается с разностью, 
вычисленной для реальных экспериментальных данных. Положение 
экспериментальной точки согласуется с гипотезой $J=\frac{1}{2}^+$.

Таким образом, было проведено измерение относительных вероятностей 
распадов $\bar{B}^0\to\Lambda_c^+\bar{p}$ 
и $B^-\to\Lambda_c^+\bar{p}\pi^-$, а также измерено их отношение, 
которые, как оказалось, поддерживает теорию, что распады прелестных 
мезонов с барионом и антибарионом в конечном состоянии предпочтительнее, 
если барион и антибарион находятся ближе друг к другу в фазовом 
пространстве.
%
Также были измерены удельные вероятности распада $B^-$ через резонансы 
$\Sigma_c(2455)^0$ и $\Sigma_c(2800)^0$. Для самих резонансов были 
вычислены массы и ширины.
%
Кроме того, было обнаружено, что спин $\Sigma_c(2455)^0$ 
равен~$\frac{1}{2}^+$.

Помимо очарованных барионных резонансов $\Sigma_c$ интерес представляют 
барионы $\Lambda_c$, имеющие нулевой изоспин. BaBar проводила их 
измерения на основе инклюзивных спектров $D^0p$, в которых $D^0$ 
восстанавливается в каналах $D^0\to\Km\pip$ 
и $D^0\to\Km\pip\pim\pip$~\cite{4-babar-lc}.
%
Спектр $D^0p$, будучи инклюзивным, включает все события без разбора, 
а значит имеет очень большой фон. На каждом этапе анализа необходимо 
проводить проверки, что наблюдаемые пики соответствуют именно 
предполагаемым процессам. Для этого сигнальные события вблизи массы 
$D^0$ сравниваются с контрольными, расположенными дальше от нее 
в спектре $K\pi$ (или $K3\pi$). Кроме того, проверяются спектры 
с $\bar{D}^0$, то есть с неправильными знаками частиц.
%
В спектрах $D^0p$ действительно наблюдаются подобные уникальные пики, 
которые, как считается, соответствуют резонансам $\LcIII$ и $\LcIV$.
%
Используя модель, состоящую из пороговой функции, фазового объема 
и функций Брейта-Вигнера, из спектра извлекаются параметры этих 
резонансов: их массы и ширины. Они оказываются равны
\[
  \begin{array}{lll}
    \LcIII: &
    m = 2881.9 \pm 0.1 \pm 0.5 \text{ МэВ}, &
    \Gamma = 5.8 \pm 1.5 \pm 1.1 \text{ МэВ}, \\
    \LcIV: &
    m = 2939.8 \pm 1.3 \pm 1.0 \text{ МэВ}, &
    \Gamma = 17.5 \pm 5.2 \pm 5.9 \text{ МэВ}. \\
  \end{array}
\]
Систематические погрешности обусловлены как моделью спектра, так 
и ограниченным знанием массы $D^0$. Полученные величины для $\LcIII$ 
хорошо согласуются с измерениями CLEO~\cite{4-cleo-lc}.

К данным результатам добавляются измерения, проведенные коллаборацией 
Belle. Помимо массы и ширины этого бариона они измеряют его 
спин-четность~\cite{4-belle-lc}. Измерения спин-четности основаны на 
угловых распределениях и проверяют три альтернативные гипотезы: спин 
$\frac{1}{2}$, $\frac{3}{2}$ и $\frac{5}{2}$. Количество событий, 
извлекаемых из экспериментального спектра масс, хоть и мало, но 
позволяет отвергнуть первые две гипотезы со значимостью 5.5 и 4.5 
стандартных отклонений, соответственно. Предпочтительной остается 
гипотеза спина $\frac{5}{2}$. Приведенные коллаборацией Belle измерения 
относительных вероятностей распада этого бариона тоже подтверждают 
данный спин, но еще и позволяют сделать заключение о четности. 
Отрицательная четность дала бы отношение парциальных ширин распада 
$\LcIII$ через $\ScII\pi$ и $\ScI\pi$ в шесть раз больше, чем 
наблюдается в эксперименте. Положительная четность, в свою очередь, 
хорошо согласуется с результатом.

Среди более недавних работ Belle можно найти наиболее точное измерение 
резонанса $\LcII$~\cite{5-belle-lc}. Его естественная ширина настолько 
мала, что на ее величину существует лишь верхний предел. Даже 
в последней работе, использовавшей полную статистику, собранную 
детектором Belle, не удается измерить его ширину с достаточной 
точностью. Большим препятствием является, очевидно, разрешение 
детектора. Единственной возможностью его учесть служит моделирование 
эксперимента методом Монте-Карло, но оно не может быть идеальным. 
Отличия в ширине гауссова разрешения между моделированием 
и экспериментом служат наибольшим источником систематической 
погрешности. Именно она и не позволяет достичь желаемой точности.

Тем не менее, статистика Belle крайне высока и позволяет измерить другие 
характеристики $\LcII$. Коллаборация использует двухмерные распределения 
по квадратам масс конечных частиц -- диаграмму Далица. Аппроксимируя 
экспериментальные данные двухмерной функцией, состоящей из нескольких 
компонент, удается извлечь числа событий определенных каналов распада 
$\LcII$. Основным конечным результатом является полноценное 
подтверждение наличия резонансных распадов бариона $\LcII$, содержащих 
$\ScI$. Их вероятности были измерены впервые и оказались равны
\[
  \begin{aligned}
    \frac{\mathcal{B}\left(\LcII\to\ScIz\pip\right)}
    {\mathcal{B}\left(\LcII\to\Lc\pip\pim\right)}
    = (5.19 \pm 0.23 \pm 0.40)\%, \\
    \frac{\mathcal{B}\left(\LcII\to\ScIpp\pim\right)}
    {\mathcal{B}\left(\LcII\to\Lc\pip\pim\right)}
    = (5.13 \pm 0.26 \pm 0.32)\%. \\
  \end{aligned}
\]

%}}}

\clearpage
\section{Многочастичные процессы $e^+e^-$ аннигиляции}
%{{{

% Изучение процессов $e^+e^- \rightarrow \pi^+\pi^-\pi^0\pi^0\pi^0$ 
% и $\pi^+\pi^-\pi^0\pi^0\eta$ при энергиях центра масс от порога до 
% 4.35~ГэВ, используя излучение в начальном состоянии.

Электрон\-позитронные ускорители, работающие в области энергии центра 
масс вблизи 10.5~ГэВ, предоставляют возможность изучения крайне редких 
распадов благодаря высокой светимости и хорошей точности теоретических 
вычислений в рамках Стандартной модели. В то же время, свойства 
процессов электрон\-позитронной аннигиляции при меньших энергиях тоже 
представляют большой интерес, особенно для процессов с легкими мезонами. 
При этом жертвовать энергией столкновения или тратить финансы на новую 
конфигурацию детектора оказывается не обязательно. Одним из возможных 
решений являются процессы с излучением в начальном состоянии.
%
Излучение в начальном состоянии включает события типа 
$ee\to{}ee\gamma\to{}X\gamma$, где $\gamma$ несет очень большой импульс 
и был рожден уже в начальном $ee$ состоянии, предшествовавшем реакции 
$ee\to{}X$.
%
Излучение в начальном состоянии позволяет исследовать широкий диапазон 
эффективных энергий центра масс ниже полной энергии центра масс $ee$.
%
Это дает возможность изучать сечения реакций при низкой энергии, 
пользуясь высокой светимостью {$B$-фабрик}. Дополнительный интерес 
обусловлен исследованиями $g-2$ мюона, для теоретических вычислений 
которых в рамках Стандартной модели необходимы сечения при низких 
энергиях.
%
Кроме того, излучение в начальном состоянии способствует адронной 
резонансной спектроскопии.
%
В работе~\cite{3-babar-5pi} изучаются процессы
$e^+e^- \rightarrow \pi^+\pi^-\pi^0\pi^0\pi^0$ и
$e^+e^- \rightarrow \pi^+\pi^-\pi^0\pi^0\eta$
при энергиях центра масс от порога до 4.35~ГэВ, используя явление 
излучения в начальном состоянии.

Набор данных, используемый для анализа, были собраны детектором BaBar, 
расположенном на $e^+e^-$ коллайдере с асимметричной энергией на SLAC 
в США. Использовались эффективные энергии центра масс вплоть до 
4.35~ГэВ, выше которой появляется фон от $\Upsilon(4S)$ резонанса. Набор 
данных соответствует интегральной светимости $468.6$~фб$^{-1}$. Среди 
основных элементов детектора можно назвать вершинный детектор, дрифтовую 
камеру, а также детекторы Черенковского излучения.

При моделировании изучаемых распадов $e^+e^- \rightarrow 
\pi^+\pi^-\pi^0\pi^0\pi^0$ и $\pi^+\pi^-\pi^0\pi^0\eta$ рассматривались 
различные цепи, в том числе включающие промежуточные мезоны 
$\omega(782)$, $f_0(980)$, $\rho(770)$, а также распады, равномерно 
распределенные по фазовому объему. Помимо сигнальных событий, наборы 
данных моделирования включают фоновые процессы как с испусканием фотона 
в начальном состоянии, так и без.

Для изучаемых конечных состояний отбор событий существенно осложнен 
ввиду наличия нескольких нейтральных пионов. Базовым критерием отбора 
было наличие двух треков пионов и как минимум семи фотонов. Фотон 
с наибольшей энергией считался испущенным электрон-позитронной парой 
в начальном состоянии. Остальные 6 фотонов группируются в 2 пары 
с массой вблизи $m_\pi$ и два независимых фотона, которые могут 
соответствовать как $\pi^0$, так и $\eta$. Каждое событие 
аппроксимируется сигнальной гипотезой $ee\to\pip\pim3\piz\gamma$ 
и фоновой гипотезой $ee\to\pip\pim2\piz\gamma$, которая имеет гораздо 
большее сечение. $\chi^2$ данного фита используется дальше для вычета 
фона.
%
Двухмерное распределение событий по инвариантной массе третьей пары 
фотонов и упомянутому $\chi^2$ показывает, что в спектре масс 
$m_{\gamma\gamma}$ более явно видны пики $\pi^0$ и $\eta$ в области, где 
$\chi^2$ мало, то есть вероятность события быть фоновым меньше. В связи 
с этим, сигнальной считается область $\chi^2$ от 0 до 60, а фоновой -- 
от 60 до 120. Двухмерные распределения по инвариантным массам 
$m_{\gamma\gamma}$ и $m_{\pip\pim2\piz\gamma\gamma}$ поддерживают 
выбранные ограничения, поскольку в сигнальных событиях наблюдаются 
отчетливые пики $\piz$ и $\eta$ вне зависимости от 
$m_{\pip\pim2\piz\gamma\gamma}$, а среди фоновых событий мезонных пиков 
нет.
%
Модельные спектры этих же величин близки к экспериментальным для всех 
рассматриваемых каналов процесса $ee\to\pip\pim3\piz$.

Далее необходимо вычислить эффективности регистрации и восстановления 
событий в зависимости от энергии центра масс. Рассматриваются 
Монте-Карло спектры $m(\gamma\gamma)$ и $m(3\pi)$ для событий 
$ee\to\pip\pim\eta$ и $ee\to\omega\piz\piz$. Из событий сигнальной 
области $\chi^2<60$ вычитаются события контрольной (фоновой) области 
больших $\chi^2$ так же, как и для экспериментальных данных. Полученные 
распределения аппроксимируются для извлечения чисел событий в пиках 
резонансов. Данная процедура повторяется для множества ячеек в спектре 
$m(\pip\pim3\piz)$, а полученная зависимость соответствует эффективности 
восстановления событий в зависимости от этой массы.

Вычисление сечения реакции $ee\to\pip\pim3\piz$ в эксперименте абсолютно 
аналогично: для ячеек в спектре $m(\pip\pim3\piz)$ определяются числа 
событий под пиком, соответствующим $\piz$ в экспериментальном 
распределении $m(\gamma\gamma)$. То же самое делается для контрольных 
событий, $\chi^2$ которых велико. Таким образом находится спектр масс 
$5\pi$ изучаемого процесса, который затем еще пригодится для нахождения 
промежуточных состояний, через которые проходит система $ee$ до $5\pi$.
%
Однако перед этим следует изучить резонансную структуру в комбинациях 
трех пионов, где ожидаются мезоны $\eta$ и $\omega$. Инвариантная масса 
$3\piz$ имеет четкий пик вблизи массы $\eta$, который виден даже 
в двухмерных спектрах как против $m(\pip\pim)$, так и против $m(5\pi)$. 
Более внимательное изучение позволяет увидеть даже одновременный 
резонанс в спектрах $m(3\piz)$ и $m(\pip\pim)$, соответствующий процессу 
$ee\to\eta\rho$. Вычисление чисел событий в пиках мезона $\eta$ 
в ячейках по $m(5\pi)$ позволяет найти сечение данного процесса.
%
Аналогичным образом исследуется спектр масс $\pip\pim\piz$, содержащий 
явный пик $\omega$. Сечение этого процесса оказывается примерно вдвое 
меньше предыдущего, что соответствует базовым представлениям об 
изоспиновой симметрии.
%
В конце концов, можно рассмотреть спектр $\pi^\pm\piz$, в котором 
ожидаются пики $\rho^\pm$. Числа событий в индивидуальных пиках спектров 
$\pip\piz$ и $\pim\piz$ достаточно велики и позволяют извлечь сечение 
в зависимости от энергии центра масс. Однако отчетливо видны события, 
соответствующие коррелированному рождению двух $\rho$ мезонов, то есть 
процессу $ee\to\rhop\rhom\piz$. Статистика для этого двойного резонанса 
уже слишком мала.

Точно такие же действия затем производятся для изучения процесса 
$ee\to\pip\pim2\piz\eta$ с единственной разницей, что в спектре 
$m(\gamma\gamma)$ аппроксимируется не пик $\piz$, а пик $\eta$. Модели 
фита выглядят подобно уже описанным, а методы совершенно одинаковы. 
Существенно более низкая статистика, однако, не позволяет настолько же 
детально изучить резонансную структуру. Тем не менее, пики 
$\omega\to\pip\pim\piz$ и $\rho\to\pi\piz$ все же наблюдаются.

Измеренные сечения процессов $ee\to5\pi$ и $ee\to4\pi\eta$ в зависимости 
от энергии в системе центра масс представляют особый интерес, поскольку 
содержат информацию о $ee$ аннигиляции не только в континууме, но 
и в областях чармониев $J/\psi$ и $\psi(2S)$. Данные спектры позволяют 
впервые измерить вероятности распадов указанных чармониев на некоторые 
промежуточные состояния мезонов, включающие $\pi$, $\omega$, $\rho$.

%}}}

\clearpage
\phantomsection
\addcontentsline{toc}{section}{Заключение}
\section*{Заключение}
%{{{

Ускорители частиц и коллайдеры позволяют исследовать поведение 
элементарных частиц в экстремальных ситуациях, которые, возможно, не так 
часто реализуются в природе на Земле, но играют важнейшую роль 
в космических процессах. Само создание вселенной обусловлено тонкостями 
взаимодействия между элементарными частицами. А исследованием этих 
тонкостей занимается современная физика высоких энергий.

Десятилетия прогресса физики ускорителей и технологий детекторов 
предоставляют возможности детального изучения крайне редких процессов, 
включая различные резонансные каналы распадов прелестных адронов. 
Объемы данных, собираемые детекторами физики высоких энергий, крайне 
велики, но и методы анализа экспериментальных данных не стоят на месте.
%
Использованные в изученных работах подходы к построению спектров, 
извлечению чисел событий и измерению параметров резонансов отражают 
современные методики проведения физических анализов.

Несомненно, особую важность имеют результаты самих анализов. 
Рассмотренные работы представляют первые наблюдения нескольких 
резонансных и многочастичных процессов, а также первые измерения свойств 
адронных резонансов, включая массы, естественные ширины и спин\-четности.
%
Полученные величины находятся в хорошем согласии с теоретическими 
предсказаниями Стандартной модели и позволяют уточнить актуальные на 
данный момент феноменологические модели адронов.

В частности, нашла подтверждение гипотеза, что распады прелестных 
мезонов с образованием бариона и антибариона в конечном состоянии более 
вероятны, если барион и антибарион находятся ближе друг к другу 
в фазовом пространстве. Это проявляется, например, в отношении 
вероятностей распадов $\bar{B}^0$ и $B^-$~\cite{2-babar-sc}:
\[
  \frac{\mathcal{B} \big( {B}{}^- \to \Lambda_c^+ \bar{p} \pi^- \big)}
  {\mathcal{B} \big( \bar{B}{}^0 \to \Lambda_c^+ \bar{p} \big)} =
  15.4 \pm 1.8.
\]
Внесение лишь одного пиона в распад $B$\-мезона увеличивает его 
вероятность более чем на порядок. Для сравнения, распады прелестных 
барионов при добавлении легких мезонов типично, если возрастают, то на 
величину от~1.5 до~4 раз~\cite{pdg}.

%}}}

\clearpage
\phantomsection
\addcontentsline{toc}{section}{Глоссарий}
\section*{Глоссарий}
%{{{

\begingroup
\def\toowide#1{%
  \makebox{%
    \hspace*{-1ex}%
    \renewcommand{\arraystretch}{.5}%
    \begin{tabular}{l}#1\end{tabular}%
    \hspace*{-1ex}%
  }%
}
\def\ipa#1{\textipa{[#1]}}
\defineshorthand{""}{\textsecstress}
\defineshorthand{'r}{\textrhoticity}

\begin{longtable}{rcl}
  absorption & \ipa{@b"zOrpS@n} & поглощение \\
  approximation & \ipa{@""prAks@"meIS@n} & приближение \\
  baryon & \ipa{"beri""An} & барион \\ % ...
  bias & \ipa{"baI@s} & искажение \\
  bottom (hadron) & \ipa{"bAd@m} & прелестный \\
  center of mass & \ipa{"send@r @f m\ae s} & центр масс \\
  charge conjugate & \ipa{tSardZ "kAndZ@g@t} & зарядово сопряженный \\
  charmed (hadron) & \ipa{"tSArmd} & очарованный \\
  coherence & \ipa{koU"hIr@ns} & когерентность \\
  collaboration & \ipa{k@""l\ae b@"reIS@n} & коллаборация \\
  collision & \ipa{k@lIZ@n} & столкновение \\
  confidence level & \ipa{"kAnf@d@ns "lev@l} & уровень доверия \\
  consistent with & \ipa{k@n"sIst@nt wID} & согласуется с \\
  constraint & \ipa{k@n"streInt} & ограничение \\
  contribution & \ipa{k@ntrI"bjuS@n} & вклад \\
  coupling constant & \ipa{"k2pliN "kA:nst@nt} & константа взаимодействия \\
  cross-feed & \ipa{kras fi:d} & перекрестный вклад \\
  cross-section & \ipa{krAs sekS@n} & сечение \\
  Dalitz plot & \ipa{"dAlits plA:t} & диаграмма Далица \\
  decay & \ipa{dI"keI} & распад \\
  degree of freedom & \ipa{dI"gri @f "fri:d@m} & степень свободы \\
  denominator & \ipa{dI"nAm@neId@r} & знаменатель \\
  density & \ipa{"dens@di} & плотность \\
  efficiency & \ipa{I"fIS@nsi} & эффективность \\
  estimate & \ipa{"est@meIt} & оценивать \\
  event selection & \ipa{I"vent s@"lekS@n} & отбор событий \\
  excitation & \ipa{""eksaI"teiS@n} & возбуждение \\
  fluctuation & \ipa{""fl2ktSu"eIS@n} & колебания \\
  frame of reference & \ipa{freIm @f "ref@r@ns} & система отсчета \\
  framework (of a theory) & \ipa{"freImw3rk} & рамки \\
  ground state & \ipa{"graUnd steIt} & основное состояние \\
  hadron & \ipa{"h\ae drA:n} & адрон \\
  half life & \ipa{h\ae f laIf} & период полураспада \\
  heavy quark & \ipa{"hevi kwArk} & тяжелый кварк \\
  hypothesis & \ipa{haI"pAT@sis} & гипотеза \\
  intermediate & \ipa{"Int@rmidi@t} & промежуточный \\
  invariant mass & \ipa{In"veri@nt m\ae s} & инвариантная масса \\
  lattice QCD & \ipa{"l\ae d@s kju si di} & КХД на решетке \\
  leading order & \ipa{"li:diN "Ord@r} & в первом приближении \\
  likelihood & \ipa{"laIklihUd} & правдоподобие \\
  linear & \ipa{"lIni@r} & линейный \\
  longitudinal & \ipa{""lAndZ@"tudIn@l} & продольный \\
  luminosity & \ipa{""lum@"nAs@di} & светимость \\
  mainly & \ipa{meInli} & преимущественно \\
  mass spectrum & \ipa{m\ae s "spektr@m} & спектр масс \\
  mass splitting & \ipa{m\ae s splIdIN} & расщепление масс \\
  matter & \ipa{"m\ae d@r} & вещество \\
  measurement & \ipa{"meZ@rm@nt} & измерение \\
  meson & \ipa{"mizAn} & мезон \\ % ...
  momentum & \ipa{m@"ment@m} & импульс \\
  natural width & \ipa{n\ae tS@r@l wIdT} & естественная ширина \\
  negligible & \ipa{"negl@dZ@b@l} & пренебрежимый \\
  neural network & \ipa{"nUr@l "nEtw3rk} & нейронная сеть \\
  normalization & \ipa{""nOrm@l@"zeIS@n} & нормировка \\
  numerator & \ipa{"num@reId@r} & числитель \\
  observation & \ipa{""Abz@r"veIS@n} & наблюдение \\
  parity & \ipa{"per@di} & четность \\
  partial width & \ipa{"pArS@l "wIdT} & парциальная ширина \\
  particle & \ipa{"pArd@k@l} & частица \\
  perturbative QCD & \ipa{p@r"t@rb@div kju si di} & теория возмущений КХД \\
  phase space & \ipa{feIz speIs} & фазовое пространство \\
  physics & \ipa{"fIzIks} & физика \\
  plane & \ipa{pleIn} & плоскость \\
  pole models & \ipa{poUl "mAd@ls} & полюсные модели \\
  polynomial & \ipa{""pAli"noUmi@l} & полином \\
  probability & \ipa{""prAb@"bIl@di} & вероятность \\
  production & \ipa{pr@"d2kS@n} & рождение \\
  projection & \ipa{prA"dZekS@n} & проекция \\
  propagation & \ipa{""prAp@"geIS@n} & распространение \\
  proper decay time & \ipa{"prAp@r dI"keI taIm} & собственное время распада \\
  proximity & \ipa{prAk"sIm@di} & близость \\
  quadratic & \ipa{kwad"r\ae dIk} & квадратичный \\
  quantum chromodynamics & \ipa{\toowide{"kwant@m \\ "kroUm@daI""n\ae mIks}} & квантовая хромодинамика \\
  quark mixing & \ipa{kwArk "mIksIN} & смешивание кварков \\
  radiation & \ipa{""reIdi"eIS@n} & излучение \\
  random variable & \ipa{"r\ae nd@m "vEri@b@l} & случайная переменная \\
  range & \ipa{reIndZ} & диапазон \\
  recoil energy & \ipa{rI"kOIl "en@rdZi} & энергия отдачи \\
  reconstruction (of a track) & \ipa{""rik@n"str2kS@n} & восстановление \\
  resolution & \ipa{rez@"luS@n} & разрешение \\
  resonance & \ipa{"res@n@ns} & резонанс \\
  rest frame & \ipa{rest freIm} & система покоя \\
  rough estimate & \ipa{r2f "est@m@t} & грубая оценка \\
  saturation & \ipa{""s\ae tS@"reIS@n} & насыщение \\
  scaling factor & \ipa{"skeIliN "f\ae kt@r} & параметр масштабирования \\
  scattering & \ipa{"sk\ae d@rIN} & рассеяние \\
  simulation & \ipa{""sImj@"leIS@n} & моделирование \\
  special relativity & \ipa{"speS@l ""rel@"tiv@di} & \toowide{специальная теория \\ относительности} \\
  threshold & \ipa{"TreShoUld} & порог \\
  transpose & \ipa{tr\ae n"spoUz} & транспонировать \\
  transverse & \ipa{tr\ae ns"v3rs} & перпендикулярный \\
  uncertainty & \ipa{2n"s3rtnti} & погрешность \\
  uniform (distribution) & \ipa{"jun@fOrm} & равномерный \\
  universe & \ipa{"jun@v3rs} & вселенная \\
  vertex & \ipa{"v3rdeks} & вершина \\
  violation & \ipa{""vaI@"leIS@n} & нарушение \\
  yield & \ipa{ji:ld} & число событий \\
\end{longtable}

\endgroup

%}}}

\clearpage
\phantomsection
\addcontentsline{toc}{section}{\refname}
\begin{thebibliography}{99}%{{{

  \bibitem{1-cdf}
    T.~Aaltonen \textit{et al.} (CDF Collaboration),
    ``Measurements of the properties of $\Lambda_c(2595)$, 
    $\Lambda_c(2625)$, $\Sigma_c(2455)$, and $\Sigma_c(2520)$ baryons,''
    Phys. Rev. D \textbf{84}, 012003 (2011).

  \bibitem{2-babar-sc}
    B.~Aubert \textit{et al.} (BaBar Collaboration),
    ``Measurements of $\mathcal{B}(\bar{B}^0 \to \Lambda_c^+ \bar{p})$ 
    and $\mathcal{B}(B^- \to \Lambda_c^+ \bar{p} \pi^-)$ and Studies of 
    $\Lambda_c^+ \pi^-$ Resonances,''
    Phys. Rev. D \textbf{78}, 112003 (2008).

  \bibitem{5-belle-sc}
    R.~Mizuk \textit{et al.} (Belle Collaboration),
    ``Observation of an isotriplet of excited charmed baryons decaying 
    to $\Lambda_c^+ \pi$,''
    Phys. Rev. Lett. \textbf{94}, 122002 (2005).

  \bibitem{4-babar-lc}
    B.~Aubert \textit{et al.} (BaBar Collaboration),
    ``Observation of a charmed baryon decaying to $D^0p$ at a mass near 
    2.94-GeV$/c^2$,''
    Phys. Rev. Lett. \textbf{98}, 012001 (2007).

  \bibitem{4-cleo-lc}
    M.~Artuso \textit{et al.} (CLEO Collaboration),
    ``Observation of new states decaying into $\Lambda_c^+\pi^-\pi^+$,''
    Phys. Rev. Lett. \textbf{86}, 4479-4482 (2001).

  \bibitem{4-belle-lc}
    K.~Abe \textit{et al.} (Belle Collaboration),
    ``Experimental constraints on the possible J**P quantum numbers of 
    the $\Lambda_c(2880)^+$,''
    Phys. Rev. Lett. \textbf{98}, 262001 (2007).

  \bibitem{5-belle-lc}
    D.~Wang \textit{et al.} (Belle Collaboration),
    ``Measurement of the mass and width of the $\Lambda_c(2625)^+$ 
    charmed baryon and the branching ratios of $\Lambda_c(2625)^+ \to 
    \Sigma_c^0\pi^+$ and $\Lambda_c(2625)^+ \to \Sigma_c^{++}\pi^-$,''
    Phys. Rev. D \textbf{107}, 032008 (2023).

  \bibitem{3-babar-5pi}
    J.~P.~Lees \textit{et al.} (BaBar Collaboration),
    ``Study of the reactions $e^+e^-\to\pi^+\pi^-\pi^0\pi^0\pi^0$ and 
    $\pi^+\pi^-\pi^0\pi^0\eta$ at center-of-mass energies from threshold 
    to 4.35 GeV using initial-state radiation,''
    Phys. Rev. D \textbf{98}, 112015 (2018).

  \bibitem{pdg}
    P.A. Zyla \textit{et al.} (Particle Data Group),
    Prog. Theor. Exp. Phys. 2020, 083C01 (2020).

\end{thebibliography}%}}}

\end{document}
