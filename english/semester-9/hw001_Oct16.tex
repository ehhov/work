\documentclass[a4paper, 12pt]{article}

% Configuration {{{
\usepackage[utf8]{inputenc}
\usepackage[T2A]{fontenc}
\usepackage[russian, english]{babel}

\usepackage[
	vmargin=1in,
	hmargin=1in
]{geometry}
\linespread{1.3}
\usepackage{xcolor}
\definecolor{allrefs}{HTML}{1010aa}
\usepackage[
	colorlinks=true,
	allcolors=allrefs
]{hyperref}
\usepackage{indentfirst}
\usepackage{graphicx}
\usepackage[multidot]{grffile}
\usepackage[labelsep=period]{caption}
\usepackage{enumitem}
\setlist{nolistsep}
\usepackage{mathtools}

%\usepackage{times}
\usepackage{multicol}
\usepackage{lipsum}
\setlength{\columnsep}{.25in}

\def\task#1{\begin{center}\it #1\end{center}}
\def\ans#1{\textit{#1}}

\newif\ifcols
\colsfalse
%}}}

\begin{document}

\noindent
Kerim Guseynov, 113M group
\hfill
Oct 16

\task{Unit 4, text, questions and main ideas}
\ifcols\begin{multicols}{2}\fi% {{{

\textbf{0. Introduction}
\\1. What are the smallest pieces of matter? \ans{Quarks.}
\\2. Who made it possible to complete the Standard Model? \ans{David Gross, David Politzer, and Frank Wilczek.}
\\3. What does the Standard Model describe? \ans{It describes the smallest objects in Nature and how they interact.}
\\ \textit{Main idea:} David Gross, David Politzer and Frank Wilczek contributed to the formulation of the Standard Model of Particle Physics, and it can help us provide a unified description of objects at all scales.

\textbf{1. The strong force explained}
\\1. What is the other name of the strong interaction? \ans{The color interaction.}
\\2. What are the main building blocks of all the matter? \ans{Protons, neutrons, and electrons.}
\\3. What did David Gross, David Politzer and Frank Wilczek discover? \ans{A property of the strong interaction which explains why quarks may behave almost as free particles only at high energies.}
\\ Though the progress in particle physics may not seem relevant, it helps us understand the fundamental forces between basic building blocks of matter, and the discovery for which David Gross, David Politzer, and Frank Wilczek are being awarded the Nobel Prize this year leads to the modern theory of the strong interaction.

\textbf{2. The Standard Model and the four forces of Nature}
\\1. What is the first force humanity found? \ans{Gravity.}
\\2. Is gravity actually strong compared to other forces? \ans{No, it's the weakest of the four.}
\\3. What exists within the Standard Model? \ans{The Standard Model describes quarks, leptons and force-carrying particles.}
\\ The four forces in nature are gravity, the electromagnetic force, the weak force, and the strong force, but gravity is too weak at particles' scales and is not taken into account within the Standard Model, which is the theory of smallest pieces of matter that form nuclei and atoms, and hence the whole Universe.

\textbf{3. The electromagnetic interaction provides light and cohesion}
\\1. What are the most common phenomena the electromagnetic interaction is responsible for? \ans{It is responsible for friction, magnetism and the fact that neither we nor objects we lay aside fall through the floor.}
\\2. What are the similarities between the electromagnetic force and gravity? \ans{The interaction strength decreases with the square of the distance and has a long range.}
\\3. Why does Quantum Electrodynamics agrees with experiments so well? \ans{The small coupling constant equals 1/137 and makes it possible to perform calculations as a series expansion.}
\\ The electromagnetic force is responsible for a lot of effects in nature, and its carriers, photons, coming from the Sun, are necessary for life on Earth; it is also described by the most successful theory, and the electromagnetic coupling constant also varies with energy.

\textbf{4. The weak interaction --- radioactive decay}
\\1. What particles carry the weak force? \ans{$W^\pm$ and $Z$ bosons.}
\\2. Why does this type of interaction has a short range? \ans{Because its carriers have large masses.}
\\3. When was the electroweak theory formulated? \ans{In the 1970s.}
\\ The weak force is carried by massive bosons, has a short range, and is responsible for some radioactive decays; it was unified with the electromagnetic force in the 1970s.

\textbf{5. The strong interaction --- charge and color}
\\1. Can we observe free quarks in experiments? \ans{No, because quarks are confined and only exist in pairs or triples.}
\\2. What charges do quarks have? \ans{They have electric charges of -1/3 or +2/3 and color charges.}
\\3. What makes the strong force so complex? \ans{Its carriers, gluons, have a color charge and hence interact with each other.}
\\ The proton and the neutron consist of quarks which by themselves cannot exist freely, have an electric charge, and a color charge; quarks and antiquarks must form color-neutral states; gluons, carrying the strong force, also have color charges, what leads to the complexity of the force.

\textbf{6. A weaker coupling sets the particles free}
\\1. Why the Feynman's calculation method cannot be directly applied to strong interactions? \ans{Because the strong coupling constant is larger than 1 and the series doesn't converge.}
\\2. Who discovered the theory with a negative beta function? \ans{Gross and Wilczek and Politzer made it independently.}
\\3. What leads to the negative beta function? \ans{The fact that gluons interact not only with quarks, but also with each other.}
\\ The interaction between nucleons can be described by pion exchange, but quarks require a different approach, which was found only in 1973 by assuming that gluons interact with each other and hence the beta function is negative, and quarks experience asymptotic freedom.


\textbf{7. The showers of particles reveal the truth}
\\1. How can other particles be produced in an electron-positron collision? \ans{Einstein's equation $E = mc^2$ allows the transformation of energy into new particles.}
\\2. When and where were three-shower electron-positron collisions observed for the first time? \ans{In the late 1970s at DESY, Hamburg.}
\\3. What carries the proton's momentum apart from quarks and what is the share? \ans{Gluons carry about a half of the proton's momentum at high energies.}
\\ A proof of the QCD theory is existence of two- and three-shower events in electron-positron collisions, since they can be calculated thanks to the asymptotic freedom of quarks; proton electron scattering also showed that half of the proton's momentum is carried by gluons.

\textbf{8. Can the forces of nature be unified?}
\\1. Does the Standard Model allow a unification of the three forces? \ans{No, it needs to be modified if this is what we expect.}
\\2. What theory unifies gravity with the other forces? \ans{The group of theories called string theories describes this unification.}
\\3. Is there any other reason to modify the Standard Model? \ans{Yes, it needs to be modified to incorporate the masses of neutrinos.}
\\Asymptotic freedom gives us hope for unification of forces, but the Standard Model has to be modified for this to be realized; one way to modify it is to introduce supersymmetric particles, which then lead to string theories; the Standard Model also fails to describe the masses of neutrinos.
\ifcols\end{multicols}\fi% }}}

\task{Ex. 1, p. 50}
\ifcols\begin{multicols}{2}\fi% {{{
  \textbf{QCD asymptotic freedom} -- асимптотическая свобода кварков -- unlike other forces, the strong coupling decreases with energy increase and hence quarks move almost freely.
\\\textbf{Fine structure constant (coupling constant)} -- постоянная тонкой структуры (константа связи) -- the constant describing the intensity of electromagnetic force.
\\\textbf{Perturbation calculation} -- вычисление по теории возмущений -- calculation of effects using a series expansion.
\\\textbf{Force carrier} -- переносчик силы -- the boson that carries the interaction between particles.
\\\textbf{Quark color charge} -- цветовой заряд кварков -- the charge of quarks affecting the strong interaction between them.
\\\textbf{Quark-antiquark particle} -- кварк-антикварковая частица -- a particle consisting of a quark and an antiquark (a meson).
\\\textbf{Supersymmetric particles} -- суперсимметричные частицы -- particles required by the supersymmetric theories.
\\\textbf{String theories} -- теории струн -- theories describing particles as different vibrations of the same entity, a string.
\ifcols\end{multicols}\fi% }}}

\task{Ex. 2, p. 50}
\ifcols\begin{multicols}{2}\fi% {{{
Sinitiro Tomonaga, Julian Schwinger, and Richard Feynman were awarded the Nobel Prize for the formulation of QED, the most successful particle physics theory.
\\Gerardus’t Hooft and Martinus Veltman received the 1999 Nobel Prize for the final formulation of the electroweak theory, unifying the electromagnetic and weak forces.
\\Hideki Yukawa was awarded the Nobel Prize for the description of nucleon-nucleon interaction via pi-mesons.
\\Kurt Sumanzik was the first to realize that the only way to achieve a reasonable theory of strong interactions is to construct a negative beta-function.
\\Gross, H. David Politzer, and Frank A. Wilczek discovered a theory having a negative beta-function.
\ifcols\end{multicols}\fi% }}}

\task{Ex. 3, p. 51}
\ifcols\begin{multicols}{2}\fi% {{{
\textbf{a)} Hadrons are a class of elementary particles subjected to the strong interactions. Hadrons are divided into baryons (in particular, protons and neutrons are baryons) consisting of three quarks and mesons (pions, kaons, and so on) consisting of a quark and an antiquark.
\\\textbf{b)} As was found in the 1960s, elementary particles consist of particles nowadays considered to be truly elementary and called quarks. The quark is a fundamental particle of the Standard Model. It has an electric charge multiple of e/3 and cannot be observed in a free state. Hypotheses are suggested according to which quarks also consist of more simple particles (prions), however they are yet to be confirmed.
\\\textbf{c)} Bosons and fermions are particles with integer (for bosons) and half-integer (for fermions) spin values. Some of fermions are quarks, electrons, muons, neutrinos, and other particles.
\\\textbf{d)} George Zweig suggested a hypothesis of quarks back in 1964. He called them ``aces'' since he supposed that only four quarks exist. Murray Gell-Mann postulated the quark model of elementary particles in 1964. The name for them Gell-Mann found in James Joyce's book ``Finnegans Wake'', where one episode has the phrase ``Three quarks for mister Mark!''.
\\\textbf{e)} Burton Richter noted that the discovery of J/$\psi$ meson was a matter of luck. By the mid 1970s was built the general theory of subatomic particles. It had only the u, d, and s quarks. It had no room for particles having at least one extra quark.
\\\textbf{f)} The discovery of the new meson was acknowledged an achievement of revolutionary scale. The new particle was so massive that it could not be made up from the ``old'' quarks. It became evident that the Standard Model needs to be extended.
\\\textbf{g)} It is now known that J/$\psi$ is a meson, a bound state of a c-quark and its antiquark. Since charm of the c-quark is 1, and of the c-antiquark is $-1$, the charm of J/$\psi$ meson is zero. Its mass estimation was repeatedly refined in experiments, and according to the latest data, it's equal to 3.0969 GeV (the mass of a c-quark is 1.29 GeV).
\\\textbf{h)} An important proof of QED was presented by collisions of electrons and their antiparticles, positrons, which led to their mutual annihilation.
\ifcols\end{multicols}\fi% }}}


\end{document}
