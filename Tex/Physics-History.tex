\documentclass[a4paper, 12pt]{article}

% Configuration {{{
\usepackage[utf8]{inputenc}
\usepackage[T2A]{fontenc} % T1 for English
\usepackage[english, russian]{babel}

\usepackage{enumitem}
\setlist{nolistsep}
\usepackage{mathtools}
\usepackage{xcolor}
\definecolor{dimblue}{HTML}{1010aa}
\usepackage[
	colorlinks=true, 
	allcolors=dimblue
]{hyperref}
\usepackage[
	vmargin=1in,
	hmargin=1in
]{geometry}
\linespread{1.3}
\usepackage{indentfirst}
\usepackage{graphicx}
\usepackage[multidot]{grffile}
\usepackage[labelsep=period]{caption}

\usepackage{titlesec}
\titleformat{\section}[hang]{\bf\centering}{\thesection.}{.5em}{}[]

%\usepackage{times} % for English
% }}}

\begin{document}

% Title Page & Table of Contents {{{
\null
\vfill

\begin{center}
	\begin{Large}
		\textbf{История и методология физики}
	\end{Large}

	\vspace{\baselineskip}

	Трубачев Олег Олегович

	\href{mailto:olegtrub@gmail.com}{olegtrub@gmail.com}
\end{center}

\vfill

Литература
\begin{enumerate}
	\item Спасский Б.И., История физики, Ч. 1 и 2 М. Высшая школа 1977
		\\ не так много конкретики
	\item Кудрявцев П.С., Курс истории физики, М. 1982
		\\ тонкий для лентяев
	\item Кудрявцев П.С., История физики, ТТ 1--3 М. Просвещение, 1956--1971
		\\ три толстых тома
	\item Кун Т., Структура Научных революций, М. АСТ 2009
		\\ всемирно уважаемая
	\item Ф. Розенбергер, История физики, ТТ 1--3, М--Л. 1934--1935
		\\ хорошо описаны древние времена, но недавние хуже
	\item М. Джеммер, Эволюция понятий квантовой механики, М. Наука, 1985
		\\ подробно и доступно, неплохо для людей, учивших уже кванты
\end{enumerate}

\clearpage

\tableofcontents

\clearpage
% }}}

\section{Предмет истории физики и зачем она нужна как учебный курс}% {{{

\hfill\textbf{Feb 10}

Излагается набор разделов, устоявшийся характер, всех разделов, 
отражающих современную физическую картину мира. Внутренне 
непротиворечивы, содержат основные аксиомы, понятия, дальше 
развивается, развитие требует непротиворечивости. Это нормально для 
раздела естествознания как физика. Кун это замечает, говорит, что 
истории не уделяется особого внимания. Отсылки к истории там только 
в плане ``вот это придумал вот тот, вот это вот тот''. Кажется, 
будто есть современное состояние науки и оно постепенно по кирпичам 
собиралось теми самыми людьми, хоть это и совершенно неправильно. 
Кумулятивный подход. Развитие происходило не так. Знать как 
развивалось на самом деле нужно для продолжения развития 
самостоятельно. Книга Куна дает критику бывшего тогда индуктивного 
подхода. Физика возникает в конце 16 века. Индуктивный 
подход -- эксперименты появляются, нужно собирать факты и их 
обобщать. Обобщать в чистом виде и аккуратно, не вовлекая ничего не 
бывшего в эксперименте. Не бывшее в эксперименте называют 
метафизикой. Декарт уникален в истории физики, он открыл 
преломление и вообще был хорошим философом. Имел метафизические 
принципы, что не характерно для эпохи, и был отличным математиком. 
Индуктивный подход был подвергнут критике. Структура физической 
теории. Компактный набор первых принципов, ключевые понятия, 
и дальше имеет место логическое построение с помощью математики. 
Системы мира Птолемея и Коперника.

% }}}

\section{Закономерности развития физики}% {{{

Нормальная наука --- нормальное естествознание --- наличие 
определенных парадигм. (парадигма -- шаблон) Есть теория и она 
общепризнана. Ее основные положения не нужно доказывать вечно. 
Далее методы постановки эксперимента. Нормальная наука возникает 
вместе с парадигмой. В 17 веке уже появляется экспериментальная 
сторона физики, но парадигма еще не устоялась. Считается 
устоявшейся в 1689 год с Ньютоном и его математической основы 
физики. Так возникает первая парадигма, хоть книга была сложная, ее 
постигали. В 18 веке уже началось такое вот использование, и теория 
точно стала общепризнанной. Кун говорит, что пока парадигмы не 
было, ученые были вынуждены писать так сказать монографии, очень 
длинно всё это разрабатывающие и описывающие с нуля с основ. 
С парадигмой уже не нужно доказывать основы каждый раз, можно 
писать привычные нам статьи. Экспериментальная наука требует 
парадигмы и сопутствующих моментов. Далее Кун заметил неравномерное 
развитие физики. До Куна было мало обобщений как физика должна была 
развиваться, без морали так сказать. Кун придумал нормальную науку, 
парадигму итд. Неравномерный характер -- возникновение разных 
парадигм, которые друг друга сменяют. И это нормально. До Куна 
такие революции научные рассматривались ненормальными, дискретность 
итд. На старое смотрели как на лженауку с такой иронией. В рамках 
нормальной науки Кун называет задачи головоломками, поскольку 
правила игры заданы, а решается задача другая. Дискретность скачка 
состоит в том, что теории несовместимы (корпускулярная теория света 
и волновая за ней) (механика Ньютона и теория относительности). Тем 
не менее, есть задачи, которые можно рассматривать с разных теорий 
и возможно даже получить одинаковые ответы. У Максвелла идея была, 
что взаимодействие распространяется, а не действует мгновенно, как 
в гравитации было. Нравилось говорить о модификации среды (поле 
и есть модификация). Фарадей не писал уравнений, Максвелл же 
старался положить в математику. А затем Герц после смерти Максвелла 
через 20 лет. В физике есть с одной стороны попытка создать теорию 
всего, но это не очень получается до сих пор, у нас есть набор 
физических теорий лишь.

У Куна нет: вот допустим сформировалась парадигма, как у Ньютона, 
ученые разбирались старались и разобрались. Стали применять куда 
хочется, например к небесным телам, сложно было применить к луне. 
Метод развился и шел успешно, развили метод возмущений, на основе 
его открывали многие планеты. Потом появился Пуанкаре, обратил 
внимание на задачу 3 тех и на неприменимость теории возмущений на 
какие-то случаи. В начале 19 века Лаплас заявил принцип 
детерминизма свой. Математики в то время доказали единственность 
решения задачи Коши (там конечно есть ограничения, но физики не 
обращали внимания на них) (тогда считалось, что мир состоит из 
молекул и атомов и их взаимодействие может быть известно) и из 
всего получалось, что будущее предопределено. Это естественно не 
так уж хорошо для всего подряд было. Затем есть нелинейное, дающее 
бифуркации и прочее, динамический хаос.

Смена парадигм говорит, что теория в физике вещь лишь временная, 
поигрались и хватит. Ньютон, Эйнштейн и многие другие стремились 
создать единую теорию всего. Стремление к единству физики остается 
неудовлетворенным до сих пор. Интуитивные моменты в развитии 
физики. Когда парадигма формируется, очень важны интуитивные 
моменты, например людей, имеющих отношение и к теории, 
и к эксперименту. Оптика утвердившаяся тогда влияла на математику 
и параллельно развивалось и теория колебаний волн и оптическое 
применение. Дальше Френель, со своей формулой про преломление 
и отражение света, говорил об эфире, но получал всё интуитивно. 
Математики верили в эфир сильнее и выводили из дифф уравнений, ибо 
в Френеля получался разрыв. Математики сделали без разрыва, но 
формула оказалась неправильная.

Ньютон держался за абсолютность движения и не говорил об 
относительности, а Мах и друзья его ввели инерциальные системы 
отсчета. Мах еще считал, что Максвелл и Больцман занимались 
метафизикой, ибо не наблюдались частицы так сказать.

Только сегодня будет лекция общего звучания, где затрагиваются 
общие вопросы, а затем будет концептуально исторический метод. 
Далее будем говорить о предыстории физики в западной античности 
и будем идти постепенно приближаясь к нашему времени. При этом 
вести последовательное изложение невозможно, поскольку времени не 
настолько много имеется. Кроме того, это бессмысленно, ибо набор 
фактов всегда скучен.

% }}}

\end{document}
