\documentclass[a4paper, 12pt]{article}

% Configuration {{{
\usepackage[utf8]{inputenc}
\usepackage[T2A]{fontenc} % T1 for English
\usepackage[english, russian]{babel}

\usepackage{enumitem}
\setlist{nolistsep}
\usepackage{mathtools}
\usepackage{xcolor}
\definecolor{dimblue}{HTML}{1010aa}
\usepackage[
	colorlinks=true, 
	allcolors=dimblue
]{hyperref}
\usepackage[
	vmargin=1in,
	hmargin=1in
]{geometry}
\linespread{1.3}
\usepackage{indentfirst}
\usepackage{graphicx}
\usepackage[multidot]{grffile}
\usepackage[labelsep=period]{caption}

%\usepackage{titlesec}
%\titleformat{\section}[hang]{\bf\centering}{\thesection.}{.5em}{}[]

%\usepackage{times} % for English
% }}}

\begin{document}

% Title Page & Table of Contents {{{
\null
\vfill

\begin{center}
	\begin{Large}
		\textbf{История и методология физики}
	\end{Large}

	\vspace{\baselineskip}

	Трубачев Олег Олегович

	\href{mailto:olegtrub@gmail.com}{olegtrub@gmail.com}

	+7 916 538 9479
\end{center}

\vfill

Литература
\begin{enumerate}
	\item Спасский Б.И., История физики, Ч. 1 и 2 М. Высшая школа 1977
		\\ не так много конкретики
	\item Кудрявцев П.С., Курс истории физики, М. 1982
		\\ тонкий для лентяев
	\item Кудрявцев П.С., История физики, ТТ 1--3 М. Просвещение, 1956--1971
		\\ три толстых тома
	\item Кун Т., Структура Научных революций, М. АСТ 2009
		\\ всемирно уважаемая
	\item Ф. Розенбергер, История физики, ТТ 1--3, М--Л. 1934--1935
		\\ хорошо описаны древние времена, но недавние хуже
	\item М. Джеммер, Эволюция понятий квантовой механики, М. Наука, 1985
		\\ подробно и доступно, неплохо для людей, учивших уже кванты
\end{enumerate}

УЧЕБНИК СПАССКОГО НУЖЕН ДЛЯ ОЦЕНКИ ОТЛИЧНО

Экзамен будет после 6--8 где-то но до конца июня. Недельки в две уложим, 
расписания изменяется. Будут билеты, список вопросов появится на почте, 
у старост, подкорректирует с учетом нас, 2 вопроса на билет. Билеты сами 
не кинут, но список вопросов в ближайшие пару дней скинут. Расписание 
смотрим в учебной части, хотелось бы преподу нашему очный. Ближе к дате 
экзамена возможно проведем консультацию какую-нибудь.

\clearpage

\tableofcontents

\clearpage
% }}}

\section{Предмет истории физики и зачем она нужна как учебный курс}
% {{{

\hfill\textbf{Feb 10}

Излагается набор разделов, устоявшийся характер, всех разделов, 
отражающих современную физическую картину мира. Внутренне 
непротиворечивы, содержат основные аксиомы, понятия, дальше 
развивается, развитие требует непротиворечивости. Это нормально для 
раздела естествознания как физика. Кун это замечает, говорит, что 
истории не уделяется особого внимания. Отсылки к истории там только 
в плане ``вот это придумал вот тот, вот это вот тот''. Кажется, 
будто есть современное состояние науки и оно постепенно по кирпичам 
собиралось теми самыми людьми, хоть это и совершенно неправильно. 
Кумулятивный подход. Развитие происходило не так. Знать как 
развивалось на самом деле нужно для продолжения развития 
самостоятельно. Книга Куна дает критику бывшего тогда индуктивного 
подхода. Физика возникает в конце 16 века. Индуктивный 
подход -- эксперименты появляются, нужно собирать факты и их 
обобщать. Обобщать в чистом виде и аккуратно, не вовлекая ничего не 
бывшего в эксперименте. Не бывшее в эксперименте называют 
метафизикой. Декарт уникален в истории физики, он открыл 
преломление и вообще был хорошим философом. Имел метафизические 
принципы, что не характерно для эпохи, и был отличным математиком. 
Индуктивный подход был подвергнут критике. Структура физической 
теории. Компактный набор первых принципов, ключевые понятия, 
и дальше имеет место логическое построение с помощью математики. 
Системы мира Птолемея и Коперника.

% }}}

\section{Закономерности развития физики}
% {{{

Нормальная наука --- нормальное естествознание --- наличие 
определенных парадигм. (парадигма -- шаблон) Есть теория и она 
общепризнана. Ее основные положения не нужно доказывать вечно. 
Далее методы постановки эксперимента. Нормальная наука возникает 
вместе с парадигмой. В 17 веке уже появляется экспериментальная 
сторона физики, но парадигма еще не устоялась. Считается 
устоявшейся в 1689 год с Ньютоном и его математической основы 
физики. Так возникает первая парадигма, хоть книга была сложная, ее 
постигали. В 18 веке уже началось такое вот использование, и теория 
точно стала общепризнанной. Кун говорит, что пока парадигмы не 
было, ученые были вынуждены писать так сказать монографии, очень 
длинно всё это разрабатывающие и описывающие с нуля с основ. 
С парадигмой уже не нужно доказывать основы каждый раз, можно 
писать привычные нам статьи. Экспериментальная наука требует 
парадигмы и сопутствующих моментов. Далее Кун заметил неравномерное 
развитие физики. До Куна было мало обобщений как физика должна была 
развиваться, без морали так сказать. Кун придумал нормальную науку, 
парадигму итд. Неравномерный характер -- возникновение разных 
парадигм, которые друг друга сменяют. И это нормально. До Куна 
такие революции научные рассматривались ненормальными, дискретность 
итд. На старое смотрели как на лженауку с такой иронией. В рамках 
нормальной науки Кун называет задачи головоломками, поскольку 
правила игры заданы, а решается задача другая. Дискретность скачка 
состоит в том, что теории несовместимы (корпускулярная теория света 
и волновая за ней) (механика Ньютона и теория относительности). Тем 
не менее, есть задачи, которые можно рассматривать с разных теорий 
и возможно даже получить одинаковые ответы. У Максвелла идея была, 
что взаимодействие распространяется, а не действует мгновенно, как 
в гравитации было. Нравилось говорить о модификации среды (поле 
и есть модификация). Фарадей не писал уравнений, Максвелл же 
старался положить в математику. А затем Герц после смерти Максвелла 
через 20 лет. В физике есть с одной стороны попытка создать теорию 
всего, но это не очень получается до сих пор, у нас есть набор 
физических теорий лишь.

У Куна нет: вот допустим сформировалась парадигма, как у Ньютона, 
ученые разбирались старались и разобрались. Стали применять куда 
хочется, например к небесным телам, сложно было применить к луне. 
Метод развился и шел успешно, развили метод возмущений, на основе 
его открывали многие планеты. Потом появился Пуанкаре, обратил 
внимание на задачу 3 тех и на неприменимость теории возмущений на 
какие-то случаи. В начале 19 века Лаплас заявил принцип 
детерминизма свой. Математики в то время доказали единственность 
решения задачи Коши (там конечно есть ограничения, но физики не 
обращали внимания на них) (тогда считалось, что мир состоит из 
молекул и атомов и их взаимодействие может быть известно) и из 
всего получалось, что будущее предопределено. Это естественно не 
так уж хорошо для всего подряд было. Затем есть нелинейное, дающее 
бифуркации и прочее, динамический хаос.

Смена парадигм говорит, что теория в физике вещь лишь временная, 
поигрались и хватит. Ньютон, Эйнштейн и многие другие стремились 
создать единую теорию всего. Стремление к единству физики остается 
неудовлетворенным до сих пор. Интуитивные моменты в развитии 
физики. Когда парадигма формируется, очень важны интуитивные 
моменты, например людей, имеющих отношение и к теории, 
и к эксперименту. Оптика утвердившаяся тогда влияла на математику 
и параллельно развивалось и теория колебаний волн и оптическое 
применение. Дальше Френель, со своей формулой про преломление 
и отражение света, говорил об эфире, но получал всё интуитивно. 
Математики верили в эфир сильнее и выводили из дифф уравнений, ибо 
в Френеля получался разрыв. Математики сделали без разрыва, но 
формула оказалась неправильная.

Ньютон держался за абсолютность движения и не говорил об 
относительности, а Мах и друзья его ввели инерциальные системы 
отсчета. Мах еще считал, что Максвелл и Больцман занимались 
метафизикой, ибо не наблюдались частицы так сказать.

Только сегодня будет лекция общего звучания, где затрагиваются 
общие вопросы, а затем будет концептуально исторический метод. 
Далее будем говорить о предыстории физики в западной античности 
и будем идти постепенно приближаясь к нашему времени. При этом 
вести последовательное изложение невозможно, поскольку времени не 
настолько много имеется. Кроме того, это бессмысленно, ибо набор 
фактов всегда скучен.

% }}}

\section{Предыстория развития физики}
% {{{

\hfill \textbf{Feb 17}

От зарождения древнегреческой натур философии до римского. Западная 
античность.

\subsection{Натурфилософия с элементами физического знания}

\paragraph{Ионийская школа}
В древней Греции была неразделимая наука натурфилософия, давшая 
философию, математику и естественные науки. Мало что до нас дошло, но 
известно, что, во-первых, в Греции представлялось, что знание должно 
быть аргументированным -- поэтому был большой успех в ... Ионийская 
школа: единое первовещество, которое может трансформироваться куда 
угодно. Фалес Милетский, Анаксимандр, Анаксимен Милетский. У Фалеса 
была началом вода, она могла даже твердые тела делать. Анаксимандр 
вводил новое абстрактное первовещество. У Анаксимена был воздух 
началом.

\paragraph{Пифагорейцы}
Пифагор Самосский (6 век до нашей эры). Школы приводили к системам 
понятий, которые дальше отразились на многих идеях современного 
естествознания. Россыпь идей надо бы упомянуть, это мы делаем... 
Пифагорейцы важны потому что мы понимаем роль числа в физике, 
количественной закономерности. Школа существовала долгое время поле 
Пифагора тоже. Идея в том, что мера всех вещей является число, и нужно 
пытаться описать мир числами. У нас другое немного. У нас вместо просто 
чисел сейчас математические теории с теоремами, аксиомами и прочим. 
У Пифагора это была именно мистика чисел. Вещь, которая была принята 
там, но физически объяснена задолго после -- связь длины волны и высоты 
тона. Пифагорейцы обратили внимание на то, что гармонично звучат только 
сочетания -- аккорды -- и там их связывали с простыми кратными 
отношениями длин струн. В этом они видели подтверждение того, что числа 
управляют миром. Тогда же считали, что планет всего 10, видим меньше, но 
10 уж слишком хорошее число, поэтому всего 10. А Земля -- шар. У них 
даже была идея движения Земли, что Земля не находится в центре всего. 
Она называлась Пифагорейской, эта идея. Потом был Аристарх Саморский, 
говоривший о таком же вращении. Но у последнего в центре было Солнце, 
а у греков всё вращалось вокруг центрального огня, и Солнце тоже. Потом 
еще считалось, что планеты издают гармонические звуки при движении.

\paragraph{Элеатская школа}
Далее происходит взрыв античной науки. Выделим основное для современной 
физики. Натурфилософия должна основываться на доказательных законах, 
аргументации. Возникла идея законов. Была идея о необходимости 
доказательства и выводов. Потом увидели, что можно было построить много 
что на основе много чего, и возникло противоборство подходов. Парменид 
говорил, что есть единое неизменное начало. Всё сущее существовало 
всегда и было неизменно, а изменение и развитие лишь кажущееся. Апории 
Зенона. Зенон выстраивал какие-то логические построения, кажущиеся 
совершенно нормальными, но приводят к противоречию. О летящей стреле: 
стрела летит с какой-то скоростью, в каждый момент времени стрела 
занимает какое-то положение, но движение стрелы не равно сумме состояний 
покоя. Другая апория про Ахилеса и черепаху: Ахилес вроде бы движется 
намного быстрее черепахи, но если это делить на какое-то такое, догонит 
ли Ахилес черепаху, получается бесконечный ряд. В общем апории Зенона 
были доказательством, что рассуждения уже тогда были глубокие, очень.

\paragraph{Плюралисты}
Они важные для нас потому что когда мы будем говорить об Аристотеле, 
возникает существование 4 стихий (а затем пяти). Это идеи Эмпедокла (5 
век до нашей эры). Его идея дожила в каком-то виде аж до Галилея. Тут 
вопрос еще до каких пор при делении тел вещество остается собой. Позиция 
Анаксагора: существование гомеомеи -- семян всех вещей, которые 
бесконечно делимы, но целостны.

\paragraph{Атомисты}
Левкипп и Демокрит (-460 -- -370 годы). Аристотель подходил критически 
к атомистам, а мы знаем только через его труды. Левкипп считается 
основателем этого дела, Демокрит тоже. Их самих трудов не дошло до нас. 
Только через Аристотеля. Поздние атомисты труды до нас дошли, на закате 
античности греческой, в эпоху римского. Было учение, как 
и Пифагорейское, существовало столетия. Идея в общем -- все состоит из 
атомов (от греческого неделимый), которые находятся в пустоте. Наличие 
этого ``ничто'', пустоты, объясняло рождение и уничтожение атомов. 
Критикам не нравилось именно существования пустоты. Существование 
одновременно и бытия (атомов) и небытия (пустоты). Дальше по поводу 
регулярности движения атомов. Где-то говорят, что говорилось 
о регулярности движения атомов, а потом Эпикур говорил о случайном 
самопроизвольном движении атомов. Еще был поэт и атомист Лукреций Карр, 
написавший поэму ``О природе вещей'', которая до нас дошла.

\paragraph{Платон}
Платон (-429 -- -347 годы). Его труды в значительной степени известны, 
он был одним из первых ученых древней греции, которого можно считать 
полноценным философом. К физике на самом деле относится мало у него, но 
все же интересно. На Платона даже ссылался Гейзенберг, на то, что 
у Платона звучат идеи о 4 стихиях (и 5 стихия эфир, небесная), тогда уже 
определенных успехов достигла математика, геометрия в частности, было 
доказано, что существует лишь 4 правильных выпуклых многогранника. И эту 
4 связал с 4 стихиями. Гайзенберг ссылался потому что атом есть набор 
каких-то там решений, дифф уравнений, математических образов. Какие были 
во времена Платона, такие он и применил, и мы аналогично делаем. Евдокс 
Книдский (408--355 до нашей эры) пытался объяснить движение планет. 
Греки считали, что самое отличное движение -- равномерное по кругу. 
Евдокс пытался комбинировать несколько таких для объяснения движения 
планет. Он вводил сферы и все такое. Земля у него покоилась, он пытался 
объяснить неравномерность движения планет.

\paragraph{Аристотель}
Аристотель (-384 -- -322) учение его оказывалось основой представления 
об окружающем мире. Аристотель был учителем Александра Македонского. 
Немного должны сказать о значении учения Аристотеля для физики. Обычно 
упускают одну вещь, очень важную. У Аристотеля возникает представление 
о том, как должна быть устроена так сказать правильная теория. На основе 
этого основалась и механика Ньютона и все остальное что захочешь. Он 
сформулировал законы (математической) логики. И мы им в общем-то 
следуем. Должен быть компактный набор первых принципов (аксиомы 
в математике, законы в физике), дальше должны быть базовые понятия. На 
этом даже основывается геометрия Евклида с аксиомами и понятиями точек 
и прочего. Далее Аристотель придумал термин физика и попытался своими 
этими сформулировать. Аристотель не противостоял проведению 
экспериментов, у него даже были попытки провести какие-то. Представления 
его конечно не идеальны были. Далее идея о представлении о причинности. 
Причинность это такое как бы очеловечевание природы. В природе не всегда 
можно говорить о причине, нужно делать это аккуратно. У Аристотеля прям 
много обсуждалось это дело. Аристотель считал, что пустоты не существует 
и даже пытался логически доказать это утверждение, и доказательство 
довольно любопытно. Если б тело существовало в пустоте, оно бы не знало 
где право лево и прочее, не знало что ему делать, а значит будет 
покоиться или двигаться равномерно или прямолинейно. Тела должны 
различать движение вниз, вверх итд. Аристотель говорил тепло холод 
сухость влажность и огонь вода земля воздух сопоставлены. Он говорил, 
что тела легки или тяжелы по природе. Говорил, что земля по природе 
тяжела, легкие воздух и особенно легкий огонь. Далее говорил, что 
у каждого тела есть и своя тяжесть, и свое место. Воздух, если на месте, 
он не давит на то, что под ним. Представление об атмосферном давлении 
возникло лишь в 17 веке. Земля истинно неподвижное тело, находится 
в центре мира. Описание движения, наблюдение движения, есть 
относительность. Относительность это правильно, но есть еще и объяснение 
движения. Он пытался сформулировать эти законы.

\paragraph{Механика Аристотеля}
Было систематическое учение Аристотеля о механическом движении, и это 
было первое систематическое объяснение. Далее ее конечно подвергают 
критике. В 17 веке конечно от этого и отталкивались. Мир разделялся на 
надлунный мир и подлунный мир. Луна считалось располагается ближе всего 
к земле. Там над ней более идеальные движения, тела погружены в эфир, 
движения равномерное по окружности. Ближе к Земле к эфиру подмешивается 
что-то, движение менее идеальное оказывается. На основе вот этого 
продолжали основываться идеи движения планет так сказать. Системы мира 
тоже. В подлунном мире есть естественные движения -- в радиальном 
направлении, снизу вверх или сверху вниз. Тут движение обусловлено тем, 
где находится место данного тела. Движение тела к естественному месту 
есть естественное движение. Это имеется в виду земля вниз, воздух и тем 
более огонь вверх. Тела в дальнейшем стремятся располагаться 
в соответствии с этой иерархией. Далее все движения остальные -- 
насильственные. Они имеют причину. Поэтому пустота не допускалась -- 
должно быть взаимодействие. В движении этого возникает некое 
противоречие и это было естественно неприятно Аристотелю. Но сперва 
напомним, что Аристотель не был против экспериментов, но просто в те 
времена было трудно говорить о каких-то количественных закономерностях 
и всем таком. Поэтому, важно, Аристотель никаких совершенно формул не 
писал. Формул не было даже настолько, что не хотели делить метры на 
секунды потому что разная размерность. Говорили о быстроте движения тел, 
но не писали ничего. Идея как происходит движение какое-либо, например 
естественное, непонятно было равноускоренно или равномерно или как 
вообще не говорилось.

\subsection{Александрийский период}

Период с 3 века до нашей эры и до 2 нашей эры. Существенный период 
предыстории развития физики. Особенным было что тут вот еще называют 
период математической физики. Эпикур, Лукреций, Карр в то время были. 
Атомизм был все еще жив несмотря на авторитет Аристотеля. Происходит 
некое отделение математики от философии. Арифметика была уже наверно, 
хоть и сложно определить момент ее возникновения. Если что арифметика 
как удовлетворяет Аристотелю логике появилась лишь в 19 веке. А другая 
область математики -- геометрия -- хорошечно себя чувствовала. Это тогда 
был образец для физико-математических теорий. Греки не так были 
заинтересованы в практическом применении, но оптика и катоптика (учение 
об отражении света) рассматривались как место применения геометрических 
схем, а не как для применения скорее. В то время были известны два 
закона -- прямолинейное распространение света и отражение света. Герон 
Аликсандрийский говорил о принципе наименьшего пути для света применение 
тогда имевшейся математики было скудным, а потом и вообще упало, что 
плохо.

\paragraph{Клавдий Птолемей}
Клавдий Птолемей (70 -- 147 годы). Один из важных ученых античной 
древности. Развитие современной физики сложно представить без 
противопоставления систем мира Коперника и Птолемея. Центром науки стала 
Александрия, хоть и писали все на греческом все еще языке, потому что 
у самих римлян не было особых успехов научных. Но центр науки был уже 
в Александрии, и Птолемей оттуда. Особенно важен один из его трудов -- 
общий обзор в 13 книгах, которые дошли не сразу до Европы. А труды 
греков дошли до наших времен через арабские записи. Подробный звездный 
каталог, точность была ограничена естественно простейшими инструментами, 
без стекол, без линз. Птолемей составил такой вот звездный каталог. 
Птолемей изложил там же систему мира. Система -- в центре находится 
земля, и все надо построить на основе равномерного кругового движения. 
Евдокс вводил равномерные шары. Птолемей говорил об эпицикле 
и диференте, эпицикл -- центр вторичной окружности, движется по первой 
окружности. Планета движется по вторичной. Это было такое равномерное 
разложение, как мы раскладываем по базису разные вещи. У Птолемея 
оказалось довольно точное разложение, хоть и без оснований возникшее. 
Прошло 1400 лет до разного этого самого нового про движения планет, 
и вот за это время не особо ушли планеты. У Коперника была существенно 
ниже точность. Недостатком, в общем, у Птолемея была безыдейность 
подхода его.

\subsection{Упадок древней физики}
Годы 150 -- 700.

\subsection{Средние века}
7 -- 15 века.

% }}}

\section{Научная революция. Экспериментальное естествознание}
% {{{

Говорится про времена 16 века и до первой трети 17 века.

\hfill\textbf{Feb 24}

\paragraph{Коперник}
Коперник решил пересмотреть систему мира Птолемея. Система мира -- 
математическая модель, позволяющая в любой момент вычислить положения 
всего, включая луну и солнце. У Птолемея была хорошая система, но на 
основе Аристотеля -- в центре земля, нужно стремиться к идеальному, 
к окружностям. Для планет приходилось комбинировать, потому что 
траектории были достаточно сложными. Эпициклы и дифференты вводились, 
см. выше. Для повышения точности. Почему была необходимость такое 
вводить вообще он не думал, просто описывал.

Коперник увидел странные закономерности -- радиус вектор одной планеты 
в эпицикле всегда совпадает с радиус вектором земля--солнце. Для верхних 
планет (Марс, Юпитер, Сатурн) по эпициклу и для нижних по деференту 
происходило с единым годичным периодом. Коперник упростил систему 
(КАК???) но уменьшил точность. Но система Коперника подтверждалась 
Аристотелем, говорившим, что нужен компактный набор исходных вещей 
и вывод всего из них. У Птолемея было много ``всемирных 
закономерностей'', и все получалось неоднозначность. Гелиоцентрическая 
система позволила даже привести к механике. Вывод всего этого смог 
сделать только Ньютон, через 150 лет, было очень сложно. В общем, 
основное преимущество в простоте.

\paragraph{Галилей}
Главный чел в создании естествознании -- Галилей 1564--1642 -- умер за 
год до рождения Ньютона. Начало и первая треть 17 века -- создание 
и возникновение современной более физики. В древнем мире были так 
сказать элементы физического знания. Галилей не особо думал 
о методологии, философии, как должна быть устроена теория. Он конечно 
рассматривал Птолемея и Коперника, но не обсуждал как должна быть 
устроена физика, думал о создании механики. Два направления -- 
астрономические исследования с использованием телескопа (сильное за 
Коперника), наблюдения носили скорее качественный характер, не 
количественный, не было партнером физики. Второе -- наука о местном 
движении (механика у поверхности Земли). Почему он решил вот разделить 
можно только гадать. Не было особых разговоров у него по поводу движения 
планет, причине движения, тяготения итд. У Галилея были сильные идеи 
противостояния схоластике. Схоластикой можно назвать даже все 
средневековое, это было основано на основе Аристотеля, 
и анти-Аристотелевский настрой Галилея был очевиден. В частности, он был 
против объяснения движения всего на основе того, что земля находится 
в центре мира. Пара слов об астрономических исследованиях. ``С 
использованием телескопа конечно сильно сказано, не знали даже законов 
преломления света, никакой теории оптических приборов не было, исходили 
из очень античных законов. Шло все от практики. Были уже очки, были 
подзорные трубы, их делали мастера, но теории не было, только практика. 
Также был уже компас, но тоже не на теории, это так к примеру. Галилей 
усовершенствовал подзорную трубу, его устройство обладало от 30 до 50 
крат увеличения, но тем не менее он сделал еще одну важную вещь. До 
этого использовали подзорную трубу только для того, а Галилей посмотрел 
в небо. Он увидел, что планеты имеют какой-то диск, а звезды нет. Еще он 
открыл фазы планет, фазы Венеры. Потом открыл горы на Луне. Оказывалось, 
что планеты похожи на Землю, и это еще сильнее било по Аристотелю с его 
словами, что Земля особенная и вообще в центре мира. Тогда можно было бы 
считать, что центров мира много, но это бред. Дальше он открыл пятна на 
Солнце. Дальше он открыл большие спутники Юпитера, 4 штуки. Если 
деятельность Коперника не встречала никакого противоречия (против 
гелиоцентрической системы даже тогда не была против, а потом стала 
против, во времена Галилея). Не только Галилею, но и некоторым 
последователям, приходилось иносказательно действовать. Систему 
солнечную пытались обосновывать как тяжелое что-то в центре и все 
вокруг. И за это даже высказывался Юпитер. Медицийские луны. Речь шла 
о противостоянии идеям упомянутым Аристотеля. Аристотелева механика 
играет принципиальную роль в описании тел, механики. Далее Галилей 
что-то там. Было сочетание, системы идей, мысленных экспериментов, 
обобщения не носили индуктивный характер. Физика уже не просто обобщала 
эксперименты. Первый этап новой науки механики -- первая треть 17 века 
-- изучение поведения тел у поверхности Земли под воздействием 
тяготения. У Галилея механика -- изучение поведения тел под 
воздействием, под тяготением итд. А откуда оно все не говорилось, закона 
всемирного тяготения не было. Ошибочные представление о работе Галилея 
-- обычно вспоминают бросание тел с Пизанской башни. Нужен был именно 
количественный эксперимент. Если бросать перышко итд, то воздух. У ядра 
будет тяготение конечно преимущественно. Нужно было хорошо измерять 
время -- тогда не было таких вот возможностей. Часов в современном 
смысле не было. Реальные вот эксперименты -- движение шаров по желобу. 
Тут уже можно что-нибудь такое вот придумать. Плохая вещь --- ускорение 
выходит не g sin alpha, потому что абсолютно твердое тело, часть энергии 
идет на поступательное, часть на вращательное. Далее желоб должен быть 
длинным, а наклон маленьким. Вообще говоря в Ньютоновской механике 
движение не раскладывается. В сложных системах нужно решать диффуры 
с многими переменными. Галилей же сказал, что раскладывать можно. 
Следующий момент был в том, что это равноускоренное движение. Галилею 
удалось систематически описать движение, как скорость будет зависеть от 
координат и времени. Причем он сделал всё это не на основе мат. анализа, 
не было бесконечно малых. Галилею приходилось решать наполовину 
эмпирически, наполовину из соображений математических. Как к понятию 
быстроты движения относился Аристотель. Никогда не делили величины 
разной размерности, Галилей уже пользовался нашей современной скоростью. 
Более того, он использовал мгновенную аж скорость. Галилей на этой 
основе сформировал принципы своей механики. Это конечно не первые 
принципы механики в современном смысле, но он смог сделать это на основе 
как мысленных, так и реальных экспериментов. Сам Галилей считал себя 
математиком, механиком, и математика играла естественно решающую роль. 
Из современных понятий там был направленный отрезок и скорость как 
вектор, хотя слово вектор не было использовано. Мгновенная скорость 
тоже, но не было бесконечно малых. Сложно ему было показать, что квадрат 
скорости пропорционален высоте, а скорость пропорциональна времени, а не 
координате. Были также разночтения. Принцип инерции у Галилея не 
сформулирован. Мы говорим, что вдали от всех тел тело будет двигаться 
без ускорения. А вот Галилей говорит, что ускорение идет к нулю при 
альфа -> к нулю. И больше ничего не говорил, наверное, а прямых 
свидетельств о его формулировке не было. Далее у Рене Декарта принцип 
инерции сформулировал. Но косвенно считается, что у Галилея был он тоже. 
Далее противостояло что Аристотелю. Аристотель не говорил никогда о том, 
как тело будет двигаться. Дальше принцип относительности, у Аристотеля 
опять же не было такого. У него было описание в адмиральской каюте 
корабля. Если окна открыты, мы видим, движения или нет корабль, а если 
закрыть, допустим волн нет, мы можем проводить внутри эксперименты, и не 
поймем ни за что стоим или движемся равномерно. Торричелли что-то там. 
Пусть у нас есть с одинаковой высотой, но разными углами, разной длины, 
то скорость в основании внизу будет одинаковой. Системы с одним телом, 
которое движется под действием силы тяжести и реакции плоскости -- 
сейчас можем так интерпретировать -- это прародитель закона сохранения 
энергии. Тем не менее тогда назывался принцип Торричелли. Дальше в те 
времена уже бесконечно малые по чуть-чуть пробирались, и можно было 
определить закон движения. Торричелли говорят первым описал маятник 
движение, независимо от Гюйгенса он вывел это. В общем у нас тут 
принципы, которые нельзя назвать конечно полноценным чем-то, но неполное 
и мы видим, что с появлением законов Ньютона ситуация кардинально 
изменилась.

\paragraph{Торричелли}
Далее Торричелли 1608--1647 пример мысленной задачи, эксперимента не было 
никакого, мысленное все. Проанализировали идеи Галилея с этими 
квадратами, горизонтальное движение равномерное, и вот получается 
парабола при движении тел под углом к горизонту. Была конечно попытка 
описать на основе этого артиллерийские вещи, но там конечно большой 
вклад от воздуха получается, и применение лишь качественное, не 
количественное. Движение тел по поверхности пытался тоже описать, но 
удалось лишь с маятником.

\paragraph{Декарт}
Рене Декарт 1597--1650 -- очень масштабная фигура. Деятельность была 
очень всесторонней. Не просто философ методолог, о метафизике Декарта 
нужно говорить . Он говорил об индукции, о необходимости 
экспериментальных дел. Декарт много всего внес, говорил, что человек 
должен в мышлении много что учесть до эксперимента, потом определенные 
общие закономерности, он считал, идут не из эксперимента, а из 
особенности познания. Эти идеи мы коротко относим, говорим о методологии 
Декарта. Что еще важно -- было в ту эпоху мощное отвержение 
Аристотелевских вещей. Идея отсутствия пустоты была у Аристотеля, 
и континуум поддерживал и очень развивал Декарт, это потом вылилось в 19 
веке в сплошную среду, гидродинамика итд. Декарт был тоже противником 
атомизма. Его объяснение среды было контратомистским. Тот же Ньютон был 
безусловно атомистом. Декарт говорил все же о сплошной среде, 
заполняющей все вообще (можем сейчас говорить даже об эфире). Еще 
у Декарта были корпускулы, они могли сталкиваться, склеиваться, 
разделяться. Дальше, если среда не очень сжимаемая, то какое вообще 
движение может быть -- вихреобразное, вихри Декарта. Механика здесь уже 
шла о взаимодействии тел, и даже поставил задачу а как же планет 
взаимодействуют. Можно было рассмотреть взаимодействие корпускул. Он 
говорил, что есть физические количественные законы взаимодействия тел. 
Декарт говорил, что взаимодействие было локальным, то есть только 
касающееся взаимодействует с корпускулой. Возникает таким образом еще 
и законы сохранения. Говорят о столкновении тел, используют законы 
сохранении. Он ввел идею сохранения количества движения -- импульса. Так 
вот Торричелли сохранял энергию, а Декарт импульс. Но вот у Декарта 
получалось не очень хорошо, потому что вектора понятия там особо не 
было. Оказалось, что возникают внутренние противоречия в сохранении 
количества движения по модулю. Но направления задано. Декарт, важно, 
говорил о том, что нужно работу Галилея развивать. Но и свое тоже вносил 
неплохое. Будем еще к Декарту возвращаться. Картезианцы. Галилей писал 
труды на итальянском. Декарт писал на латинском, его имя было бы 
Картезий, и вот картезианцы. Закон преломления света Декарта--Снелиуса, 
коммуникации были затруднены, работы были независимы.

\paragraph{Задача соударения тел}
Задача соударения тел, быстро осознали тогда ученые, имела большое 
значение. В эти же дни формируется Лондонское королевское общество. Роль 
его очень большая. Сразу был объявлен конкурс на задачу соударения тел. 
Участвовали три человека Гюйгенс, Ренн и Уоллес. Уоллес развил абсолютно 
неупругий удар. Гюйгенс рассмотрел абсолютно упругий удар. Особенно 
нецентральные удары такие очень сложные. Так вот Гюйгенс рассмотрел для 
начала центральный удар. Рассматривал имевшиеся законы и применил их 
неожиданным может быть способом. Через системы разные отсчета. Среди 
других трудов сильных был Horologium Oscillatorium. Показал, что маятник 
изохронный, но только при малых отклонениях. При циклоиде где-то там при 
конечной амплитуде колебаний уже строго изохронно получалось. Гюйгенс 
однако понимал, что практически удобнее малые колебания использовать. 
Потом он рассматривал физический маятник. Потом в том же сильном труде 
он рассматривал центростремительное ускорение. Тогда же понималось, что 
скорость это направленный отрезок, и понимали, что ускорение возникает 
не только при движении по прямой, но и при поворотах. В частности 
в равномерном движении по окружности. Вывод Гюйгенса в точности 
совпадает с школьным выводом центростремительного ускорения через 
подобие треугольников.  Далее было видно уже, что движение планет 
чистыми окружностями не обойдется, но вступят законы Кеплера скоро.

\paragraph{Возникновение теории гравитационного взаимодействия.}
Система мира Кеплера (1570--1630). Законы Кеплера 1609--1619, идея 
всеобщего тяготения Кеплера + мистицизм. Учитель, руководитель Кеплера 
Тихо Брага был сторонник Птолемея, противник Коперника. Но Кеплер 
воплотил с модификациями гелиоцентрическую систему. Движение планет 
близко к окружности, эллипс с близким к нулю эксцентриситетом, причем 
Солнце не в центре, а вообще в фокусе. Подпортило репутацию Кеплера 
может быть еще и мистификация, он говорил о душе планет. Он считал, что 
воздействие планет идет конечно от солнца, но пропорциональность 
1/радиус, не в квадрате, из-за цилиндрической симметрии. И потом когда 
Декарт сформулировал идею взаимодействия. Попытки объяснения так вот 
возникали. Идея тяжести, возникла идея также вихрей эфирных, как раз 
считалось объясняют такое движение планет. То есть вихри эфира такое 
делают, концепция близкодействия. А вот еще дальнодействия тоже была 
у Роберваля, но у него не было математической картины полной. Еще 
особенно важно было в Италии, где плохо относились к Копернику,  было 
знаменито иносказание. Были ученые итальянские, которые пытались 
применить идеи Роберваля и уточнить, применить дальнодействие от Солнца 
к описанию планет и медицийских лун (которые там у Юпитера говорили про 
спутники большие). Была идея динамического равновесия в поле тяготения 
и в связи с тем, что отталкивает дальше. Это было до публикации Гюйгенса 
о центробежной силе. Кстати законы Кеплера применялись еще к этому 
самому про Юпитер и его спутники, и успешно, поэтому больше подкрепления 
получали. В общем много было сделано до Ньютона.

\paragraph{Механика континуума}
Механика континуума -- аэростатика и гидростатика. Концепция 
атмосферного давления. Сопротивляется система вытягиванию поршня. Тогда 
разряжения эти использовались для создания фонтанов, это можно было 
объяснять страхом пустоты, идеей Аристотеля, но только на 10 метров. 
Торричелли тогда сказал, что атмосфера все же давит на все под ней, 
и поэтому выше не может подняться столб воды. Знаменитый опыт был 
с ртутью. Прибор Торричелли измерял давление в общем. И второй аргумент 
был, что при поднятии в гору этого прибора он показывал все меньше 
и меньше. Почему тоже непонятно. Галилей смог вывести экспоненциальную 
барометрическую формулу.

% }}}

\section{Математические начала натуральной философии}
% {{{

1643--1727
Ньютон -- Кембридж. Возглавил потом кафедру Лукасовской математики, 
читал по оптике. Бесконечно малое, мат анализ. Член парламента стал 
после хорошей революции. 

Развитие классической механики, вклад Ньютона в это дело. 
Математические начала натуральной философии. Это по сути физика. По 
традиции так уж называют это дело, такими словами. Многие физики 
стремились назвать теорию всего, например даже Эйнштейн, но не вполне 
это получалось. Но вот направление мысленное конечно очень важно. 
Принципиально было слово математические. Он не мог ограничиться 
качественной эмпирической моделью. Качественно многие вещи были понятны 
и до Ньютона. Но хотелось бы не просто рассуждения, а прям объяснение 
какое-нибудь. Что-то он там дождался смерти Гука, и потом только 
опубликовал. Конкретно что было сделано Ньютоном. Предыстория была 
такая, что Ньютон публиковать не любил, по математике он не хотел 
публиковать, его уговаривали, потом по механике планет тоже. Галлей был 
хорошим другом, один из основателей Лондонского королевского общества. 
Он познакомился с трудами раньше и способствовал публикации. Лейбниц 
носил натур-философский характер. В эти годы в 17 веке не было такой 
системы ссылок на предшественников. Публиковали не статьи по сути, 
а монографии. Тем не менее Ньютон четко делает отсылки на Галилея, 
Декарта, Гюйгенса, Ренна, Уоллеса. Это была не абстрактная математика, 
а просто конкретно решенные некоторые задачи. Ньютон был заинтересован 
в измерениях каких-то там. Икар измерил радиус Земли в широтном 
направлении.

\paragraph{Введение}
Содержание математических начал натуральной философии. У Кудрявцева, 
если хочется, можно тоже глянуть. Этот труд, в общем, состоит из 
введения и трех томов. Во введении в общем и сформулированы законы 
Ньютона. Написан труд на латинском языке, а потом в 18 веке было уже 
издание на английском, а на русский -- только в 20 веке. Доказательства 
математики в общем было все в геометрической форме. В начале он дает 
определения, во введении, в частности, количество материи. Эти 
определения были философскими немного скорее. Плотность у него была 
степень заполнения атомами пустоты. Масса -- количество материи -- была 
произведением объема на плотность. Количество движения было тоже. 
Понятие силы не очень отделялось от энергии тогда. Приложенная сила -- 
близко к обычной нашей силе. Сейчас мы говорим, что взаимодействие тел 
определяется центральными силами. Иначе не будет закона сохранения 
импульса строго выполняться. Ньютон аккуратно подходит к этому, 
упоминает наличие центральных сил, но не говорит, что все взаимодействия 
должны быть центральными. Еще были ускорительная и движущая силы.

Дальше поучение. Следствие структуры механики, о чем говорили 
и предшественники, был принцип относительности. Ньютон специфически 
подходил к относительности. Он говорил, что есть какая-то абсолютная 
система отсчета, неподвижная итд. Он придавал большое значение этой 
системе, связывал к господом, и пытался определить наше движение 
относительно такой системы. Невозможно абсолютное движение зафиксировать 
на данном этапе, он говорил, но надеялся, что появится возможность. Это 
вот высказывание подвергалось критике, но положительной. Мах как раз 
в этой связи сказал, что нужно формулировать, проанализировать, 
уточнить. Откорректировал Ньютона тоже, сказал, что нет смысла вводить 
абсолютное движение, а потом Ланге, последователь Маха, сформулировал 
уже что надо. Первый закон говорил, что тело движется равномерно 
и прямолинейно, пока на него ничего не действует. Второй гласит, что 
изменение количества движения пропорционально движущей силе и происходит 
по направлению той прямой, по которой эта сила действует. Третий закон 
не говорит о той идее центральности сил. Третий закон говорил лишь что 
силы равны и противоположны по направлению. Следствие 2 -- принцип 
параллелограмма сил -- зачаток векторного сложения. Следствие 4 -- роль 
центра масс. Следствие 3 -- закон сохранения импульса. Следствие 5 -- 
принцип относительности, нет разницы система покоится или движется 
равномерно и прямолинейно.

\paragraph{Том первый}
Том первый -- - движении тел. Почему Галлей стимулировал публикацию -- 
ему нравилось решение конкретных физических задач. Ньютон дальше их 
собственно решал. Они все, решения, содержатся почти все в первом томе, 
о движении тел. О движении тел что там было. Краткий ответ -- основная 
задача там была задача Кеплера. Решается она не так уж просто, а тем 
более с учетом того, что не было такого уж математического анализа. Том 
сам разделялся на отделы. Ньютон смог в 7 разделе сведенную уже 
к одномерному случаю задачу проанализировать методом геометрическим. Он 
проводил разные кривые, рассматривал под ними площади. Вычисляет 
квадратуру вот так геометрически и получает решение полноценное. 
В первом томе решается задача Кеплера, в общем. Далее в разделах 12+ 
есть задача о притягательных силах сферических тел. Движение луны или 
движение тела вблизи поверхности земли. На поверхности Земли действует 
та же самая сила, что и удерживает Луну. Здесь Ньютону нужно уже 
и количественное согласие хорошее, чтобы это все доказать. Причем радиус 
земли стал известен вдруг, и Ньютон смог довести до числа. Для этого 
всего удобна теорема Гаусса, но Ньютон смог и без нее решить. 
Современники же очень скептически к этому относились, говорили, что 
чтобы проверить, надо копать шахту в центр Земли, или найти планету 
с дыркой внутри. Если есть центральные силы вблизи большого тела, тела 
могут отскакивать (закон отражения с плоскостью падения и углами) 
и могут проникать (получается что-то типа закона преломления). В общем 
в 14 разделе получилось таким образом даже обоснование. Гримальди 
говорил о дифракции ввиду размытия границ. Ньютон видел, что здесь тоже 
можно использовать законы механики. Он считал, что всю физику вообще 
можно перевести на закон механики.

\paragraph{Том второй}
Второй том -- влияние среды. Ньютон естественно говорил об эфире. 
Свойства поля конечно нельзя вывести из механики Ньютоновской. С 19 веке 
было важно не только, что доказали теоремы о механике и об обыкновенных 
дифф. ур. Но и продвинулись в направлении анализа частных производных 
и остального, что было необходимо конечно для сплошной среды. Но 
характер зачаточный был даже в 19 веке, не то что в 17 при Ньютоне. Но 
Ньютон рассмотрел движение тел в среде, он рассмотрел вопрос 
гидростатики, интегрировал так сказать прошедшие труды в свой том. И еще 
была идея анализа силы лобового сопротивления, сила у него была 
пропорциональна скорости в первой степени. Решает он снова 
геометрически. Если время нарастает арифметически, то скорость убывает 
геометрически. Дальше он говорил, что может быть сила сопротивления 
равна какой-то комбинации первой степени и второй степени. 
Геометрический метод интегрирования снова дает какой-то ответ, но 
Ньютону это было сложнее, и Ньютон решает применить численное 
интегрирование (метод Ньютона!). Как раз тут вот оно и предложено. 
Показывает он также, что если есть движение как по типу Кеплера, но 
в среде, то происходит падение на центр, движение по спирали. Далее надо 
иметь в виду, что газов особо тогда не было известно. Но был закон Бойля 
-- экспериментально. Бойль сформулировал закон, который не 
рассматривался конечно как у нас изотермическое итд, поскольку 
термометров не было. И вообще ничего, кроме закона Бойля, о давлении 
воздуха не было известно. И на основе этого дела Ньютон смог обосновать 
барометрическую формулу, введенную Галлеем. Чисто теоретиков кстати 
тогда не было, но эксперименты у Ньютона были в оптике. Дальше было 
известно о гармонических колебаниях каких-то от этого самого дела 
и в связи с этим говорилось что-то там про скорость звука. Но тогда не 
было степени гамма где-то. Спор с Декартом был о возможности вывести 
законы Кеплера из эфира.

\paragraph{Том третий}
Третий том начал -- система мира и правила рассуждения в физике. Уже 
в том виде, как в 17 веке представляли. Птолемей уже был историей, а вот 
законы Кеплера вызывали интерес. И было видно, что применимы они даже не 
только к планетам, а вот еще и к кометам, как была комета Галлея. Того 
самого Галлея. Это уже была теория возмущений вот тут введена, была 
задача двух тел, а потом разложение по неточечности двух тел. Здесь же 
еще был приведен факт единства природы гравитации с помощью уточнения 
гравитации земли. Камни на земле испытывают ту же гравитацию, что 
и луна. Не нужно принимать причин сверх тех, что истинны и достаточны. 
Гипотез не измышляют.

Развитие механики в 17--19 веке. Очень много всего было сделано, 
конечно, поэтому многотомные монографии. Подход геометрический был 
неэффективным, уже в 17 веке Декарт, в частности, заметил, что 
аналитические методы удобнее. Эйлер 1765 год, у него были разные дифф. 
уравнения, сначала были с естественным представлением координат, 
естественным движением. Потом механика твердого тела и что-то еще. 
Дальше была аналитическая механика Лагранжа 1788. Дальнейшее развитие 
небесной механики, появлялись более точные данные. Надо было 
корректировать методом теории возмущений, который кстати не был доказан, 
что сходится, но был применен тем не менее. Применить значит смогли, 
обнаружили, что хотелось бы, чтобы было у кого-то несколько спутников. 
А потом они были обнаружены. Успешное развитие и применение механики 
конечно произвело впечатление. К тому времени, начало первая треть 19 
века, появилась математическая теорема, доказывающая единственность 
решение задачи Коши дифф. уравнений. Лаплас представлял что-то там. 
Пуанкаре увидел, что плохо дело обстоит со сходимостью теории 
возмущений, показал, что принципиально не интегрируемая эта задача трех 
тел. В 20 веке уже обсуждались динамический хаос и другие вещи. А еще 
в 19 веке как уже сказано было, сформулировали нормально полноценно 
закон про инерциальные системы отсчета.

% }}}

\section{Физика электрических и магнитных явлений}
% {{{

18 век называют периодом невесомых -- Ньютоновские подходы, гравитация 
итд. Невесомые флюиды скорее применимо тут, они стали утверждаться как 
раз в 18 веке, это по сути вещества, компоненты, которые обеспечивали 
взаимодействия, о которых мы собираемся говорить: электричество, тепло, 
оптика далее, гравитация. Гравитация универсальное взаимодействие, 
и универсальность -- главный признак. Другие виды явлений 
последователями Ньютона трактовались аналогично гравитационным, но не 
универсальным. Для взаимодействия в тело должен был зайти 
соответствующий флюид (электрический, магнитный итд). Это общий подход 
такой, нас более конкретное интересует. Гильберт опубликовал в 1600 году 
труд о магните, магнитных телах, и о большом магните -- Земле. Он 
распространяет подходы, которые Галилей использовал для местного 
движения, вблизи Земли. Кроме того он утверждает то самое про Землю, 
и мы спорить не будем. Тогда уже компасы и другие вещи тоже 
существовали, так что чуть менее неожиданно. Самое главное нужно сказать 
следующее, хотя в одну книгу объединены магнитные и электрические, 
Гильберт говорит, что это разные вещи. Понятие электромагнетизма 
сформировалось лишь в первой трети 19 века, потребовалось 200 лет. 
Главные отличия их -- электричество более универсальное, было достаточно 
лишь потереть -- а магнитное гораздо более редкое, но оно совершенно 
постоянное. Он считал, что магнитные полюса совпадают с географическими. 
Он еще заметил, что магнитные полюса ходят парами. Причем тогда даже не 
было понятия знака заряда, обсуждалось лишь притяжение такое вот. Дальше 
интересно вот что, история электричества до середины 45 46 года, 150 лет 
развития электричества. Контрпример того, что заинтересовано в чем-то, 
если проводятся эксперименты, значит хорошо. Если эксперименты 
проводятся без особого смысла, особо результатов не будет вообще. 
Несмотря на то, что какие-то вещи были открыты при таком метании, потом 
при нормальном подходе их приходилось заново открывать. Далее был 
поворотный момент 1745 год -- Лейденская банка. Развитие не было 
прямолинейным и постепенным. Открытие Лейденской банки перевернуло все, 
был пересмотрен концепт Гильберта о слабости электричества. В результате 
определенных манипуляций, обернуть банку фольгой, налить воду, в воду 
вставить гвоздь. Не было тогда никаких единиц заряда итд, было просто 
видно, что происходит что-то существенное, можно было замкнуть по 
гвоздю, и получалась искра, очень сильная. Тогда и был пересмотр дела. 
Сразу же начались разные публичные демонстрации. Через пару лет 
выяснилось, что нужно лишь две проводящие плоскости и диэлектрик. В 1747 
дошел таким образом до плоского конденсатора. В 18 веке уже в России 
образовалась академия наук в Питере, сначала там были иностранные 
ученые, но среди них был и замечательный физик, приятель Ломоносова, 
Рихман. Как только была открыта Лейденская банка, Рихман проводил 
попытки измерить степень электризации, но с помощью нити, электроскопа, 
отклонение нити. Через несколько лет некоторая предварительная теория, 
концепция, сформировалась. Это не была парадигма, как в механике, это 
там были лишь базовые понятия. И они возникли в результате Лейденской 
банки. Два исследователя -- Бенджамин Франклин и Ульрих Теодор Эпинус. 
Центром исследований была все еще Европа, но и американец смог 
сформировать вот это. Было выяснено, что такое распространенное дело как 
грозы, связаны именно с электричеством. Хотя раньше были даже мнения, 
что это сгорание серы. Это Франклин сделал. Дальше была идея, что 
электрический заряд есть какой-то флюид, и идея не лишь что проникая 
дает взаимодействие, но и невесомость, было уже известно о положительных 
и отрицательных зарядов. Франклин говорил о едином флюиде -- избыток 
и недостаток дает знак -- и таким образом заряд сохраняется, закон 
сохранения заряда. Далее еще что нас возвращает обратно в Питер, 
в академию наук. Ульрих Теодор Эпинус работал там, получал зарплату от 
России, он был не только физиком, но и скорее математиком. Он наладил 
службу шифрования. На досуге он мог заниматься электричеством. Он 
независимо от Франклина сформировал некоторые идеи -- два типа заряда, 
сохранение, ... Но его особенность была в том, что он обратил внимание 
на схожесть электричества и гравитации, сказал, что они все действуют на 
большие расстояния. Ньютонианскую позицию сформировал Эпинус, таким 
образом. Дальше конечно все развивалось, был противоположный подход, не 
очень нравилось что одна жидкость лишь. Была точка зрения, что есть две 
жидкости, положительная и отрицательная -- Роберт Симмер. Дальше у этих 
товарищей возникают количества зарядов. Далее был и противоположный 
подход, хотя нельзя было отрицать гравитацию от Ньютона, но были 
попытки. Попытки создать такую гидродинамическую теорию гравитации. 
И аналогично было с собственно электричеством. Одними из них были Эйлер 
и Ломоносов. Ломоносов занимался тогда химией скорее, но на физику тоже 
хватало какое-то время. Эйлер в своих письмах немецкой принцессе пытался 
объяснять все с помощью движения флюидов, концепции близкодействия. Есть 
движение некой среды, эфира, происходит некая модификация среды, и это 
влияет на поведение вещей вокруг. Кинетическая теория.

Дальше как стало выясняться что можно много что сделать 
с электричеством, настало мнение, что пропорциональность 1/r to the 2. 
Об этом еще говорил Эпинус когда проводил аналогию с гравитацией. Нужно 
было подтвердить это на эксперименте в общем. Пристли, Джозеф проводил 
эксперименты какие-то, базовые. Дальше был Кавендиш, но не опубликовал 
свои дела. Дальше был Кулон с 6 мемуаров по электричеству, и в 1889 по 
магнетизму. Особенным был сам метод Кулона, крутильные весы. Говорят, 
что есть крутильные деформации, формула была как раз получена Кулоном 
впервые, коэффициент пропорционален модулю сдвиговой деформации, тем 
меньше, чем меньше четвертая степень диаметра нити, и чем длиннее нить 
тем меньше. На самом деле не было тогда никакой теории упругости, потому 
что не было теории диффуров. Эти работы Кулона по выводу коэффициента 
жесткости несли очень инновационный характер. Что очень важно, Кулон 
достиг очень большой точности. В 1800 Кавендиш смог тем же методом 
измерить гравитационную постоянную, вот какая высокая точность была. 
Считалось, что электрические явления в духе Гильберта, были разными 
вещами, но уже в конце 18 века Кулон чувствовал, что сходство какое-то 
есть. В мемуарах по магнетизму было сказано, что между магнитными 
монополями тоже должно было бы быть единица на радиус в квадрате. Вообще 
тогда и математика развивалась неплохо, особенно относительно функций 
многих переменных. Частные производные присутствовали в трудах Лагранжа 
Даламбера. В начале 19 века выяснилось, что для 1/r2 можно ввести 
потенциал, который так сказать разлит по всему пространству, и из него 
можно получить силу. Теория потенциала образовывалась. Основные моменты 
в виде закона Кулона, закона сохранения заряда, понятия заряда уже были. 
Дальше физика постоянных и квазистационарных токов. Это совершенно 
особое явление, обращение к этой новой физике состоялось на рубеже 
веков. Гальвани ставил известные опыты с лягушкой, с лапками. Гальвани 
таким образом говорил о животном электричестве, электрофизиологии. Но 
был человек, который засомневался в заявлениях Гальвани, это Алессандро 
Вольта, он сказал, что бывают два типа проводников, жидкие и твердые, 
и когда мы сочетаем такие вещи, получаем что-то дающее постоянную 
электризацию, он создал такой вот источник тока, гальванический элемент. 
В 1800 году это устройство возникло. С одной стороны возникало это 
устройство, понималось, что есть два знака, плюс и минус, но вместо того 
чтобы просто от трения сделать, оно пропадало, то гальванический элемент 
был такой вот постоянный

% }}}

\section{Развитие теории тепла}
% {{{

Тепло и холод  ощущаются конечно нами всегда. Аристотель говорил, что 
есть исконно теплые и исконно холодные вещи, но ситуация начала меняться 
в 17 веке, в начале века формируются две точки зрения, обе допускали 
интенсивность. Первая -- тепло есть вещество тепла, возникает сохранение 
тепла и представление, что чем больше теплорода в теле, тем горячее, чем 
меньше тем холоднее. Но количественных измерений не удалось сделать, 
хоть иногда и вспоминают теплоскоп по типу барометра, много воздуха под 
низким давлением, узкая трубочка и шарообразный сосуд, за счет 
расширения вода поднималась, но это все было неоднозначно и не было 
законов расширения итд. Важно было что возникла идея во времена Галилея, 
а вот тут появилось устройство дающее возможность такую вещь делать 
создавать, прошло больше ста лет 100 лет. Также нужно сказать, что тогда 
во времена чуть позже во времена 17 века возникает и другая теория. 
Кинетическая концепция тепла, что все заполнена флюидом, но его сложные 
движения приводят не только к электричеству, но и теплоте, вот его 
сложные движения приводят к. Так объяснялись и эфир и оптические явления 
итд. Это конечно проблем не решило, но вот они эти два знаменитых 
подхода к природе тепла. Ближе к 1740 годам возникли какие-то 
термометры, до них не удавалось. Там было три термометрических шкалы, 
с специфическими формами сосудов. Ртуть оказалась наиболее эффектной, 
поэтому в науке больше всего используется. 1742 году ртутный термометр 
Цельсия, в 1740 году спиртовой термометр Реомюра, а потом еще был 
Фаренгейт со своими смесями воды и солей. Удалось приборы сделать, 
почему? Потому что нужна изоляция, нужно качество жидкостей, чтобы 
расширению не мешали пары воздуха и прочее. Обратим внимание на вот еще 
что. Христиан фон Вольф, учитель Ломоносова в Германии, высказал, что 
соль холоднее воды. Очень скоро приборы измерения тепла удалось удачно 
применить, наука испытала большой толчок. В 1740 году Георг Рихман 
смешивал жидкости и определял итоговые температуры. Он в конце выдал 
формулу без теплоемкостей, с чисто массами, очевидно были отклонения от 
нее, но он говорил о тепловых потерях. В 1784 году появилась 
теплоемкость, а вскоре после Рихмана скрытая теплота фазового перехода 
была измерена в Англии и Германии. Джозеф Блэк формирует парадигму 
учения о теплоте, возникает уже система и какие-то законы. Только 
с 2 половины 19 века идея теплорода начала побеждать кинетическую идею. 
Блэк говорил о существовании двух мер теплоты, количества тепла 
и температура. Жидкий теплород, сообщающиеся сосуды, какие-то вещи шире 
других, для наполнения одинаковой высоты требуется разное количество 
жидкости. Сохранение теплорода тоже возникло, и важно что и хорошо... 
Следы кое-чего можно было увидеть и у Рихмана, и у Блэка. Это 
необратимость. Немного на периферии было, у них была итоговая 
равновесная так сказать температура, выравнивание, течение определенное 
строгое. Если есть колебания в идеальной жидкости, они не прекратятся, 
надо ввести трение, диссипацию, необратимость. Но вернемся туда. 
Ломоносов еще придерживался кинетической, Эйлер тоже, но почему она 
начинает сдавать позиции? Она количественно очень сложно могла бы 
объяснять эти вещи, и наоборот базовые закономерности легко объяснялись 
теплородом.

Теперь мы завершили такой предварительный разговор, переходим 
к кинетической теории, атомизм, молекулярно-кинетическая теория. Эйлер 
Ломоносов Бернулли были сторонниками кинетической модели. Работы по 
тепловому балансу носили предварительный характер, поэтому было более 
равноправие между двумя теориями теплоты. У Ломоносова были особенности 
такие, из-за химии Ломоносов был большим сторонником атомизма, 
картезианские, эклектические. Теплоту и холод он считал скрытым 
движением нечувствительных физических частиц. Чем более интенсивное 
движение, тем больше будет тепло. Ломоносов выводит закон Бойля в опытах 
теории упругости воздуха в 1748 году. Наш современный вывод не могли 
тогда сделать, ибо не было распределений по чему-либо, не было концепции 
газов и воздуха как смеси газов не было так много. Только потом уже были 
обнаружены что?. Молекулярно-кинетическая теория менее развита. Давление 
было связано с кинетическим движением, а тепло было связано 
с вращательным. Частицы могли зацеплять друг друга и так передавать 
тепло.

Дальше обратим внимание на молекулярно кинетические концепции, в 19 веке 
уже близко к нашему развивалась, труды Ломоносова не были сильно 
известны в Европе, но Бернулли были, поэтому имела место 
молекулярно-кинетическая теория вообще. Но был еще один натур философ, 
идеи которого существенно повлияли на создателей физики 19 века. 
Хорватский товарищ Рожер Йозеф Боскович. Он говорил, что конечно есть 
атомистическая теория вещества, атомы есть, дискретное конечное 
количество, но это лишь как точки пространства. Атомы были лишь центрами 
дальнодействия сил. Дальнодействующие силы существуют повсюду, а атомы 
-- безразмерные точки центры дальнодействующих сил. Была не только 
притяжение и отталкивание, но на малых расстояниях он вводил почти 
синусоидальную структуру, конечное, но большое количество равновесных 
точек. Это объясняло наличие твердых тел, статическое равновесие. 
Интересно тем, что все пространство было заполнено силами, это приводило 
к собственно электромагнитному полю. Боскович произвел большое 
впечатление на Фарадея.

Упадок кинетической теории конечно случился. В начале в середине 18 века 
был баланс. Математических теорий не было для молекулярно-кинетических 
теорий, а ньютонианский подход позволил так объяснить вещи. Последнюю же 
каплю в пользу ньютонианского подхода привело открытие лучистого тепла. 
В конце 1790 году обнаружили фокусировку тепловых лучей зеркалом, через 
пустоту распространялось тепло. Законы распространение лучистого тепла 
очень напоминали оптику, а она была корпускулярной тогда. Поток 
теплорода через пустоту. А изменение было в начале 19 века.

% }}}

\section{Электричество и магнетизм до середины 19 века}
% {{{

Луиджи Гальвани увидел движение лягушачьих лапок под действием 
электричества -- животное электричество. Потом Алессандро Вольта 
сказал, что лапки это лишь индикатор. Потом он сказал, что возникание 
положительных и отрицательных зарядов происходит лишь только уже при 
контакте металлов -- контактная разность потенциалов. Потом выяснилось 
особая роль двух типов проводников. Проводники металлические 
и проводники жидкие. Вода, тем более что химический состав ее в деталях 
не рассматривался, считалась проводником. Электролитов тогда не 
рассматривали. Если есть жидкий проводник тоже возникает поляризация, 
но потом еще и возникает постоянный какой-то постоянная разность 
потенциалов. Короче создали гальванический элемент. Эффект -- 
постоянная модификация среды, которую мы называем сейчас 
электричеством. В то время была известна лишь лейденская банка, 
у которой импульсы очень быстрые. А тут начали быть видны постоянные 
длительные эффекты. Не сразу считали даже, что это был ток. Сначала 
считали, что это просто разность потенциалов, видели искры на 
контактах, видели, что искры не прекращаются. Потом видели даже 
вольтову дугу -- мощный разряд, который относительно постоянен. Дальше 
при замыкании цепей с гальваническим элементом тоже виделись, 
наблюдался тепловой эффект. Но в таком физическом анализе мешало 
отсутствие количественной характеристики. Надо отметить, что другие 
изменения имеют место, но сейчас в связи с электролизом вспоминают 
законы Фарадея, но эти законы могли возникнуть только после силы тока 
и сравнения с химическим действием. Но химические эффекты, как 
выделение газов, наблюдались. 

Далее было магнитное действие тока. Эрстед обратил внимание, что когда 
цепь включалась, магнитная стрелка изменяла положение. Сам Эрстед не 
очень преуспел в количественном измерении, видел как от расстояния 
зависит, от интенсивности тока. А потом Био и Савар измерили, увидели 
зависимость 1/r. Потом поняли, что можно свернуть в катушку и получить 
магнит 1810-е Араго. Потом изучались взаимодействующие токи -- Андре 
Мари Ампер 1820--26 годы. Физические перегородки между имевшимися 
науками тогда немного ломались. Это было появление общего 
электромагнетизма. Именно Ампер привел к этому, и годы -- его трудов на 
эту тему. Ампер говорил, что нет отдельно электрических и отдельно 
магнитных вещей. По поводу постоянных магнитов он говорил, что 
обусловлены они молекулярными токами. Тогда атомизм и все такое вообще 
были очень популярны. Здесь помимо всеобщности идеи связи электричества 
и магнетизма, была идея о силе взаимодействия токов. Силы считались 
центральными и дальнодействующими. У Ампера силы действовали центрально, 
пропорциональность была 1/r2, входила сила тока, кроме того Ампер вводил 
туда функцию от углов. Более того, если проинтегрировать по формуле его, 
то совпадение с нашей в общем-то будет. Еще вообще Ампер говорил 
о движении там в токе, но он не рассматривал и не мог заряды.

Затем вскоре было открытие закона Ома -- Георг Симон Ом 1825--1827. 
Особенным был метод Ома. Метод был такой же как у Кулона, с крутильными 
весами. Подключалось это дело к гальваническому кому? хз. Как можно 
проверить закон Ома, как Ом получил вообще? Он брал нормированный 
источник тока, термопару, помещал два конца в таящий лед и кипящую воду 
-- строго сто градусов -- и соединял в такие батареи если хотел. Достиг 
высокой точности, также были введены на основе его методов измерения, 
были введены количественные характеристики -- сила тока и напряжение. 
Теперь вот не хватает закона электромагнитной индукции для нашей 
современной почти что версии. Нашелся потом Фарадей, 1831, его опыты не 
были особо количественными, были качественными. А вот в 1845 году Франц 
Нейман смог уже записать в математическом виде, записал закон. Фарадей 
еще и не смог потому что не очень был силен в математике... не имел 
естественнонаучного образования. В изначальном законе говорилось, что 
изменения потока приводят к появлению тока (и писал еще и сопротивление 
туда, а мы сейчас говорим об ЭДС). Таким образом возникали 
квазистационарные токи и их основные законы.

\subsection{Электродинамика Гаусса-Вебера}

Дальше электродинамика Гаусса Вебера. У Максвелла природа тока особо не 
рассматривается, не говорится, что ток есть движение зарядов. А вот 
Вильгельм Вебер говорил, что есть в проводнике заряды плюс и минус и они 
движутся и проводник остается нейтральным, но имеет движение зарядов. 
Имеется плотность тока, которую можно посчитать в силу тока, в законы 
Ома, Ампера, Био-Савара. А с микроскопической точки зрения говорилось, 
что ток есть произведение заряда на концентрацию на среднюю скорость. 
Еще одно принципиальное отличие -- чтобы взаимодействия зарядов были 
более Ньютоновскими, входила в это скорость относительная. Выражение для 
силы взаимодействия зарядов объединяло у Вебера как закон Кулона для 
неподвижных зарядов, а потом еще содержало скорость для описания 
движения. У Вебера была зависимость силы от скорости. И дальше 
зависимость была даже от ускорения, дающей закон Фарадея. Единственное 
что мы имели вдруг константу а. И вдруг эта константа оказалась 
скоростью света. Удалось поставить опыт Вебера и Колерауша, если по цепи 
протекает заряд ..... И по результату эксперимента была измерена 
константа а, возникла скорость света. И тогда же появилась система 
единиц Гаусса.

% }}}

\section{Развитие оптики до середины 19 века}
% {{{

Довольно много было открытий в 17 веке. В геометрической оптике не было 
никакой скорости света. Олаф Ремер смог в астрономических наблюдениях 
увидел скорость света в 1675. Он наблюдал один из Галлилеевых спутников 
Юпитера через диск. Должен быть по идее строго периодический процесс, он 
наблюдал что эти процессы чуть модулируются, зависит от того 
приближается Земля к Юпитеру или удаляется. По радиусу орбиты Земли он 
смог посчитать это дело. Ньютон в своих трудах говорил, что сначала 
считал, что свет это колебания эфира (как считал Декарт), а потом 
все-таки считал, что это частицы, распространяющиеся как раз со 
скоростью света той. Потом Ньютон открыл интерференцию, кольца Ньютона. 
Дальше была дифракция -- Гримальди -- она представляла собой расплывание 
границы объекта от точечного источника. Ньютон говорил, что это 
взаимодействия частиц с телом. Ньютон пытался развить оптику с учетом 
цвета, а потом при разговоре о Гюйгенсе и Гуке, у них конечно было 
волновое представление, но не как у нас волны. У них они были 
бесцветные, какие-то потоки солитонов, как мы бы сказали.

В 18 веке было тоже несколько вещей. Аберрация света звезд. Фотометрия 
появилась к концу 18 века. Джеймс Бредли измерил аберрацию света завезд 
и тут получил немного уточнил скорость света. Дальше Юнг и Френель уже 
составляют более такую волновую оптику, говорят о зависимости цвета от 
частоты. Аналогия со звуковыми волнами была у Эйлера. Видели что длина 
волны маленькая у видимого света конечно, но вот можно было растянуть 
так сказать, это Юнг. Говорили еще что применение теории упругости 
к оптике, но тогда совершенно не было теории упругости, были отдельные 
задачи, которые были решены не очень наверно обоснованным методом, и тем 
более нужны были дифференциальные уравнения в частных производных, но не 
было такого тогда, в самом начала 1800 19 века. Была аналогия со звуком, 
а это продольные волны. Потом из двойного лучепреломления, исландского 
шпата, который видел еще Гюйгенс, вдруг увидели, что есть поляризация. 
Этот камень служил каким-то индикатором этого дела. Говорил Лаплас, что 
у света есть стороны. У продольных волн такого быть не могло. Это было 
основной проблемой. Чисто сложилось так, что мы используем принцип 
Гюйгенса-Френеля, а не Юнга. Теория дифракции Френеля в 1815, 1818 
годах. Юнг потом сказал о поперечности колебаний, потом просоединился 
и Френель. Так выгодно удобно получилось еще потому что интерференция 
в разных поляризациях не пересекается. Потом были два цикла работ 
Френеля. Это теория дифракции на плоской границе (стекло, вода, воздух), 
учитывались базовые соотношения, что поперечные волны, что конечная 
скорость, упругость среды разная, а плотность одинаковая (почему не 
наоборот -- нипочему). Тогда еще не было последовательной теории 
волновой, по сути дела Френель не пользуется волновыми четкими 
уравнениями. Грин только уже писал все математически через волновые дела 
в одной среде, в другой, потом граничные условия. Это он там сформировал 
естественные граничные условия. Эксперимент вот показал, что Грин неправ 
однако. У Френеля было нехорошо -- странные граничные условия, возникали 
фазы, направления движения, приходилось допускать разрыва нормальной 
компоненты смещения среды. Появление электромагнитной теории света 
ничего не поменяло тогда. И вот второй труд -- кристаллооптика. Френель 
показал, что возможен аж даже двухосный кристалл, с особыми точками, 
конической дифракции. Потом была формула частичного увлечения эфира 
движущейся средой. Скорость света относительно движущейся среды, формула 
как при релятивистском сложении скоростей. На самом деле была большой 
проблемой стыковки теории Френеля с теорией упругости механической. 
Здесь появились знаменитые уравнения Навье, а потом Коши еще. Оказалось, 
что для упругой среды надо вводить два коэффициента, для продольной 
и поперечной. Выключить продольные волны не удавалось вообще, помогла 
модель Мак Куллаха, разрешавшая объемное сжатие эфира. Утверждение 
волнового подхода в начале 18 века было совершенно особенным событием, 
которое произвело особое впечатление. Поэтому период называют революцией 
в оптике.

% }}}

\section{Концепция электромагнитного поля Фарадея-Максвелла}
% {{{

\hfill \textbf{Mar 24}

В какой-то мере это была эпоха восстановления кортезианства, идей 
Декарта. Это связано с появлением адекватного математического метода. 
Утверждаются методы доказательства теорем, которых не было раньше. Мат. 
анализ появился в результате действия Ньютона и Лейбница, но в начале 
19 века стал утверждаться. Также пояляется анализ функций многих 
переменных. Частные производные конечно появляются в 18 веке, но там 
много противоречий, в теории Эйлера. Но вот начиная с 20-х годов 19 
века различные интегральные теоремы возникают, утверждаются, 
осмысливаются различные дифференциальные формы, дифф. операции. Мы 
сейчас используем ротор, градиент, дивергенция, но они появляются тут, 
хоть и обозначения в конце 19 века лишь. Сначала не писали этих роторов 
итд, писали сразу выражение их в частные производные. Это еще и привело 
к развитию концепции электромагнитного поля. Идеи Декарта включали 
в себя несколько важнейших положений, объяснение всех кого-то через 
движения, и теория близкодействия. Волновая оптика приводит 
к гидродинамике и прочим теориям континуума. Тогда было принято не 
просто записать систему, а дать аналогию из макроскопического дела. 
Нужно было быть похожим на понятную макро среду. В оптике по этой 
причине возникали разные проблемы, например касательно продольных 
колебаний. Пытались убрать продольную компоненту не только в уравнении, 
но и придумать модель, среду, где оно так получается. В тех ранних 
гидродинамике, теории упругости, были разные несостыковки. Неделю назад 
мы рассматривали Гаусса-Вебера, дальнодействующая вещь, но объяснить все 
бывшее тогда можно было и другим способом, ввести векторный потенциал. 
Работая после уже Вебера и Гаусса, в 1850-е, он записывал потенциалы 
и записывал поля, в Германии даже рано довольно стали использовать это 
все. Сама концепция электромагнитного поля не случилась бы без Майкла 
Фарадея. В начале 1 половины 19 века. Особенность Фарадея была такая 
природная интуиция, он проводил очень много экспериментальных 
исследований, но математикой особо не пользовался и не записывал 
математические законы потом. Он описывал полностью точно все свои 
эксперименты, и вдохновленный его идеями Максвелл смог все переписать 
в математику. Сначала у Фарадея был Хемфри Деви руководителем, он 
начинал с исследований природы разных видов электричества, и Фарадей уже 
тогда приходит к выводу, что все виды электричества имеют единую 
природу. Напомним, что в конце 18 века все электричество по сути было 
лишь от трения. После этого возникает еще и гальваническое 
электричество, трения там нет, относительная устойчивость есть. Затем 
появляются термопары, тоже следствие идей Алессандро Вольта. Потом 
пьезоэлектричество. В общем Фарадей в своей экспериментальной 
деятельности доказывал единство всех этих видов. Вторая идея была 
гораздо ближе к концепции э-м поля. Важным является близкодействие, от 
точки к точке, и среда влияет сильно. У Фарадея как раз появляется идея 
этих констант, электрическая и магнитная проницаемость. Закон э-м 
индукции также качественно если есть ток, гальванический ток, то в среде 
возникает электронапряженное электротоническое состояние пространства. 
И когда состояние меняется, в пространстве возникает электричество, то 
есть напряженность. То есть ЭДС индукции возникает прям в пустом 
пространстве, не нужен никакой провод, провод только как индикатор. Само 
пространство изменяет свое состояние. Эля электрического поля Фарадей 
вводит понятие силовых линий, линий сил. Многочисленные исследования 
Фарадея по химическому действию тока. Электролиз, зависимость между 
величиной тока и интенсивностью реакции. Эти законы Фарадея конечно 
важны, но также важно что тогда развивалась химия неплохо, появлялась 
идея состава и прочее. Много возможностей уже было состав установить. 
Понятие тока, идеи влияния электричества, возникают в 30-е годы. Но надо 
сказать осторожно относились коллеги ученые к Фарадею. Конечно 
эксперименты играли решающую роль в развитии, но  его обобщения, причем 
в отсутствие формул, немного отталкивали. Фарадей пошел дальше, он после 
уже своей деятельности стал формулировать идею, что есть единая основа 
--- эфир --- который проявляет себя как электрическое и магнитное поле, 
и это тот же самый эфир, что и был в оптике. Прямых доказательств он не 
получил, но четко высказывал эту мысль. Однако косвенное доказательство 
единства электричества и оптики он получил --- эффект Фарадея. 
Говорилось, что если есть какая-то среда (сейчас мы скажем с отсутствием 
зеркальной симметрии), то если эту систему погрузить в магнитное поле, 
то если пропустить поляризационный свет через это дело, будет вращаться 
плоскость поляризации. Поворот видимо пропорционален концентрации, 
пропорционален величине магнитного поля. В этом Фарадей и видел связь 
электромагнитных и оптических явлений. Открытие эффекта Фарадея конечно 
имело большое значение.

\subsection{Деятельность Максвелла}

Максвелл, Джеймс Клерк --- 1831--1879 --- за короткую жизнь успел 
создать основу двух разделов классической физики, которые сейчас входят 
во все курсы теоретической физики --- электродинамика 
и молекулярно-кинетическая теория в таком развитом виде. В начальном 
виде можно и других вспомнить, но в молекулярной кинетике конечно 
Максвелл сыграл ключевую роль. Но сейчас говорим об электродинамике 
Максвелла. В 1855-56 годах опубликовал несколько статей, собранных 
в О Фарадеевых силовых линиях. Название о многом уже говорит, на основе 
идей Фарадея, у которого были локальные идеи объяснения электрических 
и магнитных явлений, Максвелл пытается записать все это в виде систем 
уравнений дифференциальных. Эту эпоху можно считать таким методическим 
трудом. Конечно принципиально нового ничего не давали сразу уравнения 
Максвелла. Максвелл записал векторный потенциал -- вектор 
электротонического состояния. Туда входила электростатика -- див Д равно 
4пи ро. Появился вектор электрического смещения, смещения эфира 
считалось тогда. Закон Био-Савара был записан в статике, без тока 
смещения. В общем по сути новых уравнений не было, была новая запись. 
Дальше в 1861-62 О физических силовых линиях. Появляются вдруг 
электромагнитные волны, это уже не лишь формализация и переписывание 
известного. Четкой системы Максвелл не пишет, причем нет даже волнового 
уравнения нет, хоть он и утверждает, что есть вот такое. Чем отличался 
подход Максвелла от нашего? У него были конечно те несколько соотношений 
как див Д, рот Н. Но у Максвелла для плотности тока сразу записан 
дифференциальный закон Ома, j = lam E. Мы конечно сейчас пользуемся 
записью как у Вебера, где j ~ v rho. Принципиальным у Максвелла было 
следующее. Конечно связать эфир с полем было уже до, а Максвелл говорит, 
что надо еще записать ток смещения, он писал через производную Е, а не 
Д, ну они тогда были пропорциональны. Но это уже следствие, а не 
причина. Самая главная была идея движения эфира, есть ток в проводниках, 
есть ток в пустом пространстве даже, смещение, и это было от эфира. Как 
только эта модификация появилась, у Максвелла возникли другие уравнения. 
В пустоте у него сразу получились поперечные волны, движущиеся со 
скоростью света. За 4 года до этого, в 1857 году, Вебер и Кольрауш 
измерили скорость этих  волн, и она оказалась равной примерно скорости 
света в пустоте. Максвелл еще говорил, что скорость в среде там чуть 
другая из-за зависимости от среды. Возникает первый способ проверить 
это. В этой серии работ Максвелл дает уже единую интерпретацию движений 
эфира. Раньше было только див Д объяснено, а теперь еще и рот Н. Заряды 
иеняются, электрические поля меняются, эфир смещяется вдоль провода, 
смещение идет поперек, и появляется идея появления этой электромагнитной 
волны. Блыо два отрицательных момента. Непоятно было как движетя эфир 
вообще, не было же продольных волн, не было сжимаемости, главная роль 
тока в том что среда увлекается током, появляется вихревое движение. 
У этого всего чуда не было механических аналогий ни с чем вообще таким. 
Тогда, вспомним, не было достаточно лишь уравнений написать, нужно было 
дать на пальцах аналогию. Дальше Максвелл даже не пытается приводить 
механические аналогии, оно и понятно, ведь есть рассуждения про свет, 
есть уже уравнения, и пусть... Сопутствующую систему координат конечно 
эфиру не построишь, но о движении эфира говорилось. В 1864 году возникла 
Динамическая теория э-м поля. У Максвелла в саму систему входили 
зависимости между В и Н, между Д и Е. В он считал моментом количества 
движения среды, Н считал вращением эфира. Д это смещение, Е локальная 
сила так сказать, возникал закон Гука... Элементом системы к тому же 
является связь j и Е через лямбду. Короче он писал j full = lambda 
E + d/dt D. Вот такое конечно не очень хорошо согласуется с законами 
разными сохранения. Лямбда больше нуля говорит, что не может быть 
никаких источников энергии, вектора Пойтинга, могут быть только стоки. 
А при лямбда равном нулю разрешены волны. И нет нигде ситуации, где были 
бы источники поля. Но понимание таких глубоких вещей как сохранение 
энергии и прочее появилось лишь в конце 19 века, так что пока что все 
было замечательно.

\hfill\textbf{Mar 31}

Уравнения в принципе линейные и могло получиться лишь поглощение волн 
в проводящих средах и распространение в средах где лямбда равно нулю. 
Также преобразование полей Максвелл приносит туда тоже включал.

Экспериментальное обоснование теории Максвелла. Измерение 
диэлектрической проницаемости. Также можно было бы дисперсию, но про нее 
было известно гораздо меньше, поэтому проницаемость. Больцман 
в 1872--1874 годах делал это, Зилов в 1875 для жидкостей. Особенностью 
было что нет необходимости экспериментальной в новой теории. Была лишь 
идея вот этого флюида, который утвердился в оптике, хотелось перевести 
на электромагнетизм. Национальное явление тоже было такое, в Англии 
к Максвеллу относились хорошо, не обсуждали, приняли сразу как важную. 
Через несколько лет появились работы по улучшению Максвелла, Пойтинг, 
Хевисайд. В Германии было немного по-другому, потому что был Вебер, 
и необходимости пересматривать не было, на Максвелла не обращали даже 
внимания. Только потом Гельмгольц строит свои теории, говорит, что можно 
прийти к подобным Максвеллу уравнениям через теорию потенциалов, теорию 
сплошной среды. Но у Гельмгольца была небольшая разница, ток смещения 
включал лишь только поляризованность, а не всю Д электрическую. Такие 
уравнения приводили к отсутствию волн в пустоте, только в среде могли 
быть. Далее был Герц в 1888--1889 годах, он проводил знаменитые 
эксперименты, но кроме того он улучшил теорию. Он вернул то что было 
у Вебера про движение зарядов (в отличие от закона Ома у Максвелла). 
Движение зарядов дает ток, но заряды мы не видим. На основе этого он 
приходит к идее открытого колебательного контура. Это и был такой вот 
первый дипольный излучатель. Перекрытие областей Е и Н. Тут возможно 
излучение наружу. Затем Лоренц и другие теоретики это дело развили, 
и к концу 19 века формируется теория дипольного излучения. Возникает 
немного путаница, классическая теория излучения у нас означает 
неквантовая, теория такая развивалась во второй половине 20 века. Потом 
был Лебедев наш МГУ, он провел опыты как у Герца, но с длнами волн 6 мм 
вместо 50 см.

Последующее развитие теории обусловлено Хевисайдов, Пойтингом, Томсоном. 
Томсон рассматривал движение заряженного шара, там энергия получалась 
нелинейно зависящей от квадрата скорости, тогда говорили что это так 
масса эффективная меняется.

% }}}

\section{Молекулярно-кинетическая теория, термодинамика, статфизика}
% {{{

История формирования молекулярно-кинетической теории, термодинамики, 
стат. физики. В начале 19 века атомизм из идеи натур-философской 
превращается в идею естественно научную, и идея подкреплена 
экспериментами, появляются какие-то параметры этих атомов. Появляются 
свидетельства существования атомов и молекул.

Эти дела шли из физики и химии. В химии был Джон Дальтон. Он по сути был 
основателем теоретической химии. Появился закон кратных весовых 
отношений. Дальтон проводит серию исследований, позволяющих обобщить 
кратности. Неделящиеся вещи могут объединяться, базовый пример H и Cl. 
На основе этих отношений Дальтон пытается ввести массы молекул. Дальше 
Гей-Люссак кроме весовых отношений говорит, что есть и объемные 
отношения. Там с H Cl возникла проблема, потому что из 1 водорода 
и 1 хлора получалось 2 HCl. Тогда не писали Н2 никогда, потому что 
одинаковые, только разные считалось можно так вместе. В игру вступил 
Амадео Авогадро и сказал, что есть двухатомные газы несмотря на 
химическое сродство. На основе этого Дальтон продолжает свои массовые 
вычисления. Дальше стал известен химический состав воздуха, нашлись 
азот, кислород, ..., и нашлась роль воды, водяных паров. У Ломоносова 
например не было возможностей связать тепловые свойства с давлением, 
вращательное и поступательное соответственно. В начале 19 века ситуация 
меняется, измеряют теплоемкости газов, видят, что для разных процессов 
разные значения. Тогда же вычисляли молярные веса на основе молярных 
масс от Дальтона. Обнаружилось еще что теплоемкость 3R у много кого, 
теплоемкость твердых тел тоже имела отношение, было странно. При всем 
этом теплота считалась тогда теплородом, тем самым флюидом, и Дальтон 
например был сторонником этого. В 1830--1840 годы только возрождается 
теория тепла кинетическая. Бенждамин Томсон видел что при сверлении 
выделяется тепло, говорили выталкивается теплород... Потом Хемфри Деви 
видел, что при трении льда о лед тоже. Но люди не хотели отказываться от 
идеи теплорода даже при тех делах, это 1799 годы. Потом возник Фурье, 
который сказал, что развита уже так сильно теория тепла, что можно 
формулировать полную математику. Но он основывался снова на теплороде... 
Он говорил поток тепла минус градиент температуры. Развиваются тогда уже 
дифференциальные формы, дифференциальные конструкции, функции многих 
переменных. Фурье успешно эту линию продолжает, вводит 
тригонометрические ряды Фурье и решает уравнения. Это 1822 и 1830 годы. 
Проводилась тогда тоже аналогия с электропроводностью. В Англии в 18 
веке понемногу появляются машины разные, их надо двигать, создали чисто 
инженерное решение тепловых машин. За 18 век они получили большое 
распространение. А в 1783 году Лазарь Карно, отец Николя Сади Карно, 
обращает внимание на исключительно полезное понятие работы. К концу 18 
века в механике все было другим. У Лагранжа не было выделения интегралов 
движения как энергия. Из практики, из инженерного дела пришли понятия 
работы и энергии и теплоты...? Пришли математические методы оценки 
эффективности этих машин. В начале 19 века также был новый период 
развития физики. Это такое возвратное движение, идея Декарта, нужно 
объяснять все движения кинетически, из какого-то вещества, эфира видимо. 
Возникла первая теория в 20-е годы, это оптика. Не могла появиться 
раньше, ибо не было математики достаточной. Были какие-то противоречия, 
но на втором плане. Дальше было установление каких-то связей между 
разделами физики: электромагнетизм, намечается деятельность Фарадея, 
возникает идея э-м поля; еще возникает идея, что тепло может переходить 
в другие виды энергии. Можно вспомнить и философские концепции этого 
времени, пожалуй здесь они не такое влияние большое оказывали.

\subsection{Майер, Гельмгольц, Джоуль}

Вот тут утвеждался постепенно закон сохранения энергии. Повлияло на 
современную молекулярно-кин теорию и стат физику. Часто говорят, что 
основной вклад принадлежал аж трем ученым, из разных стран. Майер 1841, 
Гельмгольц 1847, Джоуль 1841--43. Они были именно физиками, физическая 
связь должна быть. Были именно обобщения эксперимента. Роберт Юлий Майер 
отличался тем, что был по образованию врачом. Гельмгольц был врачом 
тоже, было популярно получать медицинского образования. Но Майер был 
изначально далек от физики, математики. Он был скорее практикующим 
врачом, судовой врач, во время плаванья судов он использовал популярный 
метод лечения больных, метод кровопускания. В процессе кровопускания он 
наблюдал что на севере и юге кровь разная получается... Можно было 
связать это качественно с энергетическим балансом в организме. Другая 
предпосылка. Майер не ставил опытов, но прочитал много литературы. 
Во-первых удалось найти состав воздуха, и найти удельные и молярные 
теплоемкости каждой компоненты (это делали во Франции). Майер 
познакомился с их работами и смог дать определенную интерпретацию 
измеренным цП и цВ. Майер обратил внимание, что разность между ними 
всегда постоянная. Он объяснил это так, что часть теплорода идет на 
совершение работы при постоянном давлении. Майер смог дать 
количественный механический что-то тепла. Работа тогда измерялась 
в чем-то странном типа кг*м. Тепло и теплоемкость вот было калории на 
моль на градус. Получается, что единица измерения, количество теплорода, 
калория оказалась однозначно связана из соотношения Майера с кг*м.

Дальше был немецкий физик и врач честно говоря, занимался проблемами 
зрения, Герман фон Гельмгольц. Он имел очень фундаментальную физическую 
и математическую подготовку, это позволило ему выдвинуться на первые 
места немецкой физики. Началось все в 1840-е. В 1847 году он опубликовал 
труд о сохранении силы. Там была и теоретическая физика, и такие 
философские обобщения. Тогда была популярна идея витализма такого. 
Энгельс например занимался критикой витализма. Это было что не только 
законы управляют деятельностью организма, была еще живая сила. Но 
Гельмгольц говорил, что нет живой силы, была бы -- был бы вечный 
двигатель. Гельмгольц возрождает идею микромодель. Рудольф Клаузиус там 
дальше еще был. Но идею необходимости атомов, молекул и такого для 
вещества. Тогда еще были проблемы с терминологией, не называли ничего 
энергией, но Гельмгольц говорил, что когда все зависит лишь от 
координат, то сохраняется та какая-то сила, которая на самом деле 
энергия. Дальше у него был анализ баланса в цепях, конденсатор, 
лейденская банка. Тогда конечно не было энергии конденсатора формулы, не 
считалась энергия такой уж важной. Формула энергии конденсатора впервые 
появилась у Гельмгольца.

И третий вот человек это Роберт Джоуль. Он не был профессиональным 
физиком, но был бизнесменом, пивоваром. Он отметился даже чуть раньше 
чем Гельмголц, тоже баланс энергии в тепловых цепях. Закон Джоуля-Ленца 
только у нас называют Ленца, а везде он просто Джоуля. Джоуль в начале 
40-х сказал, что если есть проводник, подчиняющийся закону Ома, то 
выделяется еще и тепло. До этого не наблюдалось потока тепла в проводах, 
а вот при подключении резистора откуда-то что-то бралось. Джоуль смог 
написать формулу с дать ей интерпретацию. Джоуль говорил, что 
в резисторе тепло прям рождается, ниоткуда не притекает, не баланс, не 
теплород. Тогда еще можно было получать ток механически, перемещениями, 
магнита например. Механическая работа давала поле, а поле рассеивало 
энергию в резисторе. Появился в общем вывод, что теплород не 
сохраняется, тепло появляется из других видов энергии. Но точность 
честно говоря хромала. Потом он познакомился с газами и повторял 
эксперименты и пришел к похожим Майеру выводам. Но критиковал Майера (и 
он довольно тяжко принимал к сердцу эту критику). Джоуль конечно был 
неправ... Научное сообщество тогда с какой-то готовностью восприняло 
отсутствие баланса теплорода, тогда уже нависала такая вещь как типа 
энергия и все такое.

\subsection{Термодинамика}

Дальше в 1848--1854 годы возникала термодинамика. В. Томсон (то самый 
лорд Кельвин). Тогда было видно после падения теплорода, что нужно 
вводить другое, энергию. Сади Карно был сторонником теплорода, говорил, 
что тепловая машина похожа на мельницу или водяное дело. Он написал 
формулу КПД эта = альфа(т1) * (т1 - т2), эти т были в градусах, не 
Кельвинах, не было Кельвина еще 20 лет до него. Основы термодинамики 
сформировались к 1854 почти что в современном понимании. Но конечно 
третьего начала ТД еще не было очень долго. Формируется закон сохранения 
энергии, представление о превращении тепла. Вот потом Томсон со своим 
Кельвином позволил написать альфа(т) = 1/Т. Рудольф Клаузиус. 
Особенность тогда была что не было обращения к деталям теплоты, но уже 
говорили, что это механическая теория теплоты. Уже начинали считать 
скрытым движением. Говорили еще что не знаем что это за скрытое 
движение, ну и хорошо, не будем трогать. Томсон же пошел дальше. Он 
сказал, что есть изолированная система, система стремится к тепловому 
равновесию, система сама по себе стремится лишь к равновесию, обратно 
никак. Понятие необратимости впервые было сформулировано Томсоном. Важно 
еще что Томсон был глубоким исследователем, говорил, что если 
рассматриваем микроскопические модели (Максвелл вместе с ним тоже 
собственно делал такие мысли), говорилось что если хотим объяснять 
микроскопически, то либо у нас круг нерешаемый, либо у микро вещей нет 
никакого трения, нет диссипации. При этом Томсон видел, что механика без 
трения вот такая имеет большой изъян, там нет стрелы времени, нет 
необратимости. Вот это противоречие впервые подчеркнул именно Томсон.

\subsection{Молекулярно-кинетическая теория}

В 18 веке уже появлялась чуть-чуть, но другим вещам служила так сказать. 
Пытались вывести закон Бойля. Потом были такие картезианцы, пытающиеся 
вывести гравитацию из такой кинетической теории. Говорилось 
о микрочастицах, летящих между телами и влияющими на них. Формулировали 
с заявляли это все очень легко потому что не знали о проблемах 
гидродинамики. Понемногу появилась и потребность газы объяснять. 
Гельмгольц конечно был, но какие-то небольшие такие. Август Крениг 
пытался вывести уравнения какие-то исходя из молекул вот таких, но 
Клаузиус видел это как-то скептически, потому что не знал никто никакие 
массы, концентрации и прочее. Клаузиус еще увидел у Кренига там p = 1/6 
NMv2/V, ошибку, ибо не 1/6, a 1/3, ибо частица отражается от стены, а не 
поглощается. Вот и Клаузиус после всего этого опубликовал в 1850-е годы. 
Еще возникают идеи применения теории вероятности 
к молекулярно-кинетической теории. Теорию вероятности вообще применяли 
давно к играм, потом стали к машинам (стала как современная статистика), 
потом Гаусс много что сделал, построил науку как более-менее 
последовательную, потом показал как можно рассматривать вероятности 
непрерывных вещей. Теория Гаусса привлекла внимание астрономов тогда, 
были погрешности измерения вот дальних тел, хотелось избавиться как-то 
от нее. Появляется теория ошибок, на основании работ Гаусса. А потом аж 
Максвелл использовал функцию ошибок. А в те 50-е годы Клаузиус размышлял 
таким образом, что надо вводить средние величины, есть среднее значение 
с, а отклонение уменьшается как примерно 1/sqrt(N). Вот со средним 
квадратом скорости возникла еще такая история, можно было измерить на 
основе той формулы давления, посчитав общую плотность среды получил 
скорости в несколько сотен м/с. И Клаузиус сказал еще тогда про длину 
свободного пробега, потому что например распространение дыма от трубки 
было медленным... Это парадокс Бейс-Баллота.

\hfill \textbf{Apr 7}

Возникновение в середине 19 века. Вторая половина 19 века это уже близко 
к формированию современной физики, например мы говорили уже об этом 
периоде при развитии электродинамики. Начало мол кин теории -- вторая 
половина 1850-х годов, когда Рудольф Клаузиус публикует работы по 
молекулярной кинетике. Сам Клаузиус считал, что не зная ничего 
о молекулах нельзя так вообще судить, но там опубликовал Август Крениг, 
и Клаузиус должен был ему отвечать. Мол кин теория послужила 
микроскопической опорой макро термодинамики. Была дискуссия 
с Бейс-Баллотом про тот дым и длину свободного пробега. Потом в 1860 
годы и после вириал и другие дела.

\subsubsection{Максвелл (тот же самый)}

Потом были исследования Максвелла в 1860-е годы. В 1859 году 
опубликовано распределение по скоростям, где он говорил, что скорости 
распределены по Гауссу с дисперсией пропорционально температуре. 
Оптическая астрономия достигла больших высот, нужно было определять 
координаты, а там погрешности во многом от воздуха, и там применять 
стали функции ошибок. Максвелл первым глубоко понял особенности 
применения теории вероятностей в мол кин теории. Особая роль температуры 
в таком кинетическом подходе в том, что она определяет дисперсию. Это по 
сравнению с Клаузиусом примерно говоря... Судьба распределения Максвелла 
немного не самая удачная. Откуда берется функция Максвелла? Считаем 
независимыми скорости по проекциям, отсюда получаются эти распределения. 
Но эта гипотеза ниоткуда не следует, Максвелл почти что догадался. 
Интуитивно приходит к ответу, а потом пытается доказать его 
актуальность. Вообще говоря вывод такой вот неправильный. 
В релятивистском случае не сохраняется аддитивность энергии по отношению 
к проекциям скоростей. Максвелл это понимал (?), и в 1860 годы работал 
над Динамической теорией газов 1866, где обсуждались что из себя 
представляет равновесие идеального газа, концепция столкновений, но куда 
более обстоятельно с теоретической стороны. Максвелл сделал гениальный 
ход -- расчеты упрощаются сильно, если сила пропорциональна на большом 
расстоянии пропорциональна r в $-5$ степени. Такой газ будет себя вести 
примерно как другой газ, принципиально то же самое. Там используется 
другая гипотеза -- не что проекции независимы, а что возникает такой 
динамический хаос. Максвелл сказал, что двухчастичная функция 
распределения распадается на произведение одночастичных. Это и есть 
независимые случайные величины, две скорости двух разных частиц, которые 
потом и сталкиваются. В таком виде даже сейчас не получается вывести. 
Получаются у него вот такие формулы, это с штрихами до и после, после 
тоже независимы.
$$ f_2(v_1, v_2) = f(v_1)f(v_2) \qquad f(v_1)f(v_2) = f(v_1')f(v_2') $$
Аналитический труд теплопроводности Фурье. Максвелл достаточно строго 
работал с теплопроводностью в равновесном состоянии, коэф 
теплопроводности получается. И из этой теории получается, что 
коэффициенты теплопроводности и вязкости связаны через температуру 
корень из нее, но не через концентрацию и давление. Потом это все 
подхватил Больцман, но и в теории Максвелла, и потом у Больцмана было 
много парадоксов, внутренних противоречий. Кроме того строго говоря эти 
все рассуждения не принесли ничего нового. Критическое отношение к мол 
кин теории вплоть до 18 19 веков, то есть 30 лет примерно, было большим. 
Отвечать на критику было сложно. Большим плюсом было что удалось 
объяснить ту самую независимость.

Томсон (лорд Кельвин) обратил внимание на принципиальную необратимость 
термодинамики и принципиальную обратимость механики. Причем трение 
в механике нельзя было ввести, чтобы не было противоречий. Потом 
обратили внимание на то, что релаксацию можно рассматривать с точки 
зрения теории вероятности. Переход от менее вероятному к более 
вероятному и есть релаксация. Конечно это было на базовом уровне таком. 
Потом Томсон в Дискуссии с Максвеллом уже пришли они к таким демонам 
Максвелла, который мог бы нарушить 2 начало термодинамики.

\subsubsection{Людвиг Больцман}

Потом был Людвиг Больцман (1844--1906), он создал уже современную 
картину. Объяснение теплового движение как квазипериодического движения 
-- очень крутая работа его, недооценена везде. В обычной механике есть 
адиабатический инвариант, этого понятия не было в принципе в это время 
в физике. В этой как раз студенческой работе Больцмана появляется 
понятие адиабатического инварианта! Максвелл хотел дойти до сути 
и принципиально глядел туда, одновременно пытаясь упростить все что 
можно, вот так появились те Максвелловские молекулы. Больцману не 
понравилось, что это такой вот частный случай, разреженный газ. Кроме 
того даже если газ идеальный, в присутствие неоднородности внешней 
воздействия, газ становится неоднородным. Больцман хотел это все более 
строго провернуть. Труды примерно в основном с 1868 до 1870. Больцман 
показал, что в равновесии (даже для многоатомных молекул) будет 
определенный баланс энергии. Он потихоньку начинает обобщать состояние 
равновесия для таких сложных конструкций. В изолированных системах есть 
микроканоническое распределение, Больцман называл эргодическим 
распределением. Там же получается эргодическая гипотеза -- среднее по 
времени, а у Больцмана возникает понятие ансамбля (одинаковые системы 
с разными начальными условиями). Вот такой ансамбль появляется как раз 
в работах Больцмана. И Максвелл замечает это понятие и приветствует его. 
Идея сначала столкновения многоатомных молекул, а потом введение 
ансамблей позволили описать стат равновесие для твердых тел, жидкостей 
итд. Канонический ансамбль Гиббса появляется как раз тут... Ансамбль... 
Ансамбль... Систему молекул можно было описать у Больцмана как систему 
точек, у которых там внутри что-то тоже какие-то взаимодействия. 
В равновесии если функция обладает большим числом переменных там 
усредняют по времени. Тут же появляются средние по ансамблю. Число 
Авогадро получил Планк, и Планк же в той же работе со спектром абс 
черного тела получил константу Больцмана. А у Больцмана была константа 
обратно пропорциональная абсолютной температуре. Результат из 
Больцмановского анализа был такой. Описание твердого тела, система 
точечных частиц, соединенных потенциальными силами, как пружинками. Они 
выполняют роль линейной части взаимодействия. Теорему 
о равнораспределении энергии... (...) Потом с твердым телом это система, 
которая разлагается на нормальные моды, которые в квантах появлялись 
в теории Дебая. И это был по сути метод Больцмана... И это довольно 
строгий метод, конечно там есть приближения типа малых отклонений, но 
вот так вот. И Больцман смог продемонстрировать правильность закона 
Дюлонга и Пти (уже был известен экспериментально). Большой минус в том, 
что вывод такой вот получился строгий, ничего кроме 3R быть не могло! 
При охлаждении газов и твердых тел наблюдалось изменение теплоемкости, 
а у Больцмана в теории не было такого вот. Почему у двухатомных газов 
5 степеней свободы конечно была загадка.

Далее появляется физическая кинетика -- первое кинетическое уравнение. 
Появляется время как аргумент функции распределения, у Максвелла не 
было, ибо было равновесие + усреднение. А Больцман обратился 
к релаксации, он пытался вывести 2 закон ТД. Появляется как раз там 
интеграл столкновений Больцмана, Н-теорема. Больцман конечно написал 
уравнение это, но оно такое сложное... Он ничего не смог сделать с ним. 
Максвеллу в то же время было известно, он осознавал, что необходимую 
роль играет теория вероятностей. Больцман считал, что это лишь 
приближение, и можно в точности вывести. Потом Больцман записал тот как 
раз функционал функции f, назвал его H и получил, что производная по 
времени от этой штуки меньше или равна нулю. И равна нулю именно 
в равновесии. И вот тут возникает парадокс Лошмидта. Если рассматриваем 
строгую механику без трения и изолированную систему, где только упругие 
столкновения со стенками, то будет эквивалентны замены t -> --t и r -> 
r, v -> --v. Так вот если мы поменяем знаки скорости, то вроде бы ничего 
не должно измениться, а вот время обращается. И тогда Больцман начинает 
понимать что дело в вероятностях. В 1877 году Больцман опубликовал ту 
самую знаменитую формулу $S \sim \mathrm{ln} W$, где W это вероятность 
макроскопического состояния. Тут Больцман понимает, что стат физика без 
теории вероятности никуда никак. Оказалось, что равновесие это то 
состояние, которое дает максимум вероятности. Число частиц фиксировано, 
и энергия тоже фиксирована (изолированная система). И сразу получается 
без анализа столкновений и прочего связь.

\subsubsection{Кризис в 1870--1890-е и дальше}

В 1870--90-е годы наступил кризис развития мол кин теории. Решить 
уравнение не получалось, ответ на парадокс Лошмидта был своеобразный (он 
просто перешел на другую систему аргументов...). Пуанкаре доказал, что 
существует конечное время, за которое система вернется 
в эпсилон-окрестность -- теорема возврата Пуанкаре. Ответ был, что время 
возврата очень большое, и вообще тут такие флуктуации. Потом были 
философские вопросы в плане тепловой смерти вселенной. Мах кроме того 
объявил все эти молекулы метафизическими.

Затем был Планк, затем в 1905 Эйнштейн. Больцман самоубился и не дожил 
до особого триумфа своей теории... Метод Гильберта-Чепмена-Энскога 
приближенное решение уравнения Больцмана. Там оказались не флуктуации, 
а локальные распределения Максвелла, а потом турбулентное или ламинарное 
уж как придется близость к вязкой гидродинамике.

% }}}

\section{Электронная теория Лоренца. Возникновение СТО}
% {{{

\hfill \textbf{Apr 14}

Теория относительности возникает в 1905 году.

\subsection{Возникновение электронной теории}

Деятельность Гендрика Антона Лоренца 1853--1928. Электрон был открыт 
в 1898 году, к теории приступил Лоренц в 1875 году. На этой стадии 
занятие Лоренца правильнее называть ионной теории. Это было продолжение 
теории Максвелла по электромагнитной природе света. В 1876 году опыт 
Роланда -- конвективный ток создается просто лишь перемещением 
микроскопического заряда, и все явления эквивалентны обычным токам 
в цепях. После этого пришлось признать микроскопическую природу тока. 
Благодаря Герцу удалось создать открытый колебательный контур, 
посмотреть на излучение. Но Лоренц поставил перед собой задачу показать 
как э-м теория света на этой новой теории уже с учетом более 
последовательной интерпретации природы тока, как можно объяснить 
оптичекие явления. Лоренц был первым теоретиком, основал кафедру 
теоретической физики в Лейденском университете (Нидерланды). Лоренц для 
объяснения оптических явлений проводил прием, характерный уже для 
современного изложения. В вакууме есть заряды, они являются источниками, 
а потом электродинамика в среде. Микроскопическое поле 
и макроскопическое поле, созданное многими зарядами итд, в усредненном 
виде. Было похоже на дела Максвелла и Больцмана при усреднении, тут так 
же с этими полями. Микро и макро поля, объяснение поляризации 
объясняется через смещение зарядов. Рудольф Клаузиус уже занимался 
поляризацией диэлектриков, вот для Лоренца тоже важно, но с точки зрения 
молекул и атомов. Затем возникает еще одна принципиально новая 
конструкция, это самосогласованное поле. Максвелл считал, что достаточно 
вводить уравнения для смещения и напряженности и для магнитных. 
У Лоренца принципиально другой подход, есть заряды, они движутся в поле, 
но сами по себе. Их движения меняются под силой Лоренца. То есть мы 
усредняем обе стороны, микро с обеих сторон от равно, получается сложная 
нелинейная система уравнений, самосогласованность нужно. Электродинамика 
Максвелла линейная, а у Лоренца вот нелинейная. В деятельности Лоренца 
можно смотреть с двух позиций. Подход из учебников наших рекомендованных 
там писались книги в середине 20 века, когда только недавно утвердилась 
СТО, и вопрос развития относительности в истории физики, и авторы строил 
свою историю физики с большим вниманием к относительности. Сейчас уже, 
во второй половине 20 века и позже, сильнее развивалась классическая 
конструкция как плазма, высокотемпературной плазме, и там вот идея 
самосогласованного поля является центральной, и теория Лоренца 
оказывается ключевой. И это мы выделяем как ключевое дело. Для решения 
оптических вопросов Лоренц вводит теор. подход для вычисления 
проницаемости, в отличие от старых лет, когда Максвелл предлагал 
использовать дисперсию, диэлектрическую проницаемость. Приходилось не 
только положительный и отрицательный заряд, но и доп. связь, 
квазиупругую силу, пружинку между ними. Это единственный архаичный 
момент в теории Лоренца. Такая же штука вводилась в теории дисперсии. 
Если просто ввести простейшую теорию упругой среды, эфира, вводить 
показатель преломления в духе Френеля получать уравнения, то дисперсия 
вообще не видна. А конечно видели ее потому что показатель преломления 
конечно зависит. Тогда это пытались объяснять через механическую теорию 
эфира. Лоренц, что важно, обратил внимание, что уравнения похожи на те 
самые дела с квазиупругим эфиром. А новое было в магнитной половине силы 
Лоренца. И это слагаемое объясняло много что, очевидно. Следующий момент 
в том что Лоренц вместе со своим коллегой Питером Зееманом в 1896 году 
(Ноб прем 1902) предсказали эффект Зеемана. Расщепление спектральных 
линий в внешнем магнитном поле. Эффект Штарка и многие другие уже были 
понятны объяснены. Зееман не только на очень ранней стадии развитии 
своей науки открыл эффект предсказанный Лоренцом. Там у Лоренца 
получалось, что из-за вклада магнитного поля возникает расщепление. 
Оттуда получается на сколько расщепляется, возникает циклотронная 
теплота. Тут конечно повезло, понятное дело, потому что на деле эффект 
Зеемана объясняется лишь квантовой теорией (только не аномальный, 
нормальный, объясняется классикой). Возникает еще одно важное следствие, 
на самом деле расщепление происходит на величину циклотронной частоты, 
которая включает отношение заряда к массе и величину магнитного поля. 
Зееман измерив что надо смог определить отношение заряда к массе. Это на 
самом деле электрон, но тогда так никто не говорил, не открыт еще 
электрон... И когда Томсон открыл электрон, измерял то же самое 
отношение, и оно абсолютно сошлось. Оказалось электрон обеспечивает 
оптические свойства атома, и электрон таким образом автоматом попал 
внутрь атома. Далее значит конечно следующим моментом было определение 
влияния движения тел, среды, источника на электродинамику движущихся 
тел, важно было понять что там происходит и постараться объяснить. 
Относительность такая появляется. Максвелл видел, что его теория лишь 
приближенно инвариантна к преобразованиям системы отсчета, хоть и где-то 
было, например в индукции электромагнитной. И Максвелл же вводит 
преобравзоание полей. Где-то там появлялось отношение v/c в квадрате. 
Такая точность была достигнута потом чуть на земле со скоростью движения 
Земли. Идея преобразования полей захватывает и Лоренца потом, ему было 
важно определить оптические явления, например эффект Доплера, потом еще 
эффект Физо (скорость света в движущейся среде), там было преобразование 
координат и локальное время. Потом у Майкельсона он там указал на 
ошибку, и с исправленной этой ошибкой потом удалось доказать отсутствие 
эфира. Сейчас же обратим внимание на электронная теория проводимости 
металлов. Когда был открыт электрон, стало понятно что в металлах ток 
обеспечивается движением ионов, пытались создать там и что-то. Сейчас 
иногда вспоминают, что в 102 году была создана теория Друде-Лоренца. Это 
очень важный этап развития физики, это были разные люди с разными 
подходами. Друде рассмотрел будто есть электронный газ, электронный газ 
и ионная решетка. Там особого поля нет считалось и рассматривали как 
обычный нейтральный газ. Друде действовал вполне в духе ранней 
молекулярно-кинетической теории, есть длина пробега, электрон 
сталкивается с ионами, возможно еще и между собой. По аналогии с теорией 
газов, где есть такая штука как диффузия, видим, что в если есть внешние 
силы, устанавливается поток. Тут очевидно производится аналогия 
с законом Ома... Такой анализ с помощью лишь длины пробега. Все эти 
формулы носят оценочный характер, можно оценить длину пробега, но это 
грубо... Лоренц же к этой же проблеме же обратился, но основывался на 
мкт на уравнении Больцмана. Это уже была физическая кинетика. Важно было 
когда это удалось Лоренцу 1902 год, когда уравнение Больцмана понемногу 
утверждаются. Для описания явлений переноса. Важно что! Не умели до 
Лоренца решать такие вот уравнения. И мы его тоже не решали на наших 
лекциях! Уравнение там Больцмана не решалось... Лоренц первым смог 
получить то приближенное, но последовательное решение. Это решение -- 
диффузия легкого газа в тяжелом. В этом пределе уравнения Больцмана там 
решается приближенно но последовательно. Потом выяснилось что уравнение 
Больцмана имеет и получше методы, Гильберта-...-... Большой прорыв 
в развитии теоретической физики, мол кин теории итд. Плотный электронный 
газ очень плотный и даже в поступательном движении нужны квантовые дела. 
Расчеты проводимости невозможно осуществить без квантовых эффектов. 
Теория оказалась неприменима, но осталась в физике как прорыв в решении 
кинетического дела.

\subsection{Открытие электрона}

События конца 19 века понятное дело, но посмотрим чуть до этого на 
развитие экспериментальных приколов в то время. Существенную роль 
сыграло открытие насоса. Целую серию исследований провел кто-то для 
актуальности атмосферного давления, вывел закон Бойля. Но там насос 17 
века похож на хз что. В 19 же веке был большой прогресс. Гейслер изобрел 
ртутный вакуумный насос. Началось все с 1855 года, насосы применялись 
все более эффективные и разные. Принципиально оказалось, что в 1870 
году, быстро довольно, догадались, что если в сосуд с вакуумом внедрить 
два катод и анод и создать разность потенциалов, возникает свечение. 
Назвали катодными лучами. Само явление было ярким, были разные 
предположения. Что-то связанное со столкновением частиц соответственно 
оптические явления, всему виной остаточный газ, возникает ионизация 
и возможно свечение. Оказалось длина пробега слишком маленькая. Тем не 
менее наблюдалось пятно какое-то на трубках, Герц открыл значит 
отклонение катодных лучей в магнитном поле, соответственно как вариант 
объяснения рассмотреть что это был поток частиц заряженных каких-то. 
В аналогичных опытах удалось накопить частицы, катодные лучи показали 
связаны с накоплением отрицательного заряда. А вот Томсон обратился 
более детально к этим катодным лучам, выяснил, что скорость этих частиц 
получается уж очень большой. Он провел систематическое исследование по 
отклонению лучей под действием близких к однородным полей, в скрещенных. 
Удалось определить отношение заряда к массе. Он заявил, что вот есть 
такая частица и вот она появляется в там и она имеет отрицательный 
заряд. Ключевым также было открытие Зеемана и эффект Зеемана как уже 
было сказано, выяснилось, что та же самая частица сидит в атоме 
и вызывает оптические спектральные атомы. Электрон оказался не только 
открыт не только в катодных лучах, но и в атоме. Заряд был вычислен 
в опытах по равновесию микрокапелек во внешнем поле электрическом. 
А потом и масса значится. Тот самый Томсон а потом еще Фицжеральд, 
Хевисайд рассмотрели движение частицы нерелятивистское (очевидно...) со 
скоростью v с зарядом q. Такая частица может иметь определенную энергию 
итд, если покоится есть электрическая какая-то. Для этого должен быть 
конечный радиус. Потом при движении кроме кинетической энергии возникает 
электрическая из-за тока, магнитного поля. Причем поле пропорционально 
энергии частицы. В младшем порядке там величина пропорциональная 
квадрату скорости. Томсон пришел к выводу что там какая-то константа на 
v\^2. Проблему представило то, что массу обычную и электромагнитную 
отделить невозможно. Потом оказалось, что если последовательно 
электромагнитную энергию рассчитывать из всех уравнений Максвелла, то 
кроме того квадрата появляются другие высшие четные степени. Оказалось 
не только вклад электромагнитной массы, затравочной массы, полная масса 
оказалось сама зависит от скорости (если мы все равно хотим пользоваться 
mv\^2/2). В общем накопились какие-то противоречия.

\subsection{Теория относительности}

Электродинамика движущихся тел. Есть две проблемы, реальная 
и относительное движение. Уже благодаря Лоренцу было понятно что-то там, 
хотя бы эффект Доплера, там именно v/c в первой степени. Тогда надо было 
объяснить почему получается один и тот же эффект если катушку двигать 
или итд. Такая видимая относительность была аргументацией Эйнштейна 
например. Потом Физо пришел со своей движущейся водой, у него получалось 
v/c в первой. И причем там не просто складывались скорости. Работает, 
оказалось, формула Френеля. Там у него были работы по кристаллооптике. 
Там была идея частичного увлечения эфира движущейся средой. Там надо 
было не просто складывать, а скорость среды еще делить на кое-что, ибо 
есть избыток плотности эффира, и именно только он увлекается, а базовая 
часть, вакуумная плотность эфира, остается неподвижна. Лоренцу не 
удается вывести из своих уравнений эффект Физо, и он говорит, что нужен 
поправочный коэффициент в случае движущейся среды. Сейчас мы знаем, что 
это преобразование времени... А он говорил о изменении темпов 
электромагнитных процессов и назвал это местное время. Тогда это не 
интерпретировалось вообще в виде проблемы измерения времени, но тем не 
менее обычно говорят, что Лоренц был сторонником абсолютно неподвижного 
эфира, но только у нас наблюдается в каком-то диэлектрике, там есть 
реально движение зарядов, связанные и свободные заряды, это как раз 
Лоренца нововведение. Теория Максвелла не была инвариантной при переходе 
в движущиеся системы. Когда есть какое-то относительное движение только 
тогда появляются эффекты первого порядка относительно v/c. Еще были 
опыты Майкельсона и Морли 1887--1892, которые не показали никакого 
эффекта. Идеи полного влечения эфира объясняли те опыты, но были 
внутренне противоречивы.

Потом были преобразования Лоренца в общем-то. Введение локального 
времени, потом лоренцево сокращение. И в присутствие сокращения этого 
сокращения для координат можно понять почему опыты Майкельсона Морли не 
показали ничего. Оказалось потом что Лоренц дал свое такое своеобразное 
преобразование ро и j, и не так как у нас. Эффекты объяснялись, но 
терялась инвариантность при переходе к системам отсчета. В игру входит 
Пуанкаре! Он до этого рассматривал фундаментальные проблемы устойчивости 
в частности солнечной системы, а вот в начале 20 века обратился 
к электронной теории Лоренца. Странно устроена теория. Он увидел 
какое-то разложение по v/c и желание объяснить эффекты какие-то. Но за 
тем чтобы объяснить полное отсутствие эффекта нужно вводить корень 
сокращение это. Пуанкаре не понравилось это дело. Наши физические 
величины должны образовывать группу относительно преобразований 
координат. Он решил в фундамент теории положить инвариантность, 
переписать основы. В 1905 году выходит работа Пуанкаре ``О динамике 
электрона.'' Есть мнение, что именно эта работа, вышедшая до теории 
Эйнштейна, и именно там появилась теория относительности. Конечно так 
трактовать не очень хорошо. Пуанкаре развивал именно теорию Лоренца. Он 
развивал не приложения динамики электрона, а переработал основы. Надо 
взять уравнения Максвелла как они есть, но добавить к ним в обязательном 
порядке принцип относительности. В этой теории симбиоз основных 
положений Максвелла уравнения взяты просто и потом идея природы тока 
Лоренца а вот динамика оказалась сильнее всего переработана. Уравнение 
динамики электрона было уже тогда записано в современном виде, сила 
Лоренца нормальная, а импульс уже релятивистский! Пуанкаре просто 
добавил корень вниз. Оказалось, что новая система Максвелла-Лоренца 
с вот таким уравнением динамики это вот первый шаг, он написал также еще 
преобразования но там конечно были не как у Минковского (Минковский 
после Эйнштейна уже), а с мнимыми единицами. Пуанкаре как математик 
заметил еще что преобразования все эти образуют группу. Раз это группа, 
значит все системы отсчета равноправны, значит никакого абсолютного 
движения нет просто нет. Возникают далее идеи такого релятивизма 
философского прикола, что все там так вот относительно. 
Пространственно-временного анализа у Пуанкаре нет, в отличие от 
Эйнштейна. Но были инварианты в $r^2 - c^2t^2$, $E\cdot H$, $E^2 - H^2$. 
Было дальше преобразование $\rho, j$.

\hfill \textbf{Apr 21}

Пуанкаре как мы помним там посмотрел на теорию электронов и прочего 
и пошел со стороны принципа относительности. Работа его называлась 
о динамике электронов. Пуанкаре также указал на групповые свойства 
преобразований Лоренца и показал инварианты.

\subsection{Альберт Эйнштейн}

Альберт Эйнштейн (1879--1955). В 1905 году выпустил К электродинамике 
движущихся тел. В этой статье, считается, были сформулированы основы 
СТО. Также был там сформулирован его подход к фотоэффекту. И третье, уже 
говорили, в 1905 году была опубликована его теория Броуновского 
движения. Подход Эйнштейна принципиально отличался от Лоренца 
и Пуанкаре. И ему скорее всего не были известны работы их в начале века, 
и точно не была известна последняя работа Пуанкаре (тоже 1905 году 
выпущена). Содержание работы было следующее. Первая часть 
кинематическая, там преобразования Лоренца выводятся из постулатов 
Эйнштейна: принцип относительности и принцип постоянства скорости света 
в вакууме. Его работа имеет универсальный характер, эта выдвинутая 
вперед кинематическая часть утверждает, что мы имеем дело с новым 
понятием пространства и времени. Сначала мы имели дело с Ньютоном, и там 
было некое абсолютное универсальное время и пространственные интервалы, 
расстояния, не меняются. Новая теория Эйнштейна основана на том, что 
возникает относительность и времени, и пространственных интервалов. 
Радикально изменяется теория пространства и времени. И причем так уж 
получилось, предсказательная способность, привела к возникновению новой 
электродинамке, но и привела к основе теории других взаимодействий. 
Через несколько лет ядерные взаимодействия в ядре сильные и слабые. 
Эйнштейн в первой части выводит исходя из линейности и кое-чего другого 
выводит в общем преобразования для координат и времени и показывает, что 
новая теория пространства и времени, новая концепция их, приводит 
к парадоксальным для того времени результатам. Возникает относительность 
длины и главное относительность одновременности! Там у Лоренца это 
рассматривалось как некое сокращение настоящих размеров тел, 
электрических систем. И время было местное лишь характерно для только 
тех систем и появляется только там. Эйнштейн же попытался ввести новую 
общую кинематику. Почему так было? Иногда даже не очень различают 
подходы Пуанкаре и Эйнштейна, и это приводит к выражениям, что Пуанкаре 
на самом деле изобрел СТО. Но не даром называются по-разному, у Пуанкаре 
лишь радикальное изменение теории Лоренца, а Эйнштейн как мы уже поняли 
прям ко всему применил. И понятное дело что применяя приколы Эйнштейна 
к электродинамике мы получили бы Пуанкаре. Подход Эйнштейна был уникален 
тем, что из одной лишь скорости света получается вон сколько что.

В том же году Эйнштейн свою теорию фотона и фотоэффекта выдвигает. 
Говорит, что фотон имеет корпускулярные свойства, и еще кстати поэтому 
Максвелл не всегда применим. И тогда публикуя одновременно работы по э-д 
движущихся тел не мог применять уравнения Максвелла для постулатов для 
объяснения особых свойств пространства и времени. В этом анализе 
демонстрируются проблемы одинаковой длины, проблемы измерения интервалов 
времени. На всех окружающих это конечно произвело большое впечатление. 
Изменение темпа хода часов тоже оказалось ярким эффектом.

А теперь вторая часть той его работы К электродинамике движущихся тел. 
У Пуанкаре например уже сразу видны групповые дела, а у Эйнштейна есть 
определенные недостатки. Он рассматривал динамику слабоускоренного 
электрона. Но там еще у Эйнштейна была особенность в том, что он 
рассмотрел обратил внимание на поперечный эффект Доплера, чисто 
релятивистский эффект. И когда измерили потом конечно большое 
впечатление тоже произвело. У Эйнштейна конечно нет еще 4-импульса, но 
есть формула для преобразования энергии при переходе, и для частоты 
похожее. Отсюда при внимательном чтении видно, что их отношение дает 
какую-то константу! И для фотона это отношение как раз постоянная 
Планка. Это не обсуждается в статье, но это вообще подтверждает его идею 
фотона.

Эйнштейн работал тогда в Германии в Берлине, и другие немецкие ученые 
были очень рады этому всему. Планк в следующем году развивает что 
написал Эйнштейн для э-м динамики, и разрыв с Пуанкаре сильно 
уменьшается. И потом дальше пошло продвижение. Когда мы рассматриваем 
другие теории, теорию ядерных сил например, там язык соударения частиц 
уже в таком релятивистском формате, и эти столкновения в 1909 году Льюис 
и Томлен. Потом рассматривали что там происходит с массой частиц 
релятивистской так сказать. Зыбкость и относительность понятий длины, 
времени, одновременности произвели впечатление на всех. И формально 
математический подход к этому всему была продвинута Германом вон 
Минковским. Минковский вводит группу преобразований, и она сразу 
в рамках теории Эйнштейна образуют 4-мерное пространство-время с аж 
псевдоримановой метрикой. И он показал что принцип относительности 
автоматом соблюдается, если пользоваться тензорными конструкциями в этом 
пространстве-времени. Вместо старых инвариантов вот получаются просто 
новые инварианты. И это все подавило весь философский релятивизм, 
который складывался в свете всех перемен.

Успешно применить к теории поля после открытия ядерных сил оказалось, 
что ядро состоит из нейтронов и протонов, а потом были открыты мезоны, 
в частности нейтральные, которые не участвуют в электромагнитных 
процессах (казалось бы). И мезоны понятное дело нестабильные, 
распадаются. Так вот время жизни измерялось и сравнивалось с ожидаемыми 
результатами Эйнштейна. И естественно все было превосходно!

% }}}

\section{Возникновение квантовой теории}
% {{{

\subsection{Теория теплового излучения. Начало квантовой теории}

Вообще говорят, что 20-й век это эпоха становления квантовой 
и релятивистской физики, вторая половина 19 века например считалась 
становлением в полном виде классической физики. Не одновременно, но 
к 1900 году была и классическая физика, и электродинамика (уже не 
Максвелла, а система частиц и поля благодаря работам Лоренца и Герца). 
Механика тоже тогда подросла, там роль инерциальных систем отсчета, 
также выяснилось, что механика не так просто и прозрачно как выглядит 
рассуждение Лапласа. Периоды 1900 1905 года рассматривают как революцию. 
Действительно грандиозное событие, формирование классической волновой 
оптике, переход к волновой оптике.

СТО и квантовая теория формировались даже не столько от трудностей 
экспериментальных (даже квантовая), но просто оказалось что классическая 
механика, стат физика и термодинамика, электродинамика, они должны 
как-то стыковаться. Несмотря на то что они как-то автономны, 
в пересечениях не должно быть противоречий. А они были. И тут появляется 
революция этого всего. Обратимся к теории теплового излучения. Известно 
что квантовая теория возникла благодаря Планку и его теории теплового 
излучения. Обратно в середине 19 века (в 18 веке были открыты 
инфракрасные вещи и их внесли в оптику). Кирхгоф обратил внимание, то 
если э-м излучение считать тд системой, то можно ввести вещи как 
спектральная плотность излучения, отношение знаменитый закон Кирхгофа 
появился, там где излучающая способность и поглощающая способность. Для 
черного тела поглощающая способность единица, а отношение их 
универсальная функция. Черное тело оказалось можно сделать сотворить 
с хорошим приближением. Изотермическая такая вот полость с отверстием. 
Второй момент важный, что удалось создать измерительное устройство для 
спектральной плотности, но это было уже позже. А вот в 1870-е оказалось 
что для отношения того самого из теории можно сделать два вывода. 
В термодинамике делают много выводов о равновесном этом самом излучении. 
В 1879 году коллега Больцмана Стефан показал, что интеграл по всему 
спектру пропорционален 4 степени температуры. Когда обратились к деталям 
работы оказалось, что вывод на сомнительных экспериментальных деталях. 
Больцман же теоретически основывался на термодинамике, но чтобы вывести 
формулу ему понадобилась система Максвелла, изспользовал там давление 
итд. Полный интеграл от того отношения получится константа и 4 степень 
температуры. А дальше Вин продолжил деятельность и выступил что-то 
как-то. Вин в общем получил закон смещения Вина, который позволил 
продвинуться в получении вида функции (появилась небольшая связь, что 
функция должна зависеть от произведения лямбда на Т, кроме там 
множителя). И эта вещь учитывала еще эффект Доплера. Больцман учитывал 
давление света, а вот Вин показал, что при учете Доплера можно получить 
это самое смещение. Дальше движемся вспоминаем результаты молекулярно 
кинетической теории, видим что температура входит в распределение 
Максвелла. И в сочетании с этим всем и с тем высказыванием Михельсона 
нашего русского ученого, что энергия и частота связаны, Вин написал 
форму спектра с экспонентой, как было у Максвелла! Обратим внимание что 
начинаются разные обозначения там везде для зависимости от длины волны 
и от частоты. И короче для малых длин волн удалось проверить что надо 
и сошлось.

А потом начинаются разные работы Планка. Неточные формулировки часто 
встречаются. Если взять теорему Больцмана о равнораспределении энергии 
по степеням свободы, можно рассматривать стоячую волну, каждая имеет 
энергию кТ, но их число не заканчивается никогда, и энергия бесконечная 
получается. Результат Релея-Джинса дошел до немецкой общественности 
с запаздыванием довольно. Проблема не была очевидна что там расхождение. 
А вот в Британии не доверяли вообще равнораспределению, и не смотрели 
всерьез на проблему. А там потом Планк заявил, что еще Кирхгоф сказал, 
что не зависит от вещества, а поэтому возьмем осцилляторы. От длины 
волны перешел к частоте, и получил, что спектральная плотность 
пропорциональна распределению энергии гармонических осцилляторов. 
И формула там выглядит довольно простой, но конечно получить ее тем 
более тогда просто не было... Там излучение формировалось тогда 
в системе зарядов, плазме, возникают непростые вещи, а тогда имелось 
в виду, речь шла о простой теории. Эта та формула связи спектральной 
плотности и распределения осцилляторов по энергиям уже была получена 
планком из Герца теории. К тому времени экспериментальные исследования 
на эту тему продвинулись уже в Берлине как раз. Для высоких частот не 
было сомнений надо брать формулу Вина. Для малых хорошо было Рубенс 
и Курльбаум (а не Релей). Планк интерполировал это все дело, и.... 
Получилось уж слишком хорошо. Надо было вывести другим каким-то методом 
получив то же самое. Планк пошел через Больцмановский подход через 
энтропию и ту Гамму, обосновал что там есть еще k. У Планка же был сразу 
готовый ответ, он воспользовался методом Больцмана так сказать 
с поправками. К 20-м годам уже выяснилось, что статистика Больцмана вещь 
такая, которой можно пользоваться, были уже Броуновские дела 
Смолуховского и Перрена. Но что там делал Планк в декабре года. Если 
возьмем полное число осцилляторов N, по ним распределяется энергия. 
Энергию он решил проквантовать... строго дискретные вещи. Размер порций 
уже не выпадал, как у Больцмана. Планк особым образом подсчитывал 
микросостояния для макросостояния. Больцман там смотрел на ячейки и они 
были различимы. А у Планка там другое. Но в общем все получилось. 
И чудом оказалось что там много мировых констант удалось определить из 
оттудова. Получилось там постоянная Планка, константа Больцмана была 
названа в честь него Планком, а из нее и число Авогадро, а оттуда и уже 
сразу заряд и масса электрона... Но в чем причина этой вот дискретности, 
она получилась уж очень неустранима. Отмахнуться от идеи было нельзя, но 
считалось, что это что-то очень специфическое. А в 1910 году Лоренц 
проанализировал детально теорию Планка, и сделал утверждение, что теория 
Планка с законами классической физики была несовместима. Планк в первой 
части работы связывал энергию осцилляторов с излучением и там было все 
как обычно, без дискретности, а там в осцилляторах уже да. А после 1910 
оказалось, что надо необходимо все уже передумывать все основы.

\subsection{Теория фотоэффекта. Идея кванта света}

Теория фотоэффекта. Первые обобщения сделал там Эйнштейн в 1905 году. 
Работа эта тоже состояла из двух частей. В первой части Эйнштейн 
обращается к выводу тому же. Он действует радикально. Рассмотрим 
равновесное тепловое излучение в полости как газ из частиц с энергией 
h nu, есть какой-то возврат к Больцману, его статистике, вроде бы 
обычный газ в равновесии. Оказалось что действительно получается что, 
выводится формула Вина. Мы сейчас знаем, что в пределе высоких частот 
излучение там и правда статистика совпадает с Больцманом. Газ тут себя 
ведет как классический газ таких частиц. Для того времени это было 
пионерское заявление. Но это только первая часть работы, хоть уже все 
перевернули и вернулись к корпускулярной теории света, но не совсем. 
Вторая часть касалась теории фотоэффекта. Да, вот есть такая корпускула, 
и электроны в металле могут поглощать энергию э-м излучения и вылетать 
и будет такой фототок. Эйнштейн не исходил из того, что нужно радикально 
пересмотреть теорию оптики, она прекрасно прогрессировала весь век 
прошедший. Фотоэффект конечно понятно не Эйнштейн открыл... Герц в 1886
обнаружил, что если посветить ультрафиолетом, искра при ионизации 
воздуха получше идет. Знаменит еще вклад нашего Столетова, который не 
просто открыл законы фотоэффекта, но и создал фотоэлемент. Там вакуумная 
конструкция, похожая на исследования фотолучей, но там катод горячий 
и разность потенциалов, здесь тоже (1889 год это все) появлялся ток, 
и интенсивность фототока пропорциональна интенсивности света. А еще 
знаменита там эта вещь с красной границей. Дальше Томсон отождествил 
носителей тока с электронами. Потом выяснилось, что те же свойства 
и у вылетающих с катода там частиц, это стали фотоэлектроны. Потом 
оказалось, что энергия фотоэлектронов не зависит от интенсивности света, 
а вдруг там частота света. Это открытие предшествовало сразу же 
публикации Эйнштейна. Формула в общем там понятное дело простая, но за 
ней физика мощная. Там заявлялось, что само излучение уже дискретно. 
Планк считал, что дискретность появляется именно прям в процессе 
поглощения излучения. И еще один есть момент, что один электрон 
поглощает один фотон. Потом конечно показали, что вероятность такая, что 
однофотонный процесс конечно. И дальше были измерения, показавшие снова 
постоянную планка. И вон там еще масса электрона снова. И Штарк в 1909 
году высказался, что фотон и электромагнитное излучение обладает 
энергией, импульсом и моментом импульса. Раз есть энергия, значит будет 
импульс... Имеют место законы сохранения как энергии, так и импульса. 
И в связи с этим получается эффект Комптона. И вот этот квантовый 
характер взаимодействия электрона и фотона подтвердился в общем.

\subsection{Идея корпускулярно-волнового дуализма света}

У Планка сам процесс излучения модифицируется, а так можно все писать 
через Максвелла и такие классические вещи. Это как-то легло ну и пусть, 
а в 1905 и далее Эйнштейн вывел формулу Вина рассматривая грубо говоря 
это все как газ частиц с энергией аш ню. Эйнштейн не был готов принять 
на веру Максвелла уравнения, считал, что дискретность появляется у света 
самого фундаментально. А во второй части фотоэффект, где тоже появляется 
постоянная Планка, и у нее уже появляется такой богатый смысл. В 1909 
году Эйнштейн переписывает формулу Планка для записи среднеквадратичной 
флуктуации энергии. Первое слагаемое там содержит постоянную планка 
в виде hv, а вторая формула там классические вещи Релея или кого еще, из 
уравнений Максвелла и только следующая. И в этой формуле видно, что 
у света есть и волновое слагаемое второе, и корпускулярная первая. 
К концу 19 века сформировалась уже победа волновой теории света, а тут 
вот такое начинается.

\subsection{Применение идеи квантов -- новая механика}

Генри Комптон Лоренц. Открытия 20 века показывают, что необходимо 
пересмотреть всю механику. Кардинально отличается от слов Планка. В 1911 
году Натансон обратил внимание, что есть разные способы подсчета 
микросостояний, моды расположения. У Эйнштейна в 1905 году там был один 
способ подсчета осцилляторов так сказать, там приемники можно различать, 
и поэтому выходит классическая такая Больцмана. А есть подсчет в духе 
Планка. Тут существенно, что приемники энергии нельзя различать, 
и получается другая статистика. Сольвеевский конгресс в ноябре 1911 года 
начались в этом году и продолжились потом под этим же именем. Тут уже 
точка зрения Планка меняется. Ученые пришли к выводу, что нужно 
модифицировать методы стат физики и модифицировать механику. В качестве 
примера рассмотрим гармонические осцилляторы, в задаче о равновесии 
системы таких осцилляторов появилась идея квантования. Говорилось, что 
подсчеты там через интегралы, и появляется теорема о равнораспределении. 
А в квантовом деле появляются ячейки фазового объема. Тогда та формула 
получается в классическом подходе с Больцманом, но вместо интегралов 
сумма та. Это первое квантование появилось вот тогда. Модифицировалась 
не сколько статистика сколько механика. Планк отказался от принципа 
универсальности классической механики. Раз появилась новая методика 
подсчета средней энергии там, то это же самое должно появляться 
в теплоемкости твердых тел. Эйнштейн в 1907 году опубликовал Теорию 
излучения Планка и теорию удельной теплоемкости. Там он по сути среднюю 
энергию продифференцировал по температуре, получил теплоемкость, получил 
что при нулевой температуре теплоемкость получается ноль.... А вот 
и ноль видели в эксперименте! Вот а там еще сказал Больцман задолго до 
Дебая, что в твердых веществах нужно учитывать много разных мод, 
появляются такие нормальные моды итд. Но дело в том что там все равно кТ 
получается, и все равно Дюлонга и Пти... А теперь идея Эйнштейна что 
нужно подходить сюда тоже квантово была подхвачена Дебаем, и все эти 
осцилляторы получаются квантовыми. Тут уже все зависит от частоты, 
и получается кубическая зависимость теплоемкости от температуры при 
малых Т, как в эксперименте. Это все в 1912 году Дебай был.

Дальше у Луи де Бройля появляются работы о волновых свойствах частиц. 
А потом в 1924 году появляется Бозе из Дакки (Индия-Бангладеш) 
предлагает новую идею подсчета состояний. И эта работа попала 
к Эйнштейну тогда же, когда и работа де Бройля. И дуализм очень хотелось 
распространить на все подряд.

\subsection{История создания атома. Возникновение квантовой механики}

Строение атома обсуждалось в молекулярно-кинетической теории, где там 
Мах высказывал недовольство. Но параллельно с этим измерялись кучи 
спектров всего подряд, закономерности были очень сложными, и их 
связывали со сложностью там атомов. В 1885 году возникает система 
спектра водорода, формула Бальмера. Потом получилось распространить на 
щелочные металлы. Потом связали частоту перехода с разностью 
одноиндексных вещей -- термов. А теперь возникает кризис, потому что 
объяснить, тем более классическим методом, не удается... Далее в 1897 
году открыли электрон. Потом эффект Зеемана и там далее все 
согласовалось, поместили электрон внутрь атома. Теперь появился Перрен 
экспериментатор и сказал, что уж очень все похоже на солнечную систему. 
Все стали критиковать что угодно стали, будет быстро электрон излучать 
и упадет на центр за буквально пол-оборота. Потом появилась еще одна 
модель Томсона того самого открывшего электрон. Называется пудинг 
с изюмом, дается объем с положительным зарядом легко проницаемый. Сразу 
получается квазиупругая Лоренцева сила линейная при удалении от центра. 
Электрон мог находиться много как в том числе и покоиться. И тут 
возникают трудности, причем довольно поздно. Возражения какие же были, 
самое главное конечно открытие атомного ядра, опыт Резерфорда. Модель 
Томсона конечно не объясняла то рассеяние альфа-частиц, которое давал 
опыт. Но люди говорили еще что рассеяние на большие углы, которые открыл 
Резерфорд, на самом деле есть результат того, что альфа-частица 
сталкивается не с одним атомом, а с многими. Это называется концепция 
многократного рассеяния. И показать что эта концепция не работает на 
самом деле не так уж просто, но в конце смогли конечно. Потом модель 
Нагаока в виде колец там Старна, вот такие дискретные чуть и прочее. Но 
это было суперсложно и ничего не получилось. Потом модель Никольсона 
включающая постоянную планка. А потом уже квантование орбитального 
момента с энергией, но а атомом было сложнее.

Теперь Бор 1913 год. Бор сначала работал Дж Дж Томсона и развивал 
в общем-то модель пудинга этого, потом перешел к Резерфорду, чьи 
результаты его впечатлили. Бор был конечно теоретиком и хотел все 
объяснить. Как-то нужно было преодолеть возражение Вина про 
неустойчивость. Дальше там размер атома не получался особо. Если просто 
взять закон Кулона и вводим разные частоты обращения получаем что 
угодно. Тут у Бора возникает идея что нужно не пытаться создать модель, 
а брать постоянную Планка, то есть не объяснение природы h через атом, 
а наоборот атом через h. Бор тут работав перед первой мировой войной эти 
принципы и формулирует. У Бора была попытка сочетать классическую 
механику с этими принципиально новыми вещами. Тут видно он пытался 
совместить вот такие несовместимые дела. Бор тут действует как Эйнштейн 
и говорит что нужно забыть по часть классики. Разница энергий электрона 
на двух состояниях дает энергию излучения... Но как сейчас в книжках 
пишут для орбитального момента такое не написал Бор сам тогда. Он писал 
что частота вращения электрона на орбите не имеет отношения к частоте 
перехода. Но Бор как раз решил, что для очень больших орбит там должно 
согласоваться, частота обращения должна совпадать с частотой излучения 
в случае ионизации атома или захвата. Эта идея позволила ему рассчитать 
дискретные вещи. Что дальше последовало. Были предсказания, хоть теория 
сама была противоречивой. Спектр сам следовал из комбинационной формулы 
Ритца. Второе это опыт Франка и Герца, дискретные уровни атомов могли 
наблюдаться экспериментально прямо. В теории Бора конечно еще была 
вложена баланс энергии электрона и излучения.

Еще можно поговорить про правила квантования. Они возникают уж очень 
рано честно говоря в истории квантовой теории. Введение ячеек фазового 
пространства и есть так сказать квантование. Конечный набор состояний 
в фазовом пространстве. Для каждой степени свободы так. Дальше как 
обобщить на несколько степеней свободы. Планк получил это в 1915 году, 
разбить на пары так сказать p, q. Для каждой степени свободы появляться 
должно квантовое число! А у Бора там от одного зависит энергия только, 
сразу стали говорить о вырождении. Дальше стали говорить об 
электрических орбитах, но там понятно ничего не подтвердилось. В общем 
появляется такой вот подход. Надо сначала смотреть классическую задачу, 
а потом наложить на нее принципы квантования.

Дальше фактически параллельно производился еще одно. Не очень понятно 
было тогда в начале 20 века что там вообще квантуется, что там такое. 
Размерность h была действие, но закона сохранения действия нет... Пауль 
Эренфест был учеником Больцмана там говорил, что в той самой дипломной 
работе Больцмана, где он хотел найти энтропию, смотрел на адиабатическое 
изменение квазипериодического процесса. В этой работе появляется 
адиабатический инвариант. Энергия не будет сохраняться тут, ибо открытая 
система, но асимптотически точно сохраняется что-то новое, что сейчас 
называют адиабатическим инвариантом. У Клаузиуса тоже была величина 
такого типа. Но Клаузиус позже это опубликовал, поэтому Больцман. 
Вспоминает об этом Эренфест в свете вопроса а что мы там квантуем. Вот 
вспомним еще 1911 там конференцию, Эйнштейн и Бор(?) вспоминают задачу 
о маятнике с изменяющейся длиной. Речь идет о том, что если мы меняем 
длину нити маятника медленно, что происходит там. Меняем за много 
периодов что происходит. Надо короче умножить период маятника на 
энергию, и это будет в общем постоянная. Оказалось что этим пронизаны 
все квантовые формулы, там есть отношение v/T, это тоже все 
адиабатические инварианты. Из этого было не просто открыт принцип, но 
и метод нахождения стационарных состояний. Метод адиабатической 
деформации системы, квантование сохраняется типа. Тут же Эренфест 
показал, что адиабатический принцип сбоит иногда. При изменении 
потенциальной энергии могут происходить радикальные преобразования, 
меняющие поведение. Переход маятник-ротатор например нехорошо 
получается, переход от финитного движения к инфинитному.

Естественно требовались какие-то обобщения после эффективности атома 
Бора для водорода и щелочных дел. Дальше развитие вот какое. Во-первых 
принцип соответствия. Бор в 1918 году обобщил свою идею ...... Как мы 
получим поляризацию, интенсивность различных линий спектра. Бор 
предложил использовать тут принцип соответствия. Дипольный момент, 
изменение момента во времени. Дипольный момент при переходе среднее 
значение дипольного момента использовать для перехода там. Принцип 
соответствия Бора нельзя путать с принцип соответствия философским! Там 
в философии говорят о согласии с прежним. Тут это эвристический принцип. 
Далее в 1916 году была еще одна работа Эйнштейна, он снова вернулся 
к закону Планка, снова выводит его, но говорит, что есть 3 вида 
взаимодействия с излучением. Соответствующий осциллятор может спонтанно 
излучать, а есть вероятность индуцированных переходов при наличии 
имеющихся, ну или будет обратный процесс поглощения. Имеет место баланс 
всех этих трех процессов. Эйнштейн говорит, что если есть уровни 
энергии, то можно не только про излучения говорить, но и вот такие 
иднуцированные и спонтанные вещи. Но здесь как раз начинает возникать 
некое противоречие в подходах Бора и Эйнштейна к новой физике. Эйнштейн 
вводил конечно эти вероятности, но считал строго, что не видим мы всех 
существующих вещей, неполная информация, ну и тут не денешься без 
вероятностного описания. Бор же ввел саму идею спонтанного излучения. Из 
принципа соответствия Бор связал с Фурье кого-то там, возникает идея 
соответствия интенсивности итд. Бор в общем говорит, что сами переходы 
имеют вероятностный характер и ничего с этим нельзя поделать.

\subsection{Создание матричной механики}

Приступим к событиям 1925 года. Создание матричной механики. С 1913 до 
1925 это развитие подхода Бора. А в первой половине 1925 года появляется 
Гейзенберг и публикует статью о квантовоменахическом истолковании 
кинематических и механических проблем. Ему не нравилось, что нужно 
решать классическую задачу, а потом накладывать квантование. И это 
сочетание противоречивых методов его расстраивало. С одной стороны он 
воспринял философски идеи Бора, что не нужно вводить новые какие-то 
скрытые параметры, но и Эйнштейн со своей СТО тоже впечатлил повлиял. 
Эйнштейн там ввел необходимость того, чтобы мы могли четко определить 
понятия и продемонстрировать процесс измерения. Вводится понятие 
наблюдаемой величины. Абсолютная одновременность и подобные вещи конечно 
можно придумать, но измерить -- никогда. Тут Гейзенберг говорит, что мы 
не видим что там электрон делает, а видим частоты переходов, 
поляризации, интенсивности линий и все. И на этом уровне нужно внедрить 
понятие квантования, а не применять к классике. И не пользоваться 
классическими вещами, которые мы не наблюдаем. Мы не видим траекторий 
и вообще всего остального! И мы не должны ими пользоваться. Наблюдается 
комбинационный принцип Ритца и два индекса переходов вместо там простых 
Фурье вещей и мод (один индекс). На таком базовом уровне уже хочется 
квантование ввести. Он хотел еще какую-нибудь новую задачу. В этой своей 
первой работе Гейзенберг рассмотрел некое возмущение гармонического 
осциллятора, но с квадратичной нелинейностью. Не меняет это частоту. На 
самом деле Гейзенберг говорит, что мы наблюдаем лишь только 
двухиндексные частоты, линии эти и коэффициенты перед ними, амплитуды. 
Дифференцирование дает просто множитель. Как работать с двумя индексами, 
как записывать производные, квадраты и т.д.... Гейзенберг как-то это 
делает, а потом выясняется, что это как матрицы! Чуть позже уже. 
Гейзенберг получает энергию не равную нулю начальную и формулу ту самую 
n+1/2.

\hfill\textbf{May 12}

Гейзенберг в 1925 году предложил новый подход к задаче об осцилляторе. 
Мы уже обратили внимание, что рассмотрел он гармонический осциллятор 
с квадратичной нелинейностью (обычный осц гарм не давал нового по 
сравнению с работой Планка). Основная идея Гейзенберга была не 
накладывать квантовые дела поверх классической задачи. Вместо этого мы 
используем величины, являющиеся наблюдаемыми (частоты, амплитуды, 
поляризации, интенсивности). Никаких координат, траекторий, импульса от 
времени у нас нет. Для наблюдаемых должны построить теорию. Вместо 
обычного Фурье-разложения координаты от времени Гейзенберг предлагает 
набор по комбинационному признаку Ритца рассматривать двухиндексные 
величины. Появляется новая степень свободы вот. И оказалось, что это 
у него удалось. Квантовое условие Зоммерфельда, где интеграл $mv^2$ по 
периоду равен $2\pi n \hbar$. По сути это правило квантования, правило 
сумм. И в каждом случае оказывалось, что мы вычисляем двухиндексные 
величины наборы такие вот путем свертки по среднему индексу. Все-таки он 
решил другую задачу, хоть и близкую к гармоническому осциллятору. Он 
получил принципиально новую вещь -- нулевую энергию $\hbar\omega/2$. Но 
сам Гейзенберг считал, что результат обусловлен тем, что мы используем 
только наблюдаемые величины и не думаем о неизвестных по типу 
траектории. Вскоре Борн и Йордан, у последнего было матричное исчисление 
развито, и они поняли, что это просто теория Эрмитовых матриц. 
Физические величины получилось надо описывать Эрмитовыми матрицами. Они 
выпустили книгу ``О квантовой механике'' и потом в том же 1925 году ``О 
квантовой механике II''. Они там написали это самое характеристичесское 
уравнение с определителем Н - lam I равен нулю, собственные значения 
энергии. Потом Винер увидел, что нужны матрицы бесконечные, и можно на 
деле использовать такой вот функциональный анализ. Вместо матриц 
получились операторы, линейные, дающие функцию в ответ на функцию. Такой 
операторный подход конечно хорошо стыковался с чисто матричным делом, 
а во-вторых мог описывать состояния с непрерывным спектром, а значит был 
более общим. Борн и Винер в 1925 том же году опубликовали дела про эти 
операторы, но вот они не додумались до вида формы оператора импульса. 
Потом виден сразу еще один альтернативный подход, это была алгебра так 
называемых q-чисел Дирака. То самое правило сумм он заметил там на деле 
имеет влияние некоммутативность относительно обычной механики. По сути 
дела оказалось, что есть такое важное понятие как скобки Пуассона. Дирак 
сказал, что вот есть классические скобки Пуассона, а есть вот квантовые 
такие, которые просто коммутатор. Это еще один подход записи от 
Гейзенберга.

При этом там были проблемы с решением задач, по сути только 
гармонический осциллятор получался. А вот в начале 1926 года Паули 
и независимо Дирак смогли решить задачу атома водорода! И кроме того 
Паули описать смог эффект Штарка -- это первый прогресс по сути после 
теории Бора. И тут оказалось не нужно никакого принципа соответствия, 
все получается логично, как раз как Гейзенберг хотел.

\subsection{Возникновение волновой механики}

Параллельно идут альтернативные процессы, не похожие на Гейзенберга, это 
возникновение волновой механики. Гамильтоновская механика была похожа на 
оптику почти что... Но вот тогда возникли новые вещи, вот фотон. Это 
оказалось сразу и волна, и частица, в равновесном тепловом излучении. 
При этом видно было, что корпускулярный и волновой подход не особо 
совместимы были. Эйнштейн считал, что поле формируется как некое среднее 
значение, но там в микромасштабе видны должны быть дискретности. 
И прыгнем чуть-чуть дальше, там вдруг обнаружили, что электронам тоже 
можно приписать волновые свойства. Бозе послал еще одну работу по 
обоснованию формулы Планка, и там Бозе показал, что если относительно 
обычных частиц с энергией аш ню взять просто новую вот такую квантовую 
статистику, то вот получится сразу формула Планка. Эйнштейн в то же 
время считал, что нужно применять даже к газам массивных частиц новую 
статистику. Вот и назвали статистика Бозе-Эйнштейна. А еще Луи де Бройль 
в 1923--1924 провел абстрактные рассуждения, что если есть энергия 
частиц и сопоставлена ей частота, то возьмем массивную частицу, поделим 
тоже на постоянную планка, увидим какую-то непонятно что дающую частоту. 
Потом видел что есть пара таких вот вещей в 4-мерном пространстве, 
энергия импульс и частота тоже не просто так гуляет, там еще волновой 
вектор. Если ввести такую вот связь энергия импульс поделить на 
постоянную планку, то видим вот такую волну. Казалось бы а где 
подтверждение существования такой волны. Де Бройль показал где такое 
доказательство, обычно там интеграл Зоммерфельда pdq и nh. Де Бройль вот 
тут показал, что длина волны укладывается целое число рад в окружности. 
Немного непонятно почему так должно быть, но это вот какое-то объяснение 
правила квантования. Но был неприятный такой момент, что волна будет 
сверхсветовая, отношение омега к к было больше скорости света. Де Бройль 
показал, что если взять групповую скорость д омега / д к, то как раз 
будет скорость частицы. Волновой пакет движется со скоростью частицы, 
частицу можно вот тут вариант рассматривать как волновой пакет (сейчас 
знаем, что не самый хороший подход так получается). Сейчас можно 
сказать, что волновые дела появились, и потом нашлось соответствие 
с квантовым газом Эйнштейна с новой этой статистикой, который должен 
обладать точно так же и волновыми, и корпускулярными свойствами. 
Волновые свойства электронов хотелось найти. Де Бройлю сразу говорили, 
что раз есть волны, значит должна быть интерференция и дифракция, 
считает ли он, что у электронов должны быть они видны. Он ответил да, 
и перед экспериментаторами встала задача. Дэвиссон и Джермер в США 
в 1927 году рассеяние электронов на монокристалле никеля. Джордж Паджет 
Томсон исследовал рассеяние электронов на тонких пленках (это сын дж дж 
Томсона). После этих открытий уже в общем-то никто не сомневался 
в волновых свойствах частиц.

Было еще одно сопровождающее квантовую механику событие. Там была идея 
оградиться от всех абсолютно ненаблюдаемых величин, хотелось убрать это 
дело, и работать только с наблюдаемыми. Шредингер поставил перед собой 
задачу понять эти вот волны Де Бройля в случае свободной частицы. 
Основания работы Шредингера. Основная работа его была Квантование как 
задача о собственных значениях, где были получены основные результаты. 
Шредингер воспользовался тем, что если вспомнить оптико-механическую 
аналогию Гамильтона. Он там работал с мембранами и чем-то еще, там 
возникала задача на собственные значения. Это скорее было обобщением дел 
Де Бройля на случай внешнего воздействия какого-нибудь. И Шредингер 
записал свое стационарное уравнение. И на основе этого хотелось решить 
атом водорода, он как раз получил. Оказалось, это намного удачнее, чем 
матричная механика. Год почти потребовался чтобы там решить, а Шредингер 
взял и получил дискретный набор собственных функций и собственных 
значений, проверить собственные значения он мог, а функции не с чем было 
сравнивать. Вот там удалось получить правильный спектр. Кроме того само 
квантование было по сути введено оператором импульса, не нужно было 
что-либо квантовать. Не нужно было решать классическую задачу, а потом 
накладывать условия квантования. Шредингер кроме того как и Паули 
показывает эффект Штарка. Далее появляется нестационарное уравнение 
Шредингера, внешнее поле, зависящее от времени, если энергия мала по 
сравнению с потенциалом каким-то, речь шла о решении методом 
последовательных приближений, но было представление, что должен быть 
вклад, связанный с производной по времени. Еще одной важной особенностью 
работы Шредингера было, что он начинал с стационарного состояния, где не 
было никакого времени вообще, и те решения полученные там потом были 
такими парами, обычное и комплексно сопряженное. Распределение заряда не 
зависит от времени для стационарного состояния. Боровские переходы 
трактовались как некая суперпозиция состояний исходного и конечного, 
и там такие биения. Там уже явная зависимость заряда от времени там 
колебалась как раз с периодом как у Бора частота была. Позже в 1925 году 
уже не только матричный подход, но вот и Шредингера дела. Потом тогда же 
смогли доказать, что подход Шредингера и Гейзенберга одинаковые, но 
к Шредингеру было проще там математически.

Дальше поехали к интерпретации квантовой механики. Шредингер сказал же, 
что там заряд был в функции собственной, связано это его заявление было 
с тем, что он получил уравнение непрерывности. Там ро и какой-то новый 
вектор S вели себя как в законе сохранения заряда. Плотность заряда 
ведет себя во времени так, что там сохраняется и вот непрерывность. Если 
рассматривать как набор волн. Нашли случай, что квантовое описание 
приводит плавно к классическому случаю. Если взять задачу гармонического 
осциллятора, волновой пакет, оказалось он не будет расплываться. Обычно 
волны де Бройля очень сильно диспергируют, но вот в квадратичном 
потенциале нет. Плотность заряда получается вот так размазана, но можно 
дать волновой пакет, и он будет колебаться с классической частотой. 
В случае с атомом водорода получалось конечно посложнее, и тут 
выяснилось, что везде почти, кроме гармонического осциллятора, пакет 
такой разваливается. Затем возникли еще проблемы, если расширять дальше 
теорию Шредингера, рассматривать конструкцию из нескольких частиц. 
В многоэлектронном атоме можно было решать методом итераций, теории 
возмущений, как в небесной механике, но нужно было считать, что не 
в трехмерном пространстве, а вот в конфигурационном, то есть чем больше 
частиц тем больше размерностей. И как тут записать плотность заряда 
немного непонятно. Это поставило жирный знак вопроса на интерпретацию 
пси функции Шредингера. В случае свободного движения тоже проблема. Для 
дифракции частиц надо было понять тоже пси эту. Возникают проблемы 
с нормировкой, непонятно как объяснить дифракционные явления, когда 
почти свободное движение частиц. Как это объяснить с точки зрения 
размазанного заряда конечно непонятно было. На основе задач с такими 
инфинитными делами. И вот в каком-то году Макс Борн сказал, что 
у Шредингера там все верно, но как решать задачи инфинитные, рассеяние 
например? Вот налетает электрон, Анализ Макса Борна теории рассеяния он 
там применил анализ сильный, который мы называем сейчас Борновское 
приближение. Там расходящиеся сферические волны, вот теория дает такую 
вещь. Как проверить? Оказывается, что электрон может быть локализован 
там, и в дифракции и прочем возникают такие максимумы, плавные. 
В экспериментах электрон как частица локализуется. Никогда никакого 
размазанного заряда не получается. Локализуется электрон в одном только 
месте. И здесь Борн придумал идею, что пси это амплитуда плотность 
вероятности. Но это не такая вероятность как в где-то там, это ампитуда. 
Как там в оптике среднее Е квадрат, так и тут надо среднее пси квадрат. 
Решение уравнения Шредингера для инфинитного случая, имеем разные 
распределения. И в 1954 году Макс Борн за эти мысли получил Нобелевскую 
премию. Потом последовали размышления, что это за пси функция, откуда 
берутся эти вещи. Де Бройль представил одно развитие этих вещей, что вот 
частицы обладают вещами какими-то типа массы итд, а есть непрерывные 
вещи вот пси функция, это волна, там есть фазовый множитель, она связана 
с частицей, а волна такая вот управляющая получается. Затем немного 
другой подход был выдвинут после появления особых точек, сингулярностей, 
и де Бройль основывался на мыслях Эйнштейна, что вот есть импульс, 
и энергии импульс только в виде локализованных вот частиц дискретных 
величин. И классические поля как у Максвелла выглядят как средние 
значения. Потом это уже было отвергнуто в пользу новых соображений 
в конце 20-х годов, но де Бройль до конца жизни стоял на своем.

\subsection{Теория преобразований}

Основы квантовой механики хотелось бы описать. Была проблема 
одновременного описания как непрерывного спектра, так и дискретного. Для 
непрерывных состояний совершенно не получалось нормировать, интеграл от 
пси в квадрате не мог давать единицу. Это было большой проблемой даже 
при учете помещения в бесконечный ящик и стремлению его к бесконечности. 
Для инфинитных короче была проблема нормировки. Большой прогресс связан 
с теорией Дирака, который понял, что вот матрицы с дискретной нумерацией 
столбцов, матричное дело это сводится к Гильбертову пространству! 
Различные записи физических величин есть запись в Гильбертовом 
пространстве. Но это только для дискретных. Дирак предложил обобщить 
Гильбертово пространство, введя в него дельта-функцию Дирака. Удалось 
там построить формально обобщенное Гильбертово пространство, и вот на 
нем все получилось описать. Но тогда дельта-функцию не удавалось ввести 
математически корректно, говорили, что внутренне противоречиво, не 
принимали такое. Смог решить это только Лоран Шварц уже после войны, 
и тогда приняли успешно. Гейзенберг в общим не только Гильбертово 
пространство вводил, он еще получил кое-что. Он показал, что если есть 
два оператора, которые не коммутируют, то есть известные нам например, 
то между ними имеет место перестановочное соотношение, и Гейзенебрг 
доказал, что из этого перестановочного соотношения следует, что при 
расчете средних значений можно еще ввести дисперсию, некое расплывание 
среднего значения, и собственно говоря имеет место соотношение 
неопределенности Гейзенберга. Он сделал из него такие вот определенные 
выводы уже в 1928 году. Он показал, что есть лишь единственное 
состояние, когда минимальные оба они, функция Гаусса, а так в общем 
случае при уточнении одного возрастает неопределенность другого. При 
данном q получается, что все p одинаково возможны. И наоборот тоже 
понятно короче... В дальнейшем из этого принципа были очень важные 
выводы сделаны Нильсом Бором, который дал вот принцип дополнительности. 
И потом был еще один вывод сделан, то в классической механике ссылаются 
на там задачу Коши, если знаем настоящее, то будущее и прошлое строго 
определены. Связь состояний присуща только к выбранным интегрируемым 
решениям, как задача Кеплера итд. Лапласовский детерминизм там короче 
итд. Бор говорил, что принцип причинности из Лапласовского детерминизма 
говорит, что если есть настоящее, знаем будущее. Бор говорил, что 
неверно, что мы знаем настоящее. Нельзя знать строго координаты 
и импульсы.

Теперь подход Бора, который считается современной интерпретацией 
в общем. Попытки опровергнуть сталкиваются с трудностями, поэтому надо 
знать, что в 1927 году в Комо в Италии впервые формулирует принцип 
дополнительности, а потом на 5 Сольвеевском конгрессе дает сильный 
доклад. Он говорит, что есть два способа описания, для фотонов например. 
Мы не можем обойтись без какого-либо одного, нужно обязательно держать 
оба. Но прикол в том, что они взаимно исключают друг друга... Бор 
сказал, что если эксперимент макроскопический, то можно минимизировать 
воздействие прибора. А вот в микро вещах нельзя исключить действие 
измерительного прибора. Фотоны и частицы там при рассеянии меняют свои 
дела, невозможно исключить влияние, объект не допускает точного 
измерения. Нельзя рассматривать систему без измерительного прибора. 
Дальше он пошел и встретил противоречие с подходом Эйнштейна и немного 
де Бройля. Он говорил, что установить объективную реальность в плане 
конкретного описания в некоторых задачах там что-то, но объективная 
реальность есть. У частицы существует импульс. И в каком-то смысле 
существует традиционная причинность. То, что нам нужно вводить средние 
значения Эйнштейн связывал с наличием каких-то скрытых параметров, 
проводил аналогию со статистической физикой например. Газ в состоянии 
равновесия по-другому не опишем, но как бы можно проследить траектории 
каждых вещей и можно ввести такую концепцию скрытых параметров. На компе 
можно знать траекторию вот частиц. Но эти параметры там не видны. 
Эйнштейн считал, что вероятность и неопределенность связаны со скрытыми 
параметрами. А вот надо понимать, что в октябре 1927 года провели три 
доклада там как раз, де Бройль говорил о волне-пилоте, попытке 
классической интерпретации. Борн и Гейзенберг говорили о вероятностной 
интерпретацией волновой функции, что было у нас в теории рассеяния 
сказано. Имеет единственно возможная интерпретация вероятностная 
пси-функции. И Бор с принципом дополнительности. Он говорил, что пси 
функция не является полным описанием, полного описания никогда нет...

Падарокс Эйнштейна-Подольского-Розена сейчас не будем.

% }}}


\end{document}

Экзамены будут по расписанию, вывешено оно в интернете. Будет аудитория 
5-40 все время. Все традиционно. Билеты в виде бумажек. В билете 
2 вопроса из списка соответствующего. Одна общая просьба ко всем, 
настоятельная. Он хочет начинать экзамен в 9 утра. Если кто-то придет 
к 10, его тоже примут, но хотелось бы, чтобы приходили к 9. Обратите 
внимание, что типа того, я хочу сдать раньше. Специальной досрочной 
сдачи не планируется. Но если бы учебная часть попросила, сделал. Что 
можно сделать здесь. Раньше можно было в принципе мне давали пачку 
ведомостей и учебной части было все равно. Но насколько понимается 
сейчас, в учебке следят за тем, чтобы ведомости закрывались в тот же 
день. Сдавать могут раньше те, у кого в конце. Вот он может так сказать, 
есть разные по расписанию дни, максимум по 15 человек примерно. 10 14 16 
28. Если кто-то придет раньше, но в этот день будет много народа, он 
		просто физически не сможет принять в тот день. В первую очередь 
		конечно по расписанию. Он не против перестановок, но во-первых не 
		должен быть массовый характер, а во-вторых хз.

Если экзамен сдает 9 человек, а придет дополнительно заранее сдавать 
пара человек, он успеет. Если уже 15 человек пришло, то дополнительных 
он вряд ли успеет.

Часть билетов он обычно изымает. Он любит ответы на фактический 
материал. Это имеется в виду, например, Ньютон мат начала натур 
философии. Конкретно люди иногда не знают вообще даже какие задачи 
решались там в книге. Если вопрос на грани спорного, вот например, были 
попытки в 90-е годы сделать так, чтобы Пуанкаре был автором СТО, а не 
Эйнштейн. Если кто-то придерживается такой точки зрения и аргументирует 
свои слова чем-то, то ставить ему 4 не станут. Но например когда там 
скорость света была измерена впервые в Земных условиях, это надо знать. 
Астрономическими способами тоже. Последовательность событий нарушается. 
Он не публиковал список вопросов так сказать на тройку, но в голове 
у него есть. Ничего супер спорного там нет.

А вот еще он прекрасно понимает, что это не особо ключевой экзамен, но 
он хочет стимулировать всех. Он достаточно легко ставит 
удовлетворительно, если человек вообще ничего не учил, пару слов может 
сказать, но особо не осмысливает, вот ему тройку. Раньше он мог себе 
позволить попросить чела прийти там через пару дней сдать еще раз, но 
сейчас это уже невозможно реализовать. Оценки хорошо, когда человек 
готовился, отлично, когда хорошо готовился. Он старается обычный 
какой-нибудь набор вопросов иметь и снижать если человек не знает не 
только по билету, но и вот на что-то такое.

Вольные вопросы, размытые, он выкидывает как раз. 15 человек, часа 3.5 
принимать экзамен. Вот в 3 приходит человек, у него не будут принимать 
конечно.

Он еще раз озвучивает дни, где уже 15 человек. НЕ ПРИХОДИТЬ. 10 июня, 16 
июня, затем 28 июня. Какие-то должны быть аргументы на неформальном 
уровне, записочка от учебной части с просьбой пораньше принять. 
Досрочная сдача будет несистемной, к каждому человеку индивидуально.

Спасский история физики
Кун структура научных революций
Макс Джеммер кванты
