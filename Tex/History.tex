\documentclass[a4paper, 12pt]{article}
\usepackage[T2A]{fontenc}
\usepackage[utf8]{inputenc}
\usepackage[english,russian]{babel}
\usepackage[onehalfspacing]{setspace}
\usepackage{calc}
\usepackage{enumitem}
\setlist{nolistsep}
\usepackage{cite}
\usepackage{cmap}

\setcounter{tocdepth}{4}
\usepackage{xcolor}
\definecolor{allrefs}{HTML}{1010aa}
\usepackage[
	unicode,
	linktoc=all,
	colorlinks=true,
	allcolors=allrefs,
	bookmarksnumbered,
	bookmarksopen,
]{hyperref}
\frenchspacing
\righthyphenmin=2
\usepackage{indentfirst}

\usepackage{titlesec}
\titleformat{\section}
[hang] % shape 
{\bf\large\centering} % text format 
{\thesection.} % label 
{0pt} % separation 
{\hspace{1ex}} % before code 
[] % after code 

\titleformat{\subsection}
[hang] % shape 
{\bf\normalsize\centering} % text format 
{\thesubsection.} % label 
{0pt} % separation 
{\hspace{1ex}} % before code 
[] % after code 


\usepackage[
%	showframe,
	vmargin=1in,
	hmargin=.6in,
]{geometry}
\usepackage{multicol}
\setlength{\columnsep}{.2in}

\def\t{\hspace*{1cm}}
\def\bdot{\textbullet\ }
\def\dateis#1{\vspace{\baselineskip}\hfill\textbf{#1}\par}


%%%%%
\begin{document}
\begin{titlepage}
\centering
\noindent
{Князев Павел Юрьевич} \hfill
{\href{mailto:pavkneazev@yandex.ru}{pavkneazev@yandex.ru}} 
\newline 
\hspace*{0pt}\hfill Шуваловский корпус, ауд. Г449


\vfill

Контрольные и проверочные работы лучше писать хорошо. 

Вопросы удобно задавать по почте. 

Использовать для написания рефератов: \href{https://www.rsl.ru}{rsl.ru}, \href{https://www.shpl.ru}{shpl.ru}.

\vfill

\textbf{Зачет}

\parbox[t]{.7\linewidth}{
Два вопроса: один более ранний по хронологии, другой позднее. Оба вопроса надо обсудить устно. Один из будет связан с материалом реферата, если вы его присылали. 

На зачете можно пользоваться лекциями.
}

\vfill
\vfill
\vfill

\end{titlepage}

\clearpage
\tableofcontents
\clearpage


\begin{multicols}{2}

\section{Введение}
\dateis{Sep 2}

История делится на древний мир, средние века, новое время, новейшее время. 476 год с падением Римской Империи принято считать концом древнего мира. Новое время начинается в конце XV века. Открытие Амирики; Вестфальский мир -- становление системы международных отношений. Есть смысл говорить о переходных периодах между делением истории. Период раннего нового времени включает Французскую революцию, декларацию прав человека, Наполеоновские войны, появление книгопечатания, появление Венского конгресса, промышленная революция. Япония вышла из закрытого состояния из-за реставрации Мейцзи. Переходный период к новейшему времени: первая мировая война, октябрьская революция. Следующая крупная точка -- вторая мировая война, образование ООН, затем холодная война.


\section{Древнее время}

Антропогенез. 
Стадиальная теория: человек прошел стадии эволюции, которые отделяются скачками. 
Современная теория говорит, что эволюция это сетевидный многоуровневый процесс. На разных этапах могли сосуществовать несколько существ. 

Современная теория начинается с австралопитека 1М лет назад. Лучше приспосабливались более мелкие организмы. 

Затем был homo habilis, человек разумный, умел создавать некоторые орудия труда. 

Периодизация
% Как уменьшить отступы?
\begin{itemize}[noitemsep]
\item Палеолит -- древний каменный век -- 3М лет до н.э. 10к лет до н.э.
\item Мезолит -- средний каменный век -- 10-9к лет до 7к лет до н.э.
\item Неолит -- новый каменный век -- 6-5к лет до 3к лет до н.э.
\end{itemize}

Стесанный с одной стороны камень чоппер, с двух сторон чоппинг. Делались обивкой -- скалывались крупные фрагменты. Были еще другие камни, сделанные ретушью -- скалывались мелкие фрагменты. Чопперами и чоппингами работали с мясом и растениями. Перфор\'Aтор. 

Направления искусства каменного века
\begin{itemize}[noitemsep]
\item Наскальная живопись
\item Прикладное искусство
\item Мелкая пластика 
\end{itemize}

Статуэтки весьма распространены. Мужики были похожи на хищных зверей, а женщины мягко намекали на плодородие. 

В пещерах с рисунками не жили, а делали различные ритуальные преобразования, например, посвящение во взрослую жизнь. 

Далее был переход от присваивающего хозяйства к производящему. Это могло произойти из-за появления орудий, знаний, нехватки ресурсов вокруг, увеличения кочующих поселений. 

Переход к неолиту осуществился в разные времена в разных частях света. Он привел к серьезным изменениям в материальных ресурсах людей и их мировоззрении. Стали применять керамику. Появляются поселения, строятся жилища, землянки и тд. 

\dateis{Sep 16}

\subsection{Государства древнего востока} Территории: междуречье, Египет, Персия, Инки, Финикийцы, Индия, Китай. Общее у них: земледелие, они все находились в долинах крупных рек, торговля. Они дали нашему миру письменность клинопись и иероглифику. Письмо было слоговое. Клинопись в Финикие, иероглифы в Египте. После записи слогов начали записывать алфавит. Жизнь у рек подтолкнула на создание орошающих систем, которые были сложными. Именно с этим моментом связывают становление крупных государств. Различия были, например, в  том, что Нил вытянут, а у Междуречья было много рукавов. Это привело к более быстрым потокам в Ниле и меньшему числу более крупных городов. Также формированию государств способствовало: счет, календарь, право и закон. Первые законы регулировали отношения собственности, группы полноправных людей, ограничение финансовых взаимодействий и прочее. Также духовная культура развивала народ и государства. 

\paragraph{Египет} Неолитическая революция -- переход от присваивающего хозяйства к производящему. Усиливалась роль металла, появлялись ограды, социальная дифференциация, различные украшения. В итоге образовывались поселения и окружающие их мелкие поселения -- номы. В них были собственные слои, правители и прочее. Они были довольно независимы. Египетская литература обширна: поучения, пророчества, сказки, истории, настенные записи в пирамидах и других строениях. 

Древний Египет делится на:
\begin{itemize}[noitemsep]
\item Додинастический период
\item Раннее царство
\item Древнее царство
\item Среднее царство
\item Новое царство
\item Новый период
\item Эллинистический период
\end{itemize}
Царь верхнего Египта убедил царя Нижнего Египта объединиться. Корона совмещала обе части в красном и белом цвете. Это раннее царство. Развивалось использование золота и меди, возникали торговые взаимодействия с соседними странами: Нубия, Сирия, другие восточные страны. Совершенствовались техники обработки земли, они были примитивные, из дерева и камня, но металл начинал вытеснять их. Пирамиды ранних династий были ступенчатые. Власть фараона обожествлялась, он считался сыном Гора. Концепция сильной власти царя сильно встраивалась в сознание граждан. Считалось, что власть всегда творит справедливость и гармонию, а такое можно творить только при сильной власти. Четвертая династия -- Хеопс. Его пирамида была уже привычного вида. Внутренние мятежи и смуты снова разбили Египет. Среднее царство закончилось после вторжения Иксосцев. В 16 веке до нашей эры началось новое царство -- массивные прямые мечи, пластинчатые доспехи, колесницы итд. Это позволило завоевывать внешний мир. Походы в основном были на юг и восток. В период 19й династии 14-12 века до нашей эры экспансии были активными. С 11-12 века Египет подвергается вторжением Ливийцев с запада, Нубийцев с юга. Египет ослаб, в нем стали господствовать чужеземные правители. История его постепенно теряла прежнее величие, однако регулярно развивалась характерная культура в отдельных частях. Многое завоеватели также принимали Египетскую культуру, язык, мифологию. 

Египетские боги были как общие, связанные с глобальной властью, так и локальные, обожествляющие различные стихии. В частности, был бог, отвечающий за разлив Нила. Центральное святилище Амона было в Мемфисе. Египтяне писали иероглифами, но они были долгими и сложными, поэтому многие упрощали начертание. 

\paragraph{Междуречье} Отделение торговли от ремесла, постепенно формировались города с собственными храмами, культами и правителями. Это были в основном города шумеров (по происхождению этого народа и его языка до сих пор ведутся споры, очень интересная информация). Параллельно с образованием городов возникало некое общее пространство. Города боролись друг с другом. Приходили внешние завоеватели. 

Ассирия развивалась из города Ашур. Развивалось земледелие, производился виноград, ячмень. Исполняющий религиозные функции чиновник постепенно отбирал полномочия у других чиновников. Ассирия переживала подъемы и упадки. После покорения захватчиками, ассирийцы смогли переработать право, создать сильную армию, в которой существовали разведка и саперы и прочие роды войск. После мятежа удалось объединить большую часть ближнего востока. Потом начали появляться признаки упадка и сепаратизм. К концу 7 века до н э напал Вавилон. 

Нововавилонское царство было основано **деями. Одним из самых известных правителей был Навуходоносор второй. Были построены ворота и штаб. Некоторых рабов даже отдавали на обучение ремеслу, давали некое условное владение итд. В середине 6 века Персы завоевали это дело. 

Финикийцы. Взаимодействовали с морем. Основывали свои поселения: например, Карфаген (на месте настоящего Туниса). 

Древняя Индия. Инкская цивилизация, вдоль реки Инк. Не знаем на каком языке они говорили. Знаем, что была письменность, сельское хозяйство. Сюда затем пришли народы с севера, из индо-европейской языковой семьи. Затем возникали касты и прочая хренота. 

Будет проверочная работа


\dateis{Sep 23}

\subparagraph{Греция}
Переодизация истории древней Греции
\begin{itemize}[nolistsep]
\item Минойский 
 \subitem Ранееминойский (30 -- 28 века до нэ)
 \subitem Среднеминойский (22 -- 18 века до нэ)
 \subitem Позднеминойский (16 -- 12 века до нэ)
\item Микенский (16 - 12 века до нэ)
\item Гомеровский, ``темные века'' (11 -- 9 века до нэ)
\item Архаический (8 --6 века до нэ)
\item Классический (5 -- 4 века до нэ)
\item Эллинизм
 \subitem Походы Александра и становление эллинистических государств (30-у годы 4 века -- 80 годы 3 века)
 \subitem Функционирование государств (80 годы 3 века -- середина 2 века)
 \subitem Кризис и захват Римом на Западе и Парфией на Востоке (середина 2 века -- 1 век до н.э.)
\end{itemize}

В древней Греции не было одного единого государства, но было множество полисов, которые друг с другом взаимодействовали. Например, велась торговля мореплавательная. Очередной важный момент в том, что почва была каменистая, но росли хорошо виноград и подобные вещи. А вот хлеб было проще купить, чем вырастить, это послужило толчком к торговле. 

3--4 т. лет назад, когда в Египте строились пирамиды, были жители острова Крит с очень интересной культурой, многоэтажными зданиями, настенными росписями, святилищами, но не было укреплений и больших стен, значит, цивилизация была мирной. Эта культура существовала до природных катаклизмов, извержения вулкана на острове Пера?. 

Еще были Микены, у которых находятся высокие стены, драгоценности и оружия. Эта цивилизация приходит в упадок, когда наступают разные другие челы, тоже греки, но более воинственные. Это темные века, письменности не было, мы знаем об этом времени только из-за Гомера, поэтому период иногда называется Гомеровским. Была безработица, люди переходили из города в город в поисках работы. Также в этот период был перенят алфавит, который потом ляжет в основу греческого. Позднее будет еще много заимствований из восточных стран. 

Тогда же начинается греческая колонизация, которая покрыла значительную часть средиземноморья греческими полисами. Главная причина -- рост экономики и полисов, а также нужда в сырье и пище. Греки колонизировали много, в том числе юг современной Италии. Развивается судоходство, возникают новые суда. В том числе очень вместительные. Кроме того, появляются монеты. 

Спарта Совет старейшин, народное собрание, коллегия эфор? -- высший орган. Система была похожа на демократическую. Женщины и мигранты (митеки?) не решали ничего. 

Полис это большой город с плотной постройкой. Это помогало защищать. Вокруг были земли, жители которых могли укрыться за стены в случае опасности. Полисное общество не было лишено противоречий. Была не только линия богатых и бедных, но и аристократия и народ, мигранты и народ. 

После вдруг появляются Персы, которые хотят завоевать Грецию и подчинить побережье. В битвах Греки пользовались рельефом местности и разными строями. К 499 году заключается некое перемирие, Персы оставляют в покое полисы. 

Эллинизм впитал в себя как Грецию, так и восток. 

\subparagraph{Рим} Периодизация
\begin{itemize}[nolistsep]
\item Царский 8 -- 6 века
\item Ранняя республика 5 --3
\item Поздняя республика 3 -- ...
\end{itemize}

\dateis{Sep 30}

\section{Средневековье}

476 год н.э. -- падение последнего Императора на западе. 

Черты средневековья: религиозные войны, крестовые походы, теоцентризм, феодальный строй (правовая сторона: взаимодействие герцогов и вассалов, экономическая сторона: взаимодействие крестьян с остальными, социальная сторона: иерархичность и сословность (молящиеся, сражающиеся, трудящиеся). большую роль играют не материальные возможности, а личные связи). 

На Римскую Империю напали германцы, с севера они расселились до современной Германии. Затем германцев теснили гунны, с которыми позднее пришли и тюркские народы, и славянские. В 5 веке н.э. граница между варварскими племенами и римом была прорвана, варвары часто жили в городах и даже служили Императору, а также устанавливать свою власть на новых территориях. По мере продвижения вглубь Римской Империи, они погружались в культуру и связывались с населением. Происходит синтез культур. И сам Рим тоже изменился: и политическая система, и экономика, и рабство. Новых источников рабв особо не было, и чтобы они не вымерли окончательно, разрешаются рабские семьи, имущество, и постепенно изменяется все рабовладение. Раб превращается в мелкого землевладельцы. 

У германцев было необходимым знать кучу своих родственников. Еще у них было право и наказания -- штрафы. Большую роль играли советы старейшин, заключались устные договоры и соглашения, сделки. Выделяется некая знать. Постепенно появляется все большее и большее расслоение. Это происходило в частности из-за синтеза с античной цивилизацией. С одной стороны были феодалы, а с другой церковь. Римляне держались тоже по-разному, консервативные изолировались, а другие взаимодействовали с варварами и давали им религию -- христианство. Именно этот синтез был очень важен в формировании нового строя. Постепенно общество все усложняется, возникают сложные отношения. Военные функции переходят во власть над обществом и наблюдается элемент собственности -- власть над собственностью. В ходе этого преобразования огромную роль играет все унаследованное варварами от Римской Империи. Они пришли не на пустое место, а восприняли античное наследие. Этот синтез в разных регионах проходил совершенно по-разному. В Италии и средиземноморских странах существовала большая сеть Римских городов и поселений, преобладало романоязычное население. Шел активный синтез между романским и германским населением. А на севере синтеза не было, хоть и постепенно, принимая христианство народы погружались в античную культуру. 

\subsection{Византия}
Создана в 330 году с перенесением Константином столицы.
Май 1453 года -- падение Константинополя под натиском турок под руководством Мехмеда II. 

На этой территории не было такого интенсивного варварского нашествия, поэтому жители были наиболее хорошо сохранившимися римлянами. Константинополь никто не мог взять штурмом до 1453 года, но тем не мене Византия часто могла проигрывать в битвах и западным странам, и восточным. Положение было нестабильным. Внутренние конфликты и противоречия еще более осложняли происходящее. Хоть византийцы и считали себя продолжателями Рима, но в них часто развивалось Греческое ядро. Постепенно, с 7 века, греческий язык становится языком права и остальных частей жизни. В 7 веке под натиском арабов Византия потеряла восточные части, где были распространены не похожие на православие направления христианства. Византийским центом оставались Балканы и Малая Азия. 

Византийская культура очень богата. Ее истоки надо искать и в Римской, и в Восточной. Сначала император избирался, до 11 веке не было никаких устоявшихся династий. Символика императора: мешочек с землей и корона с золотым обручем и ниспадающими с него жемчужинами. 

В 11 веке начались династии. 

При переворотах императора ослепляли или убивали, а восстающему человеку после помазания прощались все грехи. 

В Византии было очень много народов, и крещенные люди были в этом равны. Некоторые императоры были армянами, но христианами. 

Армия Византии была блочным образованием. А в позднее время частенько обращались к наемникам. Большая проблема была в создании сильной конницы для противостояния тюркам. 

Константинополь содержал до миллиона человек, в десятки раз больше, чем Европейские города. В Византии были и другие города, которые я не услышал. Производство было регламентировано, были цеха с жесткими ограничениями продукции сверху. Цехи были в некоторых производствах зависимы. Верхушке цеха не давали подняться особо высоко. Только Византия могла поддерживать стандарты Римского времени. Ткани были роскошные. Есть легенда о том, как в Византию из Китая попал шелк. Византия в итоге стала монополистом. Никаких ограничений на производство и стоимость на западе не было. Сами Византийцы иногда делали уступки: европейские товары появлялись на рынках, и причем были дешевле. Это давало громадный удар в экономику. 

Византии было очень тяжело сдерживать натиск как с запада, так и с востока. Сначала со стороны арабов, в затем и востока. Затем османская угроза, с которой сражаться было жутко трудно из-за владения балканами. Хотя европейцы предприняли ряд походов для ослабления османской империи, это не особо и помогло. Нужно отметить, что после этого в 16 и 17 веках, когда турки осаждали Вену, начинается новый период отношений Европы и Османии. 

\dateis{Oct 7}
\section{Древний Китай}
Реки Ян-Цзы и Хуанхэ привычные границы восточной азии. Север и юг весьма не похожи друг на друга. К северу от Хуанхэ умеренный климат, к югу субтропический. Муссонский климат в приморских областях. На севере иногда бывают морозы, доходят суровые ветры из Сибири и пески Монголии. Монокультура на севере просо, на северо-востоке пшеница, на юге рис. Рис был весьма сконцентрирован на юге тогда, поскольку китайцы были сильно привязаны к полям. Поля требуют особый уход, и общины поддерживали эти сложные системы. На севере все усложняется не только культурой, но и соседством со степями -- кочевым миром. Им приходилось развивать сложную систему обороны и в том числе Великую Китайскую стену. 

Еще стоит отметить, что хоть Китайская цивилизация одна из древнейших, мы знаем не очень много -- из данных археологов и книг писателей. Начало китайской истории 1--2 тыс. лет до н. э. -- период Чжоу. Одна из легенд образования государства -- поднебесный отправил мудрых Яу и каких-то Шунь с потоками, потом появляется Юнь и становится правителем. 

Активно распространяются ремесла и другой промысел -- оружие и ритуальные сосуды. Находят стены жилищ, глина, вплоть до 6 метров. За ними жили ремесленники, постепенно специализирующиеся. Письменность определяют по гадательным костям. Социальная структура усложняется. Находят могилы родственников, большие, 10 метров. И обычных общинников, неглубокие. Это говорит о социальном расслоении. 

Выделяются правители, которые были еще и жрецами. Затем, в 10 веке до н э приходит династия Чжоу на смену Инь. Появляется бронзолитейное производство, колесницы, письменность. В ходе борьбы между минигосударствами Цинь победила. По многим причинам. 

5--4 век до н э -- развитие в Китае не только воюющих между собой частей, но и появление школ как религиозного, так и другого характера. В частности, учение о благородном человеке в Конфуцианстве. Идея была в возвращении к благим порядкам прошлого. Моизм -- Мо Цзы -- необходимость всеобщей любви -- противопоставление семейной любви. Легисты -- Шан Ян -- пребывает в царство Шинь, там проходит ряд реформ с введением полноценного права и подобных вещей. 

Затем становится понятно, что объединение Китая ну очень выгодно. К 221 году до н э создается централизованное государство после завершения жоской войны. Император становится полновластным и наследственным правителем страны. Проводится унификация денег -- более эффективное налогообложение. Династия Цинь просуществовала недорого. Кризис усиливался, во время этого возвысились другие Хань и Чу товарищи. Дальнейшие реформы планировали уменьшение налогов и оставление земель за теми, кто их захватил. Далее Китай сталкивается с другими захватчиками. Были и вооруженные столкновения, и попытки заключить союз.

Восстание желтых повязок. 

Династии: Сурь, Тан, Сун. 

Китайцы потерпели поражение от Арабов. Было восстание итд, даже императору пришлось покинуть столицу. 


\section{Ранние арабы}
Возникновение ислама. Коран. Сумы (означает пример или образец -- сборник рассказов о деяниях, намерениях и высказываниях Мухамеда). Надо иметь в виду условия возникновения ислама. Иран и византия воевали еще в 3 веке до н э. 

Халифы -- верховная власть после пророка.

\dateis{Oct 14}
В Галлии франки составляли 30\% населения. Романских было больше. Синтез романских и германских товарищей. Так сформировались и непонятно что, потому что лектор стал внезапно очень тихо говорить. Наследство передавалось только мужчинам. К ранним королям относились не вполне серьезно, воины могли, например, не хотеть делиться награбленным. 

Франция -- Карл Великий. Завоевал много и получил от Рима звание Императора восточной Европы, стал первым европейским императором. 

\dateis{Oct 21}
Развитое средневековье: середина 11 до 1? века. Между ранним и поздним средневековьем. 

Развитие городов и товарно-денежных отношений развило ремесленничество и отделило его от сельского хозяйства полностью. Города разрастаются концентрически, радиальная застройка. Улицы были узкие, дома были до 2-3 этажей, росли вверх, поскольку городская земля была дорогой. 

Крестовые походы. Заподно-европейские страны, крестьяне. 1096 -- 1270. Воздействие -- торговля с ближним востоком, пряности, красители, ткани. Знакомство с достижениями арабской культуры и науки -- Аристотель и другие греческие достижения. 

Социально-экономическое развитие и становление сословий. Формирование границ государств. 

Konetz.

\dateis{Oct 28}
Предпосылки географических открытий: феодализм, некуда деть некоторое дворянство (особенно актуально для Испании); сухопутные и караванные пути приходят в упадок из-за событий середины 15 века -- взятие Константинополя, османская империя, доступ к путям был закрыт; недостаток драгоценных металлов; заимствование компаса и астролябии; возникновение карт побережий. Страны пионеры: Испания, Португалия. Португальцы пытались завоевать север Африки, но конница не была приспособлена. А попытки обогнуть побережье Африки привели к достижению южного мыса Африки. Параллельные попытки достижения Индии -- Колумб. Португалец однажды достиг Индии, обогнув Африку. Испания и Португалия даже пытались осуществить некий раздел земель. Освоение земель длилось очень долго и перешло в завоевание. 

Изобретение книгопечатания. Начало резко изменившего всю культуру процесса. Стало возможным читать книги гораздо большему числу людей. Рынок книгопечатания, подготовленный средневековьем с переписыванием книг, резко расширился после печати. 

Зарождение капиталистического уклада в современной экономике. В конце прошлого тому века люди были как бы сами по себе, но имущество не было постоянным. Капитал -- совокупность материальных и нематериальных ценностей, которая используется для получения прибыли. Капитализм -- система, в которой прибыль получает владелец капитала, а работники продают свой труд за зарплату. Мануфактура -- обычное рукоделие, но с разделением труда. Мануфактура была централизованной и рассеянной -- производство сосредоточенно под одной крышей; закупка сырья и развоз его по трудящимся. Второе выгоднее. В Англии, Нидерландах и частях Германии развитие самое быстрое. 

Реформация -- пересмотр некоторых положений католической церкви. Начинается в Германии, Мартин Лютер, находился под влиянием ученых гуманистов, в 1517 году выступил открыто со своими идеями -- против продажи индульгенции. 21 октября 1517 года на дверях университетской церкви он прибил свои 95 тезисов, призванных доказать, что Папа заблуждается и все должны одуматься. В 1520 году Мартин Лютер совершенно полностью отрезает путь примирения с церковью, и тут вдруг появляется книгопечатание, все больше людей читают их и занимают свою позицию. Сами идеи: главная в том, что человек спасается только лишь верой, в этой вере он не нуждается в посредниках (Папа, священник), значит нет необходимости в них, они ничем не отличаются от мирян; число таинств от 7 урезалось до 2; библия признавалась единственным самым важным посланием, труды священников и прочее не являлись первостепенными при толковании веры; необходимо перевести библию на немецкий язык; богослужение на латыни нужно заменить на немецкие проповеди; отказаться от почитания икон, они отвлекали от существа веры, надо было вообще убрать их из церквей. Император Карл(?) призвал Лютера куда-то там для примирения, но он ответил, что не может изменить свое мнение. В разделившихся людях появлялись как обычные товарищи, так и радикальные. Говорили (Томас Люцер(?)), что надо восстановить всеобщее благосостояние в социальном равенстве, жгли замки, монастыри, делили имущество поровну, к таким людям присоединялись даже рыцари. Это восстание было подавлено через год, а предводитель казнен. Так возникло протестантство -- протест против решений Императора. Была проведена секуляризация, перевод на немецкий, заменены церкви. Далее был заключен религиозный мир -- право решать владельцам земель их религию, признание реформации. 

Другие лидеры реформации -- Жан Кальвин, странствовал по Европе, пока не осел в Женеве. Женева выросла на торговле. У него была вера в божественное предопределение -- заранее определено, кто спасен, а кто здохнет. И никак нельзя изменить это предопределение. Считалось, что мирской успех есть индикатор избранности. И церковь должна быть построена снизу, выборами. Пастор должен уметь трактовать библию, исходя из одного лишь ее текста. 

Инквизиция существовала гораздо раньше, с 13 века, Римская и Испанская. Основной целью инквизиции было раскаяние или наказание, но масштабы убийств были около 7\% или меньше я не услышал. Еще одна цель -- контроль печати, запреты книг. Реформация подстегивала религиозные чувства. Дьявол обретает в это время свой характерный вид рог и ног, представление о дьяволе унифицируются в единый и страшный образ врага. Формируется представление о триаде -- дьявол, колдовство и их комбинация ведьма. Последнюю ``ведьму'' сожгли в 18 веке. 

Возникновение новой формы монархии -- абсолютная монархия. На что она опиралась, какой был аппарат. Габсбурги -- огромные владения во многих частях Европы. Земли в Испании, Италии, Германии, Чехии, Австрии, Нидерландах, плюс колонии. Говорили, что над владениями никогда не заходит солнце. Управлять ими было очень сложно. Карл отвергается от короны и владения передаются сыновьям. В Испании Филипп, ему еще кусок южной Италии. Эти земли становятся центрами торговли и производства, и плюс еще находятся на громадных путях торговли. Кальвинизм распространялся оттуда и возобладал даже в части Европы. Карл и Филипп ответили жестко -- Карл ввел в Нидерланды инквизицию, а Филипп начал гонения на них. Многие даже крупные целы были казнены. Также дичь происходила с налогами. Образовались Гёзы -- отряды недовольных. Это движение ширилось, пока у него не нашелся лидер Вильгельм Ар какой-то из Германии. Его борьба с Испанской армией привела к хорошему финалу и свободе протестантства на севере владений. Там было решено не выбирать нового монарха, образовалась республика (Республика Соединенных Провинций), располагалась на севере современных Нидерландов. На юге Нидерландов до 1609 году прекратилась война, а в 1649 году признание Испанией протестантства. Все больше и больше Мануфактур были централизованы из-за малости городов, переход на технические культуры, приспособление хозяйства. Происходит отвоевание земель у моря. Использование земли шло интенсивно. Северные Нидерланды торговали со всеми вообще: Америка, Индия, Африка, Япония. Урбанизации и росту городов способствовала практика привлечения мигрантов (приходили из-за веротерпимости). Иммигрировали те, кто скрывался от религиозных преследований. Даже католики, хоть и не могли особо заседать в главе городов, не подвергались никаким гонениям, унижениям и так далее. Равноправия между религиями не было, было больше протестантов, но была стабильность в обществе. В Нидерландах не было особой цензуры, там даже печатали книги на других языках для стран с цензурой. 

Во Франции она(кто) приобрела неоднозначный характер -- долгие религиозные войны. Католики и букеноты(?). Королевская реформация. 

В Англии тоже всякие особенности были. 

Konetz.

\dateis{Nov 11}
Информация по экзамену придет на почту старосте, результаты контрольных тоже придут. Как-то из них будет понятно что-то по поводу оставшихся лекций. 

Государство в новое время -- отличия от средневекового. В средневековом государь дает защиту, а потом функции расширяются. Образуется чиновничество (раньше король путешествовал и проверял страну, не было столицы), появилась необходимость формирования столицы. В 17-18 веках появляется новое государство. 

Основные изменения -- разрешение конфликта между светской и церковной властью. Папа итд не претендует на полное правление. Войны становятся государственным делом. Войны становятся из личного дела королей в дело государства. Требует большего числа ресурсов, потребность в регулярной армии. Переход к сбору постоянных налогов. Сеньор теперь тратит гораздо больше на войну, государство приходит к схеме сбора налогов. Новое административное, финансовое итд правление. Структурируется власть, назначаются люди. Судебная власть тоже, она решает гораздо больший спектр споров. Еще одна трансформация о размежеваниях Европы, ослабление церкви и усиление религиозного начала в жизни людей. Церковь уже не смотрится как единое тело, все крестьяне не видят себя одним большим целым. Различные ветки господствуют в разных странах, это влияет на самосознание. Англия отстаивает свою церковь, а Ирландцы остаются католиками, опирая на религию самосознание. 

Складывался абсолютизм (это слово появляется после фр. революции для описания происходящего тогда). Современники говорили об абсолютной власти, сильной власти. Эта концепция показывала не негатив и деспотизм, а некую ограниченную власть, хоть и пределы расширялись. По концепции король действует без какого-то земного контроля. Когда правители государств начинают воспринимать императора не как что-то великое, а как одного из них, у императора пропадает власть. В абсолюте были формулировки: ``король свободен от закона'' (король, являясь сувереном своего государства, может менять законы. хоть он и ограничен чем-то), ``король в своем королевстве -- император''. 

В разных странах эволюция власти прошла очень по-разному. В Испании власть короля крепилась. В Нидерландах люди восстали и сделали республику, без какой-то крутой власти. 

Рассмотрим становление монархии в Англии. Елизавета приняла много полезных вещей: поддержка торговли, дарование хартий, право торговать в разных частях света. В 1600 году основана компания, которая торговала с востоком -- Индией. Эта экспансия стала причиной конфликта с другими странами, особенно Испанией. После в Англии появились пуритане (pure), они углубляли еще больше англ. реформу религии взамен католицизму. Карл I распускал парламент и хотел править сам, а потом в 1637 году произошло восстание, чтобы его побороть, в 1640 году появился парламент, но его быстро распустили. В ноябре 1640 года собрался еще один парламент из-за неудачи в войне. Этот парламент стремился ограничить власть короля. Постепенно формировались преобразования, упразднялись некоторые органы власти. В 1641 году парламент выступает с программой требующей ограничить власть монарха и согласовывать дела с парламентом. Король отправляется на север Англии и объявляет парламенту войну. Это гражданская война, у нее были свои крутые полководцы. В 1649 году Карл I был казнен и с этого времени Англия становится республикой. 

В то же время внутри революционного лагеря появляются расхождения. Власть все еще в руках армии. Один генерал занял Лондон и сказал вернуть монархию -- реставрация Стюартов. Сына Карла I принесли на место монарха. Он, Карл II, усиливал личную власть монарха и власть церкви и в конце даже распустил парламент. После смерти Карла приходит Яков II, который отличался от брата приверженностью католичеству вместо англиканства. Поначалу с ним были готовы мириться, потому что все надеялись на его скорую смерть из-за возраста, тогда власть бы перешла к протестантскому человеку далее. Но потом у Якова родился сын, это усилило недовольство элиты. Это позволило Вильгельму решительно двигаться в Англию и защищать протестантство. Яков, у которого не было сторонников, бежал вместе с армией. Вильгельм пришел в Лондон. Это называют мягкой революцией ( в то время слово революция означало возвращение обратно на круги своя ). 

Вильгельм произвел много изменений, в том числе в финансовой сферы. Вильгельм нуждался в деньгах из-за войн. Королю надо было брать много займов итд -- появляется выражение ``все королевское, но долг национальный''. Появились законы веротерпимости, различные тем не менее для разных религий. Формируется оппозиция и умеренная, и радикальная. В том числе появляются сторонники Якоба, стремящиеся вернуть на место его. 

После смерти Вильгельма Стюарта власть идет к Анне. 

В 1688 году возникает сильная потребность в осмыслении всех происходивших чего-то. В 1689 году идут под интересным заголовком обоснования свершенных перемен публикуется трактат Джона Локка. О распределении властей. С Локка часто отсчитывают начало эпохи просвещения. Также после работ Ньютона, успешного применения математики к описанию материальной действительности, люди думают, что можно так же успешно описать общество. 

В каждой стране просвещение имело национальную специфику. 

Приобретая новые владения в Америке, Англичане опирались на колонистов, потомков переселенцев. 

Там происходили англо-французские войны. Их называли сначала войной короля Вильгельма, потом войной короля Георга. Колонисты надеются 

\dateis{Nov 18}

Политика метрополии в отношении североамерикански колоний 1760-е годы. 
\begin{enumerate}
\item Запрет переселяться за Аллеганы 1763
\item Гербовый сбор (1765 до 1766)
\item Указы о размещении в Сев Америке англ военных соединений
\item Перевод губернаторов на содержание метрополии
\item Акты Тауншенда
\item Декларация независимости
  \subitem У Локка естественные права: жизнь, свобода и собственности.
  \subitem У Джефферсона это стремление к счастью.
  
  \subitem Также народный суверенитет. Сам народ -- источник власти. 
  
  \subitem Этот документ лег в основу законодательства законодательства -- статья конфедераций, конституция (1788)
  
\item Война за независимость 1777 год -- первое серьезное поражение. 
\item 1783 году мир между англ. и колониями. Независимость США
\end{enumerate}

Вопрос об управлении, политической системе, Конвент в Филадельфии -- спор о том, какое устройство будет у США: сторонники конфедеративного; федеративного. В ходе длительных дискуссий власти сошлись на федеративной форме. Решив проблему, выстроили в конституции систему двухпалатного представительного органа -- верхняя и нижняя (выбирается на выборах, пропорционально населению штата). Полномочия федерального правительства и отдельных штатов обсуждались долго. Налоги решили может вносить конгресс ....? Рабство было решено сохранить, но со временем статус менялся в связи с торговыми запретами. Запрещался ввоз новых рабов из Африки. 
Были также старающиеся отменить рабство люди. 

Наконец, любопытно и то, что с этого времени после завоевания независимости пошел длительный период становления и приспособления идей к практике. Даже конституцию долго модифицировали штаты (как минимум 9 из 13 были должны). Позднее, поскольку споры по конституции не утихали, было решено вносить изменения в виде поправок. Удалось нейтрализовать критику таким образом. Первые десять поправок: билль о правах. В них были гарантии основных гражданских свобод (слова, ношения оружия итд). После этого конституция уже вступила. В дальнейшем поправки остались важным элементом конституции. 

Теперь французская революция. В наше время в СССР эта революция называлась великой и буржуазной, но это только СССР. Называют фр. револ. 18 века. Взгляд следующий: спорят насколько радикальным был характер ее и ее причинах и последствиях. Неслучайно ее называли по-разному в  разные времена. Параллели проводились часто и в 20-х и в 30-х годах. 

От англ. револ. отличий много: во фр. амбиции -- порвать с прошлым полностью, у англ. стремились вернуть старое. Фр. рев. смела монархов 17-18 веков. Нет жестких хронологических рамок: кто-то начинает с первого Бурбона -- Генриха 4, кто-то с эпохи возрождения. Власть имела ряд ограничений. Однако нельзя сказать, что непосредственно до революции во Франции было все настолько плохо, что прям было необходимо свергнуть все. Численность людей увеличивалась, развивалась промышленность, конструируется ряд вещей, которыми франция была знаменита, была большая мода на францию. Французский был языком мирового общения -- ученые, интеллигенция итд. Прокладываются дороги, строятся города. Экономика развивается, но проблем в ней тоже много. Первая -- сельхоз. Дробление земли не давало прийти к крупному производству. Сохранялись обязательства, хоть и считалось, что у дворянства была лишь треть земель, порядок вызывал у крестьян раздражение. Хотя сеньор не собирал налоги, что-то сохранилось. Касательно буржуазии вещь тоже непростая. Французы под этим словом понимали не промышленников, а людей и горожан итд. Буржуа это самостоятельные горожане, платящие налоги, живущие за счет себя итд. Более того, многие дворяне тоже не были довольны. Кто-то смотрел на золотой век фр., кто-то на англичан с их ограничениями. 

Вспомним генеральные штаты -- одно сословие - один голос --. Последние они были созданы в 1614 году. Еще один элемент -- парламент. к концу 18 века они почти целиком состояли из дворян. В Париже недворян было 10\%. Еще одно противоречие -- финансовое. Идея финансовой реформы была одной из важнейших в причинах революции. Также во многих колониях франция помогала восставшим колониям, эти попытки отомстить англии уничтожили бюджет. Было непонятно чем покрыть расходы, которые были больше доходов. На место министра финансов привели нового человека, он, желая узнать мнение людей, никого не спросив, опубликовал бюджет. 

Законодательная власть самостоятельная стала. 
26 августа -- декларация прав человека и гражданина. Свобода от угнетения и подобные вещи. 

Реформы покончили с независимостью провинций. Франция делится на департаменты: 83 штуки, распределенные по географическим местам. Также реформа: раньше все снимали шляпу, а король в шляпе оставался, то теперь король приседает???. Орган выборный и избирается народом -- платящими налоги мужчинами. Граждане делились на активных и пассивных (платящих и нет). Многие думали, что это уже конец революции, страна успокоится итд. Но противоречий много, недовольство проявляются городские низы, но не бездельники, а работающие люди. Много кто короче был недоволен характером преобразования. Вертикальные кто-то там. 

Один швейцарец Марат путешествовал по европе. Сначала он писал сентиментальные романы, потом переехал в Англию, но тк они не были в восторге от советов короля, его что-то там закончилось. Также пытался принять участие в науке. В частности на публике. Но неудачно. А вот в политике все было замечательно, он был друг народа. Все бездельники депутаты жрут за счет народа => надо двинуть революцию. Радикальные товарищи писали, что надо продолжить революцию, ибо не полностью удалась. В свете этого возникает поляризация. Возникали политические клубы. Один из знаменитых заседал в монастыре каком-то. Выражают свои мнения и спорят о судьбах революции. По мере дальнейших событий клуб раскалывается по отношению к монархии. Другом клубе феньян??? заседают как раз конституционные монархи. Они считали революцию завершенной. 

В разных клубах люди сидят в разных местах в зависимости от политических взглядов: справа конституционная монархия, слева радикальная революция. Монтоньяры сидели выше всех, на горе, они сильно топили за революцию. Там и нарабатывал очки лидер Горы. Дальнейшие проблемы полностью раскалывают людей в полярные политические лагери. 

Затем короля казнят, Франция становится республикой. 

Правительство из 12 человек. Выглядел, как человек старого порядка (парик, очень хорошо одевался итд), но у него была такая репутация защитника народа, что никто не шел против авторитета. Однажды он даже сказал, что он не защитник народа, а сам народ. Был также инвалид, но очень яркий и жесткий политик. Шутили, что он одной ногой в могиле. Метаньяры начинают политику и выдают очень большой размах. Хоть он и использовался в виде устрашения. Казнили 35-40 т. человек, причем они были не особо священники или особо дворяне. Большинство как раз представители 3-го сословия. Жертвами оказались многие, кого сложно заподозрить в контрреволюционных намерениях. 1794 года летом готовились новые чистки. Террор все сильнее увеличивал масштаб свой. Тем не менее как раз этим летом часть конвента организует заговор. Сан Жунс выходит на арену, а его перебивают итд. Правительство свергает самого себя, никто не чувствует себя в безопасности. После террора к власти приходят люди, понимающие, что жить по-старому Франция уже не будет. Практику террора очень старались забыть, но полностью невозможно было. Также готовят новую конституцию для очеркивания этой эпохи. Происходит еще заговор, другие движения. Они двигают термодерьенцев что-то там сократить. Конституция писалась с главной задачей сделать исполнительную власть слабой и подвластной законодательной. Законодательное собрание тоже должно было обновляться на треть. Но это не придало стабильности. Проводят чистки. 

Было необходимым присутствие внешнего человека, который бы исправил все. Внимание пало на французских полководцев. Наполеон, считали, смог бы успокоить народ и дать правительству просто править. Но он не стал марионеткой в руках политиков, поступил по-своему, распустил всех вообще и стал править в качестве консула. В каком-то году начинает Империю, заканчивая Французскую республику. 


\dateis{Nov 25}

\section{Англия и промышленная революция}

Вторая половина 18 века, англия. Промышленный переворот. Предпосылки и черты. Влияет на политику, культуру, социум. Самая важная сфера разумеется экономическая. 

Черта и новация -- массовое машинное производство, переход к фабрике от мануфактуры. Изменение в экономике, страна перестает быть аграрной. Термин промышленный переворот появился в 19 веке. Революцией называют из-за сдвига и перемен, революция резкая обычно, а тут долго, но все же изменения очень мощные. Промышленное развитие стоит на фундаменте 17 18 веков, те, кто изучает эту эпоху, чаще всего говорят о предпосылках и причинах. Также, значение важно. Именно эта революция делает новую историю по-настоящему новой. 

Первый вопрос: почему именно англия? В 17 веке развитая страна была ??, а в начале 18 не происходило никаких изменений, а к концу 19 англия уже стала мастерской мира. Основной характер -- переход от мануфактуры к фабрике. Казалось бы, везде люди занимаются своим делом, согласуются итд. Но в фабрике используется не ручной труд, а машинный. Почему же англия? Откуда пошли деньги на переворот? Какие трансформации повлекла революция?

Вспоминаем англию конца 17 века. Здесь своя схема сельского хозяйства. Когда людей стало больше, еды оказалось недостаточно, перешли на рыночную схему сельского хозяйства. Стало выгодно вкладываться в эту отрасль. За 80 лет англия перестает быть аграрной страной, города вырастают. Как раз с точки зрения лорда, люди, переселяющиеся в города, которых становится все больше, становятся покупателями. Спрос велик -- можно поднять зарплаты. Накапливаются достаточно большие суммы, которые можно вложить и в хозяйство, и в предприятия. В селах было много нововведений: новые средства. Также англия торговала очень много, это тоже дало сильный вклад. Еще в англии все топится углем. 

Улучшались технологии получения металлов. 
Разрабатываются многие машины, двигатели, пароходы. Причем в англии, в отличие от европы, использовали не древесину, а уголь. А пароходы дали возможность перевозить больше. Железные дороги связали разные куски англии. 

В социальном плане на фоне изменений и появления пролетариата формируются идеи классовой борьбы и прочее. С бедностью много где пробовали справляться общественные организации и церкви. Эта проблема приобретает новый вид во время революции. Было много людей, готовых работать за малую плату, их занимали в промышленности. 

В остальной европе не было угля из-за природных ресурсов. 

Растет продолжительность жизни в начале 20 века. Изменяется отношение к детям. Раньше как к маленькому взрослому, относились спокойно к детской смертности, сейчас ребенка растят по-другому. В связи с ростом качества жизни, надо водить в гос. школу, делать с ним уроки и прочее. Рождаемость несколько падает, но возрастает ответственность родителей. Больше детей из-за школ общаются с другими детьми и осознают некую отдельность свою от взрослых, они начинают смотреть по-новому. Меняется структура семьи. Раньше говорили о большой семье с поколениями, тетями, дядями итд, воспринимается как производственная ячейка большого количества поддерживающих друг друга родственников. В течение 20 века побеждает малая семья. 

Меняется и рацион европейцев, становится больше мяса, почти вдвое больше. А сахара вчетверо больше. 


\section{Отношения в Европе}

Часть завоеваний и революций где-то остаются, но с другой стороны собственные системы. 1906 год плакаты вызвали недовольства множества еврогосдарств, которые с небольшой охотой присоединялись. Баланс требовал огромных затрат от франции. 

После поражений Наполеона, битвы народов под салями, окончательно меняется европа. Впитывает черты, приобретенные ей в течение непростого революционного и наполеоновского времени. Европейцы стремились к устойчивому и прочному миру, но это очень непросто. 

ЙЦУКЕНГШЩЗХЪ

ФЫВАПРОЛДЖЭ

ЯЧСМИТЬБЮ,

Франция возвращалась к границам 1790-х годов. Австрия возвращала себе 

Запрет работорговли. 

Борьба за независимость Испанских и Португальских колоний в южной америке. 

\dateis{Dec 2}

\section{Идеологии}

Национализм. До 19 века никто не считал себя в первую очередь членом нации, а не государства или чего-либо еще. Нас сейчас интересует: что представлял из себя национализм, какие трансформации он испытывал со временем, как проявлялся в разных странах и локациях. Человечество со временем осознало, что люди -- члены наций. Национальное сознание нужно было укреплять и укоренять. Ради этого работали интеллектуалы, конструировали идею и подобное. Другой подход -- нации существовали всегда, и в древности, и в средние века. Но и эти признают, что для развития наций очень важен 19 век, когда элиты выступили побудителями к национальному сознанию. Человек всегда задает один вопрос -- кто я? Возможные ответы: имя, фамилия, группа, курс, нация, гражданство. Сильнейшие конкуренты национального сознания -- религиозное, локальное (привязанность к определенной локации, локальные связи, в средневековье было очень сильно, но по мере складывания и централизованных государств, и всего остального, люди чувствуют себя все более частью большого чего-то, государства). Очередной важный период -- французская революция. Земля провозглашается национальной. Это уже формирование современного понимания. Также формируются этнические нации. 

Государство влияет пропагандой, печатью. Создавались символы государств и распространялись. Появляются на монетах и прочих делах. Например, в Англии тетку с трезубцем была. До этого у англичан был жирный чел.

Национализм был за консерватизм. 

Как изменилась европа после парижского мира, по которому россия была сильно ослаблена. Наполеон III должен был держать непростую ситуацию во Франции, где никакие люди не успокоились и никакие сторонники не ушли. Ему нужны были победы и военные успехи для авторитета и усиления собственной власти. Но как? Была неудачная интервенция в Мексику, но кончилась она поражением. А вот северная Италия, где также стоял вопрос в объединении страны, весьма привлекал Наполеона III. Там было сардинское королевство. И правителя там привлекало объединение Италии. Но добившись своей цели, Наполеон вышел из войны, оставив Сардинское королевство в очень тяжелом положении. Смущение было довольно сильным. Им воспользовались и предложили жителям присоединиться к Сардинскому королевству. И многие охотно согласились. Венеция до 1866 года оставалась под Австрийцами. Объединение оказалось успешным. Но часть земель оказалась, к сожалению, за пределами Италии. 

В Америке в 18?? (после 1854) году побеждает кандидат из республиканской партии Авраам Линкольн, южные товарищи, желавшие оставить рабов, захотели отсоединиться и образовать свою страну. Но потом образовались всякие схватки. Сначала перевес военных действий был у Южан, а неудачи Северов до 63 года были неудачи. Но потом они применили несколько мер, чтобы получить сторонников. Приняли акт, согласно которому каждый гражданин США платил 10\$, получал землю и работал на ней 10 лет, а потом получал в собственность. Потом произошла полная отмена рабства, рабы объявлялись свободными, но им не предоставлялись земли, которые они обрабатывали. Был принят закон об измене, который позволили уничтожить опасных деятелей в правительстве. 

После 1863 года ресурсы юга сильно истощились, а у севера появились люди. Итак, в 1865 году, когда были взяты крупнейшие города конфедерации, южане капитулировали, и закончилась гражданская война. Вопрос о рабстве был решен в пользу севера. Но надо было как-то интегрировать вновь юг в политическое поле. Линкольн был в 65 году убит, но воссоединять все еще надо было, и экономику тоже. До 1877 года это длилось, называлось реконструкция. Удалось интегрировать юг в американское политическое пространство, не выходя за рамки конституции. 

Нельзя также не вспомнить про объединение Германии. Земли Австрийских Габсбургов были многонациональные: чехия, венгрия, польша, румыния, хорватия, сербия. Безусловно было трудно так все приварить. Другой вариант был не трогать Австрию при объединении, объединять под началом Пруссии. ``В политике важно прослушать слова господа, схватиться за пальто и все такое''. Отто фон Бисмарк чувствовал это дело. Он был очень жестким, но во многих вопросах старался искать компромисс. Любил собак какой-то породы. Пруссия тех времен помогает России. Потом возвращает Чехии земли, принадлежащие ей по конвенции. Потом Бисмарк предлагает Австрии объединиться. В войне выступают вместе, а завоеванные земли делят между собой. Бисмарк видит, что надо нанести удар по австрии. Повод появился в конце ???? года, когда Австрия хотела получить Ватикан. Пруссия предложила помощь Италии. Ситуация для Австрии была не самая удачная, всем было на нее пофигу. Но другие страны радовались этому и ждали затяжную войну, а она оказалась быстрой. Пруссия победила в июле 1866 года. Австрия после этого отказалась от гегемонии на земли Германии. Потом возникла дуалистическая монархия Австро-Венгрия. Венгрия получила значительную автономность. 

После вывода Австрии из игры оставался еще один потенциальный противник -- Франция с Наполеоном III. Она хоть и упустила свой шанс, но может препятствовать объединению Германии. Происходят сложные маневры, в которых Бисмарк в столкновениях с Францией планирует переиграть Наполеона в политике, объявив агрессивной страной. В 1870 году Испанское правительство приглашает на вакантный престол принца Леопольда из Пруссии. Франции это не нравится очень. Бисмарк решает воспользоваться и спровоцировать конфликт с Францией. Наполеон считал, что война будет полезна для его престижа и вообще раз надо продолжать давление на Пруссию, надо что-то там с престолом. 

В начале сентября 1899 года Французская армия что-то там. Итог войны очень плачевный. Франция провозглашается республикой, там начинается движение с Парижской коммуной. Коммуна пала, но чуть ранее подписанный во Франкфурте мир между Германией и Францией, полностью описывал новые дела. Пруссия стала Германской Империей, но еще не совсем полностью были объединены территории. Более того, во Франции должны были быть войска. 


Теперь все-таки идеологии. Большинство идеологий что-то там. Важно, что их развитие происходит в период промышленной революции. Многие считали индивида выше общества, что приносит благо одному, приносит благо всем. Идея индивидуальной чего-то стали использоваться в сфере способствования промышленности и служила мотивации. Либералы говорили, что жизнь -- постоянная конкуренция, а государство должно обеспечить равные возможности и прочее. Именно при таком подходе рациональные обоснования получали и гуманические тенденции, так и что-то ещ. Либерализм безусловно эволюционировал. В основном по отношению к порядку вещей. Сначала к торговой свободе, а потом уж вообще к необходимости изменений. Век реформации -- государство должно провести реформы и сделать уступки буржуазии, чтобы сохранить социальный строй и сохранить прогресс. Важным и крутым становится Марксизм, сформулированный в 1850х. В 60з создают первый интернационал. Важно, что, если для либералов важна идея прогресса, улучшающий качество жизни, марксисты говорят, что в капитализме заложены противоречия и должен быть решительно устранен для уничтожения неравенства. Но и Марксизм сильно эволюционировал, особенно после смерти Маркса и Энгельса. Например, считали, что мир после них уже изменился, поэтому надо не жестко, а просто создавать социалистические партии и действовать там. Были и сильные оппоненты таких, как Ленин. Он конечно считал, что мир изменился, но все равно необходима решительность и революция. 

В Англии был кто-то, который противопоставлял Англию Франции. В Англии было все мягко, а во Франции жестко. На первом месте была собственность. Сохранять четкие ограничения для сохранения ценностей и порядка вещей в обществе. Реформы с сохранением преемственности. 

\dateis{Dec 9}

В следующий раз будут произнесены итоги контрольных и определена дата зачета.

\section{XX век}

\subsection{Колонии}
В прошлом веке Британия была одной из самых крупных и мощных держав, но к концу столетия имели потенциал для торможения развития, поскольку другие страны, прежде не имевшие промышленность, строили все на новом, а в Британии должны были пользоваться старым. Для этих других стран было все проще, они быстро догнали, внедряя новые и прогрессивные технологии, и все чаще и чаще на товарах виднелись надписи made in Germany. И качество этих товаров было настолько крутым, что надпись иногда подделывали англичане. Также были и другие стремящиеся выбиться в лидеры державы, как экономически, так и политически, военно, колониально. В колониальном деле новым игроком стали объединившиеся Германия и Италия. Они часто брали не очень выгодные земли, но существенно их закрепляли за собой. В рамках венской системы было не всегда просто решить возникающие проблемы.

В колониальном деле в Англии были идеи построить железную дорогу, соединяющую что-то там. Потихоньку продвижение вглубь африки осуществлялось. В африке еще была Франция. Державы сталкивались и между собой, и с местными. Англия была еще в современной Канаде. 

В США, Нидерландах и даже в России ехали на войну добровольцы, войну с Англией где-то в Африке, вроде. Постепенно Англия начинает выходить из блестящей изоляцией, например, заключает союз с Японией. Именно Япония из-за реставрация Мей-цзы -- вестернизации. Принятия культурных инноваций, одежды, конституции. Япония смогла тоже перейти к промышленности и, не без помощи Англии, стать очень крутой в этом. В частности это привело к нашему поражению в Русско-Японской войне. Другие страны дальнего востока не могли такое делать, поскольку было много факторов к этому не располагавших. Китай становится в одно время каким-то там чем-то. Британия не просто выходит из изоляции, но и пытается договориться с основными соперниками -- Францией и Россией. Были попытки урегулирования колониальных вопросов по поводу средней азии, например. 

\subsection{Первая мировая война}
В Европе были ссоры и не по поводу колоний. Из-за балканского полуострова, в частности. Там были и национально освободительные движения, и европейские интересы. 
Осень 1912 года -- восстание албанцев против Турецкого господства. Против Османской Империи идут Сербия, Болгария и кто-то еще. Летом 1913 года против Болгарии началась еще одна война, она проиграла. За этими войнами стояли большие Европейские страны. Болгарию поддерживала Германия, например. 

У Германии был план убить Францию и Россию по одиночке. Первая мировая война последовала за Австро-Венгерским ультиматумом. Постепенно расширяется круг участников войны. Три фронта -- балканский, еще, еще какой-то. Болгария и Италия, Россия потом из войны уходит, но в 1917 году на смену ей пришел еще кто-то. Это война и правда мировая, потому что велась между многими товарищами, и даже не только в европе, а еще в колониях. 

Война была на истощение, где-то это привело к революции, где-то удалось обойтись уступками и мерами жесткой экономики, но к тому моменту, когда в июле 1918 года союзники переходят в наступление, державы были истощены. Больше всего, конечно, были истощены Германия и Австро-Венгрия. Последняя вообще распалась, а кайзер Германии сбежал в Нидерланды. 

После всех дел созывается конференция для обсуждения нового устройства Европы. В обсуждении не участвовала Россия. 

В Европе на руинах Империй образовалось множество государств. Чехия, Польша, много кого-то там еще. Османская империя будет поделена дальше. В Турецком случае была конкуренция в конце октября 1918 года и пала под силами каких-то стран, был какой-то мир и соглашение. По Сербскому договору (заключен тоже во Франции). Что-то уходило Греции. В Анкаре в апреле 1919 года собирается что-то там во главе с челом, который бросает вызов договору. Он вытисняет грецию с малой азии и добивается того, чтобы город называли Стамбулом. Вскоре после османской империи устраняется и халифат. Запрещается многоженство, переходят на западный календарь и одежду. 

\subsection{Мировой экономический кризис}

Причины: 
\begin{itemize}
\item первая мировая война и действие стран-победителей (зависимость мира от Америки)
\item Кризис перепроизводства 
\item Безработица
Что-то еще 
\end{itemize}

Глубина -- сильное сокращение роста производства. Кризис стал глобальным и отразился даже на Британии. Продолжительность очень большая даже несмотря на выработку мер по началу кризиса, до начала Второй мировой даже не удалось выйти. 

Мы не успеем рассмотреть на многие способы выхода из кризиса, модели, но посмотрим на происходящее в затронутых странах -- как развивались нацистская Италия и Германия. 

Вспомним в чем черты тоталитарных государств. Диктатура с доминирующей партией и без оппозиции. Партия стремится к сращиванию себя с государством. Поглощения аппарата. Безраздельной власти. Еще одна черта -- унификация жизни, достижение единообразия. Наличие сильного крутого вождя. Что нужно хорошо помнить: такие режимы основывались на своих моделях и стремились перестроить общество по ним, что часто ломало общественные принципы; они апеллировали к массам и призывали к мобилизации, слово массы чрезвычайно важно; исключительность нации; попытка создать режим экономической самодостаточности. 

Несправедливый мирный договор относительно Италии -- она претендовала на Югославию, и даже что-то получила, но хотела больше. И еще претендовала на те же прибрежные земли. Поэтому, несмотря на то, что Италия была победителем в первой мировой, в ней образовывался фашизм. История была связана с Бенито Муссолини -- он был в социалистической партии, но идеи несколько отличались от нее, потому что там считали, что в войну вмешиваться не нужно. 

Фашисты представляли собой мелкую партию, и постепенно набирались и люди, и идеи. Они критиковали и стран победителей, отсутствие мощи у парламента и прочее. Фашисты не просто активно пропагандировали, но и создавали боевые единицы -- отряды самообороны. Они Служили в том числе для запугивания. Поход в Рим был, после него Муссолини возглавил власть. Он в 20х годах изменял избирательную систему, и подстраивал многое другое под фашистов. 

Был однако критик фашизма, он указывал на какие-то вещи там, а потом его похитили и убили. Это увеличило антифашистские движения, но не чересчур. 

Потом вообще все места заняли фашисты, остальные партии распустили. Ввели тайную полицию и остальное. 

Фашисты хотели достичь классового чего-то путем создания корпораций, которые объединяли предпринимателей и рабочих. 

Во внешней политике Муссолини пытался создать Империю. Несмотря на некоторые успехи в колониях, Италия не была достаточно мощной пошла на мир с Германией. Не стоит забывать здесь и то, что их объединяло антикоммунистическое направление. 

Касательно Германии тоже видим неудовлетворенность регулированием, и территориально все стало плохо. Еще в Германии очень сильно действовал кризис. Он сильно ударил по социальной сфере. Правительство стремилось преодолеть это, сокращая пособия по безработице и прочие дела. Причины успеха фашизма: 
\begin{itemize}
\item централизованная и жесткая структура и внутрипартийная дисциплина, принцип фюрерства, вождизма, пресекались все попытки; 
\item программа, несмотря на противоречивость, очень привлекала людей, во-первых критикой версальского мира, люди считали, что именно эта партия способна вывести страну из дел; 
\item идея расового превосходства немцев; 
\item единство разных социальных групп, нация возводилась в высшую ценность.
\end{itemize}

Единоличное лидерство в стране. На фоне кризиса виден рост популярности партии. В 1932 году на второй срок переизбирался генерал первой мировой, довольно яркое олицетворение стабильности. Он, конечно, побеждает, но довольно близко и Гитлер. Также были коммунисты, но у них неоч хорошо. В январе 1933 года Гитлер что-то там делает важное. Когда это происходит, когда соглашается президент распустить рейхстаг, эти выборы  должны были принести власть в стане. В это время начинается активное давление на другие партии. 
Потом Гитлер объявляет себя главой. 

Во внутренней политике государство создает мощную сферу урегулирования. Создание отраслевых корпораций, объединяющих корпорации. Они контролировались партией и государством. Они могли устанавливать порядки экономического регулирования. Очень выгодным было для фашистов введение всеобщей трудовой повинности, и для мужчин, и для женщин. В достаточно короткий срок ушла безработица, строились дороги и всякие другие вещи для войны. 

Потом Германия заявляет, что не считает себя обязанной исполнять какие-то правила мирного договора. И даже ставит свои условия. Снова вводится всеобщая воинская повинность. Германия начинает готовиться к войне. 

Но почему Версальская система приводит в такое сложное и противоречивое состояние? Четыре. 1 конфликтность из-за экономического кризиса. 2. Возникновение в нескольких местах желания агрессивно пересмотреть условия мира. Это еще и Япония. Япония говорит, что не признает условия и выходит из лиги наций. 3. 4. В Испании вспыхивает гражданская война. Англия и Франция поддерживают восстания, а СССР наоборот. 


\dateis{Dec 16}

Периодизация идет по боевым действиям в Европе. С 39 до 1 июня 41 (странная война сен 39 до июня 40). От контрнаступления до до ноября 42 года. Потом период коренного перелома и после с 44 года до 9 мая 45. Период окончательного . С 9 мая 45 по 2 сен 45 -- тогда капитулировала Япония. 6 и 10 августа -- сброшены бомбы на Хирасима и Нагасаки. 

Важные черты: дипломатика и сопротивление. Декларация объединенных наций (янв 42) положила начало функционированию ООН. Конференции большой тройки -- тегеранская (ноя-дек 43), ялтинская (фев 45), потсдам (июль-авг 45). Движение сопротивления: оно делилось на пассивное и активное. Эмигрировавшие деятели сопротивления, забастовки, пассивная помощь. Активные занимаются в раздаче листовок, шпионстве или даже партизанстве. Разные были и ресурсы. Где-то правительство, где-то национальные объединения типа внутри стран. В послевоенное время в Европе представители сопротивлений смогли добиться многого и обладали авторитетом. Коллаборационисты наоборот были либо наказаны, либо не имели ничего особого. После победы над фашизмом сторонники тех теорий недолюбливались.

Итоги: большинство государств понесли очень большие потери (всего 40.8М человек, у Германии 13М, что 40\% населения Германии); начало распада колониальной системы (Британские: индия и пакистан, африка, азия, Франция: большинство колоний обрело независимость, Италии: перешли к союзникам, а затем отдана свобода). В Африке с 1949 до 1980 года освободились почти что все территории. В 1960 очень большая территория освободилась, поэтому 1960 -- год Африки. Между странами в рядах правителей, в элитах, стал вопрос -- по какому типу они пойдут, что будут делать? Могли участвовать в интеграции с Западом и развивать капитализм, а другой -- принятие социализма и ориентация на СССР, а третий -- неприсоединение, стремление отдалиться. Колониальная система распадалась не только за счет предоставления независимости, но и в ходе колониальных войн. Индонезия. Алжир. 

У Франции проблем хватало -- во Вьетнаме? терпят поражение, признают независимость. Внутри страны тоже были большие потери в экономике, а еще и политическая система была шаткая. Было много партий, за власть боролись и социалисты, и коммунисты. Правые партии тоже были, в частности голлисты, поклонники Шарля де Голя. Шарль де Голь пересмотрел конституцию. В новой центром политической системы был президент с широкими правами в исполнительной ветви, а избирается на 7 лет довольно сложной системой. Франция выходит из военной организации НАТО, оставаясь только в политической. Из Парижа в Брюссель переезжает штаб НАТО. де Голь не раз подчеркивал необходимость европейской интеграцией, но не тупое объединение. Особая роль государства в экономике важна для экономической интеграции. История подошла к концу из-за студенческих беспорядков и забастовок рабочих и служащих, нужны были реформы. Де Голь распускает парламент и проводит новые выборы в июне 68 года. Там уступает партии Порядок. 


Судьба Германии. Она была разделена на зоны оккупации. ФРГ -- Британская, Американская и Французская зоны. СССР занял восток. На западе во власть встает христианский демократ, сторонник синтеза и западных демократических ценностей. Социальное хозяйство -- экономическое чудо. Власть западной не отрицала разъединения Германии, но пыталась показать, что у них путь круче, и таким образом объединить. 

Другие лидеры. Латинская Америка. Испания проиграла все войны против колоний и лишилась территорий. Одним из довольно больших островов там была Куба. Там было очень много сахара, 90\% него шло в Америку. Революция из-за генерала с диктатурой. Возглавляет ее Фидель Кастро. Его основная цель -- ликвидация зависимости от иностранного капитала. После захвата столицы происходят преобразования. 

Диктатура скорее возникла не вдруг, а созрела внутри. Во внешней политике были хорошие отношения СССР и контракты на закупку итд. Проводится аграрная реформа, поддержка крестьянства. 

Показательно Было бы рассмотреть еще и страны, которые укрепили вектор развития на Штаты. Надо помнить Чили. Левое правительство. 


Британия. Постепенно в европу присоединяются страны не входившие ранее. Британии преграждал путь Голь. В Британии были консерваторы и лейбористы. Каждая партия в общем-то имеет преемственность. Черчилль не отменял дела лейбористов. Обществу все больше не нравились лейбористы в 70х годах, появлялась оппозиция. Растет инфляция, нужен контроль за денежной массой. В защиту всех дел выступает Маргарет Тетчер. Тогда все ждали разворотов, как делали предшественников, но он не хотела так. Она долго оставалась непонятной всем, никто не знал, на что способна. Она бросает вызов бывшему представителю консервантов. Выигрывает 130 на 119. Она называет свои намерения неоконсерватизм. Заключается в том, что нужно сократить социальные расходы, контроль над денежной массой. Не отказ государства от экономики, оно из экономики не уходит, но в ряде сфер передает дело бизнесу. Все остается под идеей, что ИП есть двигатель экономики. Ей удалось сократить инфляцию. С профсоюзом жестко. Все стачки были запрещены. Преимущество при приеме на работу для членов профсоюза. Потом она ввела плату за право голосовать -- налог. Это не понравилось многим в консервах, и ей пришлось сойти. 


%%% 
\subsection{Страны востока}
Корея была разделена на северную и южную. Южная строила капитализм, а северная социализм. В Китае была гражданская война. Мао Цзы Дун. Большой скачок и культура революции. с одной стороны он смог соединить китай под своей властью. тайвань взял. В 76 году со смертью Мао прагматики Преобразования прагматизма? Сяо Пин. Политика реформ и от что-то. Реформировался не социалистический строй, а экономическая система. Раньше было очень централизовано и спланировано наперед. Решили сделать рынок,  но с глобальным контролем сверху. 




















































\end{multicols}














































































\cleardoublepage
\hspace*{0pt}
\cleardoublepage
\appendix
\section{smith}
smith smith smith smith smith smith smith smith smith smith smith smith smith smith smith smith smith smith smith smith smith smith smith smith smith smith smith smith smith smith smith smith smith smith smith smith smith smith smith smith smith smith smith smith smith smith smith smith smith smith smith smith smith smith smith smith smith smith smith smith smith smith smith smith smith smith smith smith smith smith smith smith smith smith smith smith smith smith smith smith smith smith smith smith smith smith smith smith smith smith smith smith smith smith smith smith smith smith smith smith smith smith smith smith smith smith smith smith smith smith smith smith smith smith smith smith smith smith smith smith smith smith smith smith smith smith smith smith smith smith smith smith smith smith smith smith smith smith smith smith smith smith smith smith smith smith smith smith smith smith smith smith smith smith smith smith smith smith smith smith smith smith smith smith smith smith smith smith smith smith smith smith smith smith smith smith smith smith smith smith smith smith smith smith smith smith smith smith smith smith smith smith smith smith smith smith smith smith smith smith smith smith smith smith smith smith smith smith smith smith smith smith smith smith smith smith smith smith smith smith smith smith smith smith smith smith smith smith smith smith smith smith smith smith smith smith smith smith smith smith smith smith smith smith smith smith smith smith smith smith smith smith smith smith smith smith smith smith smith smith smith smith smith smith smith smith smith smith smith smith smith smith smith smith smith smith smith smith smith smith smith smith smith smith smith smith smith smith smith smith smith smith smith smith smith smith smith smith smith smith smith smith smith smith smith smith smith smith smith smith smith smith smith smith smith smith smith smith smith smith smith smith smith smith smith smith smith smith smith smith smith smith smith smith smith 













































\end{document}
