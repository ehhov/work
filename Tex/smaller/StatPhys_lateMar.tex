\documentclass[a4paper, 12pt]{article}

\usepackage[utf8]{inputenc}
\usepackage[T2A]{fontenc}
\usepackage[english, russian]{babel}

\usepackage{mathtools}
\linespread{1.3}
\usepackage[margin=1in]{geometry}

\usepackage{xcolor}
\definecolor{dimblue}{HTML}{1010aa}
\usepackage[colorlinks=true, allcolors=dimblue]{hyperref}

\let\straightGamma\Gamma
\let\Gamma\varGamma
\let\intorig\int
\def\int{\intorig\limits}
\def\d{\,\mathrm{d}}
\def\bigt{t \gg 1/\Gamma}

\begin{document}

\noindent
Грибков Д., Гусейнов К., Кравченко В. 
\hfill
\today
\vskip\baselineskip

\noindent\textbf{\large Вопрос}
Показать, что при $\bigt$ $\overline{(x-x_0)^4}\sim t^2$.

Пусть на систему с вязким трением действует сила $F(t)$, тогда она подчиняется уравнению Ланжевена
$$ \dot{p} + \Gamma p = F(t). $$
При начальном условии $p(0) = p_0$ решение для $p$ будет 
\begin{equation}
p(t) = p_0\, e^{-\Gamma t} + \int\limits_0^t F(t')\, e^{-\Gamma(t-t')} \d t'.
\label{eq:p_solution}
\end{equation}

Для задачи на определение $x(t)$ имеем
$$
\left\{
\begin{array}{l}
\dot{x} = p/m, \\
x(0) = x_0.
\end{array}
\right.
\Rightarrow
\ \ 
x(t) = x_0 + \frac{1}{m}\int_0^t p(t')\d t'
$$
%
\begin{equation}
\begin{aligned}
x(t) &= x_0 + \frac{p_0}{\Gamma m}\left(1-e^{-\Gamma t}\right) + \frac{1}{m} 
\int_0^t\!\d t' \int_0^{t'}\!\d t'' F(t'') e^{-\Gamma (t' - t'')} \\
 &= x_0 + \frac{p_0}{\Gamma m}\left(1-e^{-\Gamma t}\right) + \frac{1}{m} 
 \int_0^t\!\d t'' \int_{t''}^t\!\d t' F(t'') e^{\Gamma(t'' - t')} \\
 &= x_0 + \frac{p_0}{\Gamma m}\left(1-e^{-\Gamma t}\right) + \frac{1}{m} 
 \int_0^t\!\d \tau F(\tau)e^{\Gamma\tau}\frac{e^{-\Gamma\tau} - e^{-\Gamma t}}{\Gamma} \\
 &= x_0 + \frac{p_0}{\Gamma m}\left(1-e^{-\Gamma t}\right) + \frac{1}{\Gamma m} 
 \int_0^t\!\d \tau F(\tau)\left(1 - e^{-\Gamma(t-\tau)}\right) \\
 &= \tilde{x} + \int_0^t F(\tau)\,g(\tau)\d\tau \text{ -- введем сокращенные обозначения.}
\end{aligned}
\label{eq:x_solution}
\end{equation}

Пусть $F(t)$ удовлетворяет 
$$ \overline{F}(t) = 0, \quad 
\overline{F(t_1)\,F(t_2)} = C_2\,\delta(t_1-t_2), \quad
\overline{F(t_1)\,F(t_2)\,F(t_3)} = C_3\,\delta(t_1-t_2)\,\delta(t_1-t_3), \quad
$$
$$
\begin{aligned}
\overline{F(t_1)\,F(t_2)\,F(t_3)\,F(t_4)} =\ & C_4\,\delta(t_1-t_2)\,\delta(t_1-t_3)\,\delta(t_1-t_4) 
+ C_2^2\Big(
\delta(t_1-t_2)\delta(t_3-t_4) \\& + \delta(t_1-t_3)\delta(t_2-t_4) + \delta(t_1-t_4)\delta(t_2-t_3)
\Big),
\end{aligned}
$$
где $C_i = \mathrm{const}$. Тогда $\tilde{x} = \bar{x}$. Поскольку $\bar{x}$ по отношению к усреднению -- просто число, 
\begin{equation}
\overline{(x-x_0)^4} = \overline{(x-\bar{x} + (\bar{x}-x_0))^4} = \overline{\Delta x^4} + 4 \overline{\Delta x^3}\,(\bar{x} - x_0) + 6\overline{\Delta x^2}\,(\bar{x} - x_0)^2 + (\bar{x} - x_0)^4, 
\label{eq:x-x0}
\end{equation}
где $\Delta x = x - \bar{x} = \int_0^t F(\tau)g(\tau)\d\tau$. 

Для определения констант $C_i$ рассмотрим центральные моменты $p$ в предположении, что при $t\to\infty$ выполняется распределение Максвелла. Согласно распределению Максвелла, 
$$\overline{\Delta p^2} = m\theta, \quad
\overline{\Delta p^3} = 0, \quad  
\overline{\Delta p^4} = 3\,\overline{\Delta p^2} = 3m^2\theta^2. $$
Согласно уравнению~(\ref{eq:p_solution}), $\Delta p = \int_0^t F(\tau) e^{-\Gamma(t-\tau)}\d \tau$,
$$ \begin{aligned} \overline{\Delta p^2} &= 
\overline{
\int_0^t\!\d\tau_1 \int_0^t\!\d\tau_2 
F(\tau_1)F(\tau_2) e^{-\Gamma(2t - \tau_1-\tau_2)}
} = 
\int_0^t\!\d\tau_1 \int_0^t\!\d\tau_2 
\overline{F(\tau_1)F(\tau_2)} e^{-\Gamma(2t - \tau_1 - \tau_2)} \\
&= C_2 \int_0^t e^{-2\Gamma(t-\tau)}\d\tau = 
C_2 \frac{1 - e^{-2\Gamma t}}{2\Gamma} \xrightarrow[t\to\infty]{} C_2/(2\Gamma)
\quad\Rightarrow\quad
\underline{C_2 = 2 \Gamma m\theta}, 
\end{aligned}$$
$$\begin{aligned}
\overline{\Delta p^3} &= 
\int_0^t\!\d\tau_1 \int_0^t\!\d\tau_2 \int_0^t\!\d\tau_3
\overline{F(\tau_1)F(\tau_2)F(\tau_3)}
e^{-\Gamma(3t - \tau_1 - \tau_2 - \tau_3)} = C_3 \frac{1-e^{-3\Gamma t}}{3\Gamma} \to \\
&\xrightarrow[t\to\infty]{} C_3/(3\Gamma) \quad \Rightarrow \quad \underline{C_3 = 0},
\end{aligned}$$
$$\begin{aligned}
\overline{\Delta p^4} &= 
\int_0^t\!\d\tau_1 \int_0^t\!\d\tau_2 \int_0^t\!\d\tau_3 \int_0^t\!\d\tau_4 
\overline{F(\tau_1)F(\tau_2)F(\tau_3)F(\tau_4)}
e^{-\Gamma(4t - \tau_1 - \tau_2 - \tau_3 - \tau_4)} 
\\ &= C_2^2\cdot 3\left(\frac{1-e^{-2\Gamma t}}{2\Gamma}\right)^2 + 
C_4 \frac{1-e^{-4\Gamma t}}{4\Gamma} \xrightarrow[t\to\infty]{} 3m^2\theta^2 + C_4/(4\Gamma) 
\quad \Rightarrow \quad
\underline{C_4 = 0}.
\end{aligned}$$

Располагая значениями $C_2 = 2\Gamma m \theta$, $C_3 = 0$, $C_4 = 0$, посчитаем центральные моменты $x$. Согласно уравнению~(\ref{eq:x_solution}), $\Delta x = \int_0^t F(\tau)g(\tau)\d\tau$,
$$\begin{aligned}
\overline{\Delta x^2} &= C_2 \int_0^t g(\tau)^2 \d \tau = 
\frac{C_2}{\Gamma^2m^2}\int_0^t\left(1-e^{-\Gamma(t-\tau)}\right)^2\d\tau =
\frac{C_2}{\Gamma^2m^2}\int_0^t\left(1-e^{-\Gamma\tau}\right)^2\d\tau =
\\&= 
\frac{C_2}{\Gamma^2m^2}\int_0^t\left(1-2e^{-\Gamma\tau}+e^{-2\Gamma\tau}\right)\d\tau =
\frac{C_2}{\Gamma^2m^2}\left( t - 2\frac{1 - e^{-\Gamma t}}{\Gamma} + \frac{1-e^{-2\Gamma t}}{2\Gamma} \right) \to \\
&\xrightarrow[\bigt]{} \frac{2\theta t}{\Gamma m},
\end{aligned}$$
$$\begin{aligned}
\overline{\Delta x^3} = C_3 \int_0^t g(\tau)^3\d\tau = 0,
\end{aligned}$$
$$\begin{aligned}
\overline{\Delta x^4} = 
3C_2^2 \left(\int_0^t g(\tau)^2\d\tau\right)^2 + C_4 \int_0^t g(\tau)^4\d\tau 
\xrightarrow[\bigt]{}
\frac{12\,\theta^2t^2}{\Gamma^2 m^2}.
\end{aligned}$$
Также, согласно уравнению~(\ref{eq:x_solution}), 
$$\bar{x} - x_0 = \frac{p_0}{\Gamma m}\left(1 - e^{-\Gamma t}\right) \xrightarrow[\bigt]{} { p_0 \over \Gamma m}.$$
Вспоминая уравнение~(\ref{eq:x-x0}), запишем 
$$
\overline{(x - x_0)^4} = {12\,\theta^2t^2 \over \Gamma^2m^2} + 6\,{2\,\theta t \over \Gamma m}\,{p_0^2 \over \Gamma^2 m^2} + \frac{p_0^4}{\Gamma^4 m^4}.
$$
При $p_0=0$ 
$$\overline{(x-x_0)^4} \sim t^2.$$
Также, если ввести медленно меняющуюся внешнюю силу, то при $\bigt$ 
$$\bar{x} - x_0 = u_0 t = -{1 \over \Gamma m} {\partial U_\text{внеш} \over \partial x}\cdot t,$$
$$
\overline{x - x_0} = {12\,\theta^2t^2 \over \Gamma^2m^2} + 6\,{2\,\theta t \over \Gamma m}\,{u_0^2 t^2 \over \Gamma^2 m^2} + \frac{u_0^4 t^4}{\Gamma^4 m^4}.
$$
Нас интересуют малые времена $t$, при не слишком большой внешней силе слагаемыми со степенями $t$ больше $2$ можно пренебречь. Тогда, снова, 
$$\overline{(x-x_0)^4} \sim t^2.$$


\vskip\baselineskip
\noindent\textbf{\large Задача}\\
Из уравнения Ланжевена получить $a(t)$. \\
Найти $\bar{a}(t)$ и рассмотреть случаи $t \ll 1/\Gamma$, $t \gg 1/\Gamma$. \\
Найти $\overline{\Delta a\,\Delta x}\,(t)$ и рассмотреть случаи $t \ll 1/\Gamma$, $t \gg 1/\Gamma$. 
\vskip.5\baselineskip

\noindent\textbf{\large Решение}\\
Уравнение Ланжевена с начальными условиями:
$$ 
\left\{
\begin{array}{l}
\dot{p} + \Gamma p = F(t) \\
p(0) = p_0
\end{array}
\right.
\Rightarrow$$ 
$$ p(t) = p_0\, e^{-\Gamma t} + \int\limits_0^t F(t')\, e^{-\Gamma(t-t')} \d t',$$
$$ a = \dot{p}/m, \qquad p_0/m = v_0 \quad \Rightarrow $$
$$\boxed{ a(t) = -v_0\,\Gamma e^{-\Gamma t} + F(t)/m - \Gamma/m\int\limits_0^t F(t')\, e^{-\Gamma(t-t')}\d t'. }$$

Поскольку мы полагаем $\overline{F}(t) = 0$, 
$$\boxed{
\bar{a}(t) = -v_0\,\Gamma e^{-\Gamma t} = 
\left\{
\begin{array}{ll}
-v_0\Gamma, & t \ll 1/\Gamma, \\
0, & t \gg 1/\Gamma. 
\end{array}
\right.
}$$

Как уже известно и как только что получено, 
$$ \Delta x = {1 \over \Gamma m} \int_0^t F(\tau) \left(1 - e^{-\Gamma(t-\tau)}\right)\d\tau, $$
$$ \Delta a = F(t)/m - \Gamma/m \int_0^t F(\tau)e^{-\Gamma(t-\tau)}\d\tau. $$

$$ \begin{aligned}
\overline{\Delta a\, \Delta x} =&\  
{1 \over \Gamma m^2} \int_0^t \overline{F(\tau)F(t)} \left(1 - e^{-\Gamma(t-\tau)}\right)\d\tau - \\
&-
{1 \over m^2} \int_0^t\!\d\tau_1 \int_0^t\!\d\tau_2
\overline{F(\tau_1)F(\tau_2)} 
\left(1 -e^{-\Gamma(t-\tau_1)}\right)
e^{-\Gamma(t-\tau_2)}
\\=&\ -{C_2 \over m^2} \int_0^t \left(e^{-\Gamma(t-\tau)} - e^{-2\Gamma(t-\tau)} \right)\d\tau
= -{C_2 \over m^2} \int_0^t \left(e^{-\Gamma\tau} - e^{-2\Gamma\tau} \right)\d\tau
\\=&\ -2{\Gamma\theta \over m} \left( 
\frac{1 - e^{-\Gamma t}}{\Gamma} - \frac{1 - e^{-2\Gamma t}}{2\Gamma} 
\right) =
-\frac{\theta}{m} \left(1 - 2e^{-\Gamma t} + e^{-2\Gamma t}\right) 
\end{aligned}$$
При $t \ll 1/\Gamma$
$$\begin{aligned}
\overline{\Delta a\, \Delta x} &= -{\theta \over m}\Big(
1 - 2\,\left(1 - \Gamma t + \Gamma^2 t^2/2 +\ldots\right) + \left(1 - 2\Gamma t + 4\Gamma^2t^2/2 + \ldots \right)
\Big) 
\\&= -{\theta \over m} \left(
1 - 2 + 1 + 2\Gamma t - 2\Gamma t - \Gamma^2 t^2 + 2\Gamma^2t^2 + \ldots
\right)
\approx -{\theta \over m}\,\Gamma^2t^2
\end{aligned}$$
При $t \gg 1/\Gamma$ 
$$ \overline{\Delta a\,\Delta x} = -\theta/m. $$

Итак, 
$$ \boxed{
\overline{\Delta a\, \Delta x} = -{\theta \over m} \left(1 - 2e^{-\Gamma t} + e^{-2\Gamma t}\right)
= \left\{
\begin{array}{ll}
-\theta\Gamma^2t^2/m, & t \ll 1/\Gamma, \\
-\theta/m, & t \gg 1/\Gamma.
\end{array}\right.
}$$

\end{document}
