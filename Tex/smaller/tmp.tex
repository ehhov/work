\documentclass[a4paper, 12pt]{article}
\usepackage{cmap}
\usepackage[T2A]{fontenc}
\usepackage[utf8]{inputenc}
\usepackage[english,russian]{babel}
\frenchspacing
\usepackage[onehalfspacing]{setspace}
\usepackage{pdflscape}
\usepackage{pgfplots}
\usepackage{amsmath}
\usepackage{amssymb}
\usepackage{mathtools}
\usepackage{enumitem}
\usepackage[
	showframe,
	vmargin=1in, 
	hmargin=1in
]{geometry}
\setlength{\columnsep}{.2in}
\linespread{1.3}
\usepackage[labelsep=period]{caption}

\setcounter{tocdepth}{4}
\usepackage[unicode,
	colorlinks,
	allcolors=blue
]{hyperref}
\frenchspacing
\righthyphenmin=2
\usepackage{indentfirst}
\usepackage{cite}
\usepackage{cmap}
\usepackage{cuted}
\usepackage{lipsum}
\usepackage{multicol}

\newlength{\wordanddescriptionlen}
\newcommand{\wordanddescription}[3][t]{
\setlength{\wordanddescriptionlen}{\widthof{#2}}
\noindent
#2\parbox[#1]{\linewidth-\wordanddescriptionlen}{#3} \\\vspace*{-.3333333\baselineskip}}

%\usepackage{slashbox}

\begin{document}

\lipsum[1]
\hypertarget{first}{\lipsum[1-2]}

\lipsum
\lipsum
\lipsum
\lipsum

\hyperlink{first}{a link to the first text}



\end{document}

\begin{strip}
\centering\large\bf
Дейтрон
\end{strip}

Потенциал 
\begin{equation}
V(r) = \left\{
\begin{array}{ll}
-V_0, & r<a,\\
0,    & r>a
\end{array}
\right.
\label{eq:V}
\end{equation}
приводит к уравнениям движения
\begin{eqnarray}
\psi'' + \frac{2m}{\hbar^2}\left(E + V_0\right) = 0, & r<a, \label{eq:Psi.r<a}\\
\psi'' + \frac{2m}{\hbar^2}\,E = 0, & r>a, \label{eq:Psi.r>a}
\end{eqnarray}
где $m=\dfrac{m_n\,m_p}{m_n+m_p}$,
и решению
\def\vk{\varkappa}
\begin{equation}
\psi = \left\{
\begin{array}{ll}
A\,\sin(kx), & r<a,\\
B\,e^{-\vk x}, & r>a,
\end{array}
\right.
\label{eq:Psi.general}
\end{equation}
где
$$
k = \sqrt{\frac{2m\left(E+V_0\right)}{\hbar^2}}, \quad \vk = \sqrt{\frac{2m(-E)}{\hbar^2}}.
$$

Условия сшивки 
\begin{equation*}
\begin{array}{rl}
A\,\sin(ka) &= B\,e^{-\vk a},\\
kA\,\cos(ka) &= -\vk\, B\,e^{-\vk a},
\end{array}
\end{equation*}
\begin{equation*}
\frac{\sin(ka)}{k\cos(ka)} = \frac{1}{k}\tg(ka) = \frac{1}{-\vk},
\end{equation*}
\begin{equation}
\tg(ka) = - \frac{k}{\vk}.
\label{eq:kkappa}
\end{equation}
\def\Eb{\mathcal{E}_b}
Введем $\Eb = -E$, $u = V_0/\Eb$ и вспомним, чему равны $k$ и $\vk$:
$$
\begin{aligned}
ka =& \sqrt{\dfrac{2m(V_0-\Eb)}{\hbar^2}}\,a = \\
=& \vk \sqrt{\frac{V_0-\Eb}{\Eb}} \, a = \sqrt{u-1}\,\vk a,
\end{aligned}
$$
\begin{equation*}
\frac{k}{\vk} = \sqrt{\frac{V_0-\Eb}{\Eb}} = \sqrt{u-1}.
\end{equation*}
Теперь 
\begin{equation}
\tg\left(\vk a\sqrt{u-1}\right) = - \sqrt{u-1}.
\label{eq:u-1}
\end{equation}

\textbf{Рассмотрим} предельный случай $\Eb\to0$.

Тогда $u\to\infty$, $\tg\to-\infty$ и, так как у дейтрона нет возбужденных уровней, аргумент $\tg$ стремится к $\left(\pi/2+0\right)$. Так как $\Eb=0$, 
$$ k^2 a^2 = \frac{2mV_0}{\hbar^2}\,a^2 = \frac{\pi^2}{4}, $$
\begin{equation} 
\begin{aligned}
V_0 &= \frac{\pi^2\hbar^2}{8ma^2} = \frac{\pi^2\cdot197^2}{8\,\dfrac{939}{2}}\,\frac{1}{a^2} = \\
&=\frac{102\text{ МэВ}\cdot\text{фм}^2}{a^2}.
\end{aligned} 
\label{eq:Eb=0}
\end{equation}
Для $a=1.5$ фм $V_0= 45$ МэВ.


\textbf{Рассмотрим} реальный случай $\Eb = 2.2 \text{ МэВ}\neq 0$.

Тогда $\vk = 0.2307 \text{ фм}^{-1}$. Уравнение~(\ref{eq:u-1}) станет
$$
\tg(\xi) = -\frac{1}{\vk a}\,\xi,
$$
которое можно решить графически.


\begin{figure*}
\centering
\begin{tikzpicture}
\begin{axis}[
	xlabel={$\xi$},
	xmin=0, xmax=3.14,
	ymin=-12, ymax=1,
	legend pos = outer north east,
]
	\addplot[blue]{ -4.33433*x };
		\addlegendentry{$a=1$ фм, {$\xi=1.705$}}
	\addplot[orange]{ -4.33433/1.5*x };
		\addlegendentry{$a=1.5$ фм, $\xi=1.764$}
	\addplot[red]{ -4.33433/2*x };
		\addlegendentry{$a=2$ фм, $\xi=1.819$}
	\addplot[domain=0:3.14,black,samples=100]{ tan(deg(x)) };
\end{axis}
\end{tikzpicture}
\caption{Иллюстрация поиска $\xi$.}
\label{fig:graph}
\end{figure*}

Исходя из рисунка~\ref{fig:graph}, получаем для $\sqrt{u-1} = {\xi}/{\vk a}$, $V_0 = \Eb\,\left(\sqrt{u-1}^2 + 1\right)$
\begin{center}
\begin{tabular}{cccc}
\hline\hline
$a$, фм & $\xi$ & $\sqrt{u-1}$ & $V_0$, МэВ \\
\hline
1 & 1.705 & 7.39 & 122 (102) \\
1.5 & 1.764 & 5.097 & 59 (45) \\
2 & 1.819 & 3.94 & 36 (26) \\
\hline\hline
\end{tabular}
\\\hfill\parbox[t]{.9\linewidth}{\flushleft в скобках указаны величины, полученные по формуле~(\ref{eq:Eb=0}).}
\end{center}







\end{document}










\begin{multline}
-\log L = -\sum\limits_{k=1}^{N_{\mathrm{obs}}}\log\left(N_1\,S_1 + N_{1\mathrm{r}}\,S_{1\mathrm{r}} + N_{1\mathrm{nr}}\,S_{1\mathrm{nr}} + N_{\mathrm{b}}\,\mathrm{Bkg}\right) +\\
 + \left(N_1 + N_{1\mathrm{r}} + N_{1\mathrm{nr}} + N_{\mathrm{b}}\right) - N_{\mathrm{obs}}\,\log\left(N_1 + N_{1\mathrm{r}} + N_{1\mathrm{nr}} + N_{\mathrm{b}}\right)
\end{multline}
\begin{multline}
L = \frac{\left(N_1+N_{1r}+N_{1nr}+N_b\right)^{N_{obs}}}
	{e^{N_1+N_{1r}+N_{1nr}+N_b}}
	\cdot
	\prod_{k=1}^{N_{obs}}\left(N_1\,S_1 + N_{1\mathrm{r}}\,S_{1\mathrm{r}} + N_{1\mathrm{nr}}\,S_{1\mathrm{nr}} + N_{\mathrm{b}}\,\mathrm{Bkg}\right)
\end{multline}

\begin{multline}
\end{multline}


\end{document}

\centering

\textbf{ANA note}: $N_\mathrm{bgr} = 8379\pm459$, $N_1 = 3862\pm79$, $N_{1\mathrm{r}} = 1792\pm120$

\textbf{results}

modified polynomial result \\
\input{/tmp/tabular-mod-poly.tex}
\vskip\baselineskip
original polynomial result \\
\input{/tmp/tabular-unmod-poly.tex}
\vskip\baselineskip
modified exponential result \\
\input{/tmp/tabular-mod-exp.tex}
\vskip\baselineskip
original exponential result \\
\input{/tmp/tabular-unmod-exp.tex}


\newpage

\textbf{variations}

modified polynomial variation in \% \\
\input{/tmp/tabular-mod-poly-var.tex}
\vskip\baselineskip
original polynomial variation in \% \\
\input{/tmp/tabular-unmod-poly-var.tex}
\vskip\baselineskip
modified exponential variation in \% \\
\input{/tmp/tabular-mod-exp-var.tex}
\vskip\baselineskip
original exponential variation in \% \\
\input{/tmp/tabular-unmod-exp-var.tex}

\newpage
\textbf{differences}

modified exp minus poly difference in \% \\
\input{/tmp/tabular-mod-diff.tex}
\vskip\baselineskip
original exp minus poly difference in \% \\
\input{/tmp/tabular-unmod-diff.tex}
\vskip\baselineskip
polynomial mod minus orig difference in \% \\
\input{/tmp/tabular-poly-diff.tex}
\vskip\baselineskip
exponential mod minus orig difference in \% \\
\input{/tmp/tabular-exp-diff.tex}


\end{document}

\clearpage
this should be
$$a = a_0 \pm \delta a \qquad b = b_0 \pm \delta b$$
$$ r = \frac{\mathrm{cov}ab}{\sqrt{Da\,Db}} = \frac{\mathrm{cov}ab}{\delta a\,\delta b} $$
Let $b = \alpha a + c$, where $\alpha = \mathrm{const}$, $c = c_0 \pm \delta c$ -- weakly dependent on $a$ random variable. 
$$Db = \delta b^2 = \alpha^2\delta a^2 + \delta c^2$$
$$r\,\delta a\,\delta b = \mathrm{cov}ab = \mathrm{M}(\alpha a^2 + ca) - a_0b_0= \alpha (a_0^2 + \delta a^2 ) + a_0 c_0 -a_0b_0= \alpha \delta a^2$$
$$\alpha = r\delta b/\delta a$$
$$\delta c = \sqrt{\delta b^2-\alpha^2\delta a^2} = \delta b\sqrt{1-r^2}$$
$$c_0 = b_0 - \alpha a_0 = b_0 - r\frac{\delta b}{\delta a}a$$

$$b = r\frac{\delta b}{\delta a}\cdot a + (b_0 - r\delta b / \delta a \,a\,\pm\,\delta b\sqrt{1-r^2})$$
If $a = x_{\pi^0} = 0.307\pm0.005$, $b = x_{\gamma} = 0.016\pm0.004$, $r=-0.43$, then $\alpha = -0.344$, $c = 0.121608\pm0.003611$, $x_\gamma = -0.344x_\pi + c$

\end{document}
