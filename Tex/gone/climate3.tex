\documentclass[a4paper, 12pt]{article}

\usepackage[utf8]{inputenc}
\usepackage[T2A]{fontenc}
\usepackage[english, russian]{babel}

\usepackage{enumitem}
\setlist{nolistsep}
\usepackage{mathtools}
\usepackage{xcolor}
\definecolor{dimblue}{HTML}{1010aa}
\usepackage[
	colorlinks=true, 
	allcolors=dimblue
]{hyperref}
\usepackage[
	vmargin=1in,
	hmargin=1in
]{geometry}
\linespread{1.3}
\usepackage{indentfirst}
\usepackage{graphicx}
\usepackage[multidot]{grffile}
\usepackage[labelsep=period]{caption}


\begin{document}

\noindent
Гусейнов Керим Демирович, физический факультет
\hfill
23 декабря 2020

\begin{center}
	\textbf{6. Радиационный форсинг RCP2.6}
\end{center}

Радиационный форсинг -- изменения внешней границы атмосферы, 
радиационный баланс (измеряется в Вт/м$^2$). Рассматривается именно 
внешняя граница атмосферы, поскольку это граница двух систем -- Земли 
и космоса. Из космоса, от Солнца, к Земле приходит радиация, которая 
падает на эту границу и частично проходит, частично отражается. 
Коэффициент отражения, очевидно, связан с составом верхних слоев 
атмосферы. Содержание различных веществ в атмосфере и, следовательно, 
в том числе и во внешних слоях, влияет на количество отражаемой 
радиации. Влияние это весьма нетривиально, и его оценки для некоторых 
веществ сопровождаются большими погрешностями. Важно заметить, что не 
все вещества повышают глобальную температуру, увеличивают форсинг, 
некоторые дают отрицательный вклад.

Прогноз климата климата Земли основывается на различных предположениях 
касательно происходящих процессов. Раньше получали календарь 
концентраций, сейчас рассматривают календарь эмиссий. При 
моделировании учитывается множество аспектов жизни человечества на 
планете, то есть используются интегральные оценочные модели. Это может 
звучать громко, но по сути в основном предполагается, что общество, 
экономика, хозяйство и прочее останутся примерно такими же, какими 
являются сейчас (с учетом, естественно, разных предположений по поводу 
энергетики и остального, что влияет на эмиссии непосредственно). 
Несколько моделей, предсказывающих климат Земли в 21 веке, 
продемонстрированы на рисунке~\ref{fig:}. Они обычно называются 
согласно цифре RCP (representative concentration pathways) в конце 
столетия: чем меньше эта цифра, тем меньше форсинг. Как следует из 
названия, RCP2.6 предполагает самый мягкий сценарий развития 
человечества, включающий переход на другие источники топлива и прочие 
усиленные попытки максимального сокращения выбросов углерода.

Нужно заметить, что даже в самом мягком сценарии, когда, как видно, 
выбросы под конец века доходят до доиндустриальных величин, 
концентрация снижается очень слабо, а в первой половине столетия даже 
растет. Это связано с тем, что углерод накапливается, но особо никуда 
не девается из атмосферы и поэтому остается надолго. Еще сильнее этот 
эффект заметен на других характеристиках климата Земли, представленных 
на рисунке~\ref{fig:rcp}. Даже в самом мягком сценарии, RCP2.6, 
средняя температура поверхности Земли снижается слабо, уровень 
мирового океана все равно поднимается, а сентябрьские льды все равно 
сокращаются (хоть и не доходят до нуля как в худшем случае RCP8.5).

\begin{figure}% {{{
	\centering
	\includegraphics[width=\linewidth]{/home/kerim/Download/emis.pdf}
	\caption{Прогнозы эмиссий и концентрации углерода в атмосфере 
	в различных сценариях.}
	\label{fig:}
\end{figure}% }}}
\begin{figure}% {{{
	\centering
	\includegraphics[width=\linewidth]{/home/kerim/Download/rcp.pdf}
	\caption{Прогнозы средней температуры поверхности Земли, 
	сентябрьского льда в северном полушарии, среднего уровня моря и pH 
	мирового океана в различных сценариях.}
	\label{fig:rcp}
\end{figure}% }}}

\end{document}
