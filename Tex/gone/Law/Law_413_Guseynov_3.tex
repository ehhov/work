\documentclass[a4paper, 12pt]{article}

\usepackage[utf8]{inputenc}
\usepackage[T2A]{fontenc}
\usepackage[english, russian]{babel}

\usepackage[
	vmargin=1in,
	hmargin=1in
]{geometry}
\linespread{1.3}
\usepackage{xcolor}
\definecolor{allrefs}{HTML}{1010aa}
\usepackage[colorlinks=true, allcolors=allrefs]{hyperref}
\usepackage{enumitem}
\setlist{nolistsep}


\begin{document}
\noindent
Гусейнов Керим, 413 гр.
\hfill 
\today


\begin{center}\bf
Конституционное право на труд
\end{center}

\textit{Безработный гражданин Иванов обратился в государственный центр занятости с просьбой предоставить ему работу по имеющейся специальности. Однако ему отказали и предложили пройти переподготовку ввиду отсутствия вакансий по соответствующей специальности. Гражданин заявил о нарушении его конституционного права на труд и обратился в суд с требованием предоставить ему рабочее место.}

Единственная регулирующая труд норма в Конституции -- Статья 37 
\begin{enumerate}
\item Труд свободен. Каждый имеет право свободно распоряжаться своими способностями к труду, выбирать род деятельности и профессию. 
\item Принудительный труд запрещен. 
\item Каждый имеет право на труд в условиях, отвечающих требованиям безопасности и гигиены, на вознаграждение за труд без какой бы то ни было дискриминации и не ниже установленного федеральным законом минимального размера оплаты труда, а также право на защиту от безработицы. 
\item Признается право на индивидуальные и коллективные трудовые споры с использованием установленных федеральным законом способов их разрешения, включая право на забастовку. 
\item Каждый имеет право на отдых. Работающему по трудовому договору гарантируются установленные федеральным законом продолжительность рабочего времени, выходные и праздничные дни, оплачиваемый ежегодный отпуск.
\end{enumerate}

Вероятно, гражданин Иванов подразумевает, что, исходя из части 1, волен выбирать род деятельности, который хочет сам, а из части 2 -- не может быть принужден выполнять нежелательную работу. 
На первый взгляд может показаться, что такое толкование статьи справедливо. 
Однако слова ``Каждый имеет право ... выбирать род деятельности ...'' относятся к вольности распоряжения собственными силами, а не праву иметь рабочее место по выбранной гражданином специальности. 
Вторая же часть статьи говорит о принудительном труде, принуждении выполнять какую-либо работу, то есть праве отказываться от работы. 
Гражданин всегда имеет право отказаться от работы, которую ему предлагают, но это не значит, что ему кто-то должен предложить другую работу взамен. 
Цель государственного центра занятости не в содействии свободному распоряжению способностями. 
Нет вообще каких-либо законов, обязывающих содействовать свободному распоряжению способностями, то есть предоставлять необходимые для этого рабочие места. 
Конституционное право гражданина Иванова на труд не нарушается в описанной ситуации. 




\newpage
\begin{center}\bf
Равенство полов 
\end{center}

\textit{Гражданка Сидорова обратилась в Конституционный Суд РФ с жалобой в связи с нарушением ее конституционных прав отдельными нормами Уголовного кодекса РФ с Семейного кодекса РФ. По мнению Сидоровой, данные нормы являются откровенно сексистскими, дискриминационными по отношению к женщинам, поскольку предоставляют им особые привилегии по сравнению с мужчинами, тем самым исходя из заведомой второсортности и неполноценности женщин и нарушая равноправие полов. Сидорова указала на то, что истинный феминизм, идеи которого она разделяет, направлен не на восполнение слабостей женщин дополнительными правами, а на признание настоящего равенства.}

Статья 19 Конституции РФ
\begin{enumerate}
\item  Все равны перед законом и судом. 
\item  Государство гарантирует равенство прав и свобод человека и гражданина независимо от пола, расы, национальности, языка, происхождения, имущественного и должностного положения, места жительства, отношения к религии, убеждений, принадлежности к общественным объединениям, а также других обстоятельств. Запрещаются любые формы ограничения прав граждан по признакам социальной, расовой, национальной, языковой или религиозной принадлежности. 
\item  Мужчина и женщина имеют равные права и свободы и равные возможности для их реализации. 
\end{enumerate}

\textbf{Уголовный кодекс}. 
\\ Статья 53.1 пункт 7: Принудительные работы не назначаются несовершеннолетним, лицам, признанным инвалидами первой или второй группы, беременным женщинам, женщинам, имеющим детей в возрасте до трех лет, женщинам, достигшим пятидесятипятилетнего возраста, мужчинам, достигшим шестидесятилетнего возраста, а также военнослужащим.
\\ Статья 57 пункт 2: Пожизненное лишение свободы не назначается женщинам, а также лицам, совершившим преступления в возрасте до восемнадцати лет, и мужчинам, достигшим к моменту вынесения судом приговора шестидесятипятилетнего возраста.
\\ Статья 58 пункт 1: Отбывание лишения свободы назначается:
\begin{itemize}
\item[а)] лицам, осужденным за преступления, совершенные по неосторожности, а также лицам, осужденным к лишению свободы за совершение умышленных преступлений небольшой и средней тяжести, ранее не отбывавшим лишение свободы, - в колониях-поселениях. С учетом обстоятельств совершения преступления и личности виновного суд может назначить указанным лицам отбывание наказания в исправительных колониях общего режима с указанием мотивов принятого решения;
\item[б)] мужчинам, осужденным к лишению свободы за совершение тяжких преступлений, ранее не отбывавшим лишение свободы, а также женщинам, осужденным к лишению свободы за совершение тяжких и особо тяжких преступлений, в том числе при любом виде рецидива, - в исправительных колониях общего режима;
\item[в)] мужчинам, осужденным к лишению свободы за совершение особо тяжких преступлений, ранее не отбывавшим лишение свободы, а также при рецидиве или опасном рецидиве преступлений, если осужденный ранее отбывал лишение свободы, - в исправительных колониях строгого режима;
\item[г)] мужчинам, осужденным к пожизненному лишению свободы, а также при особо опасном рецидиве преступлений - в исправительных колониях особого режима.
\end{itemize}
Статья 59 пункт 2: Смертная казнь не назначается женщинам, а также лицам, совершившим преступления в возрасте до восемнадцати лет, и мужчинам, достигшим к моменту вынесения судом приговора шестидесятипятилетнего возраста.

\textbf{Семейный кодекс} не предоставляет особых прав женщинам. 

\null \hfill \textbf{Сам ответ на вопрос} \hfill \null

Из процитированного видно, что Уголовный кодекс дает разные права женщинам и мужчинам в некоторых ситуациях. 
Возраст, при котором некоторые законы вступают в силу, не совпадает для мужчин и женщин. 
Пожизненное лишение свободы не назначается женщинам никогда. 
Мужчины отбывают лишение свободы в более строгих колониях за менее тяжкие преступления. 
Помимо перечисленного, Уголовный кодекс предоставляет также дополнительные права женщинам в особых обстоятельствах: при беременности или наличии детей до определенного возраста. 
Особые права женщин в период беременности обоснованы, однако наличие детей не сказывается на правах мужчин и тем самым выделяет женщин. 
Это выделение напрямую говорит, что, с точки зрения Уголовного кодекса, забота о детях -- задача женщин. 
Возложение такой обязанности на женщин неминуемо влечет за собой ограничение их прав на другие роды деятельности. 

В свою очередь, Конституция гарантирует равенство прав и свобод независимо от пола и запрещает любые ограничения прав по каким-либо признакам. 
Лично для меня сложно выделить введенное право гражданина, прочитав, например, Статью 57 пункт 2 УК РФ. 
Однако, очевидно, что право или обязанность человека этой статьей введены, и очевидно, что при формулировке, различающей мужчин и женщин, введенные права или обязанности будут также различать мужчин и женщин. 

Таким образом, можно с уверенностью назвать процитированные статьи УК РФ сексистскими и с меньшей уверенностью заявить, что они нарушают Статью 19 Конституции России. 



\newpage
\begin{center}\bf
Роль и единство федерации
\end{center}

К Федерации относятся теперь:
\begin{itemize}
\item[71 г --] Организация публичной власти. 
\item[71 е --] Научно-технологическое развитие; установление единых правовых основ системы здравоохранения, системы воспитания и образования, в том числе непрерывного образования. 
\item[71 м --] Обеспечение безопасности личности, общества и государства при применении информационных технологий, обороте цифровых данных. 
\item[71 т --] Установление ограничений для замещения государственных и муниципальных должностей, должностей государственной и муниципальной службы, в том числе ограничений, связанных с наличием гражданства иностранного государства либо вида на жительство или иного документа, подтверждающего право на постоянное проживание гражданина Российской Федерации на территории иностранного государства, а также ограничений, связанных с открытием и наличием счетов (вкладов), хранением наличных денежных средств и ценностей в иностранных банках, расположенных за пределами территории Российской Федерации.
\end{itemize}

Стало общей заботой РФ и ее субъектов:
\begin{itemize}
\item[72 п.1 д --] Сельское хозяйство. 
\item[72 п.1 e --] Молодежная политика. 
\item[72 п.1 ж --] Обеспечение оказания доступной и качественной медицинской помощи, сохранение и укрепление общественного здоровья, создание условий для ведения здорового образа жизни, формирования культуры ответственного отношения граждан к своему здоровью. 
\item[72 п.1 ж.1 --] Защита института брака как союза мужчины и женщины; создание условий для достойного воспитания детей в семье, а также для осуществления совершеннолетними детьми обязанности заботиться о родителях. 
\end{itemize}

Ни одно из изменений не дает больше прав субъектам России. Наоборот, некоторые вещи стали относиться к одной лишь Федерации, а некоторые и к РФ, и к ее субъектам. 

Мой субъект Федерации -- город федерального значения Москва. Сложно представить себе проблему Москвы, которая возникает из-за каких-то недостатков в федеративном устройстве России. Кроме того, децентрализация в Москве невозможна, поскольку фактически отсутствует система местного самоуправления: органы наделены незначительными второстепенными полномочиями. Единственное интересное направление -- укрепление роли местного самоуправления. С дополнительными правами оно могло бы, вероятно, лучше заботиться о ЖКХ, здравоохранении и др. Но это не относится к федеративному устройству. 





\newpage
\begin{center}\bf
Нетематические поправки 
\end{center}

Статья 67.1 п.3 относится скорее к главе 2, поскольку говорит о правах особой группы людей. Однако в этом пункте также упоминается понятие правды и России приписывается обязанность ее защищать, так что от главы 3 не чересчур далеко. 

Статья 75 п.5 относится к главе 2, поскольку труд граждан и его вознаграждение есть права граждан, к тому же минимальный размер оплаты труда уже упоминался в статье 37. 

Статья 75 п.6 может относиться к главе 1, поскольку содержит общие замечания касательно пенсии (статья 7); или к главе 2, поскольку некоторые особенности пенсионной системы являются правами граждан (статья 39). Однако глава 1 не может быть изменена, так что остается только второй вариант. 

Статья 75 п.7 относится к главе 2, поскольку определяет права человека на упоминающиеся в содержании элементы. 






\end{document}
