\documentclass[a4paper, 12pt]{article}

\usepackage[utf8]{inputenc}
\usepackage[T2A]{fontenc}
\usepackage[english, russian]{babel}

\usepackage[
	vmargin=1in,
	hmargin=1in
]{geometry}
\linespread{1.3}
\usepackage{xcolor}
\definecolor{allrefs}{HTML}{1010aa}
\usepackage[colorlinks=true, allcolors=allrefs]{hyperref}


\begin{document}

\noindent {Гусейнов Керим, 413 гр.} \hfill {\today}

\begin{center}
\bf Конспект речи Ф. Лассаля о сущности конституции 
\end{center}

\def\paste{
Здесь текст, который должен занимать одну двадцатую часть страницы. Ровно столько я должен написать о каждой странице речи. Или чуть больше. 
}
%\paste\paste\paste\paste
%\paste\paste\paste\paste
%\paste\paste\paste\paste
%\paste\paste\paste\paste
%\paste\paste\paste\paste
% 0
Что такое конституция? В чем состоит сущность конституции? Собрания законов и формальные юридические определения не дают удовлетворительного ответа. 
Они содержат лишь признаки конституции, по которым ее можно узнать, но не говорят о ее сущности. Попробуем ее понять в сравнении с однородными вещами. 
Конституция и закон по сущности однородны, но она должна быть более, чем законом, быть более крепкой и неизменной. Можно сказать, что конституция есть основной закон. 
Основной закон должен быть основой всех законов, причиной их существования в их форме. И фактические отношения силы делают законы тем, чем они есть.

Представим, что все законы пропали. Их бы нельзя было сделать произвольными. Если отнять у монарха власть, он ответит армией. Крупные землевладельцы ответят помощью монарха. 
% 5
Попытки вернуть средневековый цеховой строй разрушат крупное производство, приведут к безработице и бедности людей и их восстаниям. 
Игнорирование интересов крупных банкиров приведет к невозможности получать крупные средства, которые бывают необходимы государству. Противоречащие национальному сознанию законы также будут невозможны. 
Лишение работников личной свободы приведет вновь к восстаниям. Всё перечисленное -- проявления фактических отношений силы. Правовая конституция есть их деликатное письменное выражение. 
Так, можно придать больший политический вес богатым людям, бесконечный вес дворянам землевладельцам, поставить короля выше конституции с помощью армии. 
Хоть армия короля и имеет довольно ограниченное число военных, она может сдерживать недовольства народа благодаря своей организованности и его неорганизованности. 
% 10

Многим кажется, что конституции присущи лишь новейшему времени. Но это не так. Во Франции задолго до революции говорили о вещах, которые король изменить не может. 
Тогда подобные устои не были где-либо записаны, а новейшее время привносит лишь письменные конституции. Причина этому лишь одна -- перемена фактических сил. 
В средневековье малочисленность населения не давала королю формировать собственных солдат, дворяне имели большую силу, конституция была сословной. 
По мере роста враждебного дворянам населения у короля возрастала сила, появлялась собственная армия, возникла абсолютная монархия. Но общество развивалось дальше. 
За 200 лет в Берлине население увеличилось в 27 раз, а армия -- только в 5. Граждане больше не хотят быть управляемой массой и восстают, обнуляя все имевшиеся законы. 
% 15
Но чтобы новая конституция была прочна, она обязана соответствовать действительным силам. После революции национальному собранию следовало преобразовать силу армии, чтобы она не служила средством против нации. 
Однако фактических изменений не произошло, и через 8 месяцев национальное собрание разогнали, а людей обезоружили. Более того, составление письменной конституции казалось гражданам самым важным шагом. 
Даже если бы национальное собрание постановило конституцию, это бы не изменило ничего. Принятая тогда королем конституция была даже близка к нужной народу. 
Но имеющий фактическую силу король изменял ее столько, сколько было нужно. Такая конституция, очевидно, умирает, и призывы держаться за нее лишь доказывают это. 
Если конституция не соответствует реальным силам, она уже мертва. Она может измениться в две противоположные стороны: под натиском фактической силы короля или под натиском очередной революции. 
% 20
Все это относится не только к военной силе, но и к служителям правосудия, чиновникам, администрации и другим организованным средствам силы общества. 



\begin{center}
\bf Соответствует ли текущая конституционная реформа представлениям Лассаля?
\end{center}

К сожалению, на момент прочтения задания знаний о реформе у меня не было никаких. Могу лишь заметить, что общественные настроения не сильно изменились от новостей о реформе конституции; не сильнее, чем от пакета Яровой, пенсионной реформы и подобных изменений. То есть правительство имеет достаточно большую силу для совершения реформы, а значит, она соответствует представлениям Лассаля. 

После прочтения 
\href{https://ru.wikipedia.org/wiki/%D0%9F%D0%BE%D0%BF%D1%80%D0%B0%D0%B2%D0%BA%D0%B8_%D0%BA_%D0%9A%D0%BE%D0%BD%D1%81%D1%82%D0%B8%D1%82%D1%83%D1%86%D0%B8%D0%B8_%D0%A0%D0%BE%D1%81%D1%81%D0%B8%D0%B8_(2020)}{статьи на Википедии}
стало понятно, что полномочия президента расширяются, где-то существенно, где-то нет, но он становится более обособленной фигурой. Однако, честно говоря, я не знаю, ограничен ли президент в исполнении своих желаний на данный момент, поскольку все законы практически единогласно принимаются, и непонятно может ли кто-то противостоять воле президента. В той же речи Лассаля есть слова ``Служители монархии – практики, а не краснобаи'', ``Государям служат гораздо лучше, чем вам!''. По моим представлениям, в России нет персоны, которой служат лучше, чем В.~В.~Путину, то есть фактическая сила находится в его руках. Письменное подкрепление существующего положения дел и является целью конституции в представлении Лассаля, так что соответствие имеется, как и было заключено раньше. 



\newpage
\begin{center}
\bf Справедливо ли заявление политической партии N?
\end{center}

\textit{Политическая партия N заявила, что государственная поддержка патриотизма, обеспеченная бюджетными средствами, является де-факто проявлением государственной идеологии и прямо противоречит Конституции России.}

Статья 13 пункты 1 и 2 Конституции РФ гласят, что в Российской Федерации признается идеологическое многообразие, а также, что никакая идеология не может устанавливаться в качестве государственной или обязательной. То есть государственная идеология противоречит Конституции. 

Идеология есть система концептуально оформленных представлений и идей, которая выражает интересы, мировоззрение и идеалы различных субъектов политики. 

Патриотизм есть нравственный и политический принцип, социальное чувство, содержанием которого является любовь к Родине и готовность пожертвовать своими интересами ради неё. 

Само понятие жертвы ради чего-либо подразумевает, что у названного чего-либо должны быть свои интересы, противоречащие интересам жертвующего. Поддержка или воспитание жертвенности людей есть проявление государственной идеологии, а значит, действительно, противоречит Конституции. 


\end{document}
