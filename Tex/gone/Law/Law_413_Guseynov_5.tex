\documentclass[a4paper, 12pt]{article}

\usepackage[utf8]{inputenc}
\usepackage[T2A]{fontenc}
\usepackage[english, russian]{babel}

\usepackage[style=russian]{csquotes}

\usepackage[
	vmargin=1in,
	hmargin=1in
]{geometry}
\linespread{1.3}
\usepackage{xcolor}
\definecolor{allrefs}{HTML}{1010aa}
\usepackage[colorlinks=true, allcolors=allrefs]{hyperref}
\usepackage{enumitem}
\setlist{nolistsep}


\begin{document}
\noindent
Гусейнов Керим, 413 гр.
\hfill 
\today

\begin{center}\bf
Горизонтальность отношений
\end{center}

\textit{Пенсионерка Иванова решила открыть банковский счет в ПАО \enquote{Сбербанк} для получения своей пенсии в размере 11\,000 рублей.}

%Почему отношения Сбербанка и бабушки горизонтальные и включены в предмет гражданского права?

Три основных принципа гражданско-правовых отношений:
\begin{itemize}
\item Автономия воли -- каждая сторона свободно и по своему усмотрению определяет, вступать ли в отношения, с кем вступать, на каких условиях. Цели и причины выбора остаются частным делом каждой стороны. 
\item Юридическое равенство -- ни одна сторона отношений не может принуждать другую к чему-либо и пользоваться привилегиями, с которыми другая сторона заведомо не согласна. 
\item Имущественная самостоятельность -- каждая сторона отношений, как правило, является обладателем какого-либо имущества, в связи с которым и вступает в отношения. На него не могут посягать другие лица. 
\end{itemize}

В описанной ситуации Иванова самостоятельно приняла осознанное решение вступить в отношения со Сбербанком, выбрав (или планируя выбрать в отделении банка) подходящие ей условия осуществления этих отношений. 
Сбербанк, в свою очередь, при обращении Ивановой будет рассматривать ее как клиента и принимать выбор со своей стороны. 
Обе стороны, действуя естественным для них в данной ситуации образом, образуют отношение, удовлетворяющее принципу автономии воли. 

Пенсионерка Иванова приняла решение открыть банковский счет без вмешательства Сбербанка или какого-либо другого частного лица, ее права не были нарушены. 
При ее обращении в Сбербанк работники банка будут принимать решения о различных характеристиках отношений с Ивановой, причем на результат будут влиять фактические данные, а не убеждения клиента. 
В процессе принятия решений со своей стороны Сбербанк также не будет нарушать права Ивановой, и в итоге она примет окончательное решение о вступлении в отношения. 
Все перечисленное подчиняется принципу юридического равенства. 

Также при принятии решений обе стороны принимают участие в распределении имущественных благ на свои и чужие. 
Иванова заинтересована в использовании предоставляемых банком услуг для работы со своей пенсией.
Банк тоже получает выгоду при обслуживании пенсионных счетов. 
Это распределение имущества происходит при обоюдном согласии и без посягательства на чужое, что удовлетворяет принципу имущественной самостоятельности. 

Отношения пенсионерки и банка горизонтальные, поскольку в рамках этих отношений обе стороны имеют одинаковые права. 


\newpage
\begin{center}\bf
Отказ персонала автосалона
\end{center}

\textit{Гражданин Сидоров в результате аварии оказался в полностью парализованном состоянии, однако находится в здравом уме и перемещается на специальном кресле при помощи своих родственников. 
Он может общаться и обозначать свои потребности системой особых мимических знаков.
Сидоров сообщил близким, что он хочет приобрести автомобиль, в связи с чем его привезли в один из автосалонов.
Однако менеджеры и консультанты салона категорически отказались заключать договор с Сидоровым, сославшись на то, что он, во-первых, ``полностью парализован и, очевидно, недееспособен'', а во-вторых ``в любом случае не сможет пользоваться автомобилем''.}
% правомерен ли отказ сотрудников автосалона? 

Термин недееспособность строго определен, и гражданин признается недееспособным только при наличии психического расстройства, в результате которого он не осознает значения своих действий или не может руководить ими. 
Очевидно, Сидоров не является недееспособным гражданином, поскольку остался в здравом уме. 
Аргумент, ссылающийся на отсутствие у Сидорова способности пользоваться автомобилем самостоятельно, тоже не обоснован, поскольку право на имущество и возможность воспользоваться предоставляемыми имуществом благами -- разные характеристики физического лица. 
Сидоров имеет право свободно выбирать предоставляемые автосалоном услуги. 
Может показаться, что принцип автономии воли позволяет обеим сторонам не соглашаться на сделку и не вступать в отношения по собственному желанию, однако продавец, являясь продавцом и имея в связи с этим дополнительные обязанности, не может отказать в обслуживании на личных основаниях. 
Таким образом, отказ сотрудников автосалона неправомерен. 

\end{document}
