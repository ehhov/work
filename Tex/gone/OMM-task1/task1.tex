\documentclass[a4paper,12pt]{article}
%\usepackage{style}

\usepackage[T2A]{fontenc}
\usepackage[utf8]{inputenc}
\usepackage[english,russian]{babel}

\usepackage{indentfirst}
\usepackage{mathtext}
\usepackage{mathtools}
\usepackage{amsmath}
\usepackage{amssymb}
\usepackage{amsfonts}

\setcounter{tocdepth}{4}
\usepackage[unicode]{hyperref}
\hypersetup{colorlinks=false}
\hypersetup{linktocpage}
\frenchspacing
\righthyphenmin=2
\usepackage{indentfirst}
\usepackage{cite}
\usepackage{cmap}
\usepackage{nicefrac}

\usepackage[labelsep=period, margin=1cm]{caption}

\usepackage{geometry}
\geometry{top=1in, bottom=1.3in, right=1in, left=1in}

\linespread{1.25}


\def\mysection#1{\vspace{\baselineskip}{\noindent\large\centering\bf#1\par}}
\def\d{\mathrm{d}}

\begin{document}

\mysection{Постановка задачи}

Необходимо решить задачу с квазилинейным уравнением переноса:
\begin{equation}
%\arraycolsep=0pt
\def\arraystretch{1.4}
	\left\{\begin{array}{ll}
		\dfrac{\partial u}{\partial t} + \dfrac{2ue^{u^2}}{1+e^{2u^2}}\,\dfrac{\partial u}{\partial x}=0,&\text{при }x\in(0,1),\ t>0,\\
		u\big|_{t=0}=1+x^2,&\text{при }x\in [0,1],\\
		u\big|_{x=0}=e^{-t},&\text{при }t\geq 0.
	\end{array}
	\right.
\label{eq:thetask}
\end{equation}
Введем равномерные сетки по $x$ и по $t$: $x_n=nh$, при $n=\overline{0, [1/h]}$; $t_k=k\tau$, при $k\in \mathbb{Z}_+$.
Рассмотрим разностную схему
\begin{equation}
\def\arraystretch{1.4}
	\left\{\begin{array}{ll}
	\dfrac{u^{k+1}_{n+1}-u^k_{n+1}}{\tau}+\dfrac{f^{k+1}_{n+1}-f^{k+1}_n}{h}=0,&\text{при }0<n\leq[1/h]-1,\ k>0,\\
	u^0_n=1+(nh)^2,&\text{при }0\leq n\leq [1/h],\\
	u^k_0=e^{-k\tau},&\text{при }k\geq 0,
	\end{array}
	\right.
\label{eq:scheme}
\end{equation}
где $u^k_n=u(x=x_n,\,t=t_n)$, $f^k_n=f(u^k_n)$, а
$f(u)=\mathrm{arctg}\left(u^2\right).$

\mysection{Аппроксимация}
Пусть $u\in C^2(0,1]\times(0,\infty)$. Тогда 
\begin{equation*}
\begin{split}
	&L_h[u]-L[u]=\,\frac{u(x+h, t+\tau)-u(x+h,t)}{\tau}+\frac{f\big(u(x+h,t+\tau)\big)-f\big(u(x,t+\tau)\big)}{h}-\\
	&-\big(u_t+f'(u)\,u_x\big)\Big|_{x=x+h\atop t=t+\tau}=\frac{u-\left(u+u_t\cdot(-\tau)+\overline{u}_{tt}\tau^2/2\right)}{\tau}+f'(u)\times\\
	&\times\frac{u-\left(u+u_x\cdot(-h)+\overline{u}_{xx}h^2/2\right)}{h}-u_t+f'(u)\,u_x=-\overline{u}_{tt}\,\tau/2-f'(u)\,\overline{u}_{xx}\,h/2=\mathcal{O}(\tau+h),
\end{split}
\end{equation*}
где $\overline{u}_{tt}$ и $\overline{u}_{xx}$ -- производные, взятые в интервалах $(t,t+\tau)$ и $(x,x+h)$ соответственно. То есть выбранный оператор $L_h$ аппроксимирует исходный дифференциальный оператор в первом порядке по $t$ и по $x$.

\mysection{Устойчивость}
Исследуем устойчивость методом гармоник. Предположим, что в задаче (\ref{eq:scheme}) при изменении начальных условий на $\delta\varphi_n=Ae^{i\alpha x_n}$ решение меняется на $\delta u^k_n$, причем $\delta u^{k+1}_n/\delta u^k_n=q$ и не зависит от $k$ и $n$. Тогда, подставив это выражение в дифференциальное уравнение, получим
$$\frac{1-1/q}{\tau}+f'(u^k_n)\,\frac{1-e^{-i\alpha h}}{h}=0$$
Обозначим $f'(u^k_n)=c$. Тогда $q=\dfrac{1}{1-\frac{c\tau}{h}(1-e^{-i\alpha h})}$, а
\begin{equation*}
\begin{split}
\big|q\big|=\frac{1}{\frac{c\tau}{h}(1+\frac{c\tau}{h})\big(\frac{h}{c\tau}+1+\frac{1}{\frac{h}{c\tau}+1}-2\cos\alpha h\big)}\leq\frac{1}{\frac{c\tau}{h}(1+\frac{c\tau}{h})\big(\frac{h}{c\tau}-1+\frac{1}{\frac{h}{c\tau}+1}\big)}=1.
\end{split}
\end{equation*}
Это значит, что в рассмотренном случае схема будет устойчива относительно начальных условий при любых $h$ и $\tau$, при которых достаточно ограничиться первым членом разложения $\partial_x f(u)$.

Предположим теперь, что в той же задаче (\ref{eq:scheme}) при изменении уже граничных условий на $\delta\psi_n=Ae^{i\alpha t_n}$ решение меняется на $\delta u^k_n$, причем $\delta u^{k}_{n+1}/\delta u^k_n=q$ и не зависит от $k$ и $n$. Снова подставив это выражение в дифференциальное уравнение, получим
$$\frac{1-e^{-i\alpha\tau}}{\tau}+f'(u^k_n)\,\frac{1-1/q}{h}=0\quad\text{и}\quad q=\frac{1}{1-\frac{h}{c\tau}(1-e^{-i\alpha\tau})}.$$
Аналогичными преобразованиями получаем, что схема при тех же ограничениях устойчива и относительно граничных условий. 

\mysection{Аналитическое решение}
Для поиска точного решения воспользуемся методом характеристик. Характеристики задачи (\ref{eq:thetask}) находятся из уравнения
$$\frac{\d t}{1}=\frac{\d x}{\frac{2ue^{u^2}}{1+e^{2u^2}}}.$$
Откуда можно представить характеристики в параметрическом виде
$$t-a=\frac{1+e^{2u^2_0}}{2u_0e^{u^2_0}}\,(x-b),\quad u_0\big|_{a=0}=1+b^2,\;\;u_0\big|_{b=0}=e^{-a}.$$
Получаются семейства прямых\vspace{.2\baselineskip}\par\noindent
\begin{equation}
t=\frac{1+e^{2e^{-2a}}}{2e^{-a}e^{e^{-2a}}}\,x+a,\quad a\geq0,\ \ \ \ \ \  t=\frac{1+e^{2(1+b^2)^2}}{2(1+b^2)e^{(1+b^2)^2}}\,(x-b),\quad b\in[0,1].
\label{eq:ch}
\end{equation}
Они представлены на рисунке \ref{fig:character}.

\begin{figure}
	\centering
	\includegraphics[width=.49\linewidth]{chrts.pdf}
	\includegraphics[width=.49\linewidth]{intersection.pdf}
	\caption{Характеристики задачи (\ref{eq:thetask}) (слева) и их пересечение при $x>1$ (справа).}
	\label{fig:character}
\end{figure}

Также из уравнений (\ref{eq:ch}) можно получить решение в неявном виде, поскольку $u(x,t)=u_0(0,a(x,t))\text{ или }u_0(b(x,t),0)$ при $t>\frac{1+e^2}{2e}\,x$ и $t<\frac{1+e^2}{2e}\,x$ соответственно. Функция для неявного представления имеет вид
\begin{equation}
\def\arraystretch{1.8}
F(x,t,u)=\left\{\begin{array}{ll}\displaystyle
\frac{1+e^{2u^2}}{2ue^{u^2}}\,x-t-\ln(u), &\text{при }t>x/e, \\\displaystyle
\frac{1+e^{2u^2}}{2ue^{u^2}}\,(x-\sqrt{u-1})-t, &\text{при }t<x/e. \\
\end{array}\right.
\label{eq:imp}
\end{equation}
Решить уравнения $F(x,t,u)=0$ в явном виде не представляется возможным. 

Производная функции $F$ вблизи точки $u=1$ принимает большие значения, поэтому использовать метод Ньютона для численного нахождения $u$ нерационально. Найденная методом деления отрезка пополам с точностью $\big|F\big|<10^{-9}$ функция $u(x,t)$ представлена на рисунке \ref{fig:impl}. Как видно, на прямой $t=\frac{1+e^2}{2e}\,x$ решение претерпевает слабый разрыв.

\begin{figure}[!p]
\includegraphics[width=\linewidth]{solution-comp-impl.pdf}
\caption{Численное решение уравнения $F(x,t,u)=0$.}
\label{fig:impl}
\end{figure}


\def\steps{для шагов $\tau=0.002,\ h=0.002$}
\mysection{Алгоритм решения}
Для численного решения задачи воспользуемся схемой бегущего счета. 
Вычисление начинается с расчета значений функции в начальных и граничных точках. Для расчета функции в новых точках понадобится решать следующее уравнение
$$ \frac{h}{\tau}\,u^{k+1}_{n+1}+f^{k+1}_{n+1}=  \frac{h}{\tau}\,u^{k}_{n+1}+f^{k+1}_{n},$$
где функция в предыдущих по времени и координате точках известна. Корень этого уравнения можно найти по методу Ньютона, приняв за начальную точку любое известное соседнее значение функции. Так, зная функцию в точке (0,$\tau$) и на прямой $t=0$, вычисляются ее значения на прямой $t=\tau$ и так далее. Само численное решение \steps\ представлено на рисунке \ref{fig:solution}. Решения при шагах в 4 и в 32 раза больших представлены на рисунке \ref{fig:bigger}. Решение для неравных по координате и по времени шагов представлено на рисунке \ref{fig:odd}. 

\mysection{Точность решения}
Было произведено несколько расчетов для сеток с одинаковыми вдоль двух осей шагами: $0.002,\,0.004,\,0.008,\,0.016,\,0.032,\,0.064$, а также единственный расчет с неравными шагами $\tau=0.064,\ h=0.002$. Погрешности решений оценивались в среднем: 
$$z=\sum\limits_{k=0}^{N_t}\sum\limits_{n=0}^{N_x}\frac{\big|u(M_n^k)-\tilde{u}(M_n^k)\big|}{N_tN_x},$$
где $u$ -- решение дифференциальной задачи, $\tilde{u}$ -- решение, найденное по неявному аналитическому выражению, а суммирование производится лишь по набору значений, присутствующих во всех сетках одновременно. Краткая информация о точности решений приведена в таблице \ref{tab:precision}. Из данных видно, что погрешность увеличивается медленнее, чем шаг. Вероятно, это происходит из-за нелинейности исходной задачи. 

\begin{table}
\centering
\caption{Точность решений при разных шагах сетки.}
\begin{tabular}{| c c | c | c |}
\hline
$\tau$& $h$ & $z$ & $z/(\tau+h)$ \\
\hline
0.064 &0.002& 0.008143&0.1234\\
0.002 &0.002& 0.001138&0.2846\\
0.004 &0.004& 0.002052&0.2565\\
0.008 &0.008& 0.003089&0.1931\\
0.016 &0.016& 0.004855&0.1517\\
0.032 &0.032& 0.008071&0.1261\\
0.064 &0.064& 0.012900&0.1008\\
\hline
\end{tabular}
\label{tab:precision}
\end{table}




\begin{figure}[!p]
\includegraphics[width=\linewidth]{{solution-comp-.002.}.pdf}
\caption{Численное решение задачи (\ref{eq:scheme}) \steps.}
\label{fig:solution}
\end{figure}

\begin{figure}[!p]
\centering
\includegraphics[width=.49\linewidth]{{solution-comp-.008}.pdf}
\includegraphics[width=.49\linewidth]{{solution-.064}.pdf}
\caption{Численные решения задачи для шагов $\tau=h=0.008$ (слева) и $\tau=h=0.064$ (справа).}
\label{fig:bigger}
\end{figure}

\begin{figure}[!p]
\includegraphics[width=\linewidth]{solution-odd.pdf}
\caption{Численное решение задачи для шагов $\tau=0.064$, $h=0.002$.}
\label{fig:odd}
\end{figure}














\end{document}