\documentclass[a4paper, 12pt]{article}

\usepackage[utf8]{inputenc}
\usepackage[T2A]{fontenc}
\usepackage[english, russian]{babel}

\usepackage{mathtools}
\usepackage{xcolor}
\definecolor{allrefs}{HTML}{1010aa}
\usepackage[colorlinks=true, allcolors=allrefs]{hyperref}
\usepackage[margin=1in]{geometry}
\linespread{1.3}

\def\mailto#1{\href{mailto:#1}{#1}}
\def\Lb{\varLambda_b}

\begin{document}

\begin{center}
{\bf Изучение многочастичных распадов $\Lb$ на~Большом адронном коллайдере }

\textit{Гусейнов Абдул-Керим Демирович}\footnote{\mailto{Kerim.Guseinov@cern.ch}}

\textit{МГУ им. М.\,В.~Ломоносова, физический факультет, кафедра общей ядерной физики}
\end{center}

% 46 words
Изучение тяжелых барионов и их многочастичных распадов важно для проверки Стандартной модели и поисков новой физики, поскольку в многопетлевых диаграммах Фейнмана возрастает влияние возможных новых частиц. 
$\Lb$ -- самый легкий прелестный барион, его распады с переходом барионного числа протону интересны еще и для изучения адронизации кварков. 
При изучении распадов по спектру инвариантных масс важную роль играет модель соответствующих вкладов в спектр. 
В зависимости от нее, конечный результат может иметь разную систематическую погрешность, быть более или менее стабильным по отношению к статистическим погрешностям экспериментальных данных. 

Исследование основано на данных, соответствующих интегральной светимости $3.0$ фб$^{-1}$, собранных детектором LHCb Большого адронного коллайдера. В работе изучаются вероятности распадов $\Lb\to D^{+}p\pi^-\pi^-$ и $\Lb\to D^{*+}p\pi^-\pi^-$ с $D^{*+}\to D^+\pi^0\,\big/\,D^+\gamma$, $D^{+}\to K^-\pi^+\pi^+$, в нормировке на канал $\Lb\to\varLambda_c\pi^+\pi^-\pi^-$, $\varLambda_c\to pK^-\pi^+$. 
Для спектров инвариантных масс $D^+p\pi^-\pi^-$ и $\varLambda_c\pi^+\pi^-\pi^-$ строятся модели вкладов основных (эксклюзивных) и инклюзивных распадов вблизи массы $\Lb$, изучается стабильность результата по отношению к характеристикам модели и по отношению к статистическим погрешностям данных. 

Анализ производится с помощью программного пакета Ostap на базе RooFit, ROOT. 


\end{document}
